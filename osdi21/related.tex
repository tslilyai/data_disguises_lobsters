%-------------------------------------------------------------------------------
\section{Background and Related Work}
%-------------------------------------------------------------------------------

Companies have developed frameworks to avoid deletion bugs (e.g., DELF~\cite{delf} at Facebook), but many
applications decorrelate only coarsely by associating remnants of data with a global placeholder for
all deleted users, or do not decorrelate at all (e.g., replacing usernames with a pseudonym).

\lyt{(cite Reddit, Lobsters, others?)}

The right to be forgotten has also been formally defined by Garg et
al.~\cite{garg}, where correct deletion corresponds to the notion of
leave-no-trace: the state of the data collection system after a user requests to be forgotten should
be left (nearly) indistinguishable from that where the user never existed to begin with. While
\sys{} uses a similar comparison, their formalization assumes that users operate
independently, and that the centralized data collector prevents one user's data from influencing
another's.

Other related works:
\begin{itemize}
    \item Deceptive Deletions for protecting withdrawn posts: https://arxiv.org/abs/2005.14113
    \item "My Friend Wanted to Talk About It and I Didn't": Understanding Perceptions of
        Deletion Privacy in Social Platforms, user survey https://arxiv.org/pdf/2008.11317.pdf;
        talk about decoy deletion, prescheduled deletion strategies~\cite{myfw}
    \item Contextual Integrity
    \item ML Unlearning
    \item k-anonymization, pseudonymization
\end{itemize}
