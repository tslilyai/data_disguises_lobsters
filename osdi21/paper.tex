%%%%%%%%%%%%%%%%%%%%%%%%%%%%%%%%%%%%%%%%%%%%%%%%%%%%%%%%%%%%%%%%%%%%%%%%%%%%%%%%
% Template for USENIX papers.
%
%%%%%%%%%%%%%%%%%%%%%%%%%%%%%%%%%%%%%%%%%%%%%%%%%%%%%%%%%%%%%%%%%%%%%%%%%%%%%%%%

\documentclass[letterpaper,twocolumn,10pt]{article}
\usepackage{usenix-2020-09}

% to be able to draw some self-contained figs
\usepackage{tikz}
\usepackage{amsmath}
\usepackage{hyperref}
\usepackage[normalem]{ulem}
\usepackage{listings, listings-rust}

\newcommand{\sys}{DeCor} 
\newcommand{\plinked}{$p_\text{linked}$} 
\newcommand{\premnant}{$p_\text{remnant}$} 
\newcommand{\uidkey}{$uid_\text{key}$} 
\newcommand{\gidkey}{$gid_\text{key}$} 
\newcommand\lyt[1]{\textcolor{green!55!blue}{[lyt: {#1}]}}
\newcommand{\tabitem}{~~\llap{\textbullet}~~}
%\renewcommand\lyt[1]{}

\definecolor{codegreen}{rgb}{0,0.4,0}
\definecolor{codegray}{rgb}{0.5,0.5,0.5}
\definecolor{codepurple}{rgb}{0.58,0,0.82}
\definecolor{backcolour}{rgb}{0.95,0.95,0.92}

\lstdefinestyle{rust}{
    %backgroundcolor=\color{backcolour},   
    commentstyle=\color{codegreen},
    keywordstyle=\color{magenta},
    numberstyle=\tiny\color{codegray},
    stringstyle=\color{codepurple},
    basicstyle=\ttfamily\scriptsize,
    breakatwhitespace=false,         
    breaklines=true,                 
    captionpos=b,                    
    keepspaces=true,                 
    numbers=left,                    
    numbersep=3pt,                  
    showspaces=false,                
    showstringspaces=false,
    showtabs=false,                  
    tabsize=2
}
\lstset{style=rust}

%-------------------------------------------------------------------------------
\begin{document}
%-------------------------------------------------------------------------------

%don't want date printed
\date{}

% make title bold and 14 pt font (Latex default is non-bold, 16 pt)
\title{\Large \bf \sys{}: Ensuring the Right To Be Forgotten (and Remembered) in Web Applications}

%for single author (just remove % characters)
%\author{
%{\rm Lillian Tsai}\\
%MIT
%\and
%{\rm Malte Schwarzkopf}\\
%Brown University
%% copy the following lines to add more authors
%\and
%{\rm Eddie Kohler}\\
%Harvard University
%} % end author

\author{
{\rm Anonymous Authors}\\
} % end author

\maketitle

%-------------------------------------------------------------------------------
\begin{abstract}
%-------------------------------------------------------------------------------
Web services increasingly face legal requirements to protect users' privacy (e.g.,\ the GDPR). These
    laws require companies to properly delete and anonymize users’ data when a user requests to
    remove their data. And beyond current legal requirements, many users would rest easier if they
    knew that they can unsubscribe from a web service at any time, without losing their data and
    with the opportunity to come back later.

    But unsubscribing users without compromising either the leaving user's privacy or the application
    experience for the remaining users is difficult. Removing too much information meets privacy
    requirements, but may yield surprising application behavior (e.g.,\ disappearing content);
    removing too little risks exposing the identity of unsubscribed users via inference attacks.

    In this paper, we propose \sys{}, a new system for safe unsubscription and resubscription of
    users in database-backed web services. \sys{} helps application developers specify fine-grained
    privacy policies to meet de-identification requirements, and goes beyond: with \sys{}, it is
    possible for users to switch between a privacy-preserving unsubscribed mode and an
    identity-revealing subscribed mode at any time without permanently losing their data. This
    facilitates important new web service paradigms, such as users granting a time-limited "lease"
    of data to a service instead of having a permanent service account.  
\end{abstract}

%-------------------------------------------------------------------------------
\section{Introduction}
%-------------------------------------------------------------------------------

\subsection{Motivation} 

Web application companies face increasing legal requirements to protect users’ data. These
requirements pressure companies to properly delete and anonymize users' data when a user requests to
\emph{unsubscribe} from the service (i.e.,\ revoke access to their personal data).
For example, the GDPR requires that any user data remaining after a user unsubscribes is
\emph{decorrelated}, i.e., cannot be (directly or indirectly) used to identify the user~\cite{gdpr}.  

In this paper, we propose \sys, a new approach to managing user identities in web applications.
\sys~meets the decorrelation requirements in the GDPR, and goes beyond: with \sys, it is possible
for users to switch between a privacy-preserving unsubscribed mode and an identity-revealing
subscribed mode at any time. This facilitates important new web service paradigms, such as users
granting a time-limited "lease" of data to a service instead of having a permanent service account.

%Furthermore, keeping identifying and personal data when no longer strictly necessary increases
%companies' liability: the GDPR and other laws mandate that companies retain only user data that is
%relevant and necessary for their applications' purposes. To increase users' control over their data
%and decrease the amount of incriminating data stored in the application at any one point, users
%should be able to freely unsubscribe from the service to enter a privacy-preserving mode, and later
%resubscribe when they wish to use the service. 

\subsection{Goals} 
\sys's goal is to provide the following properties while preserving an application's 
semantics: 
\begin{description} 
    \item[Decorrelation.] Decorrelation ideally guarantees that it is impossible to distinguish
        between two records formerly associated with the same unsubscribed user and two records from
        different unsubscribed users.  
    %\item Deletion Correctness: Deletion of user records should correctly conform to application
        %semantics: for example, post deletion could remove the post and its underlying comments, or
    %simply anonymize the post and keep all content accessible.  
    \item[Resubscription.] Users should be able to easily switch between a privacy-preserving unsubscribed mode 
       and an identity-revealing subscribed mode, without permanently losing their application data.  
\end{description}

\sys~must implement these properties while ensuring (1) performance comparable to today’s
widely-used databases, and (2) easy adoption (decorrelation should be
automated without needing to modify application schemas or semantics).

\subsection{Decorrelation Guarantees} 
We address applications in which application data consists of \emph{data records} and computations
(such as aggregations) that may be performed over these data records. Data records are considered
sensitive, private data records when they contain \emph{user identifiers}; for example, a row in a
table containing a column of user IDs would be a user's private data record. Data records containing
multiple, potentially different, user IDs are considered shared data records private to the
identified users. We refer to data records belonging to unsubscribed users as \emph{remnants}.

Perfect decorrelation is achieved when queries to the application reveal no information allowing an
observer to determine if two distinct user data records belong to the same user who has since
unsubscribed (are \emph{linked}), or belong to two different (real or unsubscribed) users. Observers
gain no information that allows them to distinguish the two scenarios. 

More formally, given user data records and a subset of $R$ of these records that are remnants, we
divide the $R$ remnants among some number $N$ of unsubscribed users.  Perfect decorrelation
guarantees that any division of the $R$ remnants among $N$ users is equally likely: each remnant is
equally likely to be linked with any other remnant. 

~\lyt{I toyed with saying that $P(N = R) = P(N \le R)$: the probability that every distinct remnant
belongs to a distinct user equals the probability that any user owns multiple remnants, but this
seemed unnecessarily complex when I believe that the above statement captures decorrelation.} 

~\lyt{Note: whether $N$ is known or not also plays into the probability analysis}

~\lyt{Note: this is the definition in which ghosts can be distinct from real users, so we're not
trying to pretend that remnants cannot be identified}

In practical settings, however, perfect decorrelation is likely impossible: for example, a reposted
screenshot may leak user IDs. Furthermore, an observer who can see application queries over time, or
search web archives, can detect when a user unsubscribes and refer to prior snapshots in which user
data records may have had the same user ID.  Given these limitations, we seek to achieve the maximum
decorrelation guarantees possible while assuming that the content of user data does not itself leak
identifying information, and that observers cannot access past application database
state.~\lyt{Note: UIDs can also include things like email addrs, phone number, etc; so the notion of
``identifiers'' might be more broad than stated here}.

\paragraph{Global Placeholder.}
A common strawman solution to decorrelation is to replace all unsubscribed user IDs with one global
placeholder ID. This means that all remnants can be identified by one distinct UID, which we call
$GP$.

A global placeholder can provide either complete privacy or no privacy at all. We first begin by
assuming an observer has complete information about how many users are in the system ($N$).
\begin{itemize}
    \item If $N=1$ and only one user has unsubscribed, all remnants can be linked back to that user's
        identity: any data record with UID equal to $GP$ belongs to that user.
    \item If all users unsubscribe, two remnants are equally as likely to belong
        to any two of the $N$ individual users as they are to belong to one single user. 
\end{itemize}
The amount of linkable information~\lyt{(need to define this)}, and the resulting deviance from
perfect decorrelation, decreases as more users unsubscribe.
Let us assume that an observer knows $N$ and $R$, and the number of data records per user is
uniformly distributed. Let $r = \frac{R}{N}$ be the number of remnants per unsubscribed user. 
Then an observer has probability $p_{\text{exact}}$ of guessing the correct combination of
records per users, where  
$$p_{\text{exact}} = \frac{(r!)^N}{R!}$$
An observer has probability $p_{\text{user}}$ of guessing the correct combination for one user where 
$$p_{\text{user}} = \frac{r!(R-r)!}{R!}$$
\lyt{TODO---non-uniform distributions, not knowing $N$, not knowing $R$?}

Using a global placeholder, however, makes resubscription challenging: users can no identify which
unsubscribed data records belong to them if all user-specific data has been erased from the
system. 

\paragraph{Data-Record Ghost IDs.}
To support \lyt{the same?}\sout{stronger} decorrelation guarantees while supporting resubscription,
\sys~generates a unique ghost user for each data remnant. 
Unlike a global placeholder, \sys~allows users to reactivate their account and undo
the decorrelation: user IDs can be linked back to a set of unique ghost IDs. 
This gives users the ability to freely unsubscribe to protect their privacy
without worrying about losing their accounts. \sys~resubscribes users by transparently propagating
updates to materialized views to expose real user identifiers in place of ghost identifiers. 

\sys~relies on coarse-grained schema annotations to establish which associations to decorrelate, and
builds a dataflow computation resulting in materialized views that answer application queries. Use
of dataflow automatically propagates the correct updates to materialized views.

\sout{These ghosts, unlike a global placeholder,
are indistinguishable from a real user in the system, ensuring that queries cannot correlate two
ghosts with a single real (albeit unsubscribed) user.
However, some linkable information is still leaked. For example, ghost users may be randomly
generated in a pattern identifiable by an observer (e.g.,\ if all ghosts have usernames which are
random numbers, or arbitrary animals, but a real user may have more human-friendly usernames).}

\lyt{Not sure how to incorporate DP or some notion of noninterference here (how do \sys's responses to
queries deviate from what an ideally decorrelated system would produce?)
DP seems to deal with the change in probability of some event X, rather than the initial probability
to begin with.
Perhaps DP would be more applicable if we were looking at the database over time? And if we had some
notion of noise. Without adding noise, the output would change dramatically when a user unsubscribes
and UIDs are replaced by GIDs, which reveals complete linkability information.}

%-------------------------------------------------------------------------------
\section{Background and Related Work}
%-------------------------------------------------------------------------------

Companies have developed frameworks to avoid deletion bugs (e.g., DELF~\cite{delf} at Facebook), but many
applications decorrelate only coarsely by associating remnants of data with a global placeholder for
all deleted users, or do not decorrelate at all (e.g., replacing usernames with a pseudonym).

\lyt{(cite Reddit, Lobsters, others?)}

The right to be forgotten has also been formally defined by Garg et
al.~\cite{garg}, where correct deletion corresponds to the notion of
leave-no-trace: the state of the data collection system after a user requests to be forgotten should
be left (nearly) indistinguishable from that where the user never existed to begin with. While
\sys{} uses a similar comparison, their formalization assumes that users operate
independently, and that the centralized data collector prevents one user's data from influencing
another's.

Other related works:
\begin{itemize}
    \item Deceptive Deletions for protecting withdrawn posts: https://arxiv.org/abs/2005.14113
    \item "My Friend Wanted to Talk About It and I Didn't": Understanding Perceptions of
        Deletion Privacy in Social Platforms, user survey https://arxiv.org/pdf/2008.11317.pdf;
        talk about decoy deletion, prescheduled deletion strategies~\cite{myfw}
    \item Contextual Integrity
    \item ML Unlearning
    \item k-anonymization, pseudonymization
\end{itemize}

%-------------------------------------------------------------------------------
\section{Prototype Design}
%-------------------------------------------------------------------------------
\label{sec:proto}

\begin{figure}[t!]
    \centering
    \includegraphics[width=0.5\textwidth]{img/releaser_arch}

    \caption{High-level \sys architecture. Developers specify grayed-out components.}
    \label{fig:arch}
\end{figure}

As shown in Figure~\ref{fig:arch}, \sys sits between the application logic and its database. \sys
models the application schema as an entity graph, and systematically applies the specified
unsubscription policy transformations during unsubscription and resubscription.

\sys records the transformations performed upon unsubscribing a user, and encrypts and stores this
log for the application. The encrypting key is secret-shared~\cite{secretsharing} among the user,
\sys, and a trusted third party so that the user can retrieve a lost key without trusting the
application or \sys.

\sys consists of 5K LoC of Rust, and supports SQL queries as well as \texttt{UNSUBSCRIBE
[user]} and \texttt{RESUBSCRIBE [user]} queries.
Although we show that \sys performs reasonably (Section~\ref{sec:perf}), \sys should support eventually
consistent, crash-recoverable unsubscription and resubscription, sharding, and multicore parallelism.

%To amortize the cost of unsubscription and resubscription, \name preemptively creates, stores, and
%links ghost parent entities to child entities if an update creates an edge that may be decorrelated.
%\name builds in-memory materialized views on top of the underlying database, exposing ghost
%entities only if the true entity has been decorrelated, and real entities otherwise. Updates
%propagate to the materialized views when the underlying database is updated. \name answers
%application queries using these materialized views, hiding the complexity of ghost entity and
%decorrelation management.



%%%%%%%%%%%%%%%%%%%%%%%%%%%%%%%%%%%%%%%%%%%%%%%%%%%%%%%%%%%%%%%%%%%%%%%%%%%%%%%%%%%%%%%%%%
\section{Implementing Disguising}
%%%%%%%%%%%%%%%%%%%%%%%%%%%%%%%%%%%%%%%%%%%%%%%%%%%%%%%%%%%%%%%%%%%%%%%%%%%%%%%%%%%%%%%%%%
\begin{table*}[t!]
\centering
\begin{tabular}{ c p{.7\linewidth} }
\textbf{Function} & \textbf{Description} \\
\hline
    \fn{ReadPrivateTokens(\symk{pd})} & Decrypts all of $p$'s private tokens produced by disguise
    $d$ using \symk{pd}. \\
    \fn{ReadGlobalTokens($d$)} & Retrieves all global tokens produced by disguise $d$. \\
    \fn{ApplyDisguise($d$,tokens)} & Applies disguise $d$, selectively composing $d$'s
    updates with prior disguises using the tokens's data. \\
    \fn{\op{d}.execute(tokens)} & Executes the disguise operation \op{d}, composing the operation
    with prior disguises using the tokens' data.\\
    \fn{ReverseDisguise($d$,tokens)} & Reverses disguise $d$ using the tokens' data.\\
    \fn{ReverseTokenOp(token)} & Reverses the data modification performed by the disguise operation
    that produced the token.\\
    \fn{StorePubKey($\pubk{p}$)} & Persistently saves the public key \pubk{p} indexed by $p$.\\
    \fn{LoadPubKey($p$)} & Retrieves public key \pubk{p} for $p$.\\
    \fn{LoadEncPrivKeyTokens($p$)} & Retrieves \tpriv{pdq'} ciphertexts for $p$.\\
    \fn{LoadEncSymKeys(caps)} & Retrieves \symk{pd} ciphertexts corresponding to the specified
    capabilities.\\
    \fn{StoreEncSymKey(\rptr{pd})} & Persistently saves the \symk{pd} ciphertext indexed by
    capability \rptr{pd}.\\
    \fn{LoadListTail}$(p,d)$ & Gets the first encrypted private token in \tokls{pd}, the list of
    tokens associated with $p$ produced by $d$.\\
    \fn{StoreListTail}$(p,d)$ & Persistently saves the first encrypted private token in \tokls{pd}
    indexed by $p$ and $d$.
\end{tabular}
    \vspace{12px}
\caption{Internal \sys functions}
\label{tab:funcs}
\end{table*}

Table~\ref{tab:funcs} describe \sys's internal functions run server-side to implement its API 
and apply or reverse disguises. 

Figures~\ref{fig:appdisg} and \ref{fig:revdisg} describe how \sys implements disguise application and
reversal respectively. Figure~\ref{fig:opexec} describes how disguise operations update application
state and produce and modify tokens, and 
Figure~\ref{fig:revtoken} describes how \sys uses a token's recorded
modification to reverse an applied disguise operation. 
Figure~\ref{fig:rpt} describes how \sys accesses the private tokens
corresponding to disguise $d$ and principal $p$ by recursively decrypting tokens and traversing
\tokls{pd}.

\sys's implementations of the remaining functions in Table~\ref{tab:funcs} simply read or write into
persistent maps indexed by principal and/or disguise.

\subsection{Composition Techniques}
\sys's algorithm for disguising and disguise reversal in most scenarios is straightforward. 

To apply disguise $d$, \sys applies each \op{d} to all
data objects satisfying \op{d}'s predicate, while also taking into account the information in
accessible tokens (to \eg predicate against undisguised versions of objects). Each \op{d} produces
one or more tokens, which \sys stores appropriately (globally, or encrypted in a \tokls{pd}).

For reversal of disguise $d$, \sys reveals the undisguised version of data stored in accessible tokens
corresponding to $d$ by updating the relevant objects $O$ in the database.

However, \sys must be careful of two potential problems:
(1) \sys cannot accidentally reveal data that must be disguised; and (2) \sys cannot let global
tokens recording updates to data $x$ leak information if $x$ is subsequently (privately) disguised.

%%%%%%%%%%%%%%%%%%%%%%%%%%%%%%%%%%%%%%%%%%%%%%%%%%%%%%%%%%%%%%%%%%%%%%%%%%%%%%%%%%%%%%%%%%
\vspace{6pt}\noindent\textbf{\emph{Preventing Accidental Data Revealing.}}
\sys records the disguised state of data $x$ in each \tdata{pd} recording a update performed by $d$
to $x$. If the state of $x$ does not match the disguised state in $d$'s token for $x$, then \sys
knows the token records an overwritten update to $x$, and refuses to reveal the undisguised state of $x$
stored in the token.

%%%%%%%%%%%%%%%%%%%%%%%%%%%%%%%%%%%%%%%%%%%%%%%%%%%%%%%%%%%%%%%%%%%%%%%%%%%%%%%%%%%%%%%%%%
\vspace{6pt}\noindent\textbf{\emph{Preventing Global Token Information Leakage.}}
%The ONLY REASON we need to update tokens is if they're global!!! If they're private, we can use the
%retroactive application method... (optimization to update our "own" tokens though)
%
Let $d_1$ be a global disguise and $d_2$ be a private disguise applied in sequence.  Consider the
scenario in which \op{d_1} updates a data object $O$, producing a global token \tdata{pd_1}. Later,
\sys determines that \op{d_2} also should update $O$.
%by checking if the corresponding \op{d_2}
%predicate matches recorded object data stored in \tdata{pd_1}.

If \sys simply performs \op{d_2}'s update to $O$ in the application database, an adversary can still
read undisguised data from \tdata{pd_1} since is it global and records a now-stale state of $O$!
%
%Furthermore, this means that \sys cannot restore the state of $O$ in the database to
%\xhist{[\app{d_2}]} when reversing $d_1$, because the revealing the data in \tdata{pd_1} would
%reveal data that $d_2$ should have disguised.
%
To prevent global tokens like \tdata{pd_1} from leaking information that should be disguised, \sys
updates the data in \tdata{pd_1} with \op{d_2}'s update. An update to a token such as \tdata{pd_1}
itself generates a (private) token \tdata{pd_2}, just as updating a data object generates a token.

Reversing $d_2$ uses this \tdata{pd_2} to reverse the modification to \tdata{pd_1}. 
If $d_1$ has already been reversed, and the database state of $O$ updated to reflect the contents
of \tdata{pd_1} (which were updated by the application of $d_2$), 
then \sys simply uses \tdata{pd_2} to reverse the current state of $O$ in the database. 

If a future $d_3$ has further updated \tdata{pd_1} in a way that conflicts with $d_2$'s update, then
upon reversal of $d_2$, \sys will notice that \tdata{pd_2} records an overwritten update to
\tdata{pd_1}, and will not revert the state of \tdata{pd_1} until the future disguise $d_3$ has been
reversed.

\lyt{Note that removal is just an interesting case of this, in which \tdata{pd_1} is removed, by a
token \tdata{pd_2} is stored in \tokls{pd_2} that saves the removal of \tdata{pd_1} for reversal.}

%\vspace{6pt}\noindent\textbf{\emph{Token Modification for Object Identification.}}
%Tokens need to refer to the correct objects even if objects have been modified by future disguises.
%Prior tokens should be revised by future disguises so that they apply correctly to the disguised
%data.
%This allows an earlier disguise to be reversed correctly.
%An easy way to avoid needing to do this is to have all objects in the DB have unique, permanent
%identifiers.



%-------------------------------------------------------------------------------
\section{Evaluation}
%-------------------------------------------------------------------------------

We evaluate \sys{} on two metrics: (1) can desired decorrelation policies be easily specified using the
provided decorrelation primitives for a range of practical applications?, and (2) can decorrelation
be supported with low performance overhead?

Our evaluation exemplifies how a suite of applications (Lobste.rs, \lyt{TODO}) can express a diverse
range of privacy policies using \sys{} with low developer effort.

\subsection{Lobste.rs}
The performance experiments reported are run on Intel Xeon E5-2660 v3 CPUs, with a
single-threaded client and \sys{}'s shim layer each pinned to a single core. \sys{} utilizes the
trawler workload~\cite{trawler} for Lobste.rs, which emulates production Lobste.rs traffic according
to a recorded production workload. 

\section{Discussion}

\lyt{Should we note somewhere that pseudoprincipals and recursive disguising are why we need asymmetric
crypto; otherwise, we could just send the symmetric key to the original user being disguised?}


%-------------------------------------------------------------------------------
\section*{Acknowledgments}
%-------------------------------------------------------------------------------

%-------------------------------------------------------------------------------
\section*{Availability}
%-------------------------------------------------------------------------------

\lyt{USENIX program committees give extra points to submissions that are
backed by artifacts that are publicly available. If you made your code
or data available, it's worth mentioning this fact in a dedicated
section.}

%-------------------------------------------------------------------------------
\bibliographystyle{plain}
\bibliography{paper}

%%%%%%%%%%%%%%%%%%%%%%%%%%%%%%%%%%%%%%%%%%%%%%%%%%%%%%%%%%%%%%%%%%%%%%%%%%%%%%%%
\end{document}
%%%%%%%%%%%%%%%%%%%%%%%%%%%%%%%%%%%%%%%%%%%%%%%%%%%%%%%%%%%%%%%%%%%%%%%%%%%%%%%%

%%  LocalWords:  endnotes includegraphics fread ptr nobj noindent
%%  LocalWords:  pdflatex acks
