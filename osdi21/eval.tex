%-------------------------------------------------------------------------------
\section{Evaluation}
%-------------------------------------------------------------------------------
We evaluate \sys{} on two metrics: (1) can desired decorrelation policies be easily specified using the
provided decorrelation primitives for a range of practical applications?, and (2) can decorrelation
be supported with low performance overhead?

Our evaluation exemplifies how a suite of applications (Lobste.rs, \lyt{TODO}) can express a diverse
range of privacy policies using \sys{} with low developer effort with low overhead.

\subsection{Example policies}
\begin{figure}
\begin{lstlisting}[language=Rust]
use GhostColumnPolicy, GeneratePolicy;
let ghost_policies = GhostGenerationPolicy::new(
  ("users", 
     [("id", Generate(Random)),
     ("username", Generate(Random)),
     ("karma", Generate(Default(0)))]),
  ...);

let edge_policies = vec![
  KeyRelationship{
    child: "stories".to_string(),
    parent: "users".to_string(),
    column_name: "user_id".to_string(),
    parent_child_decorrelation_policy: Decor,
    child_parent_decorrelation_policy: NoDecorRetain,
  },
  KeyRelationship{
    child: "taggings".to_string(),
    parent: "tags".to_string(),
    column_name: "tag_id".to_string(),
    parent_child_decorrelation_policy: 
        NoDecorSensitivity(0.25),
    child_parent_decorrelation_policy: NoDecorRetain,
  },
  ...];

 ApplicationPolicy {
    entity_type_to_decorrelate: "users",
    ghost_policies: ghost_policies,
    edge_policies: edge_policies,
 }
\end{lstlisting}
    \label{fig:policy}
    \caption{Excerpt of the Lobsters Application Policy}
\end{figure}

We provide several variants of an application policy for Lobsters. Each requires specifying at most 17 key
relationships and at most 5 ghost entity generation policies (see Figure~\ref{fig:policy} for an example
of what these policies). The total number of lines that a developer must write range from \lyt{TODO}
to \lyt{TODO} LoC (current policy is 190LoC).

The first policy replicates what the behavior of the current Lobsters deployment upon user
unsubscription, with the exception that users are replaced with ghost users rather than a global
placeholder. This requires specifying \texttt{Decor} policies for all edges from entities with users
as parents, resulting in the developer writing 9 edge policies and one ghost generation policy (for
ghost users). While this policy does nothing significant, splitting a user into multiple ghosts as
compared to a global placeholder allows the user to resubscribe by identifying their ghost
counterparts, whereas a global placeholder does not. Furthermore, splitting users into multiple
users as compared to grouping them together upon unsubscription may benefit privacy: global
placeholder users indicate that any child of the placeholder belongs to some deleted user, whereas
ghost users may more successfully mask which children have been abandoned.

The second policy decorrelates all user-related edges, but additionally desensitizes tags and  

\subsection{Performance}
\paragraph{Experimental Setup.}
The performance experiments reported are run on Intel Xeon E5-2660 v3 CPUs, with a
single-threaded client and \sys{}'s shim layer each pinned to a single core. \sys{} runs on top of 
MariaDB (v10.4.13). 

We compare the application execution performance on four systems, all of which configure MariaDB to
store the database on a ramdisk to avoid disk IO overheads.
\begin{enumerate}
    \item \texttt{NoShim}: queries are directly issued to a MariaDB instance;
    \item \texttt{Shim}: queries are intercepted by the shim layer,
        which does nothing more than send queries to the MariaDB instance and
        return the results back to the client;
    \item \texttt{ShimParsing}: queries are intercepted the shim layer, parsed into a
        SQL AST (which is discarded), and then sent to the MariaDB instance; the shim then returns the results to the
        client. 
    \item \texttt{\sys{}}: queries are intercepted by the shim layer and parsed into a SQL AST. \sys{}
        then processes the parsed query, potentially introducing ghost entities and sending additional queries to the MariaDB backend.
        \sys{} then returns results to the client.
\end{enumerate}

We compare \texttt{NoShim} and \texttt{Shim} to define the cost of query interception; \texttt{Shim}
and \texttt{ShimParsing} to define the cost of query parsing; and \texttt{ShimParsing} and
\texttt{\sys{}} to define the cost of adding decorrelation and recorrelation support.  \texttt{Shim}
provides an upper bound for \sys{}'s (single-threaded) performance, as the costs of query parsing
and decorrelation are not necessarily fundamental.

\paragraph{Lobste.rs Performance}
\sys{} utilizes the trawler workload~\cite{trawler} for Lobsters, which emulates production
Lobsters traffic according to a recorded production workload. The database initially begins with 1.5K
users, 10K stories, and 30K comments (1/4 the size of the the real Lobsters deployment), and issues
queries according to the recorded traffic distribution. The query distribution skews towards reads
(recent and frontpage stories) over writes (votes, commenting, and posting of stories); the
workload assumes that users with popular content are also more active and likely to post stories,
comment, and vote.

The benchmarks perform 10000 actions, each performing on average 8 application queries. 
We measure (1) overall throughput, (2) the distribution of application query
latencies, and (3) the query multiplication per application query performed by \sys{}.

The first benchmark measures \sys{}'s performance during \emph{normal execution}, namely when only
read and update actions performed by Lobsters are executed.

The second benchmark measures \sys{}'s performance with 10\% of the actions being unsubscription. If
an unsubscribed user is chosen to perform an action, this user is first resubscribed.
\texttt{NoShim}, \texttt{Shim}, and \texttt{ShimParsing} simply delete or insert the chosen user on
unsubscription or resubscription respectively; this provides a very conservative estimate of the
amount of work these baseline systems would need to do if a user withdrew their data from the system.

\lyt{overall throughput numbers here}



