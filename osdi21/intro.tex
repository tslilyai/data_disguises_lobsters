%-------------------------------------------------------------------------------
\section{Introduction}
%-------------------------------------------------------------------------------

\subsection{Motivation} 

Web application companies face increasing legal requirements to protect users’ data. These
requirements pressure companies to properly delete and anonymize users' data when a user requests to
\emph{unsubscribe} from the service (i.e.,\ revoke access to their personal data).
For example, the GDPR requires that any user data remaining after a user unsubscribes is
\emph{decorrelated}, i.e., cannot be (directly or indirectly) used to identify the user~\cite{gdpr}.  

In this paper, we propose \sys, a new approach to managing user identities in web applications.
\sys~meets the decorrelation requirements in the GDPR, and goes beyond: with \sys, it is possible
for users to switch between a privacy-preserving unsubscribed mode and an identity-revealing
subscribed mode at any time. This facilitates important new web service paradigms, such as users
granting a time-limited "lease" of data to a service instead of having a permanent service account.

%Furthermore, keeping identifying and personal data when no longer strictly necessary increases
%companies' liability: the GDPR and other laws mandate that companies retain only user data that is
%relevant and necessary for their applications' purposes. To increase users' control over their data
%and decrease the amount of incriminating data stored in the application at any one point, users
%should be able to freely unsubscribe from the service to enter a privacy-preserving mode, and later
%resubscribe when they wish to use the service. 

\subsection{Goals} 
\sys's goal is to provide the following properties while preserving an application's 
semantics: 
\begin{description} 
    \item[Decorrelation.] Decorrelation ideally guarantees that it is impossible to distinguish
        between two records formerly associated with the same unsubscribed user and two records from
        different unsubscribed users.  
    %\item Deletion Correctness: Deletion of user records should correctly conform to application
        %semantics: for example, post deletion could remove the post and its underlying comments, or
    %simply anonymize the post and keep all content accessible.  
    \item[Resubscription.] Users should be able to easily switch between a privacy-preserving unsubscribed mode 
       and an identity-revealing subscribed mode, without permanently losing their application data.  
\end{description}

\sys~must implement these properties while ensuring (1) performance comparable to today’s
widely-used databases, and (2) easy adoption (decorrelation should be
automated without needing to modify application schemas or semantics).

\subsection{Decorrelation Guarantees} 
We address applications in which application data consists of \emph{data records} and computations
(such as aggregations) that may be performed over these data records. Data records are considered
sensitive, private data records when they contain \emph{user identifiers}; for example, a row in a
table containing a column of user IDs would be a user's private data record. Data records containing
multiple, potentially different, user IDs are considered shared data records private to the
identified users. We refer to data records belonging to unsubscribed users as \emph{remnants}.

Perfect decorrelation is achieved when queries to the application reveal no information allowing an
observer to determine if two distinct user data records belong to the same user who has since
unsubscribed (are \emph{linked}), or belong to two different (real or unsubscribed) users. Observers
gain no information that allows them to distinguish the two scenarios. 

More formally, given user data records and a subset of $R$ of these records that are remnants, we
divide the $R$ remnants among some number $N$ of unsubscribed users.  Perfect decorrelation
guarantees that any division of the $R$ remnants among $N$ users is equally likely: each remnant is
equally likely to be linked with any other remnant. 

~\lyt{I toyed with saying that $P(N = R) = P(N \le R)$: the probability that every distinct remnant
belongs to a distinct user equals the probability that any user owns multiple remnants, but this
seemed unnecessarily complex when I believe that the above statement captures decorrelation.} 

~\lyt{Note: whether $N$ is known or not also plays into the probability analysis}

~\lyt{Note: this is the definition in which ghosts can be distinct from real users, so we're not
trying to pretend that remnants cannot be identified}

In practical settings, however, perfect decorrelation is likely impossible: for example, a reposted
screenshot may leak user IDs. Furthermore, an observer who can see application queries over time, or
search web archives, can detect when a user unsubscribes and refer to prior snapshots in which user
data records may have had the same user ID.  Given these limitations, we seek to achieve the maximum
decorrelation guarantees possible while assuming that the content of user data does not itself leak
identifying information, and that observers cannot access past application database
state.~\lyt{Note: UIDs can also include things like email addrs, phone number, etc; so the notion of
``identifiers'' might be more broad than stated here}.

\paragraph{Global Placeholder.}
A common strawman solution to decorrelation is to replace all unsubscribed user IDs with one global
placeholder ID. This means that all remnants can be identified by one distinct UID, which we call
$GP$.

A global placeholder can provide either complete privacy or no privacy at all. We first begin by
assuming an observer has complete information about how many users are in the system ($N$).
\begin{itemize}
    \item If $N=1$ and only one user has unsubscribed, all remnants can be linked back to that user's
        identity: any data record with UID equal to $GP$ belongs to that user.
    \item If all users unsubscribe, two remnants are equally as likely to belong
        to any two of the $N$ individual users as they are to belong to one single user. 
\end{itemize}
The amount of linkable information~\lyt{(need to define this)}, and the resulting deviance from
perfect decorrelation, decreases as more users unsubscribe.
Let us assume that an observer knows $N$ and $R$, and the number of data records per user is
uniformly distributed. Let $r = \frac{R}{N}$ be the number of remnants per unsubscribed user. 
Then an observer has probability $p_{\text{exact}}$ of guessing the correct combination of
records per users, where  
$$p_{\text{exact}} = \frac{(r!)^N}{R!}$$
An observer has probability $p_{\text{user}}$ of guessing the correct combination for one user where 
$$p_{\text{user}} = \frac{r!(R-r)!}{R!}$$
\lyt{TODO---non-uniform distributions, not knowing $N$, not knowing $R$?}

Using a global placeholder, however, makes resubscription challenging: users can no identify which
unsubscribed data records belong to them if all user-specific data has been erased from the
system. 

\paragraph{Data-Record Ghost IDs.}
To support \lyt{the same?}\sout{stronger} decorrelation guarantees while supporting resubscription,
\sys~generates a unique ghost user for each data remnant. 
Unlike a global placeholder, \sys~allows users to reactivate their account and undo
the decorrelation: user IDs can be linked back to a set of unique ghost IDs. 
This gives users the ability to freely unsubscribe to protect their privacy
without worrying about losing their accounts. \sys~resubscribes users by transparently propagating
updates to materialized views to expose real user identifiers in place of ghost identifiers. 

\sys~relies on coarse-grained schema annotations to establish which associations to decorrelate, and
builds a dataflow computation resulting in materialized views that answer application queries. Use
of dataflow automatically propagates the correct updates to materialized views.

\sout{These ghosts, unlike a global placeholder,
are indistinguishable from a real user in the system, ensuring that queries cannot correlate two
ghosts with a single real (albeit unsubscribed) user.
However, some linkable information is still leaked. For example, ghost users may be randomly
generated in a pattern identifiable by an observer (e.g.,\ if all ghosts have usernames which are
random numbers, or arbitrary animals, but a real user may have more human-friendly usernames).}

\lyt{Not sure how to incorporate DP or some notion of noninterference here (how do \sys's responses to
queries deviate from what an ideally decorrelated system would produce?)
DP seems to deal with the change in probability of some event X, rather than the initial probability
to begin with.
Perhaps DP would be more applicable if we were looking at the database over time? And if we had some
notion of noise. Without adding noise, the output would change dramatically when a user unsubscribes
and UIDs are replaced by GIDs, which reveals complete linkability information.}
