%-------------------------------------------------------------------------------
\section{Introduction}
%-------------------------------------------------------------------------------

\subsection{Motivation} 

Web application companies face increasing legal requirements to protect users’ data. These
requirements pressure companies to properly delete and anonymize users' data when a user requests to
\emph{unsubscribe} from the service (i.e.,\ revoke access to their personal data).
For example, the GDPR requires that any user data remaining after a user unsubscribes is
\emph{decorrelated}, i.e., cannot be (directly or indirectly) used to identify the user~\cite{gdpr}.  

In this paper, we propose \sys, a new approach to managing user identities in web applications.
\sys{} meets the decorrelation requirements in the GDPR, and goes beyond: with \sys, it is possible
for users to switch between a privacy-preserving unsubscribed mode and an identity-revealing
subscribed mode at any time. This facilitates important new web service paradigms, such as users
granting a time-limited ``lease'' of data to a service instead of having a permanent service account.

%Furthermore, keeping identifying and personal data when no longer strictly necessary increases
%companies' liability: the GDPR and other laws mandate that companies retain only user data that is
%relevant and necessary for their applications' purposes. To increase users' control over their data
%and decrease the amount of incriminating data stored in the application at any one point, users
%should be able to freely unsubscribe from the service to enter a privacy-preserving mode, and later
%resubscribe when they wish to use the service. 

\subsection{Goals} 
\sys's goal is to provide the following properties while preserving an application's 
semantics: 
\begin{description} 
    \item[Decorrelation.] Informally, ideal decorrelation guarantees that it is impossible to distinguish
        between two records formerly associated with the same unsubscribed user and two records from
        different unsubscribed users.  
    %\item Deletion Correctness: Deletion of user records should correctly conform to application
        %semantics: for example, post deletion could remove the post and its underlying comments, or
    %simply anonymize the post and keep all content accessible.  
    \item[Resubscription.] Users should be able to easily switch between a privacy-preserving unsubscribed mode 
       and an identity-revealing subscribed mode, without permanently losing their application data.  
\end{description}

\sys{} must implement these properties while ensuring (1) performance comparable to today’s
widely-used databases, and (2) easy adoption (decorrelation should be
automated without needing to modify application schemas or semantics).

\subsection{Threat Model.}
We address applications in which application data consists of \emph{data records} and computations
(such as aggregations) that may be performed over these data records. Data records are considered
sensitive, private data records when they contain \emph{user identifiers (UIDs)}. For example, the
user table ID field, usernames, or phone numbers are all UIDs, and a table row with any of
these as columns is a private data record. Data records may also contain \emph{indirect 
identifiers (IIDs)}, which include application metadata such as the time of posting or category of
post.

We assume that all user data records contain a user table ID field, a
numerical user key \uidkey{} that is unique to each user, and which ties all data records
to a particular user.
Data records containing multiple \uidkey{}s are
considered shared data records private to the identified users. 

We refer to data records belonging
to unsubscribed users as \emph{remnants}.

We make the following assumptions of an attacker whose goal is to break decorrelation and reveal
identifying information about data remnants:

\begin{itemize}
    \item An attacker can perform only those queries allowed by the application API: an
attacker can access the application only via its public interface. 
\lyt{Alternatively, an attacker could perform arbitrary queries on some public subset of the
application schema (e.g., all tables other than the mapping table, or all tables marked with some
compliance policy); arbitrary queries over the
entirety of the table are out of scope, unless ``private'' tables are removed and stored by
unsubscribing users.}

\item An attacker cannot perform application queries to the past or search web archives:
information from prior application snapshots may reveal 
exactly how data records were decorrelated from unsubscribed users. 

\item An attacker cannot gain identifying information from arbitrary user-generated content (for
    example, a reposted screenshot, or text in user posts or comments).  Decorrelation seeks to
        remove identifying information from only IIDs (e.g., date of postings,
        database ID columns), or controllable UIDs (e.g., a birthday or email address)
        \lyt{There needs to be a clearer definition of what is ``arbitrary'' and what is
        ``controllable''}
\end{itemize}

\subsection{Decorrelation Guarantees} 
\paragraph{Decorrelation Techniques.}
A common strawman decorrelation technique replaces all unsubscribed users' \uidkey{}s with
one \emph{global placeholder} user, essentially collapsing all unsubscribed users' data into one pool.
This means that all remnants can be identified by one distinct \gidkey{}.
Other UIDs in data records (such as emails, phone numbers) are also replaced by global placeholders
(such as a default email address) that are assigned to this global user.
Using a global placeholder, however, makes resubscription challenging: users can no identify which unsubscribed data records belong to them if all user-specific data has been erased from the
system. 

An alternative technique used by \sys{} generates a unique \emph{ghost user} for each data remnant,
essentially splitting unsubscribed users' data into individual pieces, and replacing one
\uidkey{} with many \gidkey{}s, one per data record owned by the user. Ghost users each have
distinct GIDs in place of UIDs (e.g., a randomly generated email
address per ghost).  Unlike a global placeholder, \sys{} allows users to reactivate their account and
undo the decorrelation: \uidkey{}s can be linked back to a set of unique \gidkey{}s.
This gives users the ability to freely unsubscribe to protect their privacy without worrying about
losing their accounts.  \lyt{Ghosts also make schema changes / changing the location of data records
easier to support.} 

We next describe how these two techniques can lead to different decorrelation guarantees.

\paragraph{Measuring Decorrelation.} 
We say that the system achieves \textbf{$(Q, \epsilon)$-decorrelation} when the probability that an
attacker can correctly determine that any two distinct remnants belong to any one user is less than
$\epsilon$, given information derivable from performing any $Q$ queries.  \lyt{Note that the
\emph{correctly} is probably essential here: An attacker who will mistakenly correlate two remnants
with some user actually faces more challenges}

Defining this probability requires calculating two others: \begin{enumerate}
    \item[\premnant{}]: 
        The probability that the attacker can determine that any data record is a remnant
    \item[\plinked{}]:
        The probability that any two remnants belong to the same individual
\end{enumerate}

%\lyt{(could also use some notion of asymptotic security here?)}

\paragraph{Calculating \premnant{}.}
The probability that an attacker can determine that a data record is a remnant depends on the
database decorrelation technique and specific application semantics.

At one extreme, an application may expose only aggregate information over data records via its
queries, which prevents an attacker to locate individual data remnants or data records at all. 

More realistically, however, the application may allow some queries to access the UIDs either
directly or indirectly. Then if decorrelation utilizes a global placeholder, an attacker can
determine with 100\% certainty which records are remnants: any record with a UID equal to the global
placeholder is clearly a remnant. However, if decorrelation instead relies on ghost users, an
attacker cannot guarantee that any revealed identifier is a ghost instead of a real (subscribed)
user. Instead, the attacker must calculate the probability that an identifier belongs to a ghost
using external knowledge about identifiers (e.g., GIDs may be randomly generated in a identifiable
pattern). 

An attacker can also use IIDs and knowledge of the distribution of real user profiles to calculate the
probability that a record is a remnant, which would be necessary in the case in which queries cannot
access UIDs directly or indirectly. Similar to how an attacker could use external knowledge about
identifiers, an attacker could use knowledge about ghost profiles or ghosted records to distinguish
them from real users.  For example, decorrelation may generate ghost profile usernames that are
random numbers or arbitrary animals, while real users may have more human-friendly usernames.

\paragraph{Decreasing \premnant{}.}
Because an attacker may use the distribution of data records in the system to pinpoint outliers and
anomalies that may be remnants, remnants may be less easily spotted if more users unsubscribe and
the system contains more remnants. 

Given the possibility for application-specific data to leak information even when identifiers are
hidden and when ghost users are used in place of a global placeholder, the decorrelation should
ensure that information exposed by remnants cannot distinguish remnants from real data records.
Creating remnants from data records should not follow a clear pattern (such as assigning global
values for usernames).  This is highly specific to the application: it may be difficult to
convincingly create user profiles for applications like Facebook and eCommerce sites, but easy in
applications such as Reddit where many users use pseudonyms and have simple usage patterns.

\paragraph{Calculating \plinked{}.}
The probability that two remnants belong to the same individual can be determined by how much
identifying information (from UIDs and IIDs) that attacker queries can reveal.
For example, if an attacker query reveals that multiple
(likely) remnants are comments on classical music, these remnants are more likely to be
correlated with the same individual than two remnants commenting on unrelated topics. If the
attacker additionally queries for the time of post of these comments, and sees that both these posts
were posted during daytime in California, the probability that they belong to the same individual
increases (since the number of people who live in California and like classical music is less than
the number of people who simply like classical music).

\paragraph{Decreasing \plinked{}.}
As with \premnant{}, increasing the number of unsubscribed users and resulting
number of remnants decreases the probability that any two remnants can be linked to any one
user. For example, if there are two records that are likely to be remnants and both are upvotes on topics related to
classical music, the probability that these remnants belong to the same unsubscribed user is high; but if
there are thousands of likely remnants and all are upvotes on topics related to classical music, it
is less likely that an attacker can tell that any two of these remnants belong to any one
unsubscribed user.~\lyt{I'm a bit unconvinced of this argument, but I feel like there is some
intuition here}.

Decreasing \plinked{} requires minimizing the amount of identifying information leaked
through queries. However, there is a fundamental tradeoff between preserving useful information for
the application and obfuscating remnant data. 

At one extreme, if decorrelation does not modify UIDs at all (or does so in a predictable manner)
and queries may return UIDs, then an attacker can learn usernames or email addresses that will
identify the user. Similarly, if IIDs are not properly modified, attackers may learn
the location of posts, time of posts, and potentially even usernames, leading to a high probability
that the user can be identified.  Note that this leads to a low \premnant{} (since the
remnant would look unmodified from any other subscribed user record), but also a high \plinked{}.

At the other extreme, if decorrelation modifies the remnant completely by generating random and
potentially meaningless values for all remnant UIDs and IIDs, then an attacker gains little identifying
information from the remnant. However, this may increase \premnant{} significantly.
Alternatively, decorrelation could simply to remove remnants completely. In either case, decorrelation limits the usefulness of remnant data to the application. 

Practically, in order to both preserve the usefulness of remnants to the application and provide
decorrelation, \sys{} will need to balance destroying remnant information with revealing identifying
information to the attacker when creating ghosts during unsubscription. 

\paragraph{Converting UIDs to GIDs, and adding noise to IIDs.}
\lyt{ML/GANs seem potentially relevant here in generating fake users}

\sys{} generates random \gidkey{}s for every unique data record belonging to one
\uidkey{}, and ensure that \uidkey{}s are randomly generated in the schema so
that \gidkey{}s are \uidkey{}s are indistinguishable.

For all other UIDs, \sys's decorrelation does the following to generate corresponding GIDs:
\begin{itemize}
    \item Phone number: randomly selects a real area code and randomly generates a 7-digit number
        \lyt{take into account distribution of phone numbers in the system}
    \item Email: randomly selects a real area code and randomly generates an amount to add to the 7-digit number, where the RNG is seeded with the \gidkey{}
    \item Username: generates a random but human-readable string (consisting of the
        concatenation of English words), using a hash of the original username and
        \gidkey{} to select words in the username.
\end{itemize}

In order to add noise to IIDs, \sys{} looks at the type of each IID column and
adds noise accordingly:
\begin{itemize}
\item Numerical Values and Dates/Times: adds a random amount equal to the
        \lyt{reversible?} hash of the UID time and \gidkey{} of the data record.
    \item 
        \lyt{TODO add more here}
\end{itemize}

The application programmer can add GID-generating functions for other application-specific UID or
IID columns via schema annotations; the programmer can also add functions to override \sys's
defaults.  These annotation should generate GIDs in a manner that minimizes both \premnant{} and
\plinked{}.

\lyt{TODO: Think about how this noising can be reversed upon resubscription / use something like
trapdoor permuatations? If the \gidkey{}s are
exposed, this might mean that an attacker could reverse the decorrelation as well... Perhaps the
original \uidkey{} needs to come into play.}

\subsection{A Simple Web Application}
To provide a concrete example of how \sys{} performs decorrelation, and how \plinked{} and
\premnant{} may be calculated to determine $Q$ and $\epsilon$, we look to a simple version of a typical social media 
application.
In this application, there are users, stories posted by users, and votes on stories with the schema
shown in Figure~\ref{fig:app}.
The application allows for story feed queries (selecting all of a story's content) and user profile
queries (selecting all of a users' content), shown in Figure~\ref{fig:appqs}.

\begin{figure*}
    \centering
\begin{minipage}[t]{0.3\textwidth}
Users Table:
\begin{itemize}
    \item $^*$ID (\uidkey{}): u64
    \item $^*$Username: varchar(255) 
    \item $^*$Email: varchar(255) 
    \item $^*$Phone: varchar(255)
\end{itemize}
\end{minipage}
\begin{minipage}[t]{0.3\textwidth}
Stories Table:
\begin{itemize}
    \item ID: u64
    \item $^*$UserID (\uidkey{}): u64
    \item Content: text 
    \item $^{**}$Category: u64
    \item $^{**}$Timestamp: datetime
    \item $^{**}$Location: decimal(18,12)
\end{itemize}
\end{minipage}
\begin{minipage}[t]{0.3\textwidth}
Votes Table:
\begin{itemize}
    \item ID: u64
    \item $^*$UserID (\uidkey{}): u64
    \item $^{**}$StoryID: u64
    \item $^{**}$Timestamp: datetime
\end{itemize}
\end{minipage}
\caption{$^*$ indicates a UID column and $^{**}$ indicates an IID column that can leak identifying information either directly or indirectly.
Arbitrary user-generated content is out of scope.}
\label{fig:app}
\hrulefill
\end{figure*}

\begin{figure*}
    \centering
\begin{minipage}[t]{0.45\textwidth}
    \centering
    Select story content:
\begin{verbatim}
SELECT users.username, 
    stories.content, stories.category, 
    stories.timestamp, stories.location, 
    COUNT(DISTINCT votes.id) as `count`
FROM stories 
    JOIN votes on stories.id = votes.storyid
    JOIN users on stories.userid = users.id
WHERE stories.id = ?
ORDER BY count;
\end{verbatim}
\end{minipage}
\begin{minipage}[t]{0.45\textwidth}
    \centering
    Select user content:
\begin{verbatim}
SELECT users.username, users.email, 
    users.phone, stories.content, 
    stories.category, 
    stories.timestamp, 
    stories.location, 
    votes.storyid, votes.timestamp 
FROM users 
    JOIN stories ON stories.userid = users.id 
    JOIN votes ON votes.userid = users.id
WHERE users.id=?;
\end{verbatim}
\end{minipage}
    \caption{Queries supported by the application}
\label{fig:appqs}
\end{figure*}
If \sys{} performs zero modifications to data records when a user unsubscribes, \premnant{} = 0 and
\plinked{} = 1 after even a single user profile query. Queries reveal all stories and votes that belong to a particular user. In other
words, \sys{} achieves $(Q \geq 1, 1)$-decorrelation.

Now suppose \sys{} uses a global placeholder strategy.  \sys{} zeroes u64 datatypes and datetimes,
and replaces varchar values by 'deleted').  If \sys{} modifies all UIDs but does not modify IIDs,
then \premnant{} = 1 after even a single user profile or story query.  An attacker needs to perform
at least two queries to identify two distinct remnant stories or votes (now associated with users with
the same zeroed or 'deleted' UIDs). 
\plinked{} is a conditional probability: given the current distribution of users' content, what is
the likelihood that two stories or votes belong to the same user?
%If 2 story remnants have the same location, then \plinked{} is at least the chance of choosing the
%correct user out of the total number of users residing in that location.
%\plinked{} of two story remnants is calculated by estimating the total number of
%people who could have posted a story at the spe
%potentially taking into account the distribution of real users in the application.
\lyt{TODO---not exactly sure how to precisely define this, even with a toy example}
Thus, \sys{} achieves $(Q \geq 2$, \plinked{})-decorrelation when modifying only UIDs with a global
placeholder strategy.
