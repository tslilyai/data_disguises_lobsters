%-------------------------------------------------------------------------------
\section{Introduction}
%-------------------------------------------------------------------------------

Web application companies face increasing legal requirements to protect users’ data. These
requirements pressure companies to properly delete and anonymize users' data when a user requests to
\emph{unsubscribe} from the service (i.e.,\ revoke access to their personal data). For example, the
GDPR requires that any data remaining after a user unsubscribes cannot be (directly or
indirectly) used to identify the user~\cite{gdpr}.  

In this paper, we propose \sys{}, which enables application developers to specify fine-grained
privacy policies to meet de-identification requirements, and goes beyond: with \sys{}, it is
possible for users to switch between a privacy-preserving unsubscribed mode and an
identity-revealing subscribed mode at any time without permanently losing their data. This
facilitates important new web service paradigms, such as users granting a time-limited ``lease'' of
data to a service instead of having a permanent service account.

Developers using \sys{} specify privacy-sensitive data and the state of such data after the user
leaves. \sys{} \emph{decorrelates} a user's data according to this specification when a user unsubscribes;
post-decorrelation application state is guaranteed to match the specification.  When a user wishes
to resubscribe, \sys{} \emph{recorrelates} the user's data with the user, restoring the user to her
original subscribed state as much as possible . 

\sys{} makes the key observation that a user's data entities (e.g.,\ posts and upvotes) can leak identifying
information in two ways: (1) directly via its content, and (2) indirectly via the correlations it has to
other data entities in the system.  
Correlations can range from obviously identifying (e.g.,\ posts correlated
with a user clearly belong to that user), to subtly perilous: posts correlated with a
particular user-generated tag most likely belong to the tag's author, and posts liked by the  
same group of users likely belong to a friend of the group. 

Many web applications (e.g.,\ Lobst.rs and Reddit) have come up with solutions to de-identify direct
content by anonymizing unique identifiers, leaving arbitrary user-generated content out of scope.
Other applications (e.g.,\ Facebook and Twitter) delete (most of the) user's data. The former
optimizes the amount of data the application keeps, but also retains all correlations between data
entities; the latter can lead to confusing semantics for the application (e.g.,\ nonsensical comment
threads) and the loss of useful application data. In both cases, accounts cannot be restored after
deletion.
\lyt{Applications such as Facebook or Twitter do support user account deactivation by hiding (most) of the
user's data, but retain all user data with their identifiers in their databases.}

Developers using \sys{} can specify policies that both anonymize unique identifiers, and remove
identifying information stemming from correlations between data entities. \sys{}'s support for
fine-grained decorrelation policies gives developers options beyond either completely retaining
these links or deleting of all of the user's data and their dependees (while still supporting
these options). The complexity of implementating such policies is hidden from the developer:
\sys{} automatically performs decorrelation in a way that allows for recorrelation upon
resubscription, while still achieving performance comparable to today’s widely-used databases and
requiring no modification of application schemas. 

\subsection{What would perfect decorrelation achieve?}
Decorrelation ideally breaks connections between data entities that would otherwise allow
an adversary to determine that two entities belong to the same (unsubscribed) user:
post-decorrelation, an adversary should not be able to distinguish between a scenario in which two data entities
have been generated by two distinct users, and one in which they were both generated by the same
(unsubscribed) user.  If two of a user's entities cannot be correlated back to the user, then at
most one of these entities may leak identifying information about the user.\footnote{Assume for a
contradition that both entities leak identifying information: then both entities are more likely to
have been generated by a particular identity than any other, contradicting our assumption.}  Thus,
if the content of individual data entities is appropriately anonymized, then perfect decorrelation prevents
identifying information from being leaked via correlations.

\subsubsection{Threat Model} 
An adversary aims to relink decorrelated entities to unsubscribed users after \sys{}
performs decorrelation. We make the following assumptions about such an adversary: 
\begin{itemize}
    \item An adversary can perform only those queries allowed by the application API, 
i.e.,\ can access the application only via its public interface. 
%\lyt{Alternatively, an adversary could perform arbitrary queries on some public subset of the
%application schema (e.g., all tables other than the mapping table, or all tables marked with some
%compliance policy); arbitrary queries over the
%entirety of the table are out of scope, unless ``private'' tables are removed and stored by
%unsubscribing users.}

    \item An adversary cannot perform application queries to the past or search web archives:
    information from prior application snapshots may reveal 
    exactly how data records were decorrelated from unsubscribed users. 

    \item An adversary cannot gain identifying information from arbitrary user-generated content (for
        example, a reposted screenshot, or text in user stories or comments). Decorrelation seeks to
        remove identifying information from user-generated data that can be enumerated or follows a
        specific pattern (e.g., a birthday or email address), and application metadata (e.g., date
        of postings, database ID columns).
\end{itemize}

\subsubsection{Modeling perfect decorrelation}
Application data is structured as tables, each containing a different \emph{data entity}, e.g.,\ a
story, user, or vote. Queries write, read from, and compute over entities.  All entities (e.g.,\
posts) that share a correlation with the same entity (e.g.,\ a user) form a \emph{cluster}
identified by that entity. In database terms, entities in the same cluster belong to the same table,
and have the same foreign key value (e.g.,\ a particular value for column \texttt{user\_id}). 

Decorrelation of a user breaks any clusters around the user into singleton clusters, each identified
by a unique ghost user. In essence, the user is exploded into many ghost users, each correlated with
only one of the user's data entities.

However, breaking up clusters around the user may not sufficiently decorrelate these entities from
the user. For example, stories belonging to the user may also cluster around a particular tag. Or
perhaps one of the user's stories is upvoted only by all of the users' friends (these upvotes
cluster around the story). These pieces of information allow an adversary to correlate a story back
to a single user, even when clusters around the user no longer exist.

The key observation here is that two types of clusters can still leak identifying information about
the user: (1) clusters identified by data entities owned by the users, and (2) clusters consisting
of the user's data entities, identified by other entities. Decorrelation recursively breaks up any
such clusters into singletons by introducing ghost entities (as was done with introducing ghost
users), thus removing any identifying information leaked from correlations between user's data
entities and other entities in the application.

More generally, decorrelation continues to recursively break up any clusters that may recorrelate a
data entity back to the identifier of a broken-up cluster. Let $A$ and $B$ be entity types. Let $a
\in A$ be the entity being decorrelated. $a$ identifies at least one cluster $B_a \subseteq B$.
Decorrelation on $a$ does the following: 
\begin{enumerate} 
    
    \item \textbf{Break direct clusters and
            recurse.} $a$ splits into ghost entities $A_g \subseteq A$, one for each entity $b\in B_a$.
            Decorrelation then runs recursively on each cluster entity $b$.
            
            Furthermore, if the $A_g$ are in a cluster identified by some entity $c$, $c$ is
            recursively decorrelated.

    \item \textbf{Break clusters around identity proxies and recurse.} If more than one of the
        $b \in B_a$ is also in a cluster that is identified by an entity $c \in C$ (an identity
        ``proxy''), then we decorrelate $c$ from its data. 
        
\end{enumerate}
For example, each story (the $b$) posted by a user $a$ belongs in a cluster $B_a$ identified by $a$.
Decorrelation step (1) reassigns each story to a ghost user. Then each story is itself decorrelated.
If stories are decorrelated into a ghost story per vote, for example, then the votes would be
decorrelated. The ghost stories would cluster by the ghost user, so this ghost user would have to
further decorrelate.

Decorrelation step (2) may find that a particular tag $c$ identifies at least two of the users'
stories (these stories belong in a cluster $B_c$ identified by $c$). The tag is then decorrelated to
ensure that nothing related to the tag may reassociate these stories with the same tag.  This
recursive decorrelation prevents any other correlations with the user's stories from leaking
identifying information about the stories' author.

\subsubsection{Why perfect decorrelation doesn't work in practice}
Perfect decorrelation takes decorrelation to the extreme, stripping any links to (real) application
data from data entities recursively related to the user via clusters. This removes as much
correlation-based identifying information as possible while keeping data entities present, but in
practice, completely decorrelated data entities may be useless to the application.  Applications may
lack a meaningful way to generate ghost entities: what does it mean for a story to become many ghost
stories?  And even if ghosts can be generated, the noise and data pollution from ghost entities may
instead affect the accuracy and semantics of the application: users may see meaningless content,
comment threads may be disjoint and scattered, and highly-ranked content may suddenly lose votes. 
%On the other extreme, however, performing no decorrelation at all fails to adequately de-identify
%user data. 
%Ghost entities are generated by replacing foreign key attributes with a unique identifier drawn
%from a random distribution; all other attributes can be replaced by default values, or
%using application-specific generators (for example, a randomly generated username or phone number).
%If foreign keys values are kept and consequently cluster the generated ghost entities, then these
%foreign entities must be decorrelated to ensure that ghosts cannot be grouped back together.
%
%\lyt{Given this ghost generation scheme, decorrelation of a story would then destroy story content
%(essentially deleting the story... the user would not get this content back). A
%smarter way to ghost stories may be to split up the content into ``pieces'', one per ghost story,
%and encrypt these pieces so that ghost stories cannot be relinked together, but the user could
%decrypt the pieces and reform the original story.}

\sys{} offers developers a way to decorrelate as much as possible \emph{while still retaining
application semantics}. As the next section describes, \sys{} provides decorrelation specification
primitives with which developers can specify which (and how) entities can be decorrelated from other
entities, and which entities must be kept clustered for
application correctness (e.g.,\ votes correlated with a story).

\subsection{Specifying Decorrelation Policies}
\lyt{FIGURE HERE} 

Decorrelation policies consist of two parts: 
\begin{enumerate}
    \item Ghost entity generation annotations on entity types (tables) for types which can be ghosted, and 
    \item Cluster policy annotations specifying how a cluster of one entity type identified (via
        foreign-key relations) by another entity type can be broken (e.g.,\ a cluster of stories
        identified by user).
\end{enumerate}

\subsubsection{Ghost Entity Generation}
Developers can choose between the following ghost generation annotations, which apply at 
per-column granularity.
\begin{itemize}
    \item \textbf{Cloned:} All ghosts have the same value as the original entity for this column.
        Recorrelation returns an entity with the original value.

    \item \textbf{Generated:} All ghosts have a generated value for this column. Generated values are
chosen to be random, a default value, or generated from the original value via a function provided by the developer.
        If the developer provides a reversible function (e.g.,\ has encrypted the original value
        with a user-specific key), then the original value is restored upon
        recorrelation; otherwise, recorrelation returns a generated value in place of the original.

        If a generated value is a foreign key into another table, either that table must 
        have a ghost generation policy, or the foreign key is set to an existing entity or placeholder ghost (e.g.,
        the foreign key is set to point to ``deleted story''). It is up to the developer to ensure
        referential integrity in the latter case.

\item \textbf{Single Clone, Generated Remainder:} One ghost has the same value as the original
        entity; all other ghosts have generated values. Recorrelation returns an entity with the original value.
\end{itemize}

\subsubsection{Cluster Policies}

\paragraph{Policy 1: Do not decorrelate.}

Application developers specify that specific types of clusters cannot be broken, thus retaining the edges between the cluster entities and the cluster's identifier entity.
Recursive decorrelation still propagates through the cluster entities. 
For example, perhaps only the sum of votes per story is ever queried by the application; clusters of
votes around stories can therefore remain without leaking identifying information.
%tags or categories may play an essential role
%for the application, and cannot be split into ghosts. Developers thus specify that clusters of stories 
%identified by tag cannot be broken up; However, if the tag is clustered around tag is not recursively decorrelated from its data.

\vspace{12pt}
\noindent Policy options:
\begin{itemize}
    \item \textbf{Remove user-related entities}: If decorrelation cannot break a cluster, then all
        cluster entities and their dependencies are removed.
    
    \item \textbf{Retain user-related entities}: The cluster should remain as-is. This option should
        only be selected if the developer knows that the cluster cannot leak information.

    \item \textbf{Achieve cluster threshold $t$}: At a high level, the cluster threshold
        $t$ limits the ability of an adversary to use an identity proxy as an alias for a user's
        identity.  \sys{} calculates what proportion of entities in these clusters belong to the
        entity being decorrelated. If the proportion is below $t$, no further actions are needed;
        otherwise, \sys{} generates enough ghosts such that the proportion drops below the
        threshold; ghosts are generated using the specified ghost generation policy for that entity
        type, using a randomly selected entity in the cluster as the original.
        This option is only possible if the developer has specified a ghost entity policy for the
        cluster entity type. 

        Note that if the ghosts are easily distinguished from actual cluster entities, there is
        little privacy benefit from this option.

        As an example, a reasonable cluster threshold might be 0.05 for clusters of stories identified by
        tag: less than 5\% of all stories identified by that tag should have been associated with an
        (unsubscribed) user. 
\end{itemize}

\paragraph{Policy 2: Decorrelate.}

Application developers specify that a cluster may be broken, and decorrelation can propagate down
these cluster edges.
For example, all clusters identified by a user (regardless of cluster type) can be removed: ghost
users are created for each cluster entity.

Resubscription (and recorrelation) removes any created ghost entities and restores any broken links.

\lyt{There could also be an option here of "decorrelate the minimum amount possible to reach a
cluster threshold"; not sure if it's really that useful.}

\begin{figure}
\begin{lstlisting}[language=Rust]
 pub enum GeneratePolicy {
     Random,
     Default,
     Custom<F>(f: F) // where F: FnMut(Column) -> Column,
 }
 pub enum GhostColumnPolicy {
     CloneAll,
     CloneOne(gp: GeneratePolicy),
     Generate(gp: GeneratePolicy),
 }
 pub type GhostPolicy = HashMap<Column, GhostColumnPolicy>;
 pub type EntityGhostPolicies = HashMap<Entity, GhostPolicy>;

 pub struct Cluster {
     cluster_entity: Entity,
     identifier_entity: Entity,
     foreign_key_name: String,
 }
 pub enum ClusterPolicy {
     NoDecorRemove(Cluster),
     NoDecorRetain(Cluster),
     NoDecorThreshold {
         c: Cluster,
         threshold: f64,
     }
     Decor(Cluster),
 }
 pub type ApplicationPolicy = 
    (EntityGhostPolicies, Vec<ClusterPolicy>);
\end{lstlisting}
    \caption{\sys{}-provided types to specify cluster and ghost generation policies.}
\end{figure}

    %\textbf{Achieve cluster threshold $t$}: 
%\sys{} generates ghost entities only enough to what proportion of entities in these clusters belong to the entity being
%        decorrelated. If the proportion is below the specified cluster threshold $t$, no further actions are needed; otherwise, \sys{} generates enough ghosts such that the proportion drops below the threshold. 
%Note that even for entities for which decorrelation is acceptable, they may only need to be
%decorrelated until their clusters meet the cluster threshold.

%At a high level, the correlation threshold limits the ability of an adversary to use an identity
%substitute as a proxy for a user's identity.  This correlation threshold is defined on a type of
%entity $C$ as follows. Let $B_a$ be the set of entities of type $B$ clustered by $a$, the entity
%undergoing decorrelation.  Let $c \in C$ be an entity that acts as an identity substitute for $a$, with
%$B_c \subseteq B_a$ clustered by $c$. Let $B_{total}$ be the set of entities of type $B$ clustered
%by $c$. The correlation threshold for the entity $c$ is the maximum value for the proportion $B_a /
%B_{total}$. Thus, lowering the threshold increases the number of entities in $B_{total}$ that belong
%to other users, and lowers the ability of the adversary to use $c$ to correlate the $b \in B_{c}$ to
%a single identity.

\subsection{Maintaining aggregate accuracy.}
\sys{} optionally allows for entities to be decorrelated without affecting queries which
specifically return aggregation results.

Queries that specifically perform aggregations and return statistical measures (e.g.,
the count of number of users in the system, or the number of stories per user), can return
significantly different results. This affects the utility of the data for the application: for
example, if the application relies on the number of stories per tag to determine hot topics, these
would be heavily changed if ghost tags were created.  In addition, the adversary may learn which
entities are ghosts: for example, an abnormally low count of stories per tag might indicate to an
adversary that these tags are ghost tags.  \lyt{But perhaps it's ok if an adversary can tell what's
a ghost, as long as it can't tell which user each ghost is correlated with.}

\sys{} stores and separately updates answers to aggregation queries;
these answers are updated when queries update the data tables, and these queries do not read from
the application tables (which may contain ghost records).

An alternate solution might analyze the aggregations performed by application queries, and then
introduce ghosts that lead to the same (or close-enough) aggregation result. For example, if a tag
is split into ghost tags, one per story associated with the tag, but the application still would
like the count of stories for this tag to be high, one of the ghost tags can be populated with many
ghost stories to retain the count of stories per tag.  \sys{} would remove any ghost stories that
were created upon recorrelation. Note that this solution 1) requires that generating ghosts is
admissible, and 2) may be impossible for certain combinations of aggregations (e.g., queries that
return both the average stories per tag and also the total number of stories).

\lyt{I don't *think* differential privacy really should be applied here, because we'd also face the
issue of running out of privacy budget. Adding noise might ensure that the impact of any one
(real/ghost) user is very little, but it has its own noise/utility tradeoff. Furthermore, the amount
of noise necessary if many ghost users are created might be too large.}
