\section{How Does \sys Apply and Reverse Disguises?}

Table~\ref{tab:funcs} describe \sys's internal functions run server-side to implement its API 
and apply or reverse disguises. 

Figures~\ref{fig:appdisg} and \ref{fig:revdisg} describe how \sys implements disguise application and
reversal respectively. Figure~\ref{fig:opexec} describes how disguise operations update application
state and produce and modify tokens, and 
Figure~\ref{fig:revtoken} describes how \sys uses a token's recorded
modification to reverse an applied disguise operation. 
Figure~\ref{fig:rpt} describes how \sys accesses the private tokens
corresponding to disguise $d$ and principal $p$ by recursively decrypting tokens and traversing
\tokls{pd}.

\sys's implementations of the remaining functions in Table~\ref{tab:funcs} simply read or write into
persistent maps indexed by principal and/or disguise.

\subsection{Using Disguise History to Delay Disguise Actions}

\noindent\textbf{\emph{Delayed Disguise Application.}}
A disguise $d_2$ may be retroactively applied to a data object $O$ revealed during reversal of
disguise $d_1$.  This occurs when $d_2$ applies to the original $O$, but not the $d_1$-disguised
$O$, and $d_1$'s tokens are inaccessible to \sys when applying $d_2$.

\sys uses the disguise history to know when to retroactively apply the
disguise when the data is revealed.  Disguising prior to restoration should be deterministic (based
on contents of data in application database, and e.g., not predicated on the current time).

\vspace{6pt}\noindent\textbf{\emph{Delayed Disguise Reversal.}}
In certain scenarios, \sys may not completely reverse a disguise $d_1$ if future disguise $d_2$
modifies the same data as $d_1$. If $d_2$ has applied a more recent, conflicting update to data
objects disguised by $d_1$, then the original data contents prior to $d_1$ cannot be restored. 

\sys logs the reversal of $d_1$ in the disguise history, and uses this history to detect if it
should reverse any conflicting updates performed by $d_1$ when $d_2$ is reversed.

\subsection{Token Migration and Modification}
\noindent\textbf{\emph{Global-to-Private Token Migration.}}
\sys requires migrating a global token to private token storage (in some \tokls{pd}) to prevent
globally accessible tokens from leaking information that should be stored as private by a subsequent
disguise.

Let $d_1$ and $d_2$ be two disguises applied in sequence.
Some \op{d_1} produces a global token \tdata{pd_1} recording an update to some data object
associated with principal $p$. \op{d_2} updates the same data object, also associated with $p$.
In the following scenarios, \tdata{pd_1} must be moved to \tokls{pd_1} when \op{d_2} executes:
\begin{itemize}
\item \op{d_1} and \op{d_2} both decorrelate the same data
\item \op{d_1} and \op{d_2} both modify the same data
\item \op{d_1} decorrelates or modified, and \op{d_2} removes the same data
\end{itemize}
If $d_2$ is reversed, \tdata{pd_1} again becomes globally accessible.

\vspace{6pt}\noindent\textbf{\emph{Token Modification to Prevent Information Leakage.}}
Let $d_1$ and $d_2$ be two disguises applied in sequence.
Consider the scenario in which \op{d_1} removes a data object $O$, but \op{d_2} would have decorrelated
or modified $O$.
\sys can determine if \op{d_2} applies to the removed $O$ data stored in global tokens by applying
the corresponding \op{d_2} predicate to the recorded token object data.

\op{d_1} produced a global token \tdata{pd_1} recording the contents of the removed $O$.
\sys should update this \tdata{pd_1} with \op{d_2}'s' decorrelation or modification update in order
to not leak information about $O$.
In addition, reversing $d_2$ should also reverse the modification to \tdata{pd_1}. If $d_1$ is
reversed, then reversing $d_2$ simply reverses the state of $O$ in the database instead of the staet
of $O$ in the global token.
%Alternatively, Rev($d_1$) could do nothing, and the token could simply be moved into the $d_2$ user-vault, meaning that O could not be reinserted and revealed until $d_2$ is reversed

\vspace{6pt}\noindent\textbf{\emph{Token Modification for Object Identification.}}
Tokens need to refer to the correct objects even if objects have been modified by future disguises.
Prior tokens should be revised by future disguises so that they apply correctly to the disguised
data.
This allows an earlier disguise to be reversed correctly.
An easy way to avoid needing to do this is to have all objects in the DB have unique, permanent
identifiers.

\vspace{6pt}\noindent\textbf{\emph{Delayed Disguise Reversal.}}
\sys will not reverse a disguise's modifications to a data objects until future disguises applied to
the same object have been reversed.
This occurs when the disguise's tokens are private and inaccessible; or cannot be reversed because
future disguises have modified, decorrelated, or removed the same data.
\sys uses the disguise history of all disguise applications/reversals to retroactively reverse
the disguise when it becomes feasible.

\subsection{Handling Interleaved Application Updates}
\noindent\textbf{\emph{Application Updates to Present Disguised Data.}}
\sys will not reveal disguised data still present in the database that has been updated by the
application. Application updates are equivalent to irreversible disguise operations.
\sys checks the current state of data against the latest disguised state
expected as recorded in the token; if they do not match, then an application update has occurred
since disguising, and \sys will not alter the data.

\vspace{6pt}\noindent\textbf{\emph{Application Updates to Removed Disguised Data}}
Application updates fail to update data removed by a disguise, which can later be revealed in
original form.
\sys requires all application updates that must immediately apply to revealed data (that is
currently disguised and removed) to be disguises themselves; otherwise, the application must handle
the possibility of non-updated data being revealed.
%\sys stores a query log of all application update queries, and reapplies them to any removed data by a disguise prior to revealing the data

%\vspace{6pt}\noindent\textbf{\emph{Note: Why would disguise reversal require tokens from prior disguises?}} 
%The application of a disguise $d$ may modify the contents of prior disguise $d'$ tokens!
%This can occur because some token \tdata{} may store removed data that, if in the database, would be
%affected by the application of $d$. When $d$ is applied, \tdata{}'s data is updated to reflect the
%modification by $d$; similarly, when $d$ is reversed, \tdata{}'s original contents are restored.


