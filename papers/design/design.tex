%-------------------------------------------------------------------------------
\section{Overview}
%-------------------------------------------------------------------------------
In normal database-backed applications, the application database contents completely capture data
held by the system. The application acts as the sole creator and updater of database contents. We  
target applications that have \emph{principals} (\ie user accounts in a \texttt{users} table) as first-class citizens; each principal $p$ owns some data objects in the database (\eg a row in an \texttt{address} table).
This is typical of social network sites such as Facebook or Reddit; and other web applications that
tie functionality to user accounts (\eg ecommerce or bank applications).

Our design considers \emph{disguisable application systems}, which consist of a typical database-backed
application and a disguising tool (\sys) operating alongside the application. This system operates
on three types of data:
\begin{enumerate}
    \item \emph{\textbf{Database Contents}}: The application creates and updates its database
        contents as in the typical non-disguisable setting. \sys also can read and write
        the database to apply and reverse disguises.
    \item \emph{\textbf{Disguise History}}: Data about which disguises have been applied or
        reversed, used to compose disguises on top of one another.
    \item \emph{\textbf{Disguise Tokens}}: Data produced as a byproduct by disguise application.
        Tokens encode information about disguise modifications to database contents, and are used by
        \sys to compose disguises on top of one another or reverse a disguise's modifications.
\end{enumerate}

%-------------------------------------------------------------------------------
\subsection{Security Goals}
%-------------------------------------------------------------------------------
We assume an attacker who, after compromising the system, can perform any actions the
application or \sys can perform, and access any data the application or \sys can access.
We assume \sys itself is honest but curious: when applying or reversing a disguise, \sys may temporarily
hold (in memory) plaintext token data, as well as private or symmetric keys, but is trusted to
forget them once the disguise action is complete. 
Finally, we assume standard security of public key and symmetric key primitives~\note{under a random
oracle model?}.
With this threat model, our goals are three-fold: 

\vspace{6pt}\noindent\emph{(1) Security of Restricted Disguise Application.}
We support disguises that optionally require authentication as a principal $p$ prior to being
applied. If so, only a client authenticated for principal $p$ can apply this disguise (and an
attacker unauthenticated as $p$ cannot).

\vspace{6pt}\noindent\emph{(2) Security of Private Disguise Tokens.} 
Tokens are optionally private to a single principal $p$, such that only the user who created
$p$'s account can authorize access to these tokens. An attacker who is not this user cannot
authorize or access private disguise tokens created prior to system compromise, even if authenticated as $p$.

\vspace{6pt}\noindent\emph{(3) Security of Private Disguise Token Metadata.}
An attacker cannot learn how many private disguise tokens created prior to system compromise are associated with a principal $p$'s data.

\vspace{6pt}\noindent\emph{Non-Goals.}
\begin{itemize}
    \item Security of private tokens or restricted disguises when an attacker has has knowledge of prior snapshots of system data.
    \item Security of restricted disguise application if an attacker compromises a principal's account
        and is authenticated as the principal.
    \item Security of private tokens and token metadata of $p$ if an attacker compromises the user
        who created $p$'s account.
    \item Security of private tokens and token metadata of $p$ if a user authorizes access to these
        tokens.
    \item Unlinkability guarantees of disguised principal data: there may remain implicit
        correlations based on undisguised content (e.g., a comment that mentions the author’s name).
\end{itemize}


%-------------------------------------------------------------------------------
\subsection{Openly Disguisable Application Systems}
%-------------------------------------------------------------------------------
In the \emph{open setting}, no data is kept hidden from \sys. In particular, the different data
types have the following access policies:
\begin{enumerate}
    \item Database Contents: accessible to both the application and \sys.
    \item Disguise History: accessible to \sys.
    \item Disguise Tokens: accessible to \sys.
\end{enumerate}
