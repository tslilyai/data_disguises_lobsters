% TODO add description of what token contents contain

%-------------------------------------------------------------------------------
\section{Threat Model}
%-------------------------------------------------------------------------------
An adversary can access any data stored as plaintext by the application or \sys, including data not
exposed via its client-facing API. We assume that an adversary cannot access prior snapshots of the
application database, or observe modifications or memory accesses performed during disguise
application. Furthermore, undisguised content and implicit correlations based on this content (e.g.,
a comment that mentions the author’s name) are out of scope.

We assume \sys is honest but curious: when applying or reversing a disguise, \sys may temporarily
hold (in memory) plaintext token data, as well as private or symmetric keys, but is trusted to
forget them once the disguise action is complete.  

Finally, we assume standard security of public key and symmetric key primitives~\note{under a random
oracle model?}.

%-------------------------------------------------------------------------------
\section{Notes about Disguise Protocols}
%-------------------------------------------------------------------------------
\textbf{\emph{Why would disguise reversal require tokens from prior disguises?}} 
Note that the application of a disguise $d$ may modify the contents of prior disguise $d'$ tokens!
This can occur because some token \tdata{} may store removed data that, if in the database, would be
affected by the application of $d$. When $d$ is applied, \tdata{}'s data is updated to reflect the
modification by $d$; similarly, when $d$ is reversed, \tdata{}'s original contents are restored.


