%%%%%%%%%%%%%%%%%%%%%%%%%%%%%%%%%%%%%%%%%%%%%%%%%%%%%%%%%%%%%%%%%%%%%%%%%%%%%%%%%%%%%%%%%%
\section{Disguise Application and Reversal Semantics}
\label{sec:comp}
%%%%%%%%%%%%%%%%%%%%%%%%%%%%%%%%%%%%%%%%%%%%%%%%%%%%%%%%%%%%%%%%%%%%%%%%%%%%%%%%%%%%%%%%%%
This section describes the semantics of disguising provided by \sys. In particular, we describe the
end state of data given some history of disguise applications and reversals.

\vspace{6pt}\noindent\textbf{\emph{Some Notation.}}
We describe disguise histories as a list of \app{d} and \rev{d} actions, where \app{d} corresponds
to the application of disguise $d$, and \rev{d} corresponds to the reversal of $d$. Time moves to
the right in the list.

For every data object $x$ in the system, \xstart describes its initial state, and
\xhist{[\app{d_1}, \dots]} describes its state after \sys has applied the history [\app{d_1},
$\dots$].


%%%%%%%%%%%%%%%%%%%%%%%%%%%%%%%%%%%%%%%%%%%%%%%%%%%%%%%%%%%%%%%%%%%%%%%%%%%%%%%%%%%%%%%%%%
\vspace{6pt}\noindent\textbf{\emph{Composing Multiple Disguise Applications.}}
Let $d_1$ and $d_2$ be two disguises, where $d_1$ is applied after $d_2$.
to produce disguise history [\app{d_1}, \app{d_2}]. 
%
Let $x$ be some data object in the system, where \xstart is its initial state, and
\xhist{[\app{d_1}, \app{d_2}]} be the final state after \sys applies $d_1$ and $d_2$.
%
If both $d_1$ and $d_2$ update $x$, what is \xhist{[\app{d_1}, \app{d_2}]}?
We consider two possible end states: 
%
\begin{enumerate}
\item[(\appcompone)] \xhist{[\app{d_1}, \app{d_2}]} reflects the application of $d_1$ \textbf{and} $d_2$ to \xstart. In other
words, at least one of the \op{d_2}s' predicates \texttt{pred} matches \xstart, and thus \op{d_2} is
applied to \xhist[\app{d_1}] even if \texttt{pred} does not match \xhist[\app{d_1}]. 
\op{d_2} updates are applied on top of any $d_1$'s' updates if the two conflict.

\item[(\appcomptwo)] \xhist{[\app{d_1}, \app{d_2}]} reflects the application of $d_1$ to \xstart, followed
by the application of $d_2$ to \xhist{[\app{d_1}]}. \op{d_2} predicates match only against
\xhist{[\app{d_1}]}, and are not aware of the original state \xstart.
\end{enumerate}

\noindent
For example, let \op{d_1} decorrelate all posts from authors predicated
on \texttt{author = Bea}, and let \op{d_2} remove all posts predicated on \texttt{author = Bea}.
%
\begin{enumerate}
\item[(\appcompone)] Both \op{d_1} and \op{d_2} update posts originally having author ``Bea'', resulting in the
removal of all posts originally with author Bea.

\item[(\appcomptwo)] Applying \op{d_1} results in all posts with author ``Bea'' having
pseudoprinicipal authors such as ``anonFox.'' \op{d_2} only knows the state of posts after
\op{d_1} occurs, and does not remove any posts because no post has author ``Bea''.
\end{enumerate}

The choice of whether \xhist{[\app{d_1}, \app{d_2}]} should result in \appcompone or \appcomptwo depends on the
specific application and is left to the developer to specify. 

In certain scenarios, however, \sys can only produce \appcomptwo.  In order to achieve \appcompone,
$d_2$ must evalute predicates against \xstart, which requires knowing what \xstart is.  To learn
\xstart, \sys must know how $d_1$ modified $x$ (\eg knowing that a post with author ``anonFox''
originally was a post with author ``Bea''); this information, however, is stored in tokens produced
by $d_1$. 

These tokens may be inaccessible \sys when applying $d_2$, as they could be stored privately for
principal $q$: a client who invokes $d_2$ while authenticated as
principal $p \neq q$, cannot access $q$'s tokens for $d_1$. 
In this case, $d_2$ can only match against \xhist{[\app{d_1}]} and achieve \appcomptwo.
This is shown in Table~\ref{tab:composeapp}.

\begin{table}[h]
\centering
\begin{tabular}{ c | c c }
    & \multicolumn{2}{c}{\textbf{\app{d_2} client access to $d_1$'s tokens?}}\\
    & \textbf{Yes} & \textbf{No} \\
\hline
    \xhist{[\app{d_1},\app{d_2}]}& \appcompone or \appcomptwo & \appcomptwo 
\end{tabular}
\vspace{6pt}

\caption{Possibilities for \xhist{[\app{d_1},\app{d_2}]} depending on whether the client invoking
    $\app{d_2}$ can access tokens from $d_1$ regarding $x$.}
\label{tab:composeapp}
\end{table}

%%%%%%%%%%%%%%%%%%%%%%%%%%%%%%%%%%%%%%%%%%%%%%%%%%%%%%%%%%%%%%%%%%%%%%%%%%%%%%%%%%%%%%%%%%
\vspace{6pt}\noindent\textbf{\emph{Semantics of Disguise Reversals.}}
We now consider composition of disguises in the presence of disguise reversals.

\begin{table}[h]
\centering
\begin{tabular}{ c | c c }
    & \multicolumn{2}{c}{\textbf{\rev{d} client access to $d$'s tokens?}}\\
    & \textbf{Yes} & \textbf{No} \\
\hline
    \xhist{[\app{d},\rev{d}]} & \xstart & \xhist{[\app{d}]}
\end{tabular}
\vspace{6pt}

\caption{Possibilities for \xhist{[\app{d},\rev{d}]} depending on whether the client invoking
    \rev{d} can access tokens from $d$ regarding $x$.}
\label{tab:composeapprev}
\end{table}

We first consider \textbf{\xhist{[\app{d},\rev{d}]}}; Table~\ref{tab:composeapprev}
illustrates how this state depends on which client invokes \rev{d}, and how $d$'s tokens for $x$
are stored.
As expected, when the client is authenticated as a principal without access to $d$'s
private tokens, then the updates applied by $d$ to $x$ cannot be reversed.

\begin{table}[h]
\centering
\begin{tabular}{ c | c c }
    \textbf{$d_1$ and $d_2$'s Updates} & \multicolumn{2}{c}{\textbf{\rev{d_1} client access 
        to $d_1$'s tokens}}\\
    & \textbf{Yes} & \textbf{No} \\
    \hline
    \textbf{Independent} & \xhist{[\app{d_2}]} & \xhist{[\app{d_1},\app{d_2}]}\\
    \textbf{Conflict} & \xhist{[\app{d_1},\app{d_2}]} & \xhist{[\app{d_1},\app{d_2}]}
\end{tabular}
\vspace{6pt}
    \caption{\xhist{[\app{d_1},\app{d_2},\rev{d_1}]} depending on whether the client invoking
    \rev{d_1} has access to $d_1$'s tokens, and whether updates from $d_1$ and $d_2$ to $x$ conflict.}
\label{tab:composeapprev1}
\end{table}

We next consider \textbf{\xhist{[\app{d_1},\app{d_2},\rev{d_1}]}}, the state of $x$ when a disguise
$d_1$ is reversed after a subsequent disguise $d_2$ has been applied. 
Table~\ref{tab:composeapprev1} illustrates how this state 
depends on the invoking client's access to $d_1$ tokens, and whether $d_1$ and $d_2$ performed
conflicting updates to $x$.

When the client is authenticated as a principal without access to $d_1$'s private tokens, then the
updates applied by $d_1$ to $x$ cannot be reversed.  However, even if the client invoking \rev{d_1}
has access to $d_1$'s tokens, \sys will fail to reverse $d_1$'s updates to $x$ if $d_2$ and $d_1$
both update $x$ in a conflicting manner. $d_1$ must not restore data that was also disguised by
$d_2$, and until $d_2$ is reversed, invocations of \rev{d_1} will not reveal \xstart.

%%%%%%%%%%%%%%%%%%%%%%%%%%%%%%%%%%%%%%%%%%%%%%%%%%%%%%%%%%%%%%%%%%%%%%%%%%%%%%%%%%%%%%%%%%
\iffalse
%\begin{table}[h]
%\centering
%\begin{tabular}{ c | c c }
%%%& \multicolumn{2}{c}{\textbf{$d_1$ Tokens for $x$}}\\
%\textbf{Client Invoking $\rev{d_1}$ }& \textbf{Global} & \textbf{Private to $p$}\\
%\hline
%{Authenticated as $q \neq p$} & \xstart & \xhist{[\app{d_1}]}\\
%{Authenticated as $p$} & \xstart & \xstart\\
%\end{tabular}
%\vspace{6pt}
%\caption{\xhist{[\app{d_1},\app{d_2},\rev{d_1},\rev{d_2}]} depending on the state of the client invoking
%\rev{d_1}, and the state of tokens from $d_1$ regarding $x$.
%We assume the client has access to tokens from $d_2$ regarding $x$.
%}
%\label{tab:composeapprev2}
%\end{table}
%
%Finally, we consider how reversals compose with other reversals by analyzing
%\textbf{\xhist{[\app{d_1},\app{d_2},\rev{d_1},\rev{d_2}]}}. 
%Table~\ref{tab:composeapprev2} illustrates these outcomes (assuming that the client invoking
%\rev{d_2} has access to $d_2$'s tokens for $x$).
%
%This is a necessary scenario to consider (whose semantics cannot be derived from
%Tables~\ref{tab:composeapprev}) or~\ref{tab:composeapprev1} because $d_2$ may have updated $x$ in a
%way that conflicted with $d_1$, so \rev{d_1} could not actually revert the update performed by $d_1$
%to $x$.  For example, \rev{d_1} cannot recorrelate $x$ that $d_1$ decorrelated, if $x$ has been
%removed by $d_2$.

\subsection{Handling Interleaved Application Updates}
\noindent\textbf{\emph{Application Updates to Present Disguised Data.}}
\sys will not reveal disguised data still present in the database that has been updated by the
application. Application updates are equivalent to irreversible disguise operations.
\sys checks the current state of data against the latest disguised state
expected as recorded in the token; if they do not match, then an application update has occurred
since disguising, and \sys will not alter the data.

\vspace{6pt}\noindent\textbf{\emph{Application Updates to Removed Disguised Data.}}
Application updates fail to update data removed by a disguise, which can later be revealed in
original form.
\sys requires all application updates that must immediately apply to revealed data (that is
currently disguised and removed) to be disguises themselves; otherwise, the application must handle
the possibility of non-updated data being revealed.
%\sys stores a query log of all application update queries, and reapplies them to any removed data by a disguise prior to revealing the data

\begin{table}[h]
\footnotesize
\centering
\begin{tabular}{ c | c c c }
& \textbf{$d_2$ Decorrelates} & \textbf{$d_2$ Modifies} & \textbf{$d_2$ Removes}\\
\hline
\textbf{$d_1$ Decorrelates}  & \xhist{[\app{d_2}]} & \xhist{[\app{d_2}]} &\xhist{[\app{d_2}]} \\
\textbf{$d_1$ Modifies} & \xhist{[\app{d_2}]} & \xhist{[\app{d_2}]}\checkmark
&\xhist{[\app{d_2}]}\checkmark \\
\textbf{$d_2$ Removes}  & \xhist{[\app{d_2}]}\checkmark & \xhist{[\app{d_2}]}\checkmark
&\xhist{[\app{d_2}]}\checkmark \\
\end{tabular}
\vspace{6pt}
\caption{\xhist{[\app{d_2}, \app{d_2}, \rev{d_1}]} when $d_1$ is global, and $d_2$ is private.}
\label{tab:revinternal}
\end{table}

\begin{table*}[h]
\centering
\begin{tabular}{ c | c c c c }
\textbf{History} & \textbf{(Global $d_1$, Global $d_2$)} & \textbf{(Private $d_1$, Private $d_2$)} &
\textbf{(Private $d_1$, Global $d_2$)} & \textbf{(Global $d_1$, Private $d_2$)} \\
\hline
\xhist{[\app{d_1}, \app{d_2}]} & \appcompone or \appcomptwo & \appcompone or \appcomptwo & \appcomptwo & \appcompone or \appcomptwo \\

\xhist{[\app{d_1}, \app{d_2}, \rev{d_2}]} & \xhist{[\app{d_1}]} & \xhist{[\app{d_1}]} & \xhist{[\app{d_1}]} & \xhist{[\app{d_1}]}\\

\xhist{[\app{d_2}, \app{d_2}, \rev{d_1}]} & \xhist{[\app{d_2}]} & \xhist{[\app{d_2}]} &
\xhist{[\app{d_2}]} & \xhist{[\app{d_2}]} or \xhist{[\app{d_1}, \app{d_2}]}\\
\vspace{6pt}
\end{tabular}

\caption{End state of object $x$ for various histories, depending on whether $d_1$ and
$d_2$ are private or global.}
\label{tab:compose}
\end{table*}

\fi
