\section{Disguise Tokens}

Tables~\ref{tab:datatokens} and \ref{tab:privtokens} describe the attributes contained in a \tdata{pd} and
\tpriv{pdp'} token respectively. Here we describe how \sys enforces access control to tokens, and
hides the number of tokens corresponding to each principal $p$ and disguise $d$.

\paragraph{Token Access Authorization.}
If a \globalop{d} produces \tdata{pd}, \sys stores \tdata{pd} in plaintext.
However, if a \privop{d} produces \tdata{pd}, \sys encrypts \tdata{pd} 
such that only $p$ can authorize access to \tdata{pd} plaintext.\
Any produced \tpriv{pdp'} is also encrypted such that only $p$ can authorize access to \tpriv{pdp'} plaintext.

\sys encrypts each token with symmetric key \symk{pd} during disguise application. 
To ensure only $p$ can authorize the decryption of its associated \tdata{pd} and \tpriv{pdp'}
ciphertexts, \symk{pd} itself is encrypted with \pubk{p} to create \ek{pd}, and \sys stores a mapping
from $p$ and $d$ to \ek{pd}. This additionally allows \sys to support \fn{LoadEncKey/LoadEncKeys}
calls.

\lyt{Not mentioning the strawman solution that encrypts each token with \pubk{p} for now.}
%An alternative strawman solution would encrypt each token with \pubk{p}; however, this requires
%sending all tokens 

\paragraph{Hiding Token Metadata.}
\sys must determine which token ciphertexts to decrypt after $p$ authorizes access by
decrypting \ek{pd} with \privk{p} and returning \symk{pd}. 
Naively, \sys could map $p$ and $d$ to all associated \fn{Enc(\tdata{pd}) and Enc(\tpriv{pdp'})} 
However, this allows an adversary to learn \emph{how many} private tokens correspond to $p$, metadata that
\sys should keep private.

An impractical strawman solution could require \sys to attempt to decrypt \emph{all} tokens with
\symk{pd}.  \sys utilizes a better solution, namely storing encrypted tokens for $p$ and $d$ in an
encrypted linked list \tokls{pd}, and only remembering the tail of the list \toklstail{pd}.  When a
new \tdata{pd} or \tpriv{pdp'} token is produced, \sys sets the token's \fn{nextEncToken} field to
point to the current \toklstail{pd} before encrypting it.

With this linked-list design, no party except for $p$ can traverse the list and determine how
many\tdata{pd} and \tpriv{pdp'} tokens exist for any $d$.

\begin{table}[t]
\centering
\begin{tabular}{ c p{.8\linewidth} }
    \fn{tokenID} & unique identifier for this token\\
    \fn{objID} & unique identifier for the data object modified by the \op{d} producing the token\\
    \fn{updateType} & decorrelate, modify, or remove\\
    \fn{oldValue} & unmodified value of object \fn{objID}\\
    \fn{newValue} & modified value of object \fn{objID}\\
    \fn{nextEncToken} & pointer to the next token in the list of private tokens produced by disguise $d$ for
    principal $p$, \texttt{NULL} if none exists or if token is global\\
\end{tabular}
\caption{\tdata{pd} Attributes}
\label{tab:datatokens}
\end{table}

\begin{table}[t]
\centering
\begin{tabular}{ c p{.8\linewidth} }
\fn{tokenID} & unique identifier for this token\\
\fn{anonPrivKey} & \privk{p'} of generated anonymous principal $p'$\\
\fn{nextEncToken} & pointer to the next token in the list of private tokens produced by disguise $d$ for
principal $p$, \texttt{NULL} if none exists or if token is global\\
\end{tabular}
\caption{\tpriv{pdp'} Attributes}
\label{tab:privtokens}
\end{table}
