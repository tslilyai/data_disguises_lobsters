%-------------------------------------------------------------------------------
\section{Setting and Goals}
%-------------------------------------------------------------------------------

\begin{table*}[t!]
\centering
\begin{tabular}{ c p{.8\linewidth} }
\textbf{Symbol} & \textbf{Description} \\
\hline
    \vspace{6pt}
$p$ & an application principal, corresponding to a user ID in the application\\
    \vspace{6pt}
$d$ & an instance of a disguise specification (developer-provided) that can be applied or revealed by a
    principal via \sys's API\\
    \vspace{6pt}
\op{d} & an operation specified as part of disguise $d$ (either a removal, modification, or decorrelation)\\
    \vspace{6pt}
\capa{pd} & capability required to grant \sys permissions to compose on top of disguise $d$'s
    updates, or reveal updates applied by disguise $d$. Permissions are associated with some
    principal $p$.\\
    \vspace{6pt}
\pubk{p} & public key of $p$, used to encrypt \tokk{pd} and \tpriv{pq} \\
    \vspace{6pt}
\privk{p} & private key of $p$ \\
    \vspace{6pt}
\addr{pd} & address at which encrypted \tokk{pd} and \tdata{pd} ciphertexts are located\\
    \vspace{6pt}
\tokk{pd} & symmetric key used to perform encrypt \tdata{pd}\\
    \vspace{6pt}
\tdata{pd} & data token storing undisguised data associated with $p$ prior to application of $d$.\\
    \vspace{6pt}
\tpriv{pq} & private key token storing the private key $\pubk{q}$ for a pseudoprincipal $q$ created
    when some disguise decorrelates data from $p$.\\
    \end{tabular}
\caption{Notation used to describe \sys's design.}
\label{tab:notation}
\end{table*}

%
Our design considers typical database-backed applications that link a library provided by our disguising
tool (\sys).
%
This library provides an API that the application uses to specify disguises, apply them, and (in some
cases), reveal disguised data with proper authorization (Figure~\ref{f:api}).
%
\sys takes care of executing the database transformations required as part of a disguise, and stores
disguised data in a secure way, subject to the threat model we describe in \S\ref{s:threat}.
%
To work with \sys, the application uses two abstractions:
\begin{itemize}
    \item Application principals, which correspond to natural persons using the application, and
	which are uniquely identifiable (\eg via a user ID).
        %
	We denote an application principal as $p \in P$, where $P$ is the set of all active and
	inactive application principals.
	%
    \item A disguise specification that consists of one or more disguises, $d \in D$, which each specify
	how \sys transforms the application database when $d$ is invoked by a principal $p$.
	%
	This results in a unique \emph{disguise instance} $d_p$; a principal can only apply a disguise
        once.
	%
\end{itemize}
%
\sys considers operates on two categories of data:
\begin{enumerate}
    \item \emph{\textbf{Database Contents}}: The application reads and writes database
        contents, which include visible, undisguised and currently-disguised data.
        \sys modifies the data database contents when it applies and reveals disguises.
    \item \emph{\textbf{Permissions}}: \sys produces a set of permissions when it applies a
        disguise. These permissions allow \sys to compose disguises on top of one another, or to
        reveal the updates done by a disguise. \ms{This sounds very operational. What \emph{are} permissions?
        I think they are capabilities that a principal can invoke to act upon disguised data.}
        %}\emph{\textbf{Disguise Tokens}}: Data generated by \sys's
        %disguise application. Tokens encode information about disguise modifications to database
        %contents, which \sys uses to compose disguises on top of one another or to reveal
        %modifications done by a disguise.
\end{enumerate}

%-------------------------------------------------------------------------------
\subsection{Threat Model}
\label{s:threat}
%-------------------------------------------------------------------------------

%
Data disguises protect user information in a web application against external observation
and service compromise.
%
An external observer is a user of the web application (authenticated or unauthenticated) who
observes information exposed through using the application.
%
A service compromise occurs when an attacker compromises the web application and potentially
gains full access to the server.
%
The attacker therefore can access any data stored, perform any actions the application can
perform, and access any information available to \sys.
%

%
A data disguise guarantees that the disguised data is hidden from any future attackers unless
explicitly revealed by an authorized principal.
%
In particular, an attacker who compromises \sys at time $t$ learns \emph{nothing but}:
\begin{enumerate}[nosep]
  \item the (plaintext) contents of the application database at or after time $t$;
  \item the disguises invoked, and the identity of the principals invoking them, after time $t$; and
  \item the results of revealing, after time $t$, disguises applied prior to $t$.
\end{enumerate}
%
We make standard assumptions about the security of cryptographic primitives: attackers cannot
break encryption and keys stored with non-colluding clients are safe.
%

%
\sys operates in an honest-but-curious setting: even if compromised, \sys faithfully executes
its protocols, but exposes all data accessed to the attacker.
%

\subsection{Security Goals}
%
With this threat model, \sys seeks to meet three security goals:
%

%
\vspace{6pt}\noindent\textbf{\emph{(1) Authorized Disguises.}}
%
%We support disguises that require authentication as an admin, or as a user principal $p$, prior to being applied (or
%revealed).
Only a client properly authenticated as a principal $p$ who is allowed to invoke $d$ can apply (and reveal)
$d_p$.
%

\vspace{6pt}\noindent\textbf{\emph{(2) Secure Disguise-generated Permissions.}}
%
%Permissions are optionally private to a single principal $p$.
%
Only a client who has the capability \capa{pd} can grant \sys permissions generated
by disguise instance $d_p$.
%
%can find the ciphertexts corresponding to $p$'s tokens generated by disguise $d$;
%\sys stores a public key \pubk{p} for every principal, and ensures that
%only a client who knows the corresponding private key \privk{p}
%can decrypt these ciphertexts.
%
%An attacker without \capa{pd} cannot access private disguise tokens created prior to time of compromise, even if authenticated as $p$.

\vspace{6pt}\noindent\textbf{\emph{(3) Privacy of Disguise History.}}
%
An attacker cannot learn the set of disguises that have disguised a principal's data.
%
An attacker only learns a disguise instance $d_p$ exists if an authenticated principal $p$ provides
\capa{pd} to \sys.
%$p$ authorizes access to private token ciphertexts for $d$ with
%\capa{pd}.

\vspace{6pt}\noindent\textbf{\emph{Non-Goals.}}
%
Under \sys's threat model, the following properties are out of scope:
%
\begin{itemize}
    \item Security of disguised data when an attacker has prior snapshots the system.
    \item Authorized disguises if an attacker authenticates as a principal $p$. \ms{this seems trivial}
    \item Security of $p$'s permissions and disguise history if $p$'s private eky, \privk{p}, is compromised.
    \item Security of $p$'s permissions from $d$ and knowledge that $d_p$ exists if an
	authorized client with \capa{pd} grants \sys access to $p$'s permissions from $d_p$. \ms{as above}
    \item Security of permission metadata such as the number of disguises applied to, or number of
	permissions associated with, a particular principal $p$.
    \item Privacy guarantees about undisguised data: if identifying data about $p$ is not covered by
        disguise specification $d$, it remains visible to attackers after $d_p$ applies.
        For example, a disguise that leaves contents of posts unmodified does not hide identifying references
	to users in that content.
    \item Hiding whether data is disguised or not. For example, if a post's author is anonymized as ``anonFox'',
        \sys leaks the fact that a post's author has been disguised.
\end{itemize}

%\vspace{6pt}\noindent\textbf{\emph{(2) Security of Private Disguise Token Plaintext.}}
%Tokens are optionally private to a single principal $p$.
%Only a client who has capability \capa{pd}
%can find the ciphertexts corresponding to $p$'s tokens generated by disguise $d$;
%\sys stores a public key \pubk{p} for every principal, and ensures that
%only a client who knows the corresponding private key \privk{p}
%can decrypt these ciphertexts.
%%
%An attacker without either \privk{p} or \capa{pd} cannot access private disguise tokens created prior to time of compromise, even if authenticated as $p$.


%\begin{table}[h]
%\centering
%    \begin{tabular}{ c c c }
%        \textbf{Data} & \textbf{\sys?} & \textbf{p?}\\
%\hline
%        Database Contents & \checkmark & \checkmark \\
%        Global Disguise Tokens & \checkmark & \checkmark \\
%        Private Disguise Tokens & & \checkmark \\
%        Set of Disguises Applied to Principal & & \checkmark \\
%\end{tabular}
%    \caption{Desired data access policies.}
%\label{tab:accpriv}
%\end{table}

