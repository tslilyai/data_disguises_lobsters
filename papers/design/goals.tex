%-------------------------------------------------------------------------------
\section{Setting and Goals}
%-------------------------------------------------------------------------------
\begin{table*}[t!]
\centering
\begin{tabular}{ c p{.8\linewidth} }
\textbf{Symbol} & \textbf{Description} \\
\hline
$d$ & a disguise that can be applied or reversed via \sys's API\\
    \vspace{6pt}
$p$ & an application principal, corresponding to a user ID in the application\\
    \vspace{6pt}
\pubk{p} & public key of $p$ \\
    \vspace{6pt}
\privk{p} & private key of $p$ \\
    \vspace{6pt}
%\sysk & secret key used by \sys to perform the first level of encryption of private tokens (of
%    any disguise and any principal) \lyt{What happens if \sysk is compromised? then a symmetric key can be
%    faked, but this key won't able to decrypt anything.... do we need validation at all?}\\
%    \vspace{6pt}
\symk{pd} & symmetric key used to perform the second level of encryption of private tokens associated with principal $p$ produced by disguise $d$\\
    \vspace{6pt}
\dk{p} & symmetric key used to encrypt the pointer to an array of \symk{pd}s for all $d$
    disguising data of principal $p$\\
    \vspace{6pt}
\op{d} & an operation of disguise $d$ (either a removal, modification, or decorrelation)\\
    \vspace{6pt}
\tdata{pd} & token storing undisguised data associated with $p$ and produced by $d$.\\
    \vspace{6pt}
\tpriv{pdq} & token storing private key $\pubk{q}$ when \op{d} decorrelates data from $p$.\\
    \vspace{6pt}
\tokls{pd} & linked list of private token ciphertexts associated with $p$ and produced by $d$
\end{tabular}
\caption{Notation used to describe \sys's design.}
\label{tab:notation}
\end{table*}

Our design considers \emph{disguisable application systems}, which consist of a typical database-backed
application with an integrated disguising tool (\sys).
We assume the following elements:
\begin{itemize}
    \item $d$: a disguise that can be applied or reversed via \sys's API. Disguises modify and remove
    application data to hide revealing information. 
    \item $p$: an application principal, corresponding to a user ID in the application. 
\end{itemize}

\noindent
%In normal database-backed applications, the application database contents completely capture data
%held by the system: the application acts as the sole creator and updater of database contents. 
A disguisable application system operates on two categories of data:
\begin{enumerate}
    \item \emph{\textbf{Database Contents}}: The application reads and modifies database
        contents, which include visible, undisguised and currently-disguised daa.
        \sys modifies the data database contents when it applies and reveals disguises.
    \item \emph{\textbf{Disguise Tokens}}: Data generated by \sys's
        disguise application. Tokens encode information about disguise modifications to database
        contents, and \sys uses them to compose disguises on top of one another or to reveal a
        disguise. 
\end{enumerate}

%-------------------------------------------------------------------------------
\subsection{Threat Model}
%-------------------------------------------------------------------------------

\begin{table}[h]
\centering
    \begin{tabular}{ c c c }
        \textbf{Data} & \textbf{\sys?} & \textbf{p?}\\
\hline
        Database Contents & \checkmark & \checkmark \\
        Global Disguise Tokens & \checkmark & \checkmark \\
        Global Disguise Token Metadata & \checkmark & \checkmark \\
        Private Disguise Tokens & & \checkmark \\
        Private Disguise Token Metadata & & \checkmark \\
\end{tabular}
    \caption{Desired data access policies.}
\label{tab:accpriv}
\end{table}

An attacker compromises the web application and gains full access to the backend server.
%
This means that the attacker can access any data stored on the server, perform any actions the
application can perform, and access any information about disguises tracked by \sys.
%
The application and the data disguising tool, \sys, are in the same untrusted domain (\ie they
have the same threat model, and compromising one also compromises the other).
%

%
\sys operates in an honest-but-curious setting: even if compromised, \sys faithfully executes
its protocols.
%
Further, when it applies or reverses a disguise, \sys may temporarily
hold (in memory) plaintext token data, but is trusted to delete it once the disguise is
complete.
%
Finally, we assume standard security of public key and symmetric key primitives.

With this threat model, our goals are three-fold: 

\vspace{6pt}\noindent\textbf{\emph{(1) Security of Disguise Access Control.}}
We support disguises that require authentication as an admin, or as a user principal $p$, prior to being applied (or
reversed). Only a properly authenticated client can apply (and reverse) selective disguises
(and an attacker authenticated as the wrong party cannot).

\vspace{6pt}\noindent\textbf{\emph{(2) Security of Private Disguise Tokens.}}
Tokens are optionally private to a single principal $p$. 
\sys stores a public key \pubk{p} for every principal, and ensures that 
only a client who knows the corresponding private key \privk{p} can access $p$'s tokens generated by
disguise $d$; an attacker without \privk{p} cannot authorize access to private disguise tokens
created prior to time of compromise, even if authenticated as $p$.

\vspace{6pt}\noindent\textbf{\emph{(3) Security of Private Disguise Token Metadata.}}
Only a client with private key \privk{p} knows how many private disguise tokens correspond to
principal $p$. An attacker cannot learn how many tokens are associated with any principal prior to
time of compromise.

\vspace{6pt}\noindent
Table~\ref{tab:accpriv} illustrates our goal's data access policies.

\vspace{6pt}\noindent\textbf{\emph{Non-Goals.}}
\begin{itemize}
    \item Security of private tokens or restricted disguises when an attacker has prior snapshots of system data.
    \item Security of disguise application restricted to $p$ if an attacker authenticates as $p$.
    \item Security of $p$'s private tokens and token metadata if \privk{p} is compromised.  
    \item Security of $p$'s private tokens and token metadata if an authorized client with \privk{p}
        grants \sys access to these tokens. 
    \item Any privacy guarantees about information left after disguise application: \sys 
        simply applies a disguise according to the developer-provided specification.
        For example, a disguise may not modify data that leaks the writer's name (e.g., by leaving post content unmodified)
    \item Hiding which application data objects have been disguised. For example, the author of a
        paper may be replaced with ``anonFox,'' leaking information that the paper has been
        disguised.
\end{itemize}


\iffalse
%Table~\ref{tab:dispriv} summarizes these changes.
\begin{table}[h]
\centering
    \begin{tabular}{ p{0.18\linewidth} p{.8\linewidth}}
        \textbf{Data} & \textbf{Disguise Effects}\\
\hline
        Database Contents & \sys modifies database contents, converting database rows to disguised
        versions according to the disguise specification.        \\
        Disguise Tokens & Each disguise database modification generates a token recording the
        modification. \sys saves global tokens in plaintext. For private tokens storing
        updates to $p$'s data, \sysencrypts them such with \pubk{p} and stores it in a bag with other private token ciphertexts.\\
\end{tabular}
\caption{Private setting effects of a disguise.}
\label{tab:dispriv}
\end{table}

%-------------------------------------------------------------------------------
\subsection{Strawman: Security in an Open Setting}
%-------------------------------------------------------------------------------
In the \emph{open setting}, no type of data is kept hidden from \sys: \sys reads and writes
application data, and token data plaintext. Table~\ref{tab:accopen} shows the
different data types access policies, and Table~\ref{tab:disopen} describes the effects of a
disguise on each type of data.

\begin{table}[h]
\centering
    \begin{tabular}{ c c c }
        \textbf{Data} & \textbf{\sys?} & \textbf{p?}\\
\hline
        Database Contents & \checkmark & \checkmark \\
        Disguise Tokens & \checkmark & \checkmark \\
        Disguise Token Metadata & & \checkmark & \checkmark \\
\end{tabular}
\caption{Open setting data access policies.}
\label{tab:accopen}
\end{table}

\begin{table}[h]
\centering
    \begin{tabular}{ p{0.18\linewidth} p{.8\linewidth}}
        \textbf{Data} & \textbf{Disguise Effect}\\
\hline
        Database Contents & \sys modifies database contents, converting database rows to disguised
        versions according to the disguise specification.        
        \\
        Disguise Tokens & Each disguise database modification generates a token recording the
        modification, which \sys saves in plaintext.\\
\end{tabular}
\caption{Open setting effects of a disguise.}
\label{tab:disopen}
\end{table}

An open setting can achieve only security goal (1) via application
authentication a principal prior to applying a disguise.
However, clearly an open setting fails to support private disguise tokens or hide token metadata: an
attacker who compromises \sys can read all generated tokens.
To meet all security goals, we move to a private setting, in which \sys uses cryptographic
primitives to render tokens inaccessible to \sys without 
\fi
