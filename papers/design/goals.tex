%-------------------------------------------------------------------------------
\section{Setting and Goals}
%-------------------------------------------------------------------------------
Our design considers typical database-backed applications that link a library provided by our disguising
tool (\sys).
%
This library provides an API that the application uses to specify disguises, apply them, and (in some
cases), reveal disguised data with proper authorization (Table~\ref{tab:client_api}).
%
\sys takes care of executing the database transformations required as part of a disguise, and stores
disguised data in a secure way, subject to the threat model we describe in \S\ref{s:threat}.

\subsection{Use Cases \sys Should Support} We describe the most beneficial and practical use cases
that \sys should support. Importantly, these use cases should be possible with minimal burden on
clients: clients should be able to apply and reveal disguises, or ask for permission to view
disguised data, by clicking one url link.

\head{User-Granularity Disguises for GDPR.}
\sys should support a user-invoked disguise to decorrelate and/or delete a user's data and account to meet the requirements of the
GDPR's right to be forgotten.

\head{User-Granularity Revealing.}
Disguises can optionally be revealable, in which users can \eg permanently restore their accounts and/or data ownership.

\head{Universal Disguises for Anonymization.}
\sys should support a universal, administrator-applied disguise to decorrelate and anonymize all users in the systems.

\head{Universal Revealing.}
Universally-applied disguises can optionally be revealed by an administrator.
\lyt{Note: universal revealing only makes sense for non-private disguises.}

\head{User-Granularity Views and Permissions.}
Users should be able to operate with ownership permissions on data previously correlated with their
account, but decorrelated by a disguise, \emph{without} changing the database views of others using
the application.


\subsection{\sys Abstractions}
To work with \sys, the application uses two abstractions:
\begin{itemize}
    \item Application principals, which correspond to natural persons using the application, and
	which are uniquely identifiable (\eg via a user ID).
        %
	We denote an application principal as $p \in P$, where $P$ is the set of all active and
	inactive application principals.
	%
    \item A disguise specification that consists of one or more disguises, $d \in D$, which each specify
	how \sys transforms the application database when $d$ is invoked by a principal $p$.
	%
	Each invocation of $d$ results in a unique \emph{disguise instance} $\delta_i$, identified by
        disguise sequence number $i$.
	%
\end{itemize}
%
\sys operates on two categories of data:
\begin{enumerate}
    \item \emph{\textbf{Database Contents}}: The application reads and writes database
        contents, which include visible, undisguised and currently-disguised data.
        \sys modifies the data database contents when it applies and reveals disguises.
    \item \emph{\textbf{Disguise Diffs}}: \sys records the set of disguise diffs---which contain the original database contents and the modifications performed to them---and uses
        diffs to compose disguises on top of one another, or to reveal the updates done by a disguise. 
        %\ms{This sounds very operational. What \emph{are} permissions?
\end{enumerate}

\noindent Disguise diff access is controlled via two capabilities: 
\begin{enumerate}
    \item \emph{\textbf{Data Capability \dcapa{p\delta_i}}}: Grants read access to disguise diff data
        associated with principal $p$ produced from applying $\delta_i$.
    \item \emph{\textbf{Locating Capability \lcapa{p\delta_i}}}: Allows locating the disguise
        diffs associated with principal $p$ produced from applying $\delta_i$, but grants no 
        access to the diffs' data.
\end{enumerate}

\noindent Possession of the pair of data and locating capabilities \pcapa{p\delta_i} is required in
practice to grant read access to disguise diff data.
Each capability in the pair alone is insufficient: an attacker who can spoof
\lcapa{p\delta_i} will not be able to access disguise diffs without
\dcapa{p\delta_i}, although they can learn that $\delta_i$ disguised some of $p$'s data;
and an attacker holding \dcapa{p\delta_i} cannot locate the disguise diffs to access without
\lcapa{p\delta_i}.

%-------------------------------------------------------------------------------
\subsection{Threat Model}
\label{s:threat}
%-------------------------------------------------------------------------------

%
Data disguises protect user information in a web application against external observation
and service compromise.
%
An external observer is a user of the web application (authenticated or unauthenticated) who
observes information exposed through using the application.
%
A service compromise occurs when an attacker compromises the web application and potentially
gains full access to the server.
%
The attacker therefore can access any data stored, perform any actions the application can
perform, and access any information available to \sys.
%

%
A data disguise guarantees that the disguised data is hidden from any future attackers unless
explicitly revealed by an authorized principal.
%
In particular, an attacker who compromises \sys at time $t$ learns \emph{nothing but}:
\begin{enumerate}[nosep]
  \item the (plaintext) contents of the application database at or after time $t$;
  \item the disguises invoked, and the identity of the principals invoking them, after time $t$; and
  \item the results of revealing, after time $t$, disguises applied prior to $t$.
\end{enumerate}
%
We make standard assumptions about the security of cryptographic primitives: attackers cannot
break encryption and keys stored with non-colluding clients are safe.
%

%
\sys operates in an honest-but-curious setting: even if compromised, \sys faithfully executes
its protocols, but exposes all data accessed to the attacker.
%

\subsection{Security Goals}
%
With this threat model, \sys seeks to meet four security goals:
%

%
\vspace{6pt}\noindent\textbf{\emph{(1) Authorized Disguises.}}
%
Only a client properly authenticated as a principal $p$ who is authorized to invoke $d$ can apply (and
later reveal) the corresponding disguise instance $\delta_i$.
%

%
\vspace{6pt}\noindent\textbf{\emph{(2) Secure Linking.}}
%
Only a client who has both locating capability \lcapa{p\delta_i} and data capability \dcapa{p\delta_i} can relink $p$
with any data previously owned by $p$ but decorrelated by $\delta_i$.
Establishing links allows the application to grant permissions to $p$ to act upon data otherwise
decorrelated from $p$.
%

\vspace{6pt}\noindent\textbf{\emph{(3) Secure Disguise Diffs.}}
%
Only a client who has both locating capability \lcapa{p\delta_i} and data capability
\dcapa{p\delta_i} can grant \sys access to the disguise diffs from disguise instance $\delta_i$.
%

\vspace{6pt}\noindent\textbf{\emph{(4) Privacy of Disguise History.}}
%
An attacker cannot learn the set of disguises that have disguised a principal's data.
%
An attacker only learns a disguise instance $\delta_i$ exists if an authenticated principal $p$ provides
locating capability \lcapa{p\delta_i} to \sys.

\vspace{6pt}\noindent\textbf{\emph{Non-Goals.}}
%
Under \sys's threat model, the following properties are out of scope:
%
\begin{itemize}
    \item Security of disguised data when an attacker has prior snapshots the system.
    %\item Authorized disguises if an attacker authenticates as a principal $p$. \ms{this seems trivial}
    \item Security of disguise changes and disguise history applied to $p$ if $p$'s private key, \privk{p}, is compromised.
    %\item Security of disguise changes and disguise history of $\delta_i$ applied to $p$ if an
        %authorized client provides \lcapa{p\delta_i} and \dcapa{p\delta_i}. \ms{as above}
    \item Security of metadata such as the number of disguises applied to, or number of
	disguise change records associated with, a particular principal $p$.
    \item Privacy guarantees about undisguised data: if identifying data about $p$ is not covered by
        disguise specification $d$, it remains visible to attackers after corresponding instance $\delta_i$ applies.
        For example, a disguise that leaves contents of posts unmodified does not hide identifying references
	to users in that content.
    \item Hiding whether data is disguised or not. For example, if a post's author is anonymized as ``anonFox'',
        \sys leaks the fact that a post's author has been disguised.
\end{itemize}

