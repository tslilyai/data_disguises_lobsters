%-------------------------------------------------------------------------------
\section{Setting and Goals}
%-------------------------------------------------------------------------------

\begin{table*}[t!]
\centering
\begin{tabular}{ c p{.8\linewidth} }
\textbf{Symbol} & \textbf{Description} \\
\hline
    \vspace{6pt}
$p, q, r$ & application principals, corresponding to a user ID in the application\\
    \vspace{6pt}
$d_1, d_2,\dots,d_i$ & disguise specifications (developer-provided) that can be applied or revealed\\
    \vspace{6pt}
$\delta_i$ & instance of disguise specification $d_i$\\
    \vspace{6pt}
\op{d} & a disguise specification operation (either a removal, modification, or decorrelation)\\
    \vspace{6pt}
\tdata{p\delta} & database change record storing undisguised data associated with $p$ prior to application of $d$.\\
    \vspace{6pt}
\capa{p\delta} & capability that grants access to database change records from $\delta$, as well as
    pre-$\delta$ correlations. Capabilities are associated with a principal $p$.\\
    \vspace{6pt}
\pubk{p} & public key of $p$, used to encrypt $p$'s undisguised data s.t. only one with the
    corresponding private can access it\\
    \vspace{6pt}
\privk{p} & private key of $p$ \\
    \vspace{6pt}
\tpriv{p}{q} & locked private key of $q$, encrypted with \pubk{p} \\
    \vspace{6pt}
\symk{p\delta} & symmetric key used to perform encrypt \tdata{p\delta}\\
    \vspace{6pt}
\addr{p\delta} & address at which $\enc(\symk{p\delta})$ and $\enc(\tdata{p\delta})$ ciphertexts are located\\
    \end{tabular}
\caption{Notation used to describe \sys's design.}
\label{tab:notation}
\end{table*}

Our design considers typical database-backed applications that link a library provided by our disguising
tool (\sys).
%
This library provides an API that the application uses to specify disguises, apply them, and (in some
cases), reveal disguised data with proper authorization (Table~\ref{tab:api}).
%
\sys takes care of executing the database transformations required as part of a disguise, and stores
disguised data in a secure way, subject to the threat model we describe in \S\ref{s:threat}.
%
To work with \sys, the application uses two abstractions:
\begin{itemize}
    \item Application principals, which correspond to natural persons using the application, and
	which are uniquely identifiable (\eg via a user ID).
        %
	We denote an application principal as $p \in P$, where $P$ is the set of all active and
	inactive application principals.
	%
    \item A disguise specification that consists of one or more disguises, $d \in D$, which each specify
	how \sys transforms the application database when $d$ is invoked by a principal $p$.
	%
	Each invocation of $d$ results in a unique \emph{disguise instance} $\delta$
        \lyt{; a principal can only apply a disguise
        once (not sure we need this clause? it may also not be accurate for something like
        universal anonymization, which may be applied as a cron job)}.
	%
\end{itemize}
%
\sys considers operates on two categories of data:
\begin{enumerate}
    \item \emph{\textbf{Database Contents}}: The application reads and writes database
        contents, which include visible, undisguised and currently-disguised data.
        \sys modifies the data database contents when it applies and reveals disguises.
    \item \emph{\textbf{Database Changes}}: \sys records the set of database changes performed when
        applying a disguise. \sys uses change records to compose disguises on top of one another, or to
        reveal the updates done by a disguise. 
        %\ms{This sounds very operational. What \emph{are} permissions?
\end{enumerate}

%-------------------------------------------------------------------------------
\subsection{Threat Model}
\label{s:threat}
%-------------------------------------------------------------------------------

%
Data disguises protect user information in a web application against external observation
and service compromise.
%
An external observer is a user of the web application (authenticated or unauthenticated) who
observes information exposed through using the application.
%
A service compromise occurs when an attacker compromises the web application and potentially
gains full access to the server.
%
The attacker therefore can access any data stored, perform any actions the application can
perform, and access any information available to \sys.
%

%
A data disguise guarantees that the disguised data is hidden from any future attackers unless
explicitly revealed by an authorized principal.
%
In particular, an attacker who compromises \sys at time $t$ learns \emph{nothing but}:
\begin{enumerate}[nosep]
  \item the (plaintext) contents of the application database at or after time $t$;
  \item the disguises invoked, and the identity of the principals invoking them, after time $t$; and
  \item the results of revealing, after time $t$, disguises applied prior to $t$.
\end{enumerate}
%
We make standard assumptions about the security of cryptographic primitives: attackers cannot
break encryption and keys stored with non-colluding clients are safe.
%

%
\sys operates in an honest-but-curious setting: even if compromised, \sys faithfully executes
its protocols, but exposes all data accessed to the attacker.
%

\subsection{Security Goals}
%
With this threat model, \sys seeks to meet four security goals:
%

%
\vspace{6pt}\noindent\textbf{\emph{(1) Authorized Disguises.}}
%
Only a client properly authenticated as a principal $p$ who is allowed to invoke $d$ can apply (and
later reveal) $\delta$.
%

%
\vspace{6pt}\noindent\textbf{\emph{(2) Secure Recorrelation.}}
%
Only a client who has the capability \capa{p\delta} can allow the application to recorrelate with $p$
any data previously owned by $p$ but decorrelated by $\delta$.
\lyt{I don't want to use permission here because the correlation doesn't directly grant the
permission---it's a piece of data that the application uses to then grant permission.}
%

\vspace{6pt}\noindent\textbf{\emph{(3) Secure Disguise Changes.}}
%
Only a client who has the capability \capa{p\delta} can grant \sys access to the disguise changes from 
disguise instance $\delta$.
%

\vspace{6pt}\noindent\textbf{\emph{(4) Privacy of Disguise History.}}
%
An attacker cannot learn the set of disguises that have disguised a principal's data.
%
An attacker only learns a disguise instance $\delta$ exists if an authenticated principal $p$ provides
\capa{p\delta} to \sys.

\vspace{6pt}\noindent\textbf{\emph{Non-Goals.}}
%
Under \sys's threat model, the following properties are out of scope:
%
\begin{itemize}
    \item Security of disguised data when an attacker has prior snapshots the system.
    \item Authorized disguises if an attacker authenticates as a principal $p$. \ms{this seems trivial}
    \item Security of disguise changes and disguise history applied to $p$ if $p$'s private eky, \privk{p}, is compromised.
    \item Security of disguise changes and disguise history of $\delta$ applied to $p$ if an
	authorized client provides \capa{p\delta}. \ms{as above}
    \item Security of metadata such as the number of disguises applied to, or number of
	disguise change records associated with, a particular principal $p$.
    \item Privacy guarantees about undisguised data: if identifying data about $p$ is not covered by
        disguise specification $d$, it remains visible to attackers after $\delta$ applies.
        For example, a disguise that leaves contents of posts unmodified does not hide identifying references
	to users in that content.
    \item Hiding whether data is disguised or not. For example, if a post's author is anonymized as ``anonFox'',
        \sys leaks the fact that a post's author has been disguised.
\end{itemize}

