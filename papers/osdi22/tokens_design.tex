%-------------------------------------------------------------------------------
\section{Design} 
%-------------------------------------------------------------------------------

\subsection{Secure Disguise Tokens Design} 
Applications save disguise metadata in the form of tokens: ownership tokens \town{pd} save information linking
principals to anonymous pseudoprincipals, and diff tokens \tdiff{pd} save information about modifications
performed during the disguise. 
We discuss here how the \sys securely stores tokens produced by applications upon disguising, and
how \sys allows applications to efficiently and securely retrieve these tokens.

%%%%%%%%%%%%%%%%%%%%%%%%%%%%%%%%%%%%%%%%%%%%%%%%%%%%%%%%%%%%%%
\head{Token Wrappers.}
In order to generically handle token data, \sys stores all tokens in token wrappers. 
Wrappers store the bytes of the token \tdiff{pd} or \town{pd} (provided
by the application), a correlated principal ID $p$, and the disguise ID $d$ of the disguise producing the
token. In addition, \sys includes a random nonce in the token wrapper to prevent known plaintext
attacks.

%%%%%%%%%%%%%%%%%%%%%%%%%%%%%%%%%%%%%%%%%%%%%%%%%%%%%%%%%%%%%%
\head{Securing Token Access.} 
When a new principal $p$ is created, \sys produces a private-public keypair, and returns \privk{p} to
the application (which then sends it to the appropriate client). \sys stores \pubk{p} associated
with principal ID $p$.

\sys secures a principal $p$'s tokens by storing them in a token wrapper, and encrypting the wrapper
with \pubk{p}. Only a client holding the private key \privk{p}---the decryption
capability---can access these tokens.

%%%%%%%%%%%%%%%%%%%%%%%%%%%%%%%%%%%%%%%%%%%%%%%%%%%%%%%%%%%%%%
\head{Securing Disguise History.} 
\sys should not leak information that $d$ disguised principal $p$ if $p$ interacts with the system
to reveal or otherwise access disguised data. In particular, \sys should avoid an adversary learning
that encrypted tokens exist for a particular $p$ and $d$, since this allows the adversary to deduce
that $d$ disguised $p$. For example, \sys should not reveal that a principal $p$ has invoked GDPR
deletion.

A strawman solution could store all encrypted tokens in one large bag, so those from
$d$ for $p$ are indistinguishable from those from $d'$ for $p'$. When \sys then gets a request for
tokens from disguise $d$ and for principal $p$, \sys attempts to decrypt (using a provided
decryption capability \privk{p}) all tokens in order to find the relevant ones.

To make this process more efficient, \sys uses \emph{locators \lcapa{pd}}. For each $p$ and $d$,
\sys stores encrypted tokens \tdiff{pd} and \town{pd} in a randomly located bag pointed to by
locator \lcapa{pd}. \sys returns all locators produced by a disguise to the application; to ensure
that locators don't leak information that disguise $d$ applied to $p$, \sys then forgets all locator
information, and the application should ensure that locators are stored external to the application
server (\eg by emailing them to the corresponding users).

Note that with this design, an adversary without access to \lcapa{pd} can learn that $n$ diffs exist
for \emph{some} $p$ and $d$, but cannot identify \emph{which} $p$ or $d$. 
%
Furthermore, an adversary with access to decryption capability \privk{p} would not need locators to
discover $p$'s disguise history: they can decrypt every token in every bag to see which bags they
can successful decrypt. Using locators just makes such an attack less efficient.
%
These scenarios are outside our threat model.

\lyt{
An alternative design might remove encrypted tokens completely (and \eg store them in external per-user
vaults or email them to the client).
This leaves no trace in \sys or the application, and would prevent the adversary from learning
anything. However, this places a large burden on the client.
Maybe put this in future work?
}


\iffalse
%%%%%%%%%%%%%%%%%%%%%%%%%%%%%%%%%%%%%%%%%%%%%%%%%%%%%%%%%%%%%%G
\subsection{Current Design: Discussion}
As the current design stands, \sys supports 1st-person and 3rd-person disguising, as well as
1st-person revealing.
%
Application developers can write GDPR-compliant and/or universal disguises with the primitives
exposed by \sys to write disguise specifications.
%
\sys uses diffs produced from disguising to reveal data when authorized to do so, and when
revealing does not revert updates made to the data since the time of disguise application.

%
\sys also supports ``Temporary Recorrelation without Database Changes'' because \sys can determine
the original owner of data as long as \sys has access to decorrelation diffs (which the client or
the application provides via the appropriate capabilities).
%
\sys can use information from decorrelation diffs to disguise decorrelated data as if it were owned
by the original user; and 
%
\sys can support the API discussed in \S\ref{s:api}, which allows applications to query \sys to
check ownership properties and grant authorized users personalized views and permissions to access
data objects.

Furthermore, \sys does this while meeting all security goals: \sys supports authorized disguises and
ensures the security of ownership claims, disguise diffs, and disguise history.

Our current design falls short, however, by failing to support ``Disguising Anonymized
Users'' which we explain next.
\fi
