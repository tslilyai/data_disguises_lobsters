%%%%%%%%%%%%%%%%%%%%%%%%%%%%%%%%%%%%%%%%%%%%%%%%%%%%%%%%%%%%%%%%%%%%%%%%%%%%%%%%%%%%%%%%%%
\section{Case Studies}
\label{s:case-studies}
%%%%%%%%%%%%%%%%%%%%%%%%%%%%%%%%%%%%%%%%%%%%%%%%%%%%%%%%%%%%%%%%%%%%%%%%%%%%%%%%%%%%%%%%%%

%
We wrote data disguises via \sys for three real applications:
Lobsters~\cite{lobsters},
WebSubmit~\cite{websubmit-rs-anon},
and HotCRP~\cite{hotcrp}.
%
We fully integrated \sys with two applications, Lobsters and WebSubmit.
%
In the following, we discuss our experience writing these disguises and integrating
\sys into the applications.
%

%%%%%%%%%%%%%%%%%%%%%%%%%%%%%%%%%%%%%%%%%%%%%%%%%%%%%%
\subsection{Lobsters}

%
Lobsters is a link-sharing and discussion platform with 14.5k users~\cite{lobsters},
written as a Ruby-on-Rails application backed by a MySQL database.
%
Lobsters supports account deletion, implemented manually by the developers in Ruby.
%
The current Lobsters deletion policy
\one{} hides the user account by flagging it as deleted (but retains it in the \fn{users} table);
\two{} deletes the users' invitations;
\three{} hides comments with negative votes by flagging them as deleted (but leaves
the comments in the database);
\four{} hides private messages by flagging them as deleted (but retains the message
in the database); and
\five{} optionally decorrelates a user's stories and comments by reassigning them to
to a special \fn{inactive-user}.
%
This policy allows users to return by reverting the ``deleted'' flags, but it is
not GDPR-compliant, as personally-identifiable data remains in the database after
account removal.
%

%
We implemented three disguises for Lobsters: the current Lobsters deletion
policy as a disguise, a GDPR-compliant account deletion disguise, and automatic data decay,
which anonymizes user's contributions after some time without activity on the site.
%
All disguises use \sys's high-level API.
%

\paragraph{GDPR-compliant Account Deletion.}
%
We changed the account deletion button in Lobsters's UI to trigger a disguise in \sys.
%
This disguise:
\one{} deletes the user account (and stores the encrypted details in \sys's store);
\two{} deletes information that's only relevant to the individual user, such as their
invitations, hats, and saved stories (and stores it encryptedly in \sys's store);
\three{} modifies story and comment content to ``[deleted content]'', but retains the
database objects in order to preserve comment threads (and nested threads) on stories;
\four{} hides private messages by marking them as deleted (but retains them in the
database) and decorrelates them from the account-deleting user; and
\five{} and decorrelates votes, stories, comments, and moderations on the user's
data by associating each remaining object with a unique pseudoprincipal.
%
This preserves application semantics for other users---\eg vote counts remain consistent
even after users delete their accounts, and other users' comments remain visible and
editable---while better protecting the privacy of deleted users.
%
Furthermore, important information such as moderations on the user's content remain in
the database, and \sys will recorrelate it if the user restores their account.
%
After the disguise is applied, Lobsters emails the user a URL that embeds the locator
produced by \sys.
%
When clicked, this URL allows the user to restore their account.
%

\paragraph{Data Decay.}
%
Our account decay disguise helps Lobsters users automatically gain privacy
after a period of inactivity (\eg a year).
%
We added a cron job that applies the data decay disguise to user accounts that have
been inactive for over a year.
%
Data decay:
\one{} deletes the user's account (and stores its details);
\two{} deletes information that's only relevant to the user, such as their invitations, hats,
and saved stories (again, storing it);
\three{} and decorrelates votes, stories, comments, and moderations on the
user's data by associating them with pseudoprincipals.
%
While similar to account deletion, data decay does not remove story or comment
content.
%
This preserves greater utility for other users, but also potentially retains
identifying information within the content.
%
Lobsters sends the user an email which informs them that their data has decayed,
and which includes URLs to reactivate or completely delete their account.
%
The URLs embed the locator produced by the data decay disguise, and take the user
to a form that requests their private key to complete the operation.
%

\paragraph{Integration.}
%
Our version of Lobsters is end-to-end integrated with \sys.
%
Lobster's Ruby code communicates with \sys through a JSON-HTTP API.
Lobster's disguises are written in 734 LoC of Rust and installed on \sys's server (most of which are
boilerplate schema annotations indicate \eg foreign key relationships).
%
We added 85 LoC of Ruby/templating in the Lobster's code base (originally 160k LoC), and generated an
automated Ruby client library using Swagger~\cite{swagger}, which Lobsters uses to invoke \sys's
API.
%
These modifications took less than one person-day for developers familiar with
\sys and Lobsters, but unfamiliar with Ruby.
%\lyt{Do we need to talk about how Lobsters uses \eg keystores, and the high-level API doesn't deal
%with this?}

%%%%%%%%%%%%%%%%%%%%%%%%%%%%%%%%%%%%%%%%%%%%%%%%%%%%%%
\subsection{WebSubmit}
\label{s:case-websubmit}

%
We integrated \sys with WebSubmit~\cite{websubmit-rs-anon} as a Rust library.
%
Our version of WebSubmit supports the two disguises described in our \S\ref{s:api}
running example: instructor-initiated answer anonymization, and GDPR-compliant
user account deletion.
%
Our modified WebSubmit sends locators and keys to users via email, and we added
new HTTP endpoints for editing disguised data and restoring a deleted account.
%

%
The original WebSubmit has 922 LoC; adding \sys's disguises required an additional 304 LoC
to specify the disguises, 254 LoC for two additional HTTP endpoints; and 809 LoC added to
existing code \lyt{(some of which were to add \eg benchmark timings)} \ms{can we subtract the
timing and logging lines? Adding 1.5x the code that was there originally isn't super compelling.}.
%
Overall, the process took 1 person-day for a developer familiar with \sys, but
unfamiliar with WebSubmit.
%

%%%%%%%%%%%%%%%%%%%%%%%%%%%%%%%%%%%%%%%%%%%%%%%%%%%%%%%%%%%%%%%%%%%%%%%%%%%%%%%%%%%%%%%%%%
\subsection{HotCRP}
\label{s:case-hotcrp}

%
HotCRP is a conference management application with a schema consisting of users
(reviewers/PC members and authors), paper submissions, reviews, comments, tags,
and per-user data such as watched papers and review ratings~\cite{hotcrp}.
%
We wrote two disguises for HotCRP: a universal conference anonymization disguise
(invoked by the PC chairs after the conference), and a GDPR-compliant account
deletion disguise.
%
Both disguises use the high-level \sys API.
%
%\sys-HotCRP also supports review editing after
%anonymization by emailing a link with the appropriate locators to anonymized users. Both disguises
%are implemented using \sys's higher-level API.
%
Universal conference anonymization decorrelates users from their submissions,
reviews, and comments, as well as per-user data such as watched papers.
%
User accounts remain in the database, but have no data.
%
Account deletion \one{} deletes the user's account; \two{} deletes information
that's only relevant to the user, such as their review preferences;
\three{} deletes their author relationships to papers; and
\four{} decorrelates the remainder of their data, such as reviews, to
individual pseudoprincipals.
%
The decorrelation of reviews makes them difficult to reassociate with one
another or with the natural principal.
%
Pseudoprincipals have suitable default properties provided by HotCRP; for example,
they are marked as disabled, so that they have no permissions and cannot log in.
%
This account deletion disguise removes a user's relationship to co-authored papers,
but does not remove the papers themselves; a different policy might go even
further and automatically delete a submission whose last author is removed.
%

An end-to-end integration with the HotCRP client is future work; however, we believe
that the required modifications would be similar to those in Lobsters.
