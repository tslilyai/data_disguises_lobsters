%%%%%%%%%%%%%%%%%%%%%%%%%%%%%%%%%%%%%%%%%%%%%%%%%%%%%%%%%%%%%%%%%%%%%%%%%%%%%%%%%%%%%%%%%%
\section{Case Studies}
\label{s:case-studies}
%%%%%%%%%%%%%%%%%%%%%%%%%%%%%%%%%%%%%%%%%%%%%%%%%%%%%%%%%%%%%%%%%%%%%%%%%%%%%%%%%%%%%%%%%%

%
We wrote data disguises via \sys for three real applications:
WebSubmit~\cite{websubmit-rs-anon}, Lobsters~\cite{lobsters}, and HotCRP~\cite{hotcrp}.
%
We fully integrated \sys with two applications, WebSubmit and Lobsters.
%
In the following, we discuss our experience writing these disguises and integrating
\sys into the applications.
%

%%%%%%%%%%%%%%%%%%%%%%%%%%%%%%%%%%%%%%%%%%%%%%%%%%%%%%
\subsection{Lobsters}

We integrated the Lobsters~\cite{lobsters} link-sharing and discussion application with \sys to
explore data disguises in another class of web applications. \sys-Lobsters supports three disguises:
the current Lobsters deletion policy\lyt{CITE}, a GDPR-compliant account deletion, and automatic
data decay, all implemented \sys's high-level API.

Note that a disguise such as universal anonymization makes less sense for an application such as
Lobsters, in which users may use the application at very different frequencies.  \sys-Lobsters'
account decay allows inactive users to automatically gain privacy after a period of unuse (\eg a
year); \sys-Lobsters emails decayed users a link with the appropriate locators that allows the user
to return to delete their account or reactive it.

The current Lobsters deletion policy
%
(1)~hides the user account by flagging it as deleted (but retains the user in the users table);
%
(2)~deletes the users' invitations;
%
(3)~hides comments with negative votes by flagging them as deleted (but retains the comment in the database)
%
(4)~hides private messages from the other party by flagging them as deleted (but retains the message
in the database);
%
(4)~and (optionally) decorrelates a user's stories and comments by referring them
to a \emph{single} pseudoprincipal with username \texttt{inactive-user}.
%
This policy allows users are able to return by simply marking user accounts as not-deleted.

GDPR-compliant account deletion
%
(1)~deletes the user account;
%
(2)~deletes information that's only relevant to the user, such as their invitations, hats, and saved
stories;
%
(3) modifies story and comment content to ``[deleted content]'' in order to preserve comment threads
(and nested threads) on stories;
%
(4)~hides private messages from the other party by marking them as deleted (but retains them in the
database);
%
(5)~and decorrelates private messages, votes, stories, comments, and moderations on the user's data
by referring each object to a \emph{unique} pseudoprincipal.
%

This preserves application semantics for other users (\eg vote counts remain consistent even if
users delete their accounts, and other users' comments remain visible and editable), while better
protecting the identity of deleted users than the current policy, which assigns all decorrelated
data to a single pseudoprincipal---yielding more information about which data may belong to a single
deleted user---and fails to be GDPR-compliant by retaining plaintext usernames. Furthermore,
important information such as moderations on the user's content remain in the database, and will be
recorrelated if the user decides to restore their account.  After \sys-Lobsters applies account
deletion, \sys-Lobsters emails the user a URL with the locator produced by the disguise, which when
clicked, allows the user to restore their account (barring application updates that occurred in the
interim).

\sys-Lobsters automatically applies data decay to user accounts that have been inactive for over a year.
Data decay
%
(1)~deletes the user account;
%
(2)~deletes information that's only relevant to the user, such as their invitations, hats, and saved
stories;
%
(3)~and decorrelates votes, stories, comments, and moderations on the user's data by referring them
to newly generated pseudoprincipals.
%

While similar to account deletion, data decay importantly does \emph{not} remove story or comment
content, resulting in both greater utility for other users but also potentially more retained
identifying information.  As with account deletion, \sys-Lobsters emails the user a URL with the locator
produced by applying data decay; when users want to reactivate to return to \sys-Lobsters or compose
account deletion on top of data decay, they can simply click the URL to regain access to their
restored data.

\sys-Lobsters is a fully E2E application that uses \sys by running \sys as a local
server that exposes \sys's API, and serializes arguments and return values as JSON to the Lobsters
application.
The original Lobsters is \lyt{TODO}LoC.
Modifications to the Lobsters code required \lyt{TODO}LoC and took \lyt{TODO} person-days.

%%%%%%%%%%%%%%%%%%%%%%%%%%%%%%%%%%%%%%%%%%%%%%%%%%%%%%
\subsection{WebSubmit}
\label{s:case-websubmit}
%
We integrate \sys into WebSubmit~\cite{websubmit-rs-anon} as a Rust library to support the two
disguises described in our running example from \S\ref{s:api}, namely intructor-initiated answer anonymization, and GDPR-compliant user account deletion.

Our version of WebSubmit uses \sys's lower-level API, and sends locators and keys to users via
email.
%
We added new HTTP endpoints for editing disguised data and restoring a deleted account.
%

The original WebSubmit has 922 LoC; adding \sys's disguises required an additional 304 LoC
to specify the disguises, 254 LoC for two additional HTTP endpoints; and 809 LoC added to
existing code \lyt{(some of which were to add \eg benchmark timings)} \ms{can we subtract the
timing and logging lines? Adding 1.5x the code that was there originally isn't super compelling.}.
%
Overall, the process took 1 person-day for a developer familiar with \sys, but
unfamiliar with WebSubmit.
%

%%%%%%%%%%%%%%%%%%%%%%%%%%%%%%%%%%%%%%%%%%%%%%%%%%%%%%%%%%%%%%%%%%%%%%%%%%%%%%%%%%%%%%%%%%
\subsection{HotCRP}
HotCRP is a conference management application with a schema containing users (reviewers/PC members and
authors), paper submissions, reviews, comments, tags, and per-user data such as paper watches and review ratings.
%
\sys-HotCRP supports similar disguises to WebSubmit: a universal conference anonymization disguise,
and a GDPR-compliant account deletion disguise.  \sys-HotCRP uses the high-level \sys API to
specify, apply, and reveal both disguises.  \sys-HotCRP also supports review editing after
anonymization by emailing a link with the appropriate locators to anonymized users. Both disguises
are implemented using \sys's higher-level API.

Universal conference anonymization decorrelates users from their submissions, reviews, and comments,
as well as per-user data such as paper watches. User accounts remain in the database but have no
correlated data.

\lyt{This is taken from hotos} Account deletion of a user
%
(1)~deletes the user account;
%
(2)~deletes information that's only relevant to the user, such as their review preferences;
%
(3)~deletes their contact author relationships to any submissions;
%
(4)~decorrelates their other data, such as reviews, by referring them to newly generated
pseudoprincipals.

After account deletion, a user's reviews remain in the system, but linked to different
pseudoprincipals, making them difficult to reassociate with one another or with the user.
%
Pseudoprincipals have suitable default values; for example, they should be disabled, ensuring they
have no permissions and cannot log in.
%
This form of account deletion removes a user's relationship to submissions, but does not remove the
submissions themselves; a different policy might go even further and automatically delete a
submission whose last author is scrubbed.

\lyt{An end-to-end integration with the HotCRP client is future work; however, we imagine that modifications similar to those in WebSubmit
would be necessary.}
