%%%%%%%%%%%%%%%%%%%%%%%%%%%%%%%%%%%%%%%%%%%%%%%%%%%%%%%%%%%%%%%%%%%%%%%%%%%%%%%%%%%%%%%%%%
\section{Case Studies}
%%%%%%%%%%%%%%%%%%%%%%%%%%%%%%%%%%%%%%%%%%%%%%%%%%%%%%%%%%%%%%%%%%%%%%%%%%%%%%%%%%%%%%%%%%

\subsection{WebSubmit}
WebSubmit is an answers-submission application written in Rust. 
Its schema consists of lectures, questions, answers, and user accounts; 
clients can create accounts and submit answers. 
\sys-WebSubmit supports two disguises: admin-anonymization of all answers, and
per-user GDPR-compliant account deletion. \sys-WebSubmit also supports account restoration after
deletion, and answer editing after anonymization, by emailing users a link with the appropriate
locators when applying these disguises.
%

%
Admin-anonymization performs per-user disguises for all users.
Each disguise decorrelates lecture answers from the user, grouping them by lecture (so all answer
for a particular lecture refer to the same newly generated pseudoprincipal).
\sys-WebSubmit implements anonymization using \sys's lower-level API.
%

%
GDPR-compliant account deletion 
%
(1)~deletes the user account, and 
%
(2)~deletes the user's answers. 
\sys-Websubmit implements account deletion (and restoration) using \sys's higher-level API.
%

%
\sys-WebSubmit is a fully-E2E application that supports additional endpoints to perform disguise
actions. We integrate \sys into WebSubmit as a Rust library because
WebSubmit is a Rust application.
The original WebSubmit is 922LoC; writing \sys-WebSubmit required an additional 304LoC
to specify disguises; 254LoC to add additional endpoints; and 809LoC added to existing code
\lyt{(some of which were to add e.g., benchmark timings)}. Overall, the process took 1-2 person-days
for a developer familiar with \sys, but unfamiliar with web application development and
WebSubmit.\lyt{Not sure how to put this into context...}

%%%%%%%%%%%%%%%%%%%%%%%%%%%%%%%%%%%%%%%%%%%%%%%%%%%%%%%%%%%%%%%%%%%%%%%%%%%%%%%%%%%%%%%%%%
\subsection{HotCRP}
HotCRP is a conference review application with a schema containing users (reviewers/PC members and
authors), paper submissions, reviews, comments, tags, and per-user data such as paper watches and review ratings.
%
\sys-HotCRP supports similar disguises to WebSubmit: a universal conference anonymization disguise,
and a GDPR-compliant account deletion disguise.  \sys-HotCRP uses the high-level \sys API to
specify, apply, and reveal both disguises.  \sys-HotCRP also supports review editing after
anonymization by emailing a link with the appropriate locators to anonymized users. Both disguises
are implemented using \sys's higher-level API. 

Universal conference anonymization decorrelates users from their submissions, reviews, and comments,
as well as per-user data such as paper watches. User accounts remain in the database but have no
correlated data.

\lyt{This is taken from hotos} Account deletion of a user 
%
(1)~deletes the user account;
%
(2)~deletes information that's only relevant to the user, such as their review preferences;
%
(3)~deletes their contact author relationships to any submissions;
%
(4)~decorrelates their other data, such as reviews, by referring them to newly generated
pseudoprincipals.

After account deletion, a user's reviews remain in the system, but linked to different
pseudoprincipals, making them difficult to reassociate with one another or with the user.
%
Pseudoprincipals have suitable default values; for example, they should be disabled, ensuring they
have no permissions and cannot log in.
%
This form of account deletion removes a user's relationship to submissions, but does not remove the
submissions themselves; a different policy might go even further and automatically delete a
submission whose last author is scrubbed.

\lyt{An end-to-end integration with the HotCRP client is future work; however, we imagine that modifications similar to those in WebSubmit
would be necessary.}

\subsection{Lobsters} 
We integrated the Lobsters~\cite{lobsters} link-sharing and discussion application with \sys to
explore data disguises in another class of web applications. \sys-Lobsters supports two disguises:
GDPR-compliant account deletion, and automatic data decay, both implemented \sys's high-level API.
\sys-Lobsters' account decay allows inactive users to automatically gain privacy after a period of
unuse (\eg a year); \sys-Lobsters emails decayed users a link with the appropriate locators that
allows the user to return to delete their account or reactive it. A disguise such as universal
anonymization makes less sense for an application such as Lobsters, in which users may use the
application at very different frequencies.

GDPR-compliant account deletion 
%
(1)~deletes the user account;
%
(2)~deletes information that's only relevant to the user, such as their invitations, hats, and saved
stories;
%
(3) modifies story and comment content to ``[deleted content]'' in order to preserve comment threads
(and nested threads) on stories;
%
(4)~and decorrelates votes, stories, comments, and moderations on the user's data by referring them
to newly generated pseudoprincipals.
%

This preserves application semantics for other users (\eg vote counts remain consistent even if
users delete their accounts, and other users' comments remain visible and editable), while better
protecting the identity of deleted users.  Furthermore, important information such as moderations on
the user's content remain in the database, and will be recorrelated if the user decides to restore
their account.  After \sys-Lobsters applies account deletion, \sys-Lobsters emails the user a URL with the
locator produced by the disguise, which when clicked, allows the user to restore their account
(barring application updates that occurred in the interim).

\sys-Lobsters automatically applies data decay to user accounts that have been inactive for over a year.
Data decay 
%
(1)~deletes the user account;
%
(2)~deletes information that's only relevant to the user, such as their invitations, hats, and saved
stories;
%
(3)~and decorrelates votes, stories, comments, and moderations on the user's data by referring them
to newly generated pseudoprincipals.
%

While similar to account deletion, data decay importantly does \emph{not} remove story or comment
content, resulting in both greater utility for other users but also potentially more retained
identifying information.  As with account deletion, \sys-Lobsters emails the user a URL with the locator
produced by applying data decay; when users want to reactivate to return to \sys-Lobsters or compose
account deletion on top of data decay, they can simply click the URL to regain access to their
restored data.

\sys-Lobsters is a fully E2E application that uses \sys by running \sys as a local
server that exposes \sys's API, and serializes arguments and return values as JSON to the Lobsters
application. 
The original Lobsters is \lyt{TODO}LoC. 
Modifications to the Lobsters code required \lyt{TODO}LoC and took \lyt{TODO} person-days.

