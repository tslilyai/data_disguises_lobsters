%%%%%%%%%%%%%%%%%%%%%%%%%%%%%%%%%%%%%%%%%%%%%%%%%%%%%%%%%%%%%%%%%%%%%%%%%%%%%%%%%%%%%%%%%%
\section{Case Studies}
%%%%%%%%%%%%%%%%%%%%%%%%%%%%%%%%%%%%%%%%%%%%%%%%%%%%%%%%%%%%%%%%%%%%%%%%%%%%%%%%%%%%%%%%%%

\subsection{WebSubmit.rs}
\lyt{I imagine we'll have this sprinkled throughout the design as the running example?}
WebSubmit is an answers-submission application. Its simple schema consists of lectures, questions,
answers, and user accounts; each user owns their account and a set of answers. We integrated \sys
into WebSubmit to support a disguise for admin-anonymization of all answers; a disguise for per-user
GDPR-compliant account deletion; account restoration after deletion; and answer editing after
anonymization.  \sys-WebSubmit is a fully-E2E application that supports additional endpoints to
perform these disguise actions, and emails users their capabilities and locators (as these
endpoints) when disguise actions are performed.

\subsection{Lobsters} 
We integrated the Lobsters~\cite{lobsters} link-sharing and discussion
application with \sys to explore data disguises in another class of web applications.

\sys-Lobsters supports two disguises: GDPR-compliant account deletion, and automatic data decay, and uses the high-level \sys API to specify, apply, and reveal both disguises.

GDPR-compliant account deletion 
%
(1)~deletes the user account;
%
(2)~deletes information that's only relevant to the user, such as their invitations, hats, and saved
stories;
%
(3) modifies story and comment content to ``[deleted content]'' in order to preserve comment threads
(and nested threads) on stories;
%
(4)~and decorrelates votes, stories, comments, and moderations on the user's data by referring them
to newly generated pseudoprincipals.
%

This preserves application semantics for other users (\eg vote counts remain consistent even if
users delete their accounts, and other users' comments remain visible and editable), while better
protecting the identity of deleted users.  Furthermore, important information such as moderations on
the user's content remain in the database, and will be recorrelated if the user decides to restore
their account.  After Lobsters applies account deletion, Lobsters emails the user a URL with the
locator produced by the disguise, which when clicked, allows the user to restore their account
(barring application updates that occurred in the interim).

Lobsters automatically applies data decay to user accounts that have been inactive for over a year.
Data decay 
%
(1)~deletes the user account;
%
(2)~deletes information that's only relevant to the user, such as their invitations, hats, and saved
stories;
%
(3)~and decorrelates votes, stories, comments, and moderations on the user's data by referring them
to newly generated pseudoprincipals.
%

While similar to account deletion, data decay importantly does \emph{not} remove story or comment
content, resulting in both greater utility for other users but also potentially more retained
identifying information.  As with account deletion, Lobsters emails the user a URL with the locator
produced by applying data decay; when users want to reactivate to return to Lobsters or compose
account deletion on top of data decay, they can simply click the URL to regain access to their
restored data.

\subsection{HotCRP}
We implement two disguises on top of the HotCRP schema: a universal conference anonymization disguise, and a
GDPR-compliant account deletion disguise.
\sys-HotCRP uses the high-level \sys API to specify, apply, and reveal both disguises.

Universal conference anonymization decorrelates users from their submissions, reviews, and comments,
as well as data such as paper watches and review ratings. User accounts remain in the database, but
are decorrelated from any data.

\lyt{This is taken from hotos} Account deletion of a user 
%
(1)~deletes the user account;
%
(2)~deletes information that's only relevant to the user, such as their review preferences;
%
(3)~deletes their contact author relationships to any submissions;
%
(4)~decorrelates their other data, such as reviews, by referring them to newly generated pseudoprincipals.

After account deletion, a user's review texts are still in the system, but they are linked to
different pseudoprincipals, making them difficult to reassociate with one another or with the user.
%
Pseudoprincipals have suitable default values; for example, they should be disabled, ensuring they
have no permissions and cannot log in.
%
This form of account deletion removes a user's relationship to submissions, but does not remove the
submissions themselves; a different policy might go even further and automatically delete a
submission whose last author is scrubbed.

We ran these disguises on a HotCRP database prepopulated with 450 total users (50 reviewers), 450 papers
(50 accepted), and 4 reviews and 4 comments per paper. 
An end-to-end integration with the HotCRP client is future work; however, we imagine that
modifications similar to those in WebSubmit would be necessary.

\lyt{TODO: editing after decorrelation}
