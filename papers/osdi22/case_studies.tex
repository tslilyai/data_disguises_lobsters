%%%%%%%%%%%%%%%%%%%%%%%%%%%%%%%%%%%%%%%%%%%%%%%%%%%%%%%%%%%%%%%%%%%%%%%%%%%%%%%%%%%%%%%%%%
\section{Case Studies}
%%%%%%%%%%%%%%%%%%%%%%%%%%%%%%%%%%%%%%%%%%%%%%%%%%%%%%%%%%%%%%%%%%%%%%%%%%%%%%%%%%%%%%%%%%

\subsection{WebSubmit.rs}

\subsection{Lobsters}
We implement a second in the Lobsters~\cite{lobsters} web application.

\subsection{HotCRP}
We implement two disguises in HotCRP: a universal conference anonymization disguise, and a
GDPR-compliant account deletion disguise.

Universal conference anonymization decorrelates users from their submissions, reviews, and comments,
as well as data such as paper watches and review ratings. User accounts remain in the database, but
are decorrelated from any data.

\lyt{This is taken from hotos} Account deletion of a user 
%
(1)~deletes the user account;
%
(2)~deletes information that's only relevant to the user, such as their review preferences;
%
(3)~deletes their contact author relationships to any submissions;
%
(4)~modifies their other data, such as reviews, to refer to newly generated pseudoprincipals
instead, thus \emph{decorrelating} the reviews from their identity.

After account deletion, a user's review texts are still in the system, but they are linked to
different pseudoprincipals, making them difficult to reassociate with one another or with the user.
%
Pseudoprincipals have suitable default values; for example, they should be disabled, ensuring they
have no permissions and cannot log in.
%
This form of account deletion removes a user's relationship to submissions, but does not remove the
submissions themselves; a different policy might go even further and automatically delete a
submission whose last author is scrubbed.
