%%%%%%%%%%%%%%%%%%%%%%%%%%%%%%%%%%%%%%%%%%%%%%%%%%%%%%%%%%%%%%
\subsection{High-Level Disguise Specifications.} 
\sys's high-level disguising API can be used for a restricted set of disguises in applications that
use SQL databases. \sys provides a disguise specification language that supports disguising and
revealing database objects corresponding to individual database rows, and can only decorrelate data
that is linked to principals directly via foreign key relationships.

Disguises specifications consist of a set of predicated disguise operation specifications that \sys knows
how to perform.  Operations take data objects as input and execute updates to
application data. \sys automatically generates records when applying each operations.

In addition to the application-aware pseudoprincipal generation policy (which is required for the
low-level API as well), developers specify, using foreign keys, how \sys can determine which
principal(s) to associate with an operation's generated database. For example, a record from a
comment removal operation should be associated with the principal whose ID is referenced by the
author column, a foreign key to the principals table.

\vspace{6pt}\noindent
\sys supports three high-level disguise specification operations:
\begin{enumerate}
    \item Modify: change an attribute of the data object.
    \item Remove: delete the data object.
    \item Decorrelate: generate a pseudoprincipal $q$, and rewrite the foreign key to original
        principal $p$ from the data object to instead point to $q$.  
\end{enumerate}

\noindent For each operation, the application developer specifies:
\begin{itemize}
    \item An associated predicate to select input objects in a SQL-like fashion.
    \item The type of operation and its arguments (e.g., which attributes of the data object to
        modify).
    \item The foreign keys to the principals table that determine which principal(s) to associate
        with the operation's records.
\end{itemize}
%
\sys then automatically generates records for each operation.
Decorrelation creates a speaks-for record \town{pd} containing the created pseudoprincipal's
ID; removal produces a diff record \tdiff{pd} with the removed object's value; and modification
produces a diff record \tdiff{pd} containing the old and new values of the modified object.

Records are stored by \sys using its lower-level secure record API.

\head{Disguise Application.}
To apply a disguise specification as disguise $d$, \sys applies each operation in the spec to
all data objects satisfying the operation's predicate. 

The \fn{ApplyDisguise} API call also allows for \sys to \emph{compose} disguise operations on top of
prior ownership changes.
If provided a decryption capability and locator \lcapa{pd'} from disguise $d'$, \sys can access all
speaks-for records
\town{pd'}. \sys thus knows if some link exists between $p$ and a pseudoprincipal $q$ generated by
disguise $d'$.

If one of $d$'s operations has a predicate to apply the operation to $p$'s data, \sys
composes the operation on top of $d'$ by also applying the operation to data of $q$.

\head{Disguise Revealing.} 
To reveal disguise $d$, \sys retrieves all records accessible with the provided decryption capability
and locator. Because \sys can understand the record bytes, \sys can restore undisguised application
data by updating the relevant object rows in the database.  However, \sys must be
careful to not accidentally reveal data that has been updated since $d$ was applied.  

All records produced by applying disguise specification operations allow \sys to check that the data
to restore is still in its disguised state. Prior to revealing each record, if the state of data does
not match the disguised state recorded by the record, then \sys knows the record records a stale and
overwritten update, and will refuse to restore the data to its undisguised state.
