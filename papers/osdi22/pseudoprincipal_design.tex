%%%%%%%%%%%%%%%%%%%%%%%%%%%%%%%%%%%%%%%%%%%%%%%%%%%%%%%%%%%%%%
\subsection{Disguising Anonymized Users}

%\begin{table*}[t!]
%\centering
    %\begin{tabular}{ p{.5\linewidth} p{.5\linewidth} }
%\textbf{API Call} & \textbf{Description} \\
%\hline
    %\fn{GetEncryptedPseudoPrincipalPrivateKeys($p$, \lcapa{pd}) $\rightarrow$
        %\fn{Vec<Enc(\pubk{p},\privk{q})>}} & Returns the set of encrypted pseudoprincipals and their
        %private keys created when decorrelating $p$'s data for disguise $d$.
    %\vspace{6pt}\\
        %\fn{GetEncryptedLocatingCaps($q$) $\rightarrow$ \fn{Vec<Enc(\pubk{q}, \lcapa{qd})>}} & 
        %Returns the set of \lcapa{qd} encrypted with
        %\pubk{q} for all $d$ that applied to $q$. Nonempty only if $q$ is a pseudoprincipal.\\
    %\end{tabular}
%\caption{Additional Client-\sys API Calls for Asymmetric Encryption}
%\label{tab:client_api_ext}
%\end{table*}

To apply one disguise on top of another in a way that supports, \eg comment removal after universal
decorrelation, \sys must support \emph{pseudoprincipal diffs}, namely diffs associated with
pseudoprincipals. 

These diffs can occur after at least one disguise has been applied: subsequent disguises may
associate diffs with pseudoprincipals. For example, after universal decorrelation, every paper or
review correlates with a pseudoprincipal user. When \sys applies universal decorrelation,
decorrelation of all papers and reviews generates correlation diffs for these pseudoprincipals.

However, we now have a problem. Diffs are encrypted with \privk{p} and stored at
\lcapa{pd}, and \sys emails \lcapa{pd} to (potentially offline) clients and does not retain
\lcapa{pd}.  However, pseudoprincipals have no corresponding real user (and, for unlinkability, \sys
cannot store which pseudoprincipal-user correspondences). Thus, \sys has no way to communicate
\lcapa{pd} to a user who speaks for pseudoprincipal $p$!
%, so pseudoprincipals' capabilities can either be insecurely stored by \sys, or lost forever!

We have three potential approaches to solve this problem. Two we see as strawmen because they fail
to meet our secury and use case goals respectively:
\begin{enumerate}
    \item \emph{Weak Security:} \sys can store \lcapa{pd} for pseudoprincipal $p$. This means
        an adversary will be able to learn $p$'s undisguised data, even if it had been disguised by
        $d$.

    \item \emph{Permanent Disguises:} \sys throws away \lcapa{pd} for pseudoprincipal $p$. This means that the
        disguise modifications are permanent, and no links (to determine ownership) between
        $p$ and other pseudoprincipals can be made. This affects temporary recorrelation, because
        decorrelation links between principals cannot be derived from discarded diffs.
\end{enumerate}

We next describe a third option, which meets our security and use case goals while trading off
efficiency and usability.

\head{Pseudoprincipal Private Keys.}
When a pseudoprincipal $q$ is generated to decorrelate data from $p$: 
\begin{enumerate}
    \item \sys generates keypair (\pubk{q}, \privk{q}) for $q$
    \item encrypts the private key of $q$ with \pubk{p} to produce $\enc(\pubk{p}, \privk{q})$
    \item stores $\enc(\pubk{p}, \privk{q})$ at location \lcapa{pd} along with $p$'s encrypted diffs from disguise $d$
\end{enumerate}
This ensures that a client with \lcapa{pd} and \privk{p} (\privk{p}) can access \privk{q}, the data
capability of pseudoprincipal $q$.  Furthermore, only a client who holds \lcapa{pd} can learn
that $p$ has decorrelated data because associated pseudoprincipal private keys are not stored
publically with $p$'s ID.

When \sys applies disguise $d'$ to further disguise pseudoprincipal $q$'s data, \sys:
\begin{enumerate}
    \item stores disguise diffs \tdiff{qd'} at location \lcapa{qd}
    \item encrypts \lcapa{qd} with \pubk{p} to produce $\enc(\pubk{p}, \lcapa{qd})$
    \item stores $\enc(\pubk{p}, \lcapa{qd})$ mapped to by principal $q$'s ID
\end{enumerate}
Note that \sys stores \lcapa{qd} ciphertexts associated with $q$'s ID, whereas for a real principal
$p$, \sys does not store any similar metadata indicating how many disguises have applied to $p$. 
This means that an adversary can learn that \emph{some} disguise has applied to and
decorrelated data from $q$. However, this falls out of scope in our threat model: $q$ is a
pseudoprincipal created from a decorrelation operation, and therefore any disguise metadata regarding $q$ has already been dissociated from
the original principal $p$'s identity.  

\head{Extending capabilities to authorize pseudoprincipal diffs access.}
With this extended design, a client speaking for $p$ that provides data and locating capabilities%\pcapa{p}{d}
authorizes access to \emph{both} $p$'s diffs for disguise $d$, \emph{and}
diffs of a pseudoprincipal $q$ created by
decorrelating $p$'s data. \lyt{I don't think we're necessarily allowing clients to speak for $q$,
but rather to perform actions using $q$'s diffs.}

Given a data capability \privk{p}, \sys can can access \privk{q} by decrypting \enc(\pubk{p},
\privk{q}).
%
With \privk{q}, \sys can decrypt all \enc(\pubk{q}, \lcapa{qd}) so that \sys has both the data and
location capabilities for all of $q$'s disguise diffs. This allows the application to reveal,
disguise, or perform application actions with access to diffs from all disguises applying to $q$.

While clearly more expensive for \sys to execute, this protocol allows \sys to recursively disguise
pseudoprincipals, yet still require only the top-level original principal capabilities to support the
desired use cases. Furthermore, this design still meets our security goals: data capabilities secure
all disguise diffs, and an adversary learns no disguise metadata for real users.

%This introduces another round-trip for any client using the API: a client speaking for $p$ that
%wants to reveal or compose on top of pseudoprincipal diffs must (1) query for all pseudoprincipal
%\privk{q} ciphertexts associated with $p$ for all \lcapa{qd} (for all $d$); (2) recursively
%query for all nested pseudoprincipal \privk{q} ciphertexts associated with $q$ for all \lcapa{qd};
%and (3) actually invoke the action with the decrypted capability pairs.
%
%This makes it difficult to implement a userflow where the user can click on a single URL to grant
%\sys diff access: in this design, the user has to first retrieve all possible URLs, and make several
%more round-trips to the application/\sys.
