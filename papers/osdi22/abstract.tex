%\begin{abstract}
\paragraph{Abstract.}
% What is the problem?
Modern web services are legally and morally called upon to provide better
data privacy for their users.
%
But developers must manually realize the necessary privacy
features today. %---a serious burden.
% Why does it matter?
As a result, privacy features are sparse or non-existent, sloppily
implemented, or---even worse---buggy.
%

% What is our solution?
\emph{Data disguises} are a new structured framework that helps
developers specify, implement, and execute privacy transformations over
user data in database-backed web applications.
%
When applied to user data, a data disguise anonymizes or removes user data,
protecting the user's privacy against attackers or accidental leaks.
%
However, the disguise also creates cryptographically-protected server-side
state, which preserves the user's ability operate on the data after its
disguising.
%

% How well does it work?
We implemented \sys, a prototype data disguising library, and used \sys to
add data disguises to two real-world web applications.
%
Our evaluation shows that adding data disguises to existing applications
is feasible with incremental changes, and that \sys has low overheads,
provides users with privacy assurances, and realizes user control over
disguised data.
%
%\end{abstract}
