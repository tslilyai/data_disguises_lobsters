\begin{abstract}
% What is the problem?
Modern web services are legally and morally called upon to provide better
data privacy for their users, but developers face high burdens in manually
realizing the necessary privacy features.
% Why does it matter?
As a result, privacy features are sparse, sloppily implemented,
or---worse---buggy or non-existent.

% What is our solution?
\emph{Data disguises} are a new structured framework that helps
developers specify, implement, and execute privacy transformations over
user data in a databased-backed web application.
%
When applied to user data, disguises anonymize or remove data, but retain
key application functionality with the aid of cryptographically-protected
server-side state.
%
Using this state, the user(s) originally associated with the data can still
operate on it via the capability-based authorization our framework provides,
even when disguises have removed identifying information from the database.
%

% How well does it work?
We implemented \sys, a prototype data disguising library, and used \sys to
add data disguises to \note{three} real-world web applications.
%
Our evaluation shows that adding data disguises to existing applications
is feasible with incremental changes, and that \sys has low overheads
while providing users with privacy assurances and control over their stored
data.
%
\end{abstract}
