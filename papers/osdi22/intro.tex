\section{Introduction}
\label{s:intro}

%% Problem statement
%
Web applications today handle copious personal information, much of it
sensitive, and store it for a long time.
%
For example, a user who booked a room at a hotel once can have sensitive
information, such as their passport and phone numbers, compromised by
an attacker years later~\cite{starwood-hack}.
%
%Users' privacy and their rights to exercise control over their data in these
%applications thus are an increasingly important concern for application
%developers.
%
Many developers want to do a good job providing privacy choices to their users,
and legal requirements (\eg the GDPR, CCPA, and others) increasingly mandate
doing so.
%
%Even beyond meeting the legal requirements, many developers are at least
%worried about the reputational damage of a large-scale data leak.
%
But doing a good job at data privacy is hard: it requires extensive manual
implementation work today, and it is easy to get wrong.
%
As a result, applications lack desirable privacy features, like \eg ways to
remove or anonymize a user's account, or for old data to ``decay'' into
lower-fidelity, less identifying representations over time.
%

%
%Furthermore, even these efforts are nontrivial, and require careful understanding and manipulation
%of the web application's database. Companies with resources such as FB have implemented internal
%systems that handle e.g., user account deletion (cite DELF); however, solutions like these fail to
%generalize and require time and money that smaller-scale companies may lack.
%

%% Our key idea: data disguising
%
This paper describes \emph{data disguising}, a new systematic framework that
provides abstractions that help developers implement correct,
expressive, and end-user-friendly privacy functionality.
%
Data disguises transform a user's data in a web application in
privacy-preserving ways, and thus reduce the damage done when database contents
are exposed to an attacker.
%
Crucially, our framework allows the application to maintain key functionality even
when a user has disguised their data, and supports reversible disguises for users
who want to return to an identity-revealing state.
%

%
% Motivation
%
To understand the need for data disguising, consider Lobsters, a Reddit-like
discussion forum~\cite{lobsters}.
%
The Lobsters developers already provide an option for a user to delete their own
account, but also permit users to change their mind and reclaim their
account later.
%
As a result, Lobster's account deletion actually merely hides data (\eg user
details, comments, private messages) by marking it as deleted, and optionally
reassigns contributions to a ``inactive user'' placeholder.
%
This allows the user to return, but provides them with no privacy against an
attacker who compromises the server while the user is in deleted state.
%
Providing this privacy would impose a high burden on the
Lobsters developers: they would need to devise their own cryptographic
server-side storage for deleted users' data,
%or have the user
%export their account data, delete it in the database, and then
%cryptographically verify on re-import that the data has not been tampered
%with.
%
%In addition, the Lobsters developers would also need to
ensure that the application continues to work for the remaining users, and
protect against attacks that leverage leftover metadata (such as the tags on
deleted users' posts) to infer a deleted user's identity.
%
Worse, developers must solve these challenges anew for each application
today.
%

%
% Overview and key ideas
%
Data disguising provides a clean set of reusable abstractions that web
applications use to implement secure and expressive privacy transformations,
based on two key idas.
%
% Two key ideas: encrypted data storage and pseudoprincipals
%
First, an application can \emph{disguise} data at any point, replacing it with
less identity-revealing information in the application database.
%
The original data under disguise remains on the server, but in encrypted form,
managed by the data disguising library.
%
After disguising, only the original application principal associated with the
encrypted data can ever \emph{reveal} it again through explicit action.
%
Second, the application uses \emph{pseudoprincipals}, which are first-class
anonymous users that serve as stand-ins for disguised application users, to
preserve functionality even after it has disguised user data.
%
Pseudoprincipals cannot be de-anonymized even by an attacker who has full
access to the server, but the original user can prove their ownership of
pseudoprincipals to access data under disguise or lift the disguse.
%
A disguised user typically---depending on the level of security and anonymity
desired---has many pseudoprincipals, up to one for each of their application
data objects under disguise.
%

%
We implemented data disguising in \sys, a reusable library that
web applications use to express, apply, and reveal data disguises.
%
Applications may have a wide range of privacy goals and policies, and they
use \sys's APIs to realize these.
%
\sys itself is agnostic to the specific disguise implemented, and the
application interacts with it through the abstractions of application-managed
principals, database diffs, and speaks-for relationships between principals.
%
Importantly, the application can rest assured that any database diff
and speaks-for relationship provided to \sys for secure storage is protected
until the original owner actively chooses to reveal it.
%
\sys uses only standard public-private key cryptography to achieve this, and
hides the cryptographic details from application developers.
%

%
% Threat model
%
\sys protects users' data under disguise against an attacker who completely
compromises the application server, and thus gains access to the full database
contents, the server's memory, and can act with full administrator privileges
in the application.
%
Data disguised prior to compromise remains secure as long as the principal
originally associated with the data does not interact with the data under
disguise stored in \sys or explicitly reveals it.
%
%Further, \sys's metadata---\eg whose data has been disguised in which way---is
%hidden from the attacker under the same assumption.
%
\sys, like other practical systems for private data storage~\cite{cryptdb, mylar,
keybase}, cannot prevent statistical inference attacks over metadata, but always
keeps private data content confidential.
%

%
% Prototype
%
We end-to-end integrated \sys with two real-world web applications to demonstrate
its practicality.
%
\sys helps Lobsters~\cite{lobsters} support GDPR-compliant account deletion with
the option to return, as well as anonymization of old contributions after some
time of user inactivity; and \sys anyonmizes a homework submission
system~\cite{websubmit-rs-anon} after the end of a class while preserving
students' ability to view and edit their homework.
%

%
% Contributions
%
In summary, our key contributions are:
%
\begin{enumerate}[nosep]
 \item Data disguising, a systematic framework for privacy transformations on
   user data in web applications that protects the user privacy against server
   compromise in the absence of actively data-revealing operations.
 \item The abstractions and APIs needed to realize data disguising in a
  practical system, including pseudoprincipals and secure server-side storage
  of data under disguise.
 \item A prototype implementation of data disguising in \sys, a reusable
  library for data disguising, and its integration in two web applications.
 \item A qualitative and quantitative evaluation of \sys's expressivity,
  security, and performance.
\end{enumerate}
%
Our empirical evaluation shows that \sys's operations have reasonable cost
and complete with sub-second latency in most cased, and that normal application
request latency in the presence of disguise and reveal operations increases
only by single-digit milliseconds.
%
%
% Limitations
%
Our prototype has some limitations, however.
%
\note{add limitations.}
%
It does not yet adapt revealed data if the application database schema has
changed while the data was disguised; we plan to use standard schema upgrade
techniques~\cite{hotcrp-autoupgrade, f1-schema-change,
automated-schema-evol-prism} to address this.
%

%
We will make \sys available as open-source software.
%


%%%%%%%%%%%%%%%%%%%%%%%%%%%%%%%%%%%%%%%%%%%%%%%%
%\subsection{Motivation}
%%%%%%%%%%%%%%%%%%%%%%%%%%%%%%%%%%%%%%%%%%%%%%%%
%To motivate data disguising, we consider how HotCRP, a popular application for conference paper and
%review management, can improve user privacy.  HotCRP currently retains data from past conferences
%indefinitely, and many conferences often enforce single- or double-blind reviews, meaning that no
%one but the reviewer should be able to correlate their reviews or paper comments with their
%identity. However, HotCRP's stores reviews indefinitely linked to the identity of the reviewer,
%leaving reviewers succeptible to unblinding in case of a data breach.
%
%To improve upon HotCRP's provided user privacy, HotCRP implement the follow disguises using \sys:
%\begin{enumerate}[nosep]
%    \item Per-user GDPR Account Deletion: removing all of a particular user's data such as reviews and papers.
%    \item Universal Decorrelation: decorrelating users from
%        their authored reviews and papers by (1) generating an
%        anonymous user, and (2) relinking the paper or review
%        to the newly created anonymous user.
%    \item Universal Comment Removal: removing all comments on papers.
%\end{enumerate}
%%%%%%%%%%%%%%%%%%%%%%%%%%%%%%%%%%%%%%%%%%%%%%%%
%Importantly, these disguises place low burden on application developers and end users.
%%
%When HotCRP disguises a user's data for universal decorrelation or comment removal, HotCRP emails
%the user a unique URL that it gets via invoking \sys's API.  When the user
%wants to ask for permission to view (or edit, if permitted) the undisguised paper or review, the
%user simply has to click the URL link and log into the application. Thus, application of universal
%disguises should not require a user to be online in order to gain the capability to apply further
%disguises or view their disguised data.
%
%Similarly, when HotCRP disguises a user's data for GDPR deletion, the invoking user receives a URL
%that allows the user to reveal the disguise, restoring their account.
