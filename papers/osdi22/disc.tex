\section{Discussion}
\label{s:disc}

%
In our prototype, \sys's encrypted store is co-located with the application
database, but an alternative design might store the encrypted bags externally.
%
For example, a more decentralized version of \sys could push database diffs and
speaks-for records to an external ``vault'' under the control of the user (\eg
an S3 storage bucket), or ask the user to download the data under disguise
and remember a hash for tamper-proof restoration.
%
This leaves no trace of the data on the server, and would prevent the attacker
from learning anything (including statistics about the data).
%
However, this design is incompatible with further disguising of
pseudoprincipal-owned data, as \sys no longer knows which natural principal's
vault to push into when the data is decorrelated.
%

%
\sys's efficiency is also affected by cryptographic design choices.
%
If \sys were to only support direct disguises of a natural principal's data, a
symmetric key that the application emails to the client and forgets would
suffice.
%
But to support pseudoprincipals and composition over decorrelated data
(\S\ref{s:composition}), \sys must use asymmetric, public-private key
cryptography.
%
Finally, \sys could pad bags and locator sets to a fixed maximum size, or with
random padding, to hide the size of data under disguise and the number of
disguises applied to a pseudoprincipal.
%
But this would be expensive, as the amount of padding would need to be
proportional to the maximum amount of data under disguise and the number of
disguises.
%
