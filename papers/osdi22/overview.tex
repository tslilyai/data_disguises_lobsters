%-------------------------------------------------------------------------------
\section{Overview and API}
\label{s:api}
%-------------------------------------------------------------------------------

\sys exposes an API (Table~\ref{tab:api}) to web applications that wish to support secure
disguising.
%
The API allows the application to start and end disguises; securely save diff and ownership token
data; and extract token data when necessary to enable post-disguise temporary recorrelation or
disguise revealing.
%
\sys also provides a higher-level convenience API (Table~\ref{tab:high_level_api}) that enables
applications to apply and reveal (a restricted set) of disguises without needing to implement the
disguise's application database modifications.

To use \sys's API, applications must provide \sys with a description of principal IDs and how to
generate new principal objects.
We next describe how an application uses \sys's API.

%%%%%%%%%%%%%%%%%%%%%%%%%%%%%%%%%%%%%%%%%%%%%%%%%%%%%%%%%%%%%%%%%%%%%%%%%%%%%%%%%%%%%%%%%%%
\head{Principal Registration.}
Every client who creates an application account for principal $p$ registers the principal's unique id 
with \sys by invoking \fn{RegisterPrincipal}. \sys generates a public-private key pair, returning private key \privk{p} to the
application. \sys saves the public key \pubk{p} with $p$'s ID.

The application handles passing \privk{p} to the client; \privk{p} will allow the user running the
client to later access $p$'s disguise tokens.

%%%%%%%%%%%%%%%%%%%%%%%%%%%%%%%%%%%%%%%%%%%%%%%%%%%%%%%%%%%%%%%%%%%%%%%%%%%%%%%%%%%%%%%%%%%
\head{Disguising and Disguise Tokens.}
When a client speaking for principal $p$ wants to apply a disguise $d$,
the application invokes \fn{StartDisguise} to signal \sys that a new disguise has begun.
The application then performs the disguise's data updates (\eg, removing or modifying
data objects). If the application wishes for the original data to be later revealed, the
application generates some custom diff and invokes \fn{SaveDiffToken} with the diff.

In the case that the application decorrelates a data object from a principal, the application
queries \sys for a new pseudoprincipal owner for the data object (to preserve referential integrity)
using \fn{CreatePseudoprincipal}. \sys returns the pseudoprincipal's ID and object data; the
application handles properly inserting the pseudoprincipal object.  If the application wishes for
the original owning principal to be temporarily or permanently recorrelated with the data object,
the application generates any ownership information it wishes to save, and invokes
\fn{SaveOwnershipToken} with both the original principal's and new pseudoprincipal's IDs.

When the disguise is complete, the application invokes \fn{EndDisguise}. This informs \sys to erase
any sensitive data, complete any token encryption operations, and to return to the application a
pair of locators that will allow \sys later to locate the (encrypted) diff and ownership tokens
produced by the disguise respectively.

The application handles passing the locators \lcapa{pd} to the user speaking for $p$; to ensure that
adversaries cannot learn that disguise $d$ applied to principal $p$, the application should not
retain locators.

%%%%%%%%%%%%%%%%%%%%%%%%%%%%%%%%%%%%%%%%%%%%%%%%%%%%%%%%%%%%%%%%%%%%%%%%%%%%%%%%%%%%%%%%%%%
\head{Disguise Revealing.}
In order to reveal the original state of data disguised by $d$ for principal $p$, the application 
needs to retrieve the diff and ownership tokens of $p$ from $d$ storing information about the undisguised
state prior to applying $d$.
This requires the application to obtain the locators \lcapa{pd} and decryption capability \privk{p}
from a client speaking for $p$.

With these in hand, the application calls \fn{GetTokensOfDisguise}, which returns the set of
(plaintext) diff and ownership tokens from disguise $d$ for principal $p$; the application then uses
the token data to reveal data as appropriate.
\lyt{Add details about how sometimes we need multiple locators $d' \neq d$ because of pseudoprincipals?}

%%%%%%%%%%%%%%%%%%%%%%%%%%%%%%%%%%%%%%%%%%%%%%%%%%%%%%%%%%%%%%%%%%%%%%%%%%%%%%%%%%%%%%%%%%%
\head{Temporary Authorization.}
The application may also wish to temporarily recorrelate users with their data by temporarily
authorizing them to act on behalf of a pseudoprincipal. For example, users can temporarily act as
anonymous users to edit their anonymized lecture answers. Furthermore, applications may wish to
\emph{compose} disguises on top of anonymized data, to \eg delete now-anonymized data that
originally belonged to $p$ when $p$ deletes their account.

\sys allows the application to query for all pseudoprincipals corresponding to a particular
principal using \fn{GetPseudoprincipalsOf}, which returns the set of corresponding pseudoprincipal
IDs. The application must provide (obtaining from the client) any locators \lcapa{pd} from 
anonymizing disguises $d$ that produced the pseudoprincipals in question, as well as the client's
decryption capability \privk{p}. This allows \sys to inspect the content of ownership tokens and
verify links between principal $p$ and pseudoprincipals stored in ownership tokens.

Once the application knows which pseudoprincipals $p'$ correspond to a principal $p$, the application can
authorize a client acting on behalf of $p$ to act on behalf of $p'$.

%%%%%%%%%%%%%%%%%%%%%%%%%%%%%%%%%%%%%%%%%%%%%%%%%%%%%%%%%%%%%%%%%%%%%%%%%%%%%%%%%%%%%%%%%%%
\head{Disguising and Revealing with the High-Level API.}
If an application developer does not want to implement the necessary data transformations for disguise
application and revealing, the developer can provide a disguise specification in a language provided
by \sys, which includes schema information and supports a restricted set of SQL database
modifications. When \fn{ApplyDisguise} is invoked with the disguise specification, \sys applies the
disguise data transformations and automatically generates the appropriate diff and ownership tokens.

Similarly, when \fn{RevealDisguise} is invoked, \sys can interpret the automatically-generated diff and
ownership tokens and perform the data modifications to restore the original state of data (given
that no subsequent updates had been performed since disguising).

Application handling of locators and decryption capabilities remains the same: \fn{ApplyDisguise}
takes optional locators and decryption capabilities to support disguise composition, and produces
locators for the applied disguise; \fn{RevealDisguise} takes locators and decryption capabilities to
locate and access tokens to reveal.

\begin{table*}[h!]
\centering
    \begin{tabular}{ p{.5\linewidth} p{.5\linewidth} }
\textbf{API Call} & \textbf{Description} \\
\hline
\fn{RegisterPrincipal($p$)} $\rightarrow$ \privk{p} & Registers $p$ as
        a principal, returning a private key \privk{p} corresponding to a principal-specific public
        key $\pubk{p}$ known to \sys.\\
    \vspace{6pt}\\
\fn{StartDisguise($d$) $\rightarrow$ ()} & Invoked to inform \sys that a disguise application is
        starting.
    \vspace{6pt}\\
        \fn{EndDisguise($d$) $\rightarrow$ (\lcapa{pd}, \lcapa{pd})} & Marks the end of applying disguise $d$.
        Returns a pair of locators required to later locate disguise $d$'s diff tokens and ownership tokens
        respectively.
    \vspace{6pt}\\
        \fn{SaveDiffToken($p$, $d$, tokenData)} $\rightarrow$ () & Securely stores tokenData in a diff token \tdiff{pd}.\\
    \vspace{6pt}\\
        \fn{SaveOwnershipToken($p$, $p'$, $d$, tokenData)} $\rightarrow$ () & Securely stores tokenData in an
        ownership token \town{pd} that saves the link from principal $p$ to pseudoprincipal $p'$.\\
    \vspace{6pt}\\
        \fn{CreatePseudoprincipal()} $\rightarrow$ ($p$, \fn{principalData}) & Returns
        the pseudoprincipal ID and data of a new pseudoprincipal. \fn{principalData} is
        generated using an application-specified pseudoprincipal-generation policy.\\
    \vspace{6pt}\\
        \fn{GetTokensOfDisguise($d$, \privk{p}, \lcapa{pd})}
        $\rightarrow$ (\{\tdiff{pd}\}, \{\town{pd}\}) & Retrieves the diff and ownership tokens produced by 
        disguise $d$ that are accessible with the provided locator 
        and decryption capability.
    \vspace{6pt}\\
        \fn{GetPseudoprincipalsOf($p$, \privk{p}, \{\lcapa{pd}\})} $\rightarrow$ \{$p'$\} & Returns
        the pseudoprincipal IDs of all pseudoprincipals linked with $p$ produced by disguises $d$.
        The decryption capability and set of locators enable \sys to access ownership tokens that
        reveal links between $p$ and its pseudoprincipals.\\
    \end{tabular}
\caption{\sys's API}
\label{tab:api}
\end{table*}

\begin{table*}[h!]
\centering
    \begin{tabular}{ p{.5\linewidth} p{.5\linewidth} }
\textbf{API Call} & \textbf{Description} \\
\hline
    \fn{ApplyDisguise($p$, $d$, DisguiseSpec dSpec, \privk{p}, \{\lcapa{pd'}\})}
        $\rightarrow$ (\lcapa{pd}, \lcapa{pd}) &
        Applies disguise $d$, returning a parit of locators required to later locate disguise $d$'s
        diff and ownership tokesn respectively. 
        \vspace{6pt}\\
        \fn{RevealDisguise($p$, $d$, \privk{p}, \lcapa{pd})} $\rightarrow$ ()& 
        Reveals disguise $d$ by reversing data stored in tokens produced by disguise $d$ accessible
        using \lcapa{pd} and \privk{p}. 
\end{tabular}
\caption{\sys's Higher-Level Convenience Library API}
\label{tab:high_level_api}
\end{table*}


\iffalse
Imagine that HotCRP disguise $d$ anonymizes all reviewers and authors of a conference by
decorrelating them from their papers and reviews respectively. As described above, HotCRP emails the
client a link when that their data has been decorrelated, which contains the 
capability in the URL. Assume that HotCRP sends a unique link for every piece of data (\eg every review
or paper); HotCRP could also send a single link for all decorrelated data.

The application normally ensures that only a client speaking for $p$ should have read-only access to
reviews for $p$'s authored papers, and read-write access to reviews $p$ wrote.

If the client speaking for $p$ wants to edit a review as it would normally, the application
(post-$d)$ will not grant the client any read/write permissions to perform actions on any reviews of
the conference: any previously correlated paper or review is now associated with pseudoprincipal
$q$.  

Instead, for the client speaking for $p$ to regain authorship or reviewership permissions of a
particular review or paper:
\begin{itemize}
    \item The client clicks on the URL emailed to the client for that piece of data when it was decorrelated, which
        opens a HotCRP page.
    \item Javascript on the webpage retrieves the client's data capability
        stored by the client's browser, and extracts the  capability and the data object the client
        wants to access from the URL.
    \item The client then must log into HotCRP as principal $p$.
    \item HotCRP invokes \sys's \fn{CapEstablishesOwnership} for the data object in question,
        using the capabilities passed in by the client.
    \item If \sys returns \fn{true}, the application permits the client to view the object with ownership
        rights (\eg ability to see reviews on authored papers, or edit their review); otherwise, the
        application does not allow the client to perform privileged actions.
\end{itemize}
\fi
