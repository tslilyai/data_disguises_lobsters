\section{Design}
\label{s:design}

\subsection{Abstractions}
%
\sys interacts with the application through three key abstractions:
\one{} principals, \two{} diffs and speaks-for records, and
\three{} record locators.
%

%
Principals are holders of cryptographic keys and locators, and correspond to
uniquely identifiable application users (\eg via a user ID).
%
An application represents principals as a record in a \fn{users} table or some
other datastore abstraction.
%
\sys refers to human application users as \emph{natural principals}.
%
Natural principals have data rights (\eg under the GDPR) and interact with
the application.
%
By contrast, \emph{pseudoprincipals} are anonymous users generated as part of
\sys's disguises, and inserted into the application's principal storage
to maintain application-level functionality while decorrelating data.
%
A pseudoprincipal may or may not observably differ from a natural principal
at the application level.
%

%
When they invoke \sys's disguise and reveal APIs, they cause \sys to encrypt,
store, and retrieve \emph{diff records} and \emph{speaks-for records}.
%
A diff record stores a set of database changes that the disguise made.
%
A speaks-for record stores links between pseudoprincipals produced by a
disguise and another principal.
%
\sys securely stores these records for the application by encrypting records
with per-principal public keys.
%
Only clients hold the private keys for natural principals, so once disguised,
a natural principal's data can only be revealed with their involvement.
%

%
\sys stores record ciphertexts from disguise $d$ for principal $p$ in a record
bag indexed by \emph{record locator \lcapa{pd}}.
%
Locators are opaque byte strings that clients use to tell \sys which bag to
use in a reveal operation.
%
The application only handles locators transiently and should not store them;
instead, its role is to communicate the locators to clients.
%
Locators are strictly a performance optimization, and \sys's security
properties are unaffected by their existence.
%

%%%%%%%%%%%%%%%%%%%%%%%%%%%%%%%%%%%%%%%%%%%%%%%
\subsection{API}
\label{s:api}

\begin{figure}[t]
%\centering
%    \begin{tabular}{ p{.5\linewidth} p{.5\linewidth} }
%\textbf{API Call} & \textbf{Description} \\
%\hline
%    \fn{ApplyDisguise($p$, $d$, DisguiseSpec dSpec, \privk{p}, \{\lcapa{pd'}\})}
%        $\rightarrow$ (\lcapa{pd}, \lcapa{pd}) &
%        Applies disguise $d$, returning a parit of locators required to later locate disguise $d$'s
%        diff and speaks-for records respectively.
%        \vspace{6pt}\\
%        \fn{RevealDisguise($p$, $d$, \privk{p}, \lcapa{pd})} $\rightarrow$ ()&
%        Reveals disguise $d$ by reversing data stored in records produced by disguise $d$ accessible
%        using \lcapa{pd} and \privk{p}.
%        \lyt{In practice this takes a set of \lcapa{} from different disguises because of
%        pseudoprincipals}
%\end{tabular}
\begin{lstlisting}[language=Rust, style=rust]
RegisterPrincipal(p: UID) -> PrivKey;
ApplyDisguise(p: UID, spec: DisguiseSpec, pk: PrivKey, locs: Vec<Locator>) -> (DisguiseID, Vec<Locator>);
RevealDisguise(p: UID, did: DisguiseID, pk: PrivKey, locs: Locator);
\end{lstlisting}
\caption{\sys's higher-level API.}
\label{f:api-high}
\end{figure}

\begin{figure}[t]
%\centering
%    \begin{tabular}{ p{.5\linewidth} p{.5\linewidth} }
%\textbf{API Call} & \textbf{Description} \\
%\hline
%\fn{RegisterPrincipal($p$)} $\rightarrow$ \privk{p} & Registers $p$ as
%        a principal, returning a private key \privk{p} corresponding to a principal-specific public
%        key $\pubk{p}$ known to \sys.\\
%    \vspace{6pt}\\
%\fn{StartDisguise($d$) $\rightarrow$ ()} & Invoked to inform \sys that a disguise application is
%        starting.
%    \vspace{6pt}\\
%        \fn{EndDisguise($d$) $\rightarrow$ (\lcapa{pd}, \lcapa{pd})} & Marks the end of applying disguise $d$.
%        Returns a pair of locators required to later locate disguise $d$'s diff records and
%        speakds-for records
%        respectively.
%    \vspace{6pt}\\
%        \fn{SaveDiffRecord($p$, $d$, recordData)} $\rightarrow$ () & Securely stores recordData in a diff record \tdiff{pd}.\\
%    \vspace{6pt}\\
%        \fn{SaveSpeaksForRecord($p$, $p'$, $d$, recordData)} $\rightarrow$ () & Securely stores recordData in a
%        speaks-for record \town{pd} that saves the link from principal $p$ to pseudoprincipal $p'$.\\
%    \vspace{6pt}\\
%        \fn{CreatePseudoprincipal()} $\rightarrow$ ($p$, \fn{principalData}) & Returns
%        the pseudoprincipal ID and data of a new pseudoprincipal. \fn{principalData} is
%        generated using an application-specified pseudoprincipal-generation policy.\\
%    \vspace{6pt}\\
%        \fn{GetRecordsOfDisguise($d$, \privk{p}, \lcapa{pd})}
%        $\rightarrow$ (\{\tdiff{pd}\}, \{\town{pd}\}) & Retrieves the diff and speaks-for records produced by
%        disguise $d$ that are accessible with the provided locator
%        and decryption capability.
%    \vspace{6pt}\\
%        \fn{GetPseudoprincipalsOf($p$, \privk{p}, \{\lcapa{pd}\})} $\rightarrow$ \{$p'$\} & Returns
%        the pseudoprincipal IDs of all pseudoprincipals linked with $p$ produced by disguises $d$.
%        The decryption capability and set of locators enable \sys to access speaks-for records that
%        reveal links between $p$ and its pseudoprincipals.\\
%    \end{tabular}
\begin{lstlisting}[language=Rust, style=rust]
RegisterPrincipal(p: UID) -> PrivKey;
CreatePseudoprincipal() -> (UID, PrincipalData);
DeregisterPrincipal(p: UID);

StartDisguise() -> DisguiseID;
EndDisguise(did: DisguiseID) -> Vec<Locator>;

// stores an encrypted diff with app-specific content
StoreDiff(p: UID, did: DisguiseID, data: Vec<u8>);
// stores encrypted mapping that p1 speaks for p2
StoreSpeaksFor(p1: UID, p2: UID, did: DisguiseID);

// get diffs stored as part of did at loc
DiffsForDisguise(did: DisguiseID, pk: PrivKey, loc: Locator) -> Vec<Vec<u8>>;
// principals that p speaks for (stored at loc)
PseudoprincipalsOf(p: UID, pk: PrivKey, loc: Locator) -> Vec<UID>;
\end{lstlisting}
\caption{\sys's low-level API.}
\label{f:api-low}
\end{figure}




\noindent
\sys exposes two APIs to applications:
%
\begin{enumerate}[nosep]
 \item A high-level API (Figure~\ref{f:api-high}) for applications that want \sys to do
   all the work of applying disguises, including database modifications, based on
   a declarative disguise specification and a database schema with foreign key
   relationships.
 \item A low-level API (Figure~\ref{f:api-low}) that provides application developers with
   additional control. This API allows custom organization of disguised data, lets the application
   make its own database modifications, and is schema-agnostic. This API requires applications
   to produce and interpret diff records, and to tell \sys about speaks-for links between
   principals and pseudoprincipals.
\end{enumerate}
%
\sys internally implements the high-level API in terms of the low-level API.
%
We primarily discuss the low-level in the following; \S\ref{s:api-high}
briefly describes the high-level API.
%

%% Malte:
%% This is only needed for the high-level API, and there it's part of the disguise spec.
%%
%To use either API, applications provide \sys with a description of how the application represents
%principals, how to interpret unique principal IDs, and how to
%generate new principal objects.
%

%
\paragraph{Example Application.}
%
We structure our discussion of \sys's API around an example application,
WebSubmit~\cite{websubmit-rs-anon}.
%
WebSubmit is a homework submission application used at a U.S.\ university.
%
It has a simple schema that consists of tables for lectures, questions, answers, and
user accounts.
%
Clients create an account, submit homework answers, and view their submissions;
course staff can also view submissions, and add/edit questions and lectures.
%
The original WebSubmit lacks privacy transformations and retains data forever.
%

%
We wrote two disguises for WebSubmit using the low-level API: \one{}
instructor-initiated anonymization of all answers (\eg at the end of the semester),
and \two{} GDPR-compliant user account deletion.
%
%Anonymization performs per-user disguises that decorrelate lecture answers from the
%user, but keeps the answers grouped by lecture (\ie one pseudoprincipal per set
%of lecture answers).
%
With \sys, WebSubmit users can restore their deleted accounts, and students can
view and edit their answers after even after class anonymization.
%

\subsection{Applying Disguises}
%
We now explain how an application (here, WebSubmit) applies disguises.
%

\head{(1) Register Principals.}
%
Clients who create a WebSubmit account generate a fresh public/private key pair
in their web browser, and provide the public key to WebSubmit.
%
WebSubmit then invokes \fn{RegisterPrincipal} with the new principal's UID $p$
and public key \pubk{p}, which \sys saves.
%

\head{(2) Start a Disguise.}
%
WebSubmit's disguises start either when a client deletes their account,
or when the instructor chooses to anonymize the class.
%
WebSubmit first invokes \fn{StartDisguise} to signal \sys that a new
disguise has begun, and receives back a disguise ID ($d$).
%

\head{(3) Apply Database Changes and Store Records.}
%
WebSubmit then applies database changes for the disguise.
%
To delete principal $p$'s account, WebSubmit first retrieves the account record
from \fn{users} and the principal's answers from \fn{answers}.
%
WebSubmit serializes the records describing $p$'s account and $p$'s answers into
a buffer, and invokes \fn{StoreDiff(}$p$, $d$, \fn{buffer)}.
%
\sys securely stores these bytes until WebSubmit retrieves them to operate on
disguised data (\S\ref{s:op-disg}).
%
WebSubmit then removes the answers from the database.
%

%
To anonymize principal $p$'s account, WebSubmit likewise retrieves the account
details and the principal's answers.
%
WebSubmit then removes the account, and decorrelates the principal's answers.
%
To decorrelate, WebSubmit:
%
\begin{enumerate}[nosep]
%
\item generates a new anonymous WebSubmit user $p_{\textrm{anon}}$ for each
  lecture--answer pair;
%
\item invokes \fn{CreatePseudoprincipal(}$p_{\textrm{anon}}$\fn{)};
%
\item updates the answers' foreign keys to \fn{users} to point to
  $p_{\textrm{anon}}$ instead of $p$; and
%
\item invokes \fn{StoreSpeaksFor(}$p$, $p_{\texttt{anon}}$, $d$\fn{)} to tell
    \sys to securely store the speaks-for link between $p$ and $p_{\texttt{anon}}$.
\end{enumerate}
%

\head{(4) End the Disguise.}
%
When the disguise is complete, WebSubmit invokes \fn{EndDisguise}.
%
This informs \sys to complete any record encryption operations and return two
locators \lcapa{pd} that will allow \sys to later locate the (encrypted) diff and
speaks-for records stored during disguise $d$.
%
\sys maintains the map from locators to encrypted record bags, but importantly
forgets which principal and disguise produced the locator.
\lyt{added}
%
WebSubmit emails these locators to the client, and then forgets them to ensure
that attackers cannot learn that $d$ was applied to $p$.
%
Finally, WebSubmit deletes the natural principal's account from the database and
invokes \fn{ForgetPrincipal} to remove \sys's state for the natural principal,
leaving no trace of their existence in the system.
%

\subsection{Operating on Disguised Data}
\label{s:op-disg}

%
WebSubmit uses \sys to operate on disguised data when \one{} a user restores their
previously-deleted account; \two{} a user view or edits their answers after the
class has been anonymized; and \three{} so that a user who deletes their account
after class anonymization can still delete their anonymized answers.
%

\head{(1) Restoring a Deleted Account.}
%
When a client wants to restore the account of natural principal $p$, the client
supplies $p$'s private key and the locators \lcapa{pd} that WebSubmit emailed to
the client when it applied the account deletion disguise $d$.
%
WebSubmit receives the key and locators, and invokes \fn{DiffsForDisguise} with
\lcapa{pd} and the private key.
%
\sys uses the private key to decrypt the records stored earlier, and returns the
set of (plaintext) database diffs, which WebSubmit then uses to return the
application database to its original state.
%
Finally, \sys and the application forget the locators and $p$'s private key.
%
In this case, $p$ has no pseudoprincipals, as account deletion removes their
answers than decorrelating them.
%
WebSubmit thus only handles diff records, but if the disguise had decorrelated
data, the application would invokes \fn{PseudoprincipalsOf} with the \lcapa{pd}
and $p$'s private key to determine which pseudoprincipals' data should be
recorrelated back with $p$.
%

\head{(2) Accessing Anonymized Answers.}
%
When a client chooses to edit their anonymized answer, the client provides $p$'s
private key and the locators \lcapa{pd} email to by WebSubmit when it applied
the anonymization disguise $d$.

When WebSubmit receives the request, it invokes \fn{PseudoprincipalsOf} with the
\lcapa{pd} and private key to determine which pseudoprincipals $p$ may speak-for.
%
Once WebSubmit confirms that $p$ may speak-for the pseudoprincipal who is
associated with the lecture answers in question in the application database,
WebSubmit authorizes $p$ to edit the answers.
%
Finally, WebSubmit forgets the private key and locators provided by the client.
%

\head{(3) Deleting Anonymized Answers.}
%
A client may choose to delete their account after the instructor anonymized the
class.
%
To apply the account deletion disguise on top of the already-disguised (anonymized)
data, the client supplies---along with the account deletion HTTP request---its
private key and the locators \lcapa{pd} that WebSubmit sent when it applied the
anonymization disguise $d$.
%
WebSubmit then invokes \fn{PseudoprincipalsOf} with the \lcapa{pd} and the private
key to determine which pseudoprincipals $p$ may speak-for, and deletes the answers
of these pseudoprincipals.
%
As before, WebSubmit then forgets the private key and locators, and also removes
the pseudoprincipals from the application database.
%
(\sys must continue to remember the pseudoprincipals, however, so that the user
can later restore their---anonymized---account.)
%

%%%%%%%%%%%%%%%%%%%%%%%%%%%%%%%%%%%%%%%%%%%%%%%%%%%%%%%%%%%%%%%%%%%%%%%%%%%%%%%%%%%%%%%%%%%
\subsection{High-Level API}
\label{s:api-high}

%
Applications with relatively simple disguises can use \sys's high-level API to
delegate the necessary database transformations, generation of diff and speaks-for
records, and pseudoprincipal management to \sys.
%
To do so, the application developer provides a declarative disguise specification in
\sys high-level domain-specific language (DSL).
%
The DSL includes schema annotations and supports a restricted set of SQL database
modifications.
%
When the application invokes \fn{ApplyDisguise} with a disguise specification, \sys
applies the specified changes, creates pseudoprincipals, generates the appropriate
diff and speaks-for records, and stores them encryptedly.
%
Similarly, \fn{RevealDisguise} tells \sys to interpret the diff and speaks-for
records and apply the database changes to restore the original state.
%
(If the disguised records have changed in the database, \sys leaves the modified,
disguised record in place rather than restoring.)
%

%
Handling of locators and private keys remains the application's responsibility:
\fn{ApplyDisguise} returns locators for the applied disguise that the
application conveys to the client; and \fn{RevealDisguise} takes locators and a
private key to locate and decrypt records to reveal.
%
Finally, \fn{ApplyDisguise} takes optional locators and decryption capabilities
to support disguise composition (\eg deletion after anonymization), which
we discuss in \S\ref{s:composition}.
%
