%-------------------------------------------------------------------------------
\section{Design} 
%-------------------------------------------------------------------------------

\subsection{Disguise Operations.} 
Because disguises are inherently application-specific, the application developer 
specifies a disguise as a 
%provides disguise
%specifications that consist of a 
set of predicated disguise operations \op{d} to perform.

Operations \op{d} of disguise $d$ take data objects as input and execute updates to application
data.  \sys automatically generates database change records when applying \op{d}. 

Developers describe which principal(s) an operation's generated database change record corresponds
to. For example, a database change record generated by removing comment may correspond to the
principal whose ID is referenced by the author column.  

Developers also specify an application-aware pseudoprincipal generation policy, namely how to
generate new user accounts in a manner that the application can handle (e.g., pseudoprincipals may
not have email addresses).

Finally, developers specify which principals are authorized to apply the disguise: enforcing access
control for disguising is left to the application.  
%\lyt{It seems most reasonable for the
%pplication to enforce AC for the disguising API the application exposes to the client?}

\vspace{6pt}\noindent
Operations come in three forms:
\begin{enumerate}
    \item Modify: change an attribute of the data object.
    \item Remove: delete the data object.
    \item Decorrelate: generate a \emph{pseudoprincipal} $q$, and rewrite the foreign key to original
        principal $p$ from the data object to instead point to $q$.  
\end{enumerate}

\noindent For each \op{d}, the application developer specifies:
\begin{itemize}
    \item An associated predicate over the application database that selects \op{d}'s input
        objects in a SQL-like fashion.
    \item The type of operation and its arguments (e.g., which attributes of the data object to
        modify).
    \item The corresponding principal(s) that should have the capability to access \op{d}'s
        generated disguise change or correlation record.
\end{itemize}

%%%%%%%%%%%%%%%%%%%%%%%%%%%%%%%%%%%%%%%%%%%%%%%%%%%%%%%%%%%%%%
\subsection{Database Diffs.}
%%%%%%%%%%%%%%%%%%%%%%%%%%%%%%%%%%%%%%%%%%%%%%%%%%%%%%%%%%%%%%
Every \op{d} produces a \emph{database diff} \tdata{pd} associated with the disguise $\delta$ and a principal $p$. 
%
Each \tdata{pd} contains the ID of the disguise, the associated principal $p$'s ID,
and a random nonce.
%
In addition to this data, a decorrelation \tdata{pd} contains the created pseudoprincipal's
ID; a removal \tdata{pd} contains the removed object's value; and a modification
\tdata{pd} contains the old and new value of the modified object.

%%%%%%%%%%%%%%%%%%%%%%%%%%%%%%%%%%%%%%%%%%%%%%%%%%%%%%%%%%%%%%
\subsection{Diff Access Control.} 

\head{Securing Access to Database Diffs.} 
\sys's design uses principal $p$'s private key \privk{p} as the data capability \dcapa{p}. \sys
secures $p$'s diffs \tdata{pd} by encrypting them with \pubk{p}. The diff's nonce ensures safety
against known-plaintext attacks. 
Only a client who knows \privk{p} can access diffs.

%\sys represents a data capability \dcapa{p} as a symmetric key specific to
%the disguise and principal.
%

%%%%%%%%%%%%%%%%%%%%%%%%%%%%%%%%%%%%%%%%%%%%%%%%%%%%%%%%%%%%%%
\head{Securing Disguise History.}
\sys's design may leak information that $d$ disguised principal $p$ because an adversary may learn
that ciphertexts for \tdata{pd} exist for a particular $p$ and $d$.  This is a problem: for example,
\sys should not store information that a principal $p$ has invoked GDPR deletion.

To avoid this problem, \sys stores an array of \tdata{pd} ciphertexts at random location pointed to
by locating capability \lcapa{pd}.  This dissociates one disguise applied to a principal $p$ from
all other disguises, so an adversary only ever learns about a single disguise if \sys gains
permission to reveal or compose upon that disguise.

Note that an adversary without access to any \lcapa{pd} can learn that $n$ diffs exist for
\emph{some} $p$ and $d$, but cannot directly identify which $p$ or $d$.  If an adversary has access
to \lcapa{pd}, the adversary can learn that $d$ applied to $p$, and the number of \tdata{pd} diffs
for that $p$ and $d$. This metadata is out of scope of our threat model.
\lyt{An alternative design might remove the encrypted data completely (and email it to the client),
so that an adversary doesn't even know of its existence. However, since we're allowing disguised
data to be distinguished from undisguised data, this seems potentially unnecessary.}

%%%%%%%%%%%%%%%%%%%%%%%%%%%%%%%%%%%%%
\head{Storage of Capabilities.}
\sys should not store \dcapa{p} or \lcapa{pd}: an adversary could then learn that $d$ has applied to
principal $p$. Thus, these capabilities must be stored externally, even when no client speaking for
$p$ is currently online.
%
%The only exception to this occurs when $p$ is a pseudoprincipal: in this case, \sys saves a
%(globally accessible) mapping from $p$ to \lcapa{pd}. This leaks the disguise history of
%pseudoprincipals, but is allowable because pseudoprincipals' data has already been decorrelated from their
%original principal owner.
%

Clients already hold \dcapa{p}: a client generates a keypair for a new principal $p$ and stores
\privk{p}, while registering \pubk{p} with \sys. \sys only stores \pubk{p}.

Because \sys generates \lcapa{pd} during disguising, \sys must communicate \lcapa{pd} to clients.
\sys emails \lcapa{pd} to the corresponding email address associated with $p$.  This allows \sys to
disguise data of principals even when no client authenticated as $p$ has an active session open.

%%%%%%%%%%%%%%%%%%%%%%%%%%%%%%%%%%%%%%%%%%%%%%%%%%%%%%%%%%%%%%
\subsection{Current Design: Discussion}
As the current design stands, \sys supports 1st-person and 3rd-person disguising, as well as
1st-person revealing.
%
Application developers can write GDPR-compliant and/or universal disguises with the primitives
exposed by \sys to write disguise specifications.
%
\sys uses diffs produced from disguising to reveal data when authorized to do so, and when
revealing does not revert updates made to the data since the time of disguise application.

%
\sys also supports ``Temporary Recorrelation without Database Changes'' because \sys can determine
the original owner of data as long as \sys has access to decorrelation diffs (which the client or
the application provides via the appropriate capabilities).
%
\sys can use information from decorrelation diffs to disguise decorrelated data as if it were owned
by the original user; and 
%
\sys can support the API discussed in \S\ref{s:api}, which allows applications to query \sys to
check ownership properties and grant authorized users personalized views and permissions to access
data objects.

Furthermore, \sys does this while meeting all security goals: \sys supports authorized disguises and
ensures the security of ownership claims, disguise diffs, and disguise history.

Our current design falls short, however, by failing to support ``Disguising Anonymized
Users'' which we explain next.

%%%%%%%%%%%%%%%%%%%%%%%%%%%%%%%%%%%%%%%%%%%%%%%%%%%%%%%%%%%%%5
%%%%%%%%%%%%%%%%%%%%%%%%%%%%%%%%%%%%%%%%%%%%%%%%%%%%%%%%%%%%%5
%%%%%%%%%%%%%%%%%%%%%%%%%%%%%%%%%%%%%%%%%%%%%%%%%%%%%%%%%%%%%5
%%%%%%%%%%%%%%%%%%%%%%%%%%%%%%%%%%%%%%%%%%%%%%%%%%%%%%%%%%%%%5
%\begin{itemize}
%\item \fn{diffID}: unique ID for this diff
%\item \fn{disguiseID}: ID of disguise $d$ being applied
%\item \fn{principalID}: associated principal $p$
%\item \fn{objID}: unique ID for the data object modified by \op{d}
%\item \fn{updateType}: decorrelate, modify, or remove
%\item \fn{oldValue}: original value of object \fn{objID}
%\item \fn{newValue}: updated value of object \fn{objID}
%\item \fn{nonce}: random value generated to prevent known-plaintext attacks
%\end{itemize}

\iffalse
\begin{figure*}[t!]
\pcb{
\<\< \\[-0.9\baselineskip]\\
\textbf{Client Authenticated as $p_0$} \< \< \textbf{\sys} \\
[0.1\baselineskip][\hline]
\<\< \\[-0.9\baselineskip]\\
\fn{capPairs} \gets \fn{LoadClientCapPairs($p_0$)}\\
% TODO get all privkeys
\fn{encPrivKs} \gets \fn{LoadClientCapPairs($p_0$)}\\
\fn{privKs} \gets \{p \mapsto \privk{p}\}\pclb
\pcintertext[dotted]{Recursively get addresses of all linked pseudoprincipals}
\fn{idsToProcess} \gets \{p\}\\
\pcwhile \fn{idsToProcess} \neq \{\}:\\
\quad q \gets \fn{idsToProcess.pop()}\\
\< \sendmessageright*{\fn{GetPseudoPrincipalEncPrivKeys($q$)}} \< \\
\<\< \fn{encPrivKs} \gets \fn{\sys.Principals[q].EncPKTokens}\\
\< \sendmessageleft*{\fn{encPrivKs}} \< \\
\quad \pcforeach \fn{encPrivK} \in \fn{encPrivKs}:\\
\quad \quad \privk{q} \gets \fn{privKeys}[q]\\
\quad \quad \tpriv{qr} \gets \dec(\privk{q}, \fn{encPrivK})\\
\quad\quad\fn{privKs.insert}(r\mapsto \tpriv{qr}.\fn{privKey}) \<\< \\
\quad\quad\fn{idsToProcess.insert($r$)} \<\< \\
\< \sendmessageright*{\fn{GetGlobalAddresses($q$)}} \< \\
\<\< \fn{globalAddrs} \gets \fn{\sys.Principals[q].GlobalAddrs}\\
\< \sendmessageleft*{\fn{globalAddrs}} \< \\
\quad\quad\fn{addrs.append(globalAddrs)}\\
\quad \pcendforeach\\
\pcendwhile\pclb
    \pcintertext[dotted]{Use addresses to get corresponding encrypted token keys}
\< \sendmessageright*{\fn{AddressesToEncTokenKeys(addrs)}} \< \\
    \<\< \fn{encsymkeys} \gets \fn{LoadEncTokenKeys(addrs)}\\
\< \sendmessageleft*{\fn{encsymkeys}}\\
\fn{symkeys} \gets \{\}\\
\pcforeach \fn{Enc(\symk{qd'})} \in \fn{encsymkeys}\\
\quad \pcif \text{authorizes access to $q$'s private tokens for $d'$}: \\
\quad\quad \privk{q} \gets \fn{privKs[q]}\\
\quad\quad \fn{symkeys.insert(\dec(\privk{q}, \symk{qd'}))}\\
\pcendforeach\pclb
\pcintertext[dotted]{Apply disguise with token keys}
\< \sendmessageright*{\fn{Disguise($d$,symkeys)}} \< \\
\<\< \{\addr{}\}\gets\fn{ApplyDisguise($d$,symkeys)} \pclb
\pcintertext[dotted]{Addresses can be emailed by \sys or returned to application}
}
\caption{\textbf{Disguise Application.}}
\label{fig:disgapp}
\end{figure*} 
\fi

%%%%%%%%%%%%%%%%%%%%%%%%%%%%%%%%%%%%%%%%%%%%
%%%%%%%%%%%%%%%%%%%%%%%%%%%%%%%%%%%%%%%%%%%%
\iffalse
\head{Decorrelation and Locked Private Keys.}
Decorrelation generates a public-private key pair \pubk{q} and \privk{q} for pseudoprincipal $q$
created during decorrelation.
\sys stores \pubk{q} associated with $q$'s user ID, and puts \privk{q} in a \emph{locked private key}
associated with the original principal $p$ and $q$, notated as \tpriv{p}{q}. \tpriv{p}{q} contains the
pseudoprincipal's ID and private key, along with a random nonce, which are all encrypted with \pubk{p}:
%\begin{itemize}
%\item \fn{principalID}: original principal $p$
%\item \fn{pseudoprincipalID}: generated pseudoprincipal $q$
%\item \fn{pseudoPrincipalPrivateKey}: \privk{q}
%\item \fn{nonce}: random value generated to prevent known-plaintext attacks
%\end{itemize}
Because \tpriv{p}{q} is encrypted with \pubk{p}, only a client who has
\privk{p} can access pseudoprincipal $q$'s private key \privk{q}.


\head{Protecting \dcapa{p} when stored by \sys.}
%\lyt{We need public keys here so that \sys does not hold onto keys that can decrypt for pseudoprincipals!!}
\sys then encrypts \dcapa{p} with public key \pubk{p}.  
%
Thus, only a client who has \privk{p} can decrypt the private key, and return in the capability pair the plaintext symmetric key that enables \sys to decrypt $p$'s database diffs produced from applying $d$.

%\sys stores all global \tdata{pd} tokens in plaintext, where any party can access it.

%%%%%%%%%%%%%%%%%%%%%%%%%%%%%%%%%%%%%
\head{Acquiring Pseudoprincipal Capabilities}
As discussed above, \sys can access locating capabilities \lcapa{qd}s for pseudoprincipal $q$ without a
client providing it. 
%
However, a client still needs to acquire \dcapa{qd} for \sys to have full capability pairs
to disguise, reveal, or otherwise act on behalf of pseudoprincipal $q$.

\sys adds a \fn{GetPrincipalLockedPrivateKeys} Client API call, which returns all locked private
keys \tpriv{p}{q} generated when decorrelating data from $p$.
A client then must prove it has \privk{p} to decrypt \tpriv{p}{q} and extract pseudoprincipal ID $q$ and \privk{q}.

With $q$ and \privk{q}, the client can get all encrypted \dcapa{qd} for all $d$, and
provide these to \sys so \sys has the full capability pair \pcapa{q}{d}. With
this pair, \sys can retrieve diffs or establish ownership for $q$'s data.

\lyt{TODO: Add an internal server-side API call that retrieves globally accessible location capabilities?}

%-------------------------------------------------------------------------------
\subsection{Does this design achieve our security goals?}
\label{sec:achievegoals}
%-------------------------------------------------------------------------------
\vspace{6pt}\noindent\textbf{\emph{(1) Authorized Disguises.}}
\sys ensures that only properly authenticated clients speaking for authorized principals can apply or
reveal a disguise, achieving security goal (1).

\vspace{6pt}\noindent\textbf{\emph{(2/3) Secure Ownership and Secure Disguise Diffs.}}
\lcapa{pd} is required to find the encrypted \dcapa{p}---a symmetric key
specific to each disguise $\delta$ and principal $p$ encrypted with \pubk{p}.
Because \sys encrypts \tdata{pd} diffs associated with $p$ with \dcapa{p}, a client
needs to first decrypt $\dcapa{p}$ with \privk{p}, and then hand the pair \pcapa{p}{d}
to \sys for \sys to decrypt of $p$'s diffs from $d$.

Thus, \sys ensures that only clients who can provide \lcapa{pd} and \dcapa{p} (and,
by consequence, have \privk{p}) can access $p$'s disguise diffs, or learn what data objects $p$ and
own. 

\vspace{6pt}\noindent\textbf{\emph{(4) Privacy of Disguise History.}}
We describe how \sys hides which disguises affect each principals' data by requiring a client to
provide a locating capability \lcapa{pd} in order to find encrypted data from disguise $d$ applied
to (non-pseudoprincipal) $p$. Because \sys does not store these locations, and the locationsare per-disguise, \sys
cannot link different disguises' encrypted data to the same principal.

\begin{table}[t!]
\centering
\begin{tabular}{ c | c c }
    & \multicolumn{2}{c}{\textbf{\tdata{pd} Token Access}}\\
\textbf{Client Proof}& \textbf{Global} & \textbf{Private to $p$}\\
\hline
    $\emptyset$ & plaintext & \\
    \privk{p} & plaintext & \\
    \dcapa{p} & plaintext & ciphertext \\
    \privk{p}, \dcapa{p} & plaintext & plaintext \\
\end{tabular}
\vspace{6pt}
\caption{Client access to disguise $d$'s tokens, depending on what proof the client can provide and the state of the tokens.}
\label{tab:access}
\end{table}

Table~\ref{tab:access} illustrates which clients can access a disguise $d$'s tokens, depending on
what the client presents as proof, and the state of the tokens.

\sys ensures that private tokens are properly secured such that only a client with private key
\privk{p} and capability \dcapa{p} can access tokens produced by disguise $d$ associated with $p$.
%
Furthermore, \sys ensures that only a client with capability \dcapa{p} can learn if a disguise $d$
applies to principal $p$.

%%%%%%%%%%%%%%%%%%%%%%%%%%%%%%%%%%%%%
\head{Storing \tpriv{p}{q} Tokens.}
\tpriv{p}{q} tokens are encrypted with \pubk{p} and the resulting ciphertexts stored in one array
corresponding to the original principal $p$. These token ciphertexts are not separated by disguise
as \tdata{pd} ciphertexts are, since \tpriv{p}{q} tokens grant the holder authorization to access
tokens from any disguise for pseudoprincipal $q$.

\sys (and equivalently an attacker) cannot learn which disguises applied to principal $p$ from
observing the array of \tpriv{p}{q} ciphertexts. However, \sys can learn how many disguises were
applied to---and in particular, decorrelated---$p$'s data, as well as the number of pseudoprincipals
in the system linked to $p$. This metadata is out of scope of our threat model.
\fi
