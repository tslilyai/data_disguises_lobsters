%%%%%%%%%%%%%%%%%%%%%%%%%%%%%%%%%%%%%%%%%%%%%%%%%%%%%%%%%%%%%%
\subsection{High-Level Disguise Specifications.} 
To use \sys's high-level disguising API, the application developer specifies disguises using a set
of predicated disguise operations that \sys knows how to perform.

Operations \op{d} of disguise $d$ take data objects as input and execute updates to application
data.  \sys automatically generates database change records when applying \op{d}. 

Developers describe which principal(s) an operation's generated database change record corresponds
to. For example, a database change record generated by removing comment may correspond to the
principal whose ID is referenced by the author column.  

Developers also specify an application-aware pseudoprincipal generation policy, namely how to
generate new user accounts in a manner that the application can handle (e.g., pseudoprincipals may
not have email addresses).

Finally, developers specify which principals are authorized to apply the disguise: enforcing access
control for disguising is left to the application.  
%\lyt{It seems most reasonable for the pplication to enforce AC for the disguising API the
%application exposes to the client?}

\vspace{6pt}\noindent
Operations come in three forms:
\begin{enumerate}
    \item Modify: change an attribute of the data object.
    \item Remove: delete the data object.
    \item Decorrelate: generate a \emph{pseudoprincipal} $q$, and rewrite the foreign key to original
        principal $p$ from the data object to instead point to $q$.  
\end{enumerate}

\noindent For each \op{d}, the application developer specifies:
\begin{itemize}
    \item An associated predicate over the application database that selects \op{d}'s input
        objects in a SQL-like fashion.
    \item The type of operation and its arguments (e.g., which attributes of the data object to
        modify).
    \item The corresponding principal(s) that should have the capability to access \op{d}'s
        generated disguise change or correlation record.
\end{itemize}
Each \tdiff{pd} contains the ID of the disguise, the associated principal $p$'s ID,
and a random nonce.
%
In addition to this data, a decorrelation \tdiff{pd} contains the created pseudoprincipal's
ID; a removal \tdiff{pd} contains the removed object's value; and a modification
\tdiff{pd} contains the old and new value of the modified object.

