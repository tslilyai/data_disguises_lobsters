%-------------------------------------------------------------------------------
\section{Related Work}
%-------------------------------------------------------------------------------
\label{sec:related}

%
Protecting personal data in web services is a long-standing research direction.
%
\textbf{Decentralized platforms} such as Solid~\cite{solid}, BSTORE~\cite{bstore},
Databox~\cite{databox} and others~\cite{diy, amber, oort, w5, blockstack}
seek to put data directly under user control.
%
These systems give users power to control access, and often delete or modify, their
own data, but they burden users with maintaining infrastructure, typically lack the
capacity for useful application features based on server-side compute, and they break
the current, ad-based application revenue model.
%
In contrast, data disguising helps application developers support privacy
transformations without changing the application data model or business model,
and allows today's server-side processing.
%

%
Other platforms attest that server-side processing respects
\textbf{user-defined policies} over their data, via cryptographic means~\cite{zeph}
or systems security mechanisms~\cite{riverbed}.
%
The application functionality is typically restricted, in terms of the feasible
operations (\eg additively homomorphic ones) and in whether data with different policies
can be combined.
%
Data disguising only provides assurances about data privacy after the application has
disguised the data, but in exchange allows unrestricted application functionality
without any overhead.
%

%
Other systems enforce developer-specified \textbf{visibility and access control policies}
based on information flow control (IFC)~\cite{static, jeeves, jif, hails, ifdb},
authorized views~\cite{oracle} or per-user views~\cite{multiverse}, and rewriting database
queries~\cite{qapla, sieve}.
%
Data disguising transforms the actual data stored, and, unlike these systems, protects
against full server compromise.
%


%
New privacy laws have led researchers to propose \textbf{database designs
for privacy compliance}.
%
These databases track the owner of data objects and erase them on user request, compliant
with \eg the GDPR.
%
They either modify the data layout~\cite{usershards} or use fine-grained information flow
tracking (IFC) to determine PII propagation and restrict access~\cite{schengendb}.
%
Data disguising can be employed for GDPR compliance, but supports broader privacy
transformations that go beyond deleting PII, and requires no ownership tracking or
fine-grained IFC.
%

%
\textbf{Systems for data deletion}, on the other hand, specifically focus on the
challenge of correctly removing a user's to provide them with post-deletion
privacy.
%
While cryptographically-guaranteed deletion is difficult and expensive~\cite{vanish},
practical systems that help developers get deletion right exist.
%
DELF~\cite{delf}, a framework for data deletion at Facebook helps developers write
correct data deletion code, and uses annotations on application edge and object types
to specify a deletion policy.
%
DELF focuses only on data deletion, while data disguising targets broader privacy
transformations, including decorrelation and anonymization.
%
Sypse~\cite{sypse} pseudonymizes user data and partitions personally identifying
information (PII) from other data.
%
Like data disguising, Sypse's design is application-oblivious, and leaves
the specification of what counts as PII to the application.
%
Instead of constantly maintaining data partitions (as Sypse does), data disguising
adapts the application database on disguising and stores encrypted data to
potentially later reveal it.
%

%
Our work shares some ideas with \textbf{encrypted storage systems}.
%
However, the deployment scenarios for these systems assume that all data is
encrypted, which imposes overheads on normal operation and requires application
functionality to live on the client.
%
Support for arbitrary and fully-oblivious operation over encrypted data would
be ideal, but while there is encouraging recent progress on search over encrypted
data and oblivious key-value stores~\cite{dory, snoopy}, the cost is too high for
practical web services.
%
However, data disguising has a similar threat model to CryptDB~\cite{cryptdb},
Mylar~\cite{mylar}, or Keybase~\cite{keybase}, and protects confidentiality of
data under disguise against similar server compromises.
%
Data disguising encrypts only users' data currently under disguise, but is
efficient and still protects inactive users' privacy.
%

%
%Finally, privacy-enhancing techniques like $k$-anonymity, $l$-diversity, and
%differential privacy~\cite{dataminingmodels, differential} can complement
%data disguising.
%%
%Disguises might add differential privacy noise to
%data left in the application DB, for example.
%


%---------------------------

\iffalse
Key ideas:
    pseudonymization
    data partitioning
    synthetic data

Similarities:
- privacy guarantees weaker than DP, no constraints on queries
- pseudonymization --> overwrite a portion or all of data for individual, difficult to reassociate
    only a small fraction of data?
    breaking up objects grouped by user identifier key
    breaking connections between tables (just like )

- try to ensure that database still useful even w/pseudonymization
    synthetic data (ghosts in \name)
- use of in-memory caches to avoid expensive joins (materialized views)

Differences
- design for \name puts data in hands of user, Sypse has a separate partitioned table (not
necessarily )
- Sypse has synthetic data generation instead of ghosts to replace data taken out
    - foreign keys point to random records
    - VS ghost records
    - not developer-specified? might lead to incorrect behavior?
- what are partitions?
    - limited access, the queries will be run against pseudonymized or synthetic data only
    - Detail Database: anything that isn't PII (pseudonymous)
        - perform analysis, data mining, etc. on these
    - PII Database: anything tied to specific individual (username, user identifiers)
        - replaces with surrogate keys
        - geological / biometric information
        - breaking up objects grouped by user identifier key (direct edge policies)
        - breaking connections between tables (non-direct edge policies)
    - backed up separately, different access control mechanisms
- assumes annotated schema, but doesn't actually describe it!!! just says requires analysis, or is
relatively straightforward
    -made automatically through a continuous analysis of the database schema and the data within it,
    visible to users
    - lack of schema discipline in practice (Database decay and how to avoid it.)
    - no understanding of multiple entities and how they relate to each other
- no support for resubscription

Partition strategies:
- split table into multiple tables, additional joins to reconnect different partitions, doubled updates, forgetting requires multiple
updates
- encrypt certain attributes, add as new attributes, encrypted keys stored in user->key mapping: joins limited
- us: we don't care about size of "PII" table --> this data is given to user

Other related work to cite:
- secure/encrypted databases(?)
- differential privacy
- Kraska SchengenDB: A Data Protection Database Proposal.
    - identify all personal data via tracking / relation graphs
    - tables and columns with PII declared w/schema notations
    - purposes restrict what PII you can access
    - sandboxing
    - also has object relation graph
    - we assume that data isn't copied, whereas they support data copies by tracking ownership
    beyond foreign key relationships
- GDPR by compliance: user shard model
- ScrambleDB: ScrambleDB: Oblivious (Chameleon) Pseudonymization-as-a-Service
- Sieve: A Middleware Approach to Scalable Access Control for DatabaseManagement Systems.

\subsection{Decoupling User Data from Web Applications}
Our proposed subscription paradigm contrasts with a model of strong privacy in web applications, in
which the user has complete data ownership (see Figure~\ref{fig:world}b). In proposals for this
model~\cite{solid, amber, w5, blockstack, bstore}, applications must compute on per-user, filesystem-esque
storage, \eg as JavaScript in users' browsers in Solid~\cite{solid}, Blockstack~\cite{blockstack} and
BStore~\cite{bstore}. Applications face the loss of the flexibility and ease of programming over SQL
databases with application-specific schemas, and users must decide on appropriate access control and
authentication policies. As a result, service-side computation and data sharing between users---a
large reason users use applications to begin with---becomes inefficient and impractical.
%
Furthermore, users are burdened with long-term data maintenance and storage: solutions that use PIKE
(\eg Blockstack~\cite{blockstack}) require users to maintain a
master private key, a cumbersome and fragile solution in which losing the key results in permanent
loss of all the user's data.
%
Finally, users must change the way they pay for services, as the current revenue model for
applications relies on access to user data.
\lyt{this might not be a bad thing, though}.

Under the subscription paradigm, applications do not need to access any data outside its own
servers, simplifying the secure storage of this data when users enter privacy-preserving mode.
Applications can operate as they do today, with the current web architecture's performance and
programming flexibility benefits. However, unlike today's model of data privacy in web applications,
users can freely reclaim their their data and privacy whenever they wish.

\subsection{Data Anonymization}

Prior research on k-anonymization, pseudonymization (more for static analysis purposes than used in
web applications).
%\url{https://amnesia.openaire.eu/index.html}
%\url{https://www.aclweb.org/anthology/2020.coling-main.32.pdf}

While psuedo/anonymization
techniques still allow for reidentification/reconstruction attacks), \name provides better
anonymization than prior approaches, which decorrelate only coarsely by associating remnants of data
with a global placeholder for all deleted users, or do not decorrelate at all (e.g., replacing
usernames with a pseudonym, as in the AOL dataset case~\lyt{TODO}%\cite{aol}).

\subsection{Differentially Private Queries}

PINQ, a data analysis platform created by McSherry~\cite{pinq}, provides formal differential privacy
guarantees for any query allowed by the platform.  Data analysts using PINQ can perform
transformations (Where, Union, GroupBy, and restricted Join operations) on the underlying data
records prior to extracting information via aggregations (e.g., counts, sums, etc.) The aggregation
results include added noise to meet the given privacy budget $\epsilon$, ensuring that analysts only
ever receive $\epsilon$-differentially-private results.  PINQ calculates the privacy loss of any
given query based on transformations and aggregations to be performed; if a particular query exceeds
a predefined privacy budget, PINQ refuses to execute the query.

Differential privacy provides a formal framework that defines the privacy loss a user incurs
when the user's data is included in the dataset. However, the setting of differential privacy
that of \name in several important ways.

First, web applications' utility often derives from visibility into individual data records. PINQ
restricts JOIN transformation and exposes information only through differentially-private
aggregation mechanisms. While sufficient for many data analyses, this approach severely hinders web
application functionality. \name supports all application queries, even ones that expose individual
records, such as \texttt{SELECT story.content, story.tag FROM stories LIMIT 10}.
However, this makes formally defining the privacy guarantees of \name more complex than
simply applying DP.  DP provides formal guarantees for results such as sums or averages because such
results are computed using well-defined mathematics, allowing us to neatly capture the total
knowledge gained by the adversary. However, in the world of web applications, the knowledge gained
via queries that may reveal individual data records cannot be so easily defined: adversaries may
learn information from the friends who liked the (ghosted) story, or the time and location the story
was posted.

Thus, we cannot, in general, use the DP approach of asking, ``by how much do answers to adversary
queries change with the presence of a users' decorrelated data?'' While we can precisely define which
data records an adversary sees with and without a user's decorrelated data, we cannot numerically
quantify the knowledge gained by the adversary in the same way as we can quantify the percentage
change to a sum or average.
Instead, \name reduces identifying information in the changes induced by a users' (decorrelated)
data, rather than reducing the amount of change itself.

This is demonstrated by how PINQ and \name differ in disguising entities associated with unique keys.
PINQ restricts JOINs to group by unique keys. As an (imperfect) example, selecting a JOIN of stories
with users via the UID would result in a record, one per UID, which represents the ``group'' of all
stories associated with a user. A normal join would create a separate record per story, each
associated with the user of that UID.

Instead of lumping all matching stories together, \name explodes the users' UID into individual
unique keys (ghosts), one per story. By PINQ's analysis of stable transformations, this leads to
unbounded stability (as the number of different results produced by queries is proportional to the
potentially very large number of stories for that user). In PINQ's DP framework, such unbounded
stability leads to unbounded privacy loss: one users' data could change aggregation results in
``significant'' ways unless huge amounts of noise were added.

\name and PINQ's approaches can be seen as complementary: for queries that are more analytic in nature (for example,
population statistics used by the web application to measure usage or trends), \name can utilize
PINQ's technique to provide differentially private guarantees (for a particular snapshot of the
dataset).  \name's decorrelation, in fact, adds more noise than necessary in these cases: a count
of unique users who posted about cats will be more imprecise because each post by a decorrelated
user is now associated with a unique (ghost) user.

\subsection{Deletion Privacy}
The right to be forgotten has also been formally defined by Garg et
al.~\cite{garg}, where correct deletion corresponds to the notion of
leave-no-trace: the state of the data collection system after a user requests to be forgotten should
be left (nearly) indistinguishable from that where the user never existed to begin with. While
\name uses a similar comparison, Garg et al.'s formalization assumes that users operate
independently, and that the centralized data collector prevents one user's data from influencing
another's.

Potentially other related works:
\begin{itemize}
    \item Deceptive Deletions for protecting withdrawn posts %https://arxiv.org/abs/2005.14113
    %\item "My Friend Wanted to Talk About It and I Didn't": Understanding Perceptions of
        %Deletion Privacy in Social Platforms, user survey https://arxiv.org/pdf/2008.11317.pdf;
        %talk about decoy deletion, prescheduled deletion strategies~\cite{myfw}
    \item Contextual Integrity
    \item ML Unlearning
    \item Database IFC (sensitive data flowing down edges?)
    \item Graph databases
\end{itemize}
\fi
