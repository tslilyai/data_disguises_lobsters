\section{Data Mask Goals \& Challenges}

Data masks all have similar goals, namely to (1) selectively retain data for application
utility or legal purposes; and (2) and modify, delete and/or decorrelate sensitive data, all while
maintaining correct application semantics.

\paragraph{Selective Retention.}
Retained data serves two purposes: to preserve legal or necessary information about an individual
user (\eg order history in e-commerce applications), and to maintains application utility for
subscribed users.  For example, Facebook and Twitter retain copies of private messages for other users in
the conversation~\cite{facebook:privacy, twitter:privacy}; Reddit retains all user posts and comments;
Lobsters retains the user's unique username and logs of the user's actions~\cite{reddit:privacy};
Amazon and Prestashop retain a user's order history~\cite{amazon:privacy, prestashop:privacy}; and
Strava retains geolocation data and any ``necessary'' information~\cite{strava:privacy}. 
Developers must ensure that masked state avoids accidentally deleting or retaining data, and convey
to the user exactly what data the mask leaves behind.

\paragraph{Data Deletion and Decorrelation.}
Masks both remove and modify individual data records, and break up \emph{structural correlations}
from or to sensitive data, which can also leak sensitive information. For example, orders tagged
with the same location likely belong to the same user, and order history can identify the user. 

Ensuring these modifications do not break application semantics, and balancing data removal with
data retention, requires developers to reason about complex data relationships and application
invariants. For example, developers implementing unsubscription in HotCRP may want to handle
potentially identifying correlations between users' papers and conflicting reviewers, which can
identify the user via the conflicts' affiliations and recent collaborations However, developers
cannot simply remove all conflicting reviewers with an unsubscribed user's paper: this incorrectly
allows paper reviews to be assigned with potential conflicts. Careful selection of which data to
retain while ensuring that sensitive data is masked through modification or removal, all without
breaking application semantics, requires application-specific expertise and a thorough understanding
of exactly how the data mask changes state.

To avoid the complexity of implementing fine-grained masks that both retain and modify data, developers often revert to coarse-grained masks (\eg delete everything). For example, HotCRP chooses to reduce utility
for other users and delete all papers correlated with unsubscribing users, rather than deal with
paper conflicts.
%
Furthermore, developers implement masks using ad-hoc methods, scattering data modifications throughout
application code. This leads to a lack of a clear, centralized specification of how the mask modifies
sensitive information and results in \eg vague privacy policies.

%Retained data should mask any sensitive data contents (\eg user identity or harmful information). In many cases, however, simply modifying data contents (\eg simply removing or replacing
%identifiers with an anonymized placeholder for de-identification), fails to sufficiently guarantee
%that the data is hidden: \emph{structural correlations} from or to retained data can still leak sensitive
%information. For example, orders tagged with the same location likely belong to the same user, and
%order history can identify the user. 
%Anonymized posts with the same flairs or custom tags can identify which user made the posts.
%, a solution that clearly would not work in a world where users frequently unsubscribe at any time.

\iffalse
To achieve \name, a system must (1) enable applications to \emph{selectively retain} unsubscribed
users' data for necessary legal or application purposes; (2) properly \emph{de-identify} the user
upon unsubscription by removing any possibly identifiable information, or decorrelating this
information from the user's identity; and (3) allow users to \emph{resubscribe} without losing their
account history or data, so unsubscription has no permanent consequences. We next describe each
requirement and the challenges to achieve it in greater detail.

\paragraph{Selective Retention.}
Retained data serves two purposes: to preserve legal or
necessary information about an individual user (\eg order history in e-commerce applications), and
to maintains application utility for subscribed users (\eg private messages
for thread coherence, or crowdsourced reviews and popular news posts).
Developers must ensure that post-unsubscription state avoids accidental data deletion or retention.
%Developers should easily determine what data to retain %post-unsubscription, and 

\paragraph{De-Identification.}
Any retained data must be properly de-identified. The surveyed applications often remove or replace
user identifiers with a placeholder user identifier. However, this can still fail to sufficiently
de-identify retained data: \emph{structural correlations} from or to retained data can still identify
unsubscribed users. For example, orders tagged with the same location likely belong to the same
user, and order history can identify the user. 
%Anonymized posts with the same flairs or custom tags can identify which user made the posts.

Detecting and correctly modifying even subtly problematic content and correlations pose technical
challenges that can prevent developers from retaining data in privacy-preserving ways, or force
developers to forgo data retention completely.  For example, HotCRP cannot just remove unsubscribing
users' identifiers: correlations between users' papers and conflicting reviewers indirectly identify
the user via the conflicts' affiliations and recent collaborations.  HotCRP chooses to reduce its
utility by completely removing the papers, rather than deal with these potentially identifying
correlations.
%, a solution that clearly would not
%work in a world where users frequently unsubscribe at any time.

These complex data transformations also make it difficult for developers to specify exactly 
what identifying information may be left behind for the user, leading to vague privacy policies.

\paragraph{Resubscription.}
Resubscription must recorrelate any retained data with the original owner, and restore any removed
data. Today, applications either retain all data and forgo de-identification entirely in order to
support resubscription, or permanently remove and anonymize data.
Supporting resubscription requires storing information necessary for recorrelation and restoring
users' accounts, but should not unduly burden either the application or user, nor add any
identifying information about unsubscribing users to the system.

%HotCRP
%users---particularly reviewers---would prefer that their blinded reviews never become publically
%associated with their identity in the case of a data compromise. 
%

%Reviewers and authors create user accounts to
%ubmit or review papers, and HotCRP supports program committee conflict detection, paper rebuttals
%nd comments, review delegation, and other features necessary for the conference review process.
%HotCRP's current privacy policy~\cite{hotcrp:privacy} allows site managers (e.g.,\ program chairs)
%to indefinitely store and distribute submissions and reviews.  This is by request of its users:
%users want to be able to reference prior conference reviews and paper metadata.  

%Currently, the only way to ensure this is to contact the HotCRP maintainers directly to manually
%remove their user profile: the HotCRP developers considered automating support for unsubscription
%and found it too cumbersome to implement. 
%
%\lyt{(This isn't exactly what I want to say---sorry to put words in your mouth, Eddie!---but something like this is
%probably useful.)} 
%
%When user accounts are removed, every piece of data ever associated with that
%user is deleted from the system: this model clearly would not work if all users freely
%unsubscribed at will, as the system would be left with no useful data for subscribed users.
%
%The desired HotCRP unsubscription behavior would both retain useful data for HotCRP and its users,
%and satisfy the desire of HotCRP users to de-identify their data. In particular, the HotCRP
%developers should be able to specify that unsubscription deletes only the user profile, and
%\emph{decorrelates} the links between the user and their reviews (and other user-associated data)
%while retaining the blinded reviews and data for other users to view. While correlations between
%users and their reviews are obviously identifying, unsubscription must also handle more subtle
%correlations: a user's reviewer conflicts can identify the user's affiliation and recent
%collaborations, and in many cases, the user themselves. Implementing unsubscription that handles
%these complex data relationships correctly would not be trivial.
%
\fi
