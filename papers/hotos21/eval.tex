%-------------------------------------------------------------------------------
\section{Case Studies}
%-------------------------------------------------------------------------------
\label{sec:hotcrp_example}

\begin{figure}[t]
    \centering
    %\footnotesize
    \begin{tabular}{@{}cccc@{}}
        \textbf{Application} & \textbf{\#Object} & \textbf{Schema} &
        \textbf{Disguise} \\
        \textbf{Disguise} & \textbf{Types} & \textbf{LoC} & \textbf{LoC} \\
    \midrule
    \lrtbf & 19 & 318 & 100 \\
    \hrtbf & 25 & 352 & 142 \\
    \hrtbfplus & 25 & 352 & 255 \\
    \hconfanon & 25 & 352 & 232 \\
\end{tabular}
    \caption{Data disguise specifications for Lobsters and HotCRP have similar complexity to
    a relational schema.
}
\label{tab:loc}
\end{figure}

%
We evaluate the ease of writing single disguises by writing disguises for GDPR deletion in
Lobsters~\cite{lobsters}, an open-source news feed application, and HotCRP~\cite{hotcrp}.
%
We consider four disguises: \lrtbf and \hrtbf implement the current account
deletion policies in the two applications~\cite{lobsters:privacy, hotcrp:privacy}.
%
\hrtbfplus specifies a HotCRP account deletion policy that balances useful data retention with
data deletion for privacy (\S\ref{design:eg}).
%
Finally, \hconfanon specifies the conference anonymization disguise for HotCRP.

\paragraph{Complexity.}
%
We would hope that writing disguises involves similar labor and difficulty as writing
relational schemas.
%
In particular, a developer should write a disguise only once.
%
Because disguises specify a set number of transformations for each object type,
disguise complexity is limited by the number of object types and relations in the
application schema.
%
Figure~\ref{tab:loc} shows that the disguise specification for our applications is indeed
comparable in size to the applications' schemas.
%

\paragraph{Performance.}
\label{sec:perf}

Data disguising faces several performance challenges.
%
First, disguise application should not unduly impact the performance of normal application queries.
\sys currently runs a disguise in one large SQL transaction, and performs transformation queries
sequentially; some batching and parallelization is possible and could improve this performance.
Furthermore, eventually consistent disguises would remove the expensive nature of a single large
transaction.  The importance of reducing the cost of disguise application depends on the rate of
disguising, which may range from rare (as in today's applications) to quite frequent (in a
privacy-supporting world where users freely disguise and reveal themselves, or where data
automatically ages out).

%
\sys must also be careful to do as little work as possible to handle disguise interdependencies.
When executing a disguise, \sys not only modifies the relevant objects, but must also
log all updates in the vault, and potentially read and reverse prior vault entries.
%
As expected, the number of queries performed by \sys to fetch and update the relevant to-be-disguised objects
grows linearly with the number of objects. This disguise overhead is unavoidable: these
modifications are crucial to ensuring proper disguise application.
However, \sys could potentially need to read from, reverse, and reapply all entries in the vault
that correspond to all previous disguises.

We evaluate the estimated cost of vault operations and disguise composition with a simple
comparison. We first invoke \sys with two independent disguises, namely \hrtbf of different
users. We then compare the cost of invoking \hrtbf after a conflicting disguise, \hconfanon, has
already been applied.
The disguises are applied to a database with 430 users (30 PC members), 450 papers, and 1400
reviews.
The unoptimized application of \hrtbf of a PC member after an independent \hrtbf application for a
   different user requires 135ms on average;
the same \hrtbf disguise applied after \hconfanon requires takes 118ms (\hconfanon itself takes
   7000ms). 

\sys avoids unnecessarily redoing decorrelation updates applied by \hconfanon when applying \hrtbf;
however, \sys
incurs overhead from temporarily recorrelating objects using vault data, so these
objects can then be correctly removed. 
While \sys's selective reintroduction of data from user vaults is not free, it is noticeable
   cheaper than completely undoing prior disguises.
\sys currently relies on manual input to analysize which operations must be selectively undone and
(not) reapplied, and we imagine that we will be able to use static analysis to replace the manual annotations.
