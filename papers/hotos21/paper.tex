\documentclass[sigconf,natbib=false,10pt]{acmart}

%% Rights management information.  This information is sent to you
%% when you complete the rights form.  These commands have SAMPLE
%% values in them; it is your responsibility as an author to replace
%% the commands and values with those provided to you when you
%% complete the rights form.
\setcopyright{acmcopyright}
\copyrightyear{2021}
\acmYear{2021}
%\acmDOI{10.1145/1122445.1122456}

%% These commands are for a PROCEEDINGS abstract or paper.
\acmConference[HotOS '21]{HotOS '21: 18th Workshop on Hot Topics in Operating Systems}{May 31--June 2, 2021}{Virtual Conference}
%^\acmBooktitle{Woodstock '18: ACM Symposium on Neural Gaze Detection,
%  June 03--05, 2018, Woodstock, NY}
%\acmPrice{15.00}
%\acmISBN{978-1-4503-XXXX-X/18/06}

%%\acmSubmissionID{123-A56-BU3}


% to be able to draw some self-contained figs
\usepackage{tikz}
\usepackage{amsmath}
\usepackage{hyperref}
\usepackage[normalem]{ulem}
\usepackage{listings, listings-rust}
\usepackage{xspace}
\usepackage{booktabs}
\usepackage{multirow}
\usepackage{wasysym}
\usepackage{subcaption}
\usepackage{enumitem}
\usepackage[utf8]{inputenc}
\usepackage[compact, small]{titlesec}

% BibLaTeX for bibliography
\usepackage[
  backend=biber,
  style=numeric-comp,
  minalphanames=3,
  isbn=false,
  sortcites=true,
  sorting=anyt,
  abbreviate=false,
  url=false,
  doi=false,
  maxnames=99,
  minbibnames=3,
  maxbibnames=99]{biblatex}
\addbibresource{paper.bib}

%\newcommand{\sys}{Couturier\xspace}
%\newcommand{\sys}{Fashionista\xspace}%Sidekick
\newcommand{\sys}{Edna\xspace}
\newcommand{\lrtbf}{\texttt{Lobsters-RTBF}\xspace}
\newcommand{\hrtbf}{\texttt{HotCRP-RTBF}\xspace}
\newcommand{\hrtbfplus}{\texttt{HotCRP-RTBF+}\xspace}
\newcommand{\hconfanon}{\texttt{HotCRP-ConfAnon}\xspace}
\newcommand{\thing}{guise\xspace}
\newcommand{\things}{guises\xspace}
\newcommand{\Things}{Guise\xspace}
\newcommand{\Thing}{Guises\xspace}

\newcommand\ms[1]{\textcolor{red!55!blue}{[malte: {#1}]}}
\newcommand\lyt[1]{\textcolor{green!55!blue}{[lyt: {#1}]}}
\newcommand\eddie[1]{\textcolor{blue!55!blue}{[E: {#1}]}}
\newcommand{\tabitem}{~~\llap{\textbullet}~~}
\newcommand{\eg}{{e.g.},\xspace}
\newcommand{\ie}{{i.e.},\xspace}


\definecolor{codegreen}{rgb}{0,0.4,0}
\definecolor{codegray}{rgb}{0.5,0.5,0.5}
\definecolor{codepurple}{rgb}{0.58,0,0.82}
\definecolor{backcolour}{rgb}{0.95,0.95,0.92}

\lstdefinestyle{rust}{
    %backgroundcolor=\color{backcolour},
    commentstyle=\color{codegreen},
    keywordstyle=\color{codepurple},
    stringstyle=\color{blue},
    basicstyle=\ttfamily\scriptsize,
    breakatwhitespace=false,
    breaklines=true,
    captionpos=b,
    keepspaces=true,
    showspaces=false,
    showstringspaces=false,
    showtabs=false,
    tabsize=2
}
\lstset{style=rust}

%-------------------------------------------------------------------------------
\begin{document}
%-------------------------------------------------------------------------------

%don't want date printed
\date{}

%%
%% The "title" command has an optional parameter,
%% allowing the author to define a "short title" to be used in page headers.
% make title bold and 14 pt font (Latex default is non-bold, 16 pt)
\title{Privacy Heroes Need Data Disguises}
%\name-ing Your Data from the Internet }
%An Invisibility Cloak for the Internet}

%%
%% The "author" command and its associated commands are used to define
%% the authors and their affiliations.
%% Of note is the shared affiliation of the first two authors, and the
%% "authornote" and "authornotemark" commands
%% used to denote shared contribution to the research.
\author{Lillian Tsai}
\affiliation{%
  \institution{MIT CSAIL}
%\email{tslilyai@mit.edu}
%\affiliation{%
%  \institution{MIT}
%  \state{}
  \country{}
}
\author{Malte Schwarzkopf}
\affiliation{%
  \institution{Brown University}
%\email{malte@cs.brown.edu}
%\affiliation{%
%  \institution{Brown University}
%  \state{RI}
  \country{}
}
\author{Eddie Kohler}
\affiliation{%
  \institution{Harvard University}
%\email{kohler@seas.harvard.edu}
%\affiliation{%
%  \institution{Harvard University}
%  \state{MA}
  \country{}
}

\begin{abstract}
    The paper proposes a new web application paradigm in which users subscribe to applications by
    granting a time-limited ``lease'' of their data, instead of having a permanent account. Users
    flexibly switch between a privacy-preserving unsubscribed mode and an identity-revealing
    subscribed mode at any time without permanently losing their data. This subscription paradigm
    strikes a balance between one extreme in which applications have complete ownership of user
    data, and another extreme in which users own and store their data at the cost of application
    utility. 

%
    Achieving this subscription paradigm is difficult. Application developers must balance
    between retaining enough information so the application correctly and usefully functions, and
    removing enough information to prevent disclosing the identity of unsubscribed users via
    inference attacks. To begin to solve these challenges, we design \sys{}, a new system that
    automates privacy-compliant unsubscription and resubscription of users in database-backed web
    applications.
\end{abstract}


\maketitle
%\vspace{-2\baselineskip}

%-------------------------------------------------------------------------------
\section{Introduction}
%-------------------------------------------------------------------------------

\subsection{Motivation} 

Web application companies face increasing legal requirements to protect users’ data. These
requirements pressure companies to properly delete and anonymize users' data when a user requests to
\emph{unsubscribe} from the service (i.e.,\ revoke access to their personal data).
For example, the GDPR requires that any user data remaining after a user unsubscribes is
\emph{decorrelated}, i.e., cannot be (directly or indirectly) used to identify the user~\cite{gdpr}.  

In this paper, we propose \sys, a new approach to managing user identities in web applications.
\sys~meets the decorrelation requirements in the GDPR, and goes beyond: with \sys, it is possible
for users to switch between a privacy-preserving unsubscribed mode and an identity-revealing
subscribed mode at any time. This facilitates important new web service paradigms, such as users
granting a time-limited "lease" of data to a service instead of having a permanent service account.

%Furthermore, keeping identifying and personal data when no longer strictly necessary increases
%companies' liability: the GDPR and other laws mandate that companies retain only user data that is
%relevant and necessary for their applications' purposes. To increase users' control over their data
%and decrease the amount of incriminating data stored in the application at any one point, users
%should be able to freely unsubscribe from the service to enter a privacy-preserving mode, and later
%resubscribe when they wish to use the service. 

\subsection{Goals} 
\sys's goal is to provide the following properties while preserving an application's 
semantics: 
\begin{description} 
    \item[Decorrelation.] Decorrelation ideally guarantees that it is impossible to distinguish
        between two records formerly associated with the same unsubscribed user and two records from
        different unsubscribed users.  
    %\item Deletion Correctness: Deletion of user records should correctly conform to application
        %semantics: for example, post deletion could remove the post and its underlying comments, or
    %simply anonymize the post and keep all content accessible.  
    \item[Resubscription.] Users should be able to easily switch between a privacy-preserving unsubscribed mode 
       and an identity-revealing subscribed mode, without permanently losing their application data.  
\end{description}

\sys~must implement these properties while ensuring (1) performance comparable to today’s
widely-used databases, and (2) easy adoption (decorrelation should be
automated without needing to modify application schemas or semantics).

\subsection{Decorrelation Guarantees} 
We address applications in which application data consists of \emph{data records} and computations
(such as aggregations) that may be performed over these data records. Data records are considered
sensitive, private data records when they contain \emph{user identifiers}; for example, a row in a
table containing a column of user IDs would be a user's private data record. Data records containing
multiple, potentially different, user IDs are considered shared data records private to the
identified users. We refer to data records belonging to unsubscribed users as \emph{remnants}.

Perfect decorrelation is achieved when queries to the application reveal no information allowing an
observer to determine if two distinct user data records belong to the same user who has since
unsubscribed (are \emph{linked}), or belong to two different (real or unsubscribed) users. Observers
gain no information that allows them to distinguish the two scenarios. 

More formally, given user data records and a subset of $R$ of these records that are remnants, we
divide the $R$ remnants among some number $N$ of unsubscribed users.  Perfect decorrelation
guarantees that any division of the $R$ remnants among $N$ users is equally likely: each remnant is
equally likely to be linked with any other remnant. 

~\lyt{I toyed with saying that $P(N = R) = P(N \le R)$: the probability that every distinct remnant
belongs to a distinct user equals the probability that any user owns multiple remnants, but this
seemed unnecessarily complex when I believe that the above statement captures decorrelation.} 

~\lyt{Note: whether $N$ is known or not also plays into the probability analysis}

~\lyt{Note: this is the definition in which ghosts can be distinct from real users, so we're not
trying to pretend that remnants cannot be identified}

In practical settings, however, perfect decorrelation is likely impossible: for example, a reposted
screenshot may leak user IDs. Furthermore, an observer who can see application queries over time, or
search web archives, can detect when a user unsubscribes and refer to prior snapshots in which user
data records may have had the same user ID.  Given these limitations, we seek to achieve the maximum
decorrelation guarantees possible while assuming that the content of user data does not itself leak
identifying information, and that observers cannot access past application database
state.~\lyt{Note: UIDs can also include things like email addrs, phone number, etc; so the notion of
``identifiers'' might be more broad than stated here}.

\paragraph{Global Placeholder.}
A common strawman solution to decorrelation is to replace all unsubscribed user IDs with one global
placeholder ID. This means that all remnants can be identified by one distinct UID, which we call
$GP$.

A global placeholder can provide either complete privacy or no privacy at all. We first begin by
assuming an observer has complete information about how many users are in the system ($N$).
\begin{itemize}
    \item If $N=1$ and only one user has unsubscribed, all remnants can be linked back to that user's
        identity: any data record with UID equal to $GP$ belongs to that user.
    \item If all users unsubscribe, two remnants are equally as likely to belong
        to any two of the $N$ individual users as they are to belong to one single user. 
\end{itemize}
The amount of linkable information~\lyt{(need to define this)}, and the resulting deviance from
perfect decorrelation, decreases as more users unsubscribe.
Let us assume that an observer knows $N$ and $R$, and the number of data records per user is
uniformly distributed. Let $r = \frac{R}{N}$ be the number of remnants per unsubscribed user. 
Then an observer has probability $p_{\text{exact}}$ of guessing the correct combination of
records per users, where  
$$p_{\text{exact}} = \frac{(r!)^N}{R!}$$
An observer has probability $p_{\text{user}}$ of guessing the correct combination for one user where 
$$p_{\text{user}} = \frac{r!(R-r)!}{R!}$$
\lyt{TODO---non-uniform distributions, not knowing $N$, not knowing $R$?}

Using a global placeholder, however, makes resubscription challenging: users can no identify which
unsubscribed data records belong to them if all user-specific data has been erased from the
system. 

\paragraph{Data-Record Ghost IDs.}
To support \lyt{the same?}\sout{stronger} decorrelation guarantees while supporting resubscription,
\sys~generates a unique ghost user for each data remnant. 
Unlike a global placeholder, \sys~allows users to reactivate their account and undo
the decorrelation: user IDs can be linked back to a set of unique ghost IDs. 
This gives users the ability to freely unsubscribe to protect their privacy
without worrying about losing their accounts. \sys~resubscribes users by transparently propagating
updates to materialized views to expose real user identifiers in place of ghost identifiers. 

\sys~relies on coarse-grained schema annotations to establish which associations to decorrelate, and
builds a dataflow computation resulting in materialized views that answer application queries. Use
of dataflow automatically propagates the correct updates to materialized views.

\sout{These ghosts, unlike a global placeholder,
are indistinguishable from a real user in the system, ensuring that queries cannot correlate two
ghosts with a single real (albeit unsubscribed) user.
However, some linkable information is still leaked. For example, ghost users may be randomly
generated in a pattern identifiable by an observer (e.g.,\ if all ghosts have usernames which are
random numbers, or arbitrary animals, but a real user may have more human-friendly usernames).}

\lyt{Not sure how to incorporate DP or some notion of noninterference here (how do \sys's responses to
queries deviate from what an ideally decorrelated system would produce?)
DP seems to deal with the change in probability of some event X, rather than the initial probability
to begin with.
Perhaps DP would be more applicable if we were looking at the database over time? And if we had some
notion of noise. Without adding noise, the output would change dramatically when a user unsubscribes
and UIDs are replaced by GIDs, which reveals complete linkability information.}

%-------------------------------------------------------------------------------
\subsection{A Data Disguising Tool}
\label{sec:tool}
%-------------------------------------------------------------------------------

A data disguising tool handles the complexity behind disguise composition, applying disguises in
sequence and generating the necessary storage operations to achieve an acceptable end state.  When
applying a disguise, a disguising tool must both not only apply the specified disguise
transformations, but also manage codependencies between the new disguise and any prior disguises. 

\lyt{New text added!}
Because disguises inherently destroy data, applying one disguise limits the transfomational
capabilities of future disguises. However, applications would benefit from the ability to apply
multiple disguises. For example, \S\ref{sec:motivation} highlights two desirable disguises in
HotCRP, namely \gdpr and \ca. \gdpr provides user privacy by removing the user's data; \ca provides
user privacy by anonymization.  As we observed, these disguises share dependencies: applying \ca
destroys information needed to remove the user's data when applying \gdpr.

The tool can overcome some of these limitations if applications optionally allow disguise reversibility,
which the tool supports using per-user vaults. These vaults store per-user revealing functions for
reversible disguises, and enable the tool to support a larger set of privacy policies
(\S\ref{sec:composition}).

%To handle inter-disguise dependencies, a disguising tool relies on (1) the structured nature of disguises to
%statically determine whether disguises share dependencies, and (2) the key abstraction of \emph{user
%vaults}, namely per-user logs of disguise updates to that user's data.  User vaults solve the issue
%that disguises inherently destroy data necessary to correctly achieve the end-state of future
%disguises by providing a secure way to store the data. A disguising tool queries the user vault to
%temporarily restore destroyed data (\eg decorrelated foreign key relationships) in order to apply
%the disguise correctly.

%As shown in Figure~\ref{fig:tool}, a disguising tool sits next to the application, and queries the
%user vaults and the application database. The application performs disguises by invoking a
%disguising tool.

%-------------------------------------------------------------------------------
\paragraph{Deploying User Vaults.}
%-------------------------------------------------------------------------------
User vaults can be flexibly configured and deployed. It remains important, however, that any
configuration of user vaults should not violate the guarantee that disguises indeed destroy data,
from the viewpoint of the application and any users.
We imagine several exciting directions to explore for designing vaults that are both
performant and secure.
%

%
Some possible configurations include a disguising tool storing vaults encrypted with a per-user key; this key
may be secret-shared using a (2, 3) threshold scheme~\cite{secretsharing} between the user, the
tool, and a trusted third party (\eg Amazon S3), so that the user can authorize a disguising tool and the
third party to restore the key if the user forgets their share.
%
The vault entries could be configured to expire after a certain time; the corresponding disguises
then become irreversible.
%
Or perhaps the vault entries are stored entirely by some third party or locally by the user, whose
server runs a corresponding interface to allow disguise tools to read and write the vault.

\lyt{Added text:}
The choice of vault deployment has serious consequences for the practicality of disguise reversal.
For example, if \gdpr is to be reversable, reveal functions must necessarily be stored in user vaults external
to the application in order to be GDPR-compliant.
While it is reasonable to imagine accessing a single user's vault to reverse \gdpr in this
deployment, complete reversal of \ca would need to retrieve reveal functions from every user's vault, an infeasible task.
%A disguising tool should, with user permission, be able to read and write the vaults: a user invoking GDPR
%\texttt{GDPR} could provide the key to their vault, and the application invoking the
%\texttt{ConfAnon} might notify each user for approval (and temporary vault access) prior to disguise
%application. 
An alternative might be a multi-tier security design: the first tier stores reveal functions of
non-GDPR disguises locally and acccessible to the disguising tool and application, while the second
tier stores reveal functions from user-invoked disguises in external, per-user encrypted storage.
%\lyt{Might want to say more about how often user vaults are
%queried---what if one disguise needs access to many user vaults, but is being done on behalf of a
%single user?}

%-------------------------------------------------------------------------------
\paragraph{Applying Disguises.}
%-------------------------------------------------------------------------------
A disguising tool applies disguises in a five-phase procedure:
\begin{enumerate}[nosep]
    \item \emph{Prepare}: execute the appropriate reveal functions of co-dependent,
        reversible disguises from the user vaults, if applicable (\S\ref{sec:composition})
        %reconcile any data dependencies between this disguise and prior disguises.
        %A disguising tool detects read-after-write dependencies between the new disguise's
        %predicates and prior disguises' updates, and, using entries in the vault, undoes any writes
        %that may affect the new disguise's predicates. As an optimization, vault entries recording
        %object removals need not be reversed.
        \item \emph{Read}: get all objects that satisfy (per-type) developer-specified predicates.
        \item \emph{Update}: modify, decorrelate, or remove objects read in step (2) according to the
        developer's specification.
    \item \emph{Record}: if the disguise is reversible, store per-user reveal functions for the
        disguise in the appropriate per-user vaults. 
        A disguising tool must be able to determine which user vault should record each modification. This can be
        developer-specified, or rely on a set of heuristics (\eg assigning ownership by traversing,
        starting from each user, the application's object graph expressed in an object-relational
        model (ORM)~\cite{orm}, or implicitly via foreign keys).
        \item \emph{Finalize}: After applying the new disguise updates, the disguising tool
            redisguises any temporarily revealed data from earlier disguises.
\end{enumerate}

%Composed disguises should achieve an end-state that combines, in some way, the end-states achieved by each disguise when applied to the original application database in isolation.
%Correct composition of multiple disguises achieves an end-state equivalent to combining the
%end-states achieved by each disguise when applied to the original application database in isolation.
%
%If a prior disguise is reversible, then a disguising tool can use user vaults to ensure that this
%prior disguise does not affect \emph{which} objects are updated
%a future disguises. 
%%In this case, a disguising tool allows developers to reason about multiple conflicting updates to
%the same object: 
%regardless of when the disguises occurred, if one disguise removes an object that the other disguise
%modified, then the removal takes precedence.
%%
%
%However, if they both modify the same object attribute, a disguising tool establishes no precedence
%between the modifications and applies them in chronological order.  Alternatively, we can imagine
%that the developer could specify a partial ordering between modifications, or our framework could
%restrict the set of possible modifications and establish a precedence order within this set.
%
%If the prior disguise is not reversible, however, then the disguising tool could prevent future
%conflicting disguise application, or perhaps a-priori prevent the application of such non-reversible
%disguises that conflict with legally required disguises such as GDPR deletion \lyt{not sure what to
%put here? May also want to include something about developer assertions}
%%: a disguise will update all objects that it would have updated if performed on
%the original, undisguised state of application data.

%-------------------------------------------------------------------------------
\section{Data Disguises}
%-------------------------------------------------------------------------------
\begin{figure}[t!]
    \centering
    \includegraphics[width=0.5\textwidth]{img/disguises}

    \caption{Disguises move the target object (in this example, a user Bob) from an identity-revealing
    guise to privacy-preserving guises. \lyt{TODO need to change from reveal?}}
    \label{fig:example}
\end{figure}

%\lyt{TODO (Frans): drive w/explicit example of table rows and foreign key relationships, too
%abstract. Start with Figure~\ref{fig:arch}?}

\begin{figure*}[t!]
    \centering
    \footnotesize
\begin{tabular}{@{}c|c|c|c@{}}
\textbf{User Transformation Spec} & \textbf{User Object} & \textbf{Guise 1} &
    \textbf{Guise 2} \\
\begin{lstlisting}[language=Rust]
"id":       IDAttribute,
"name":     Gen(Random),
"active":   Gen(Default(false)),
"darkmode": CopyAll,
"notifs":   CopyOnce+Gen(Default(false)),
"tag_id":   GenForeignKey,
\end{lstlisting}
    &
\begin{lstlisting}[language=Rust]
"id":       19,
"name":     BobParr,
"active":   true,
"darkmode": false,
"notifs":   true,
"tag_id":   11
\end{lstlisting}
&
\begin{lstlisting}[language=Rust]
"id":       295,
"name":     MrIncredible,
"active":   false,
"darkmode": false,
"notifs":   true,
"tag_id":   81483
\end{lstlisting}
&
\begin{lstlisting}[language=Rust]
"id":       918,
"name":     SuperDad,
"active":   false,
"darkmode": false,
"notifs":   false,
"tag_id":   15592
\end{lstlisting}
\end{tabular}
    \caption{Creating two guises of an example user (of a synthetic application schema).}
    \label{fig:guises}
\end{figure*}

%
The key idea behind \emph{data disguising} is to associate multiple \emph{guises} with a target data
object (\ie a row in a relational database). Guises vary in how they reveal identities or preserve privacy.
%
Objects move between different guises by means of disguises---a set of constrained privacy
transformations.

Figure~\ref{fig:example} illustrates this with the example of user account deletion.
%
When his account is active, user Bob's profile is associated with his true identity and all his
contributions to the site (an identity-revealing guise).
%
When Bob deletes his account, his profile and contributions move to different, privacy-preserving
guises: his name has been anonymized, his email address has been redacted, and his contributions
have been decorrelated and attributed to individual, unidentified user guises.
%
Decorrelation makes it seem as if a different user provided each of Bob's contributions. This allows
these contributions to be retained, while preventing an observer from correlating these
contributions back to Bob's identity, and while preserving referential integrity.

Disguises transform a guise by modifying it at per-attribute (\ie per-column) granularity, or
splitting it into multiple guises in order to decorrelating from objects that reference it (via \eg foreign-key relationships).
Guises can also be removed entirely. 
At any given moment, an application's data object graph comprises a mix of identity-revealing guises
and privacy-preserving ones. Disguises modify, split, and/or combine individual guises when triggered.

The application developer writes a disguise specification for each privacy transformation needed
in the application.
We assume that:
\begin{enumerate}[nosep]
  \item developers use their domain knowledge to write correct and complete disguises;
    %\lyt{a bit worried about ``complete'' here}
  \item application code handles the different guises appropriately (\eg in
    displaying them); and
  \item different guises of the same object have the same structure (\eg they can be
    rows in the same table).
\end{enumerate}
%
A data disguising tool takes the disguise specification and turns it into storage operations that
achieve the desired application data state.

%
%\lyt{Not sure if we need this paragraph anymore.}
%Data disguising builds on the existing structure of web applications.
%
%Web applications are often structured as object graphs, either explicitly~\cite{tao, delf},
%through an object-relational model (ORM)~\cite{orm}, or implicitly via foreign keys (edges)
%between tables (vertices) in relational databases.
%
%Data disguises transform this object graph.

\subsection{Writing Disguises}
\label{sec:disguises}
Developers must specify both the desired transformations---the end-state the disguise should
ensure---and a predicate for each transformation that selects the objects to be transformed.
\lyt{This predicate is specified as a read-only SQL WHERE clause}. For example, developers can
specify that all email addresses be obfuscated in an anonymous manner for users who are reviewers in
HotCRP.  When writing a disguise, developers can reason about its specification in isolation;
interactions between different disguises is handled by the disguising tool
(\S\ref{sec:composition}).

To create a guise from an object, developers specify how to transform attributes of the
object (\eg table column values) into guise attributes.
%
Figure~\ref{fig:guises} shows an example, producing guises for user objects.
%
User objects have identifier \texttt{id}; a reference \texttt{tag\_id}
associates the user via a foreign key constraint to tag objects.
%

%
Guises always have unique, random identifiers.
%
Developers choose how to create other guise attributes, selecting from among the following:
%
\paragraph{(1) Copy object content.}
%
Guises of the same object all share the object's attribute values.
%
If the attribute is a reference attribute (\eg a foreign key column), all guises will refer to the same object.
%
%
Copying allows developers to retain the object's content, without worrying about how to
synthesize attribute values for guises.
%
%However, this should only be chosen if guise attribute
%values cannot be generated, or if this attribute says little about the true identity of the
%entity.
For example, in Figure~\ref{fig:guises} the \texttt{darkmode} attribute is copied in
all guises.
%; the \texttt{darkmode} attribute reveals very little about the underlying user's
%identity.

\paragraph{(2) Generate new content.}
%
To create new attributes, developers specify whether the guise's value should be random,
a default value, or generated from the object's attribute value via a custom function (\eg hashing 
the value).
%
Figure~\ref{fig:guises} illustrates an example of random (\texttt{name}) and default
(\texttt{active}) generated value attributes.
%
%
Creating new guise reference attributes (\eg new foreign key relationships) requires
creating a new guise for the referenced object in order to maintain referential
integrity;
the data disguise rewrites the reference to point to the new guise.
%
In Figure~\ref{fig:guises}, creating two user guises requires creating two
tag guises, and the tag guises' identifiers become the user guises' foreign keys.
%

\paragraph{(3) Copy object content, but only once.}
%
One guise copies the attribute value from the object, but all other guises generate new
values (as described above).
%
\texttt{notifs} in Figure~\ref{fig:guises} illustrates how the attribute is copied once.
%
This enables the application to retain the original object semantics (\eg a count of how many
users want notifications) without creating duplicates.
%

%%%%%%%%%%%%%%%%%%%%%%%%%%%%%%%%%%%%%%%%%%%%%%%%%%%%%%%%%%%%%%%%%%%%%%%%%%%%%%%%%%%%%%%%%%
\section{Disguising and Revealing Semantics}
\label{sec:comp}
%%%%%%%%%%%%%%%%%%%%%%%%%%%%%%%%%%%%%%%%%%%%%%%%%%%%%%%%%%%%%%%%%%%%%%%%%%%%%%%%%%%%%%%%%%
This section describes the semantics of disguising provided by \sys. 
In particular, we describe the end state of a data object given some history of disguise
applications and reveals.

\vspace{6pt}\noindent\textbf{\emph{Some Notation.}} We describe disguise histories as a list of
\app{d_i} and \rev{d_i} actions, where \app{d_i} corresponds to the application of disguise $d_i$, and
\rev{d_i} corresponds to the reveal of $d_i$'s diffs. Time moves to the right in the list.

For every data object $x$ in the system, \xstart describes its initial state, and
\xhist{[\app{d_1}, \dots]} describes its state after \sys has applied the history [\app{d_1},
$\dots$].


%%%%%%%%%%%%%%%%%%%%%%%%%%%%%%%%%%%%%%%%%%%%%%%%%%%%%%%%%%%%%%%%%%%%%%%%%%%%%%%%%%%%%%%%%%
\vspace{6pt}\noindent\textbf{\emph{Composing Multiple Disguise Applications.}}
Let $d_1$ and $d_2$ be two disguises, where $d_1$ is applied after $d_2$.
to produce disguise history [\app{d_1}, \app{d_2}]. 
%
Let $x$ be some application data object, where \xstart is its initial state, and
\xhist{[\app{d_1}, \app{d_2}]} is the final state after \sys applies $d_1$ and $d_2$.
%
If both $d_1$ and $d_2$ update $x$, what is \xhist{[\app{d_1}, \app{d_2}]}?
We consider two possible end states: 
%
\begin{enumerate}
    \item[(\appcompone)] \xhist{[\app{d_1}, \app{d_2}]} reflects the application of
        \emph{both $d_1$ and $d_2$} to \xstart. 

        In other words, if an \op{d_2}'s predicate \texttt{pred} matches \xstart, then \op{d_2} is
        applied to \xhist[\app{d_1}], even if \texttt{pred} does not match
        \xhist{[\app{d_1}]}. 

    \op{d_2} updates are applied sequentially after $d_1$'s updates if the two modify the
        same object attribute.

\item[(\appcomptwo)] \xhist{[\app{d_1}, \app{d_2}]} reflects the application of $d_1$ to \xstart, followed
by the application of $d_2$ to \xhist{[\app{d_1}]}. When deciding whether to apply to $x$,
        \sys matches \op{d_2}'s predicate only against
\xhist{[\app{d_1}]}, and not against the original state \xstart.
\end{enumerate}

\noindent
For example, let \op{d_1} decorrelate all posts from authors predicated on \texttt{author =
Bea}, and let \op{d_2} remove all posts predicated on \texttt{author = Bea}.
%
\begin{enumerate}
\item[(\appcompone)] Both \op{d_1} and \op{d_2} update posts originally having author ``Bea'', resulting in the
removal of all posts originally with author Bea.

\item[(\appcomptwo)] Applying \op{d_1} results in all posts with author ``Bea'' having
pseudoprinicipal authors such as ``anonFox.'' \op{d_2} only knows the state of posts after
\op{d_1} occurs, and does not remove any posts because no post has author ``Bea''.
\end{enumerate}

The choice of whether \xhist{[\app{d_1}, \app{d_2}]} should result in \appcompone or \appcomptwo depends on the
specific application and is left to the developer to specify. 

In certain scenarios, however, \sys can only achieve end state \appcomptwo. In order to achieve \appcompone,
$d_2$ must evalute predicates against \xstart, which requires knowing the value of \xstart.  To learn
\xstart, \sys must know how $d_1$ modified $x$ (\eg knowing that a post with author ``anonFox''
originally was a post with author ``Bea''); this information, however, is stored in in the diffs produced
by $d_1$. 

\sys may not have access to these diffs when applying $d_2$: a client who invokes $d_2$ while authenticated as
principal $q \neq p$, and who does not provide the relevant capabilities%y pairs \pcapa{pd_1}, 
cannot access $p$'s diffs for $d_1$. 

In this case, $d_2$ can only match against \xhist{[\app{d_1}]} and achieve end state \appcomptwo.
This is shown in Table~\ref{tab:composeapp}.

\begin{table}[h]
\centering
\begin{tabular}{ c | c c }
    & \multicolumn{2}{c}{\textbf{Access $d_1$'s diffs while \app{d_2}?}}\\
    & \textbf{Yes} & \textbf{No} \\
\hline
    \xhist{[\app{d_1},\app{d_2}]}& \appcompone or \appcomptwo & \appcomptwo 
\end{tabular}
\vspace{6pt}

\caption{Possibilities for \xhist{[\app{d_1},\app{d_2}]} depending on whether \sys has access to diffs from $d_1$ regarding $x$.}
\label{tab:composeapp}
\end{table}

%%%%%%%%%%%%%%%%%%%%%%%%%%%%%%%%%%%%%%%%%%%%%%%%%%%%%%%%%%%%%%%%%%%%%%%%%%%%%%%%%%%%%%%%%%
\vspace{6pt}\noindent\textbf{\emph{Semantics of Disguise Reversals.}}
We now consider composition of disguises in the presence of disguise reversals.

\begin{table}[h]
\centering
\begin{tabular}{ c | c c }
    & \multicolumn{2}{c}{\textbf{Access to $d_i$'s diffs while \app{d_i}?}}\\
    & \textbf{Yes} & \textbf{No} \\
\hline
    \xhist{[\app{d_i},\rev{d_i}]} & \xstart & \xhist{[\app{d_i}]}
\end{tabular}
\vspace{6pt}

\caption{Possibilities for \xhist{[\app{d_i},\rev{d_i}]} depending on whether \sys has 
    access to diffs from $d_i$ regarding $x$.}
\label{tab:composeapprev}
\end{table}

We first consider \textbf{\xhist{[\app{d_i},\rev{d_i}]}}; Table~\ref{tab:composeapprev}
illustrates how this state depends the authorization a client invoking \rev{d_i} provides to
\sys to access $d_i$'s diffs for $x$.
As expected, the updates applied by $d_i$ to $x$ cannot be revealed when the relevant diffs are
inaccessible.

\begin{table}[h]
\centering
\begin{tabular}{ c | c c }
    \textbf{$d_1$ and $d_2$'s Updates} & \multicolumn{2}{c}{\textbf{Access 
        to $d_1$'s diffs while \rev{d_1}?}}\\
    & \textbf{Yes} & \textbf{No} \\
    \hline
    \textbf{Independent} & \xhist{[\app{d_2}]} & \xhist{[\app{d_1},\app{d_2}]}\\
    \textbf{Conflict} & \xhist{[\app{d_1},\app{d_2}]} & \xhist{[\app{d_1},\app{d_2}]}
\end{tabular}
\vspace{6pt}
    \caption{\xhist{[\app{d_1},\app{d_2},\rev{d_1}]} depending on whether \sys has
    access to $d_1$'s diffs when revealing $d_1$, and whether updates from $d_1$ and
    $d_2$ to $x$ modify the same data.}
\label{tab:composeapprev1}
\end{table}

We next consider \textbf{\xhist{[\app{d_1},\app{d_2},\rev{d_1}]}}: the end state of $x$ when a disguise
$d_1$ is revealed after a subsequent disguise $d_2$ has been applied. 
Table~\ref{tab:composeapprev1} illustrates how the result
depends on \sys's access to $d_1$ diffs, and whether $d_1$ and $d_2$ performed
conflicting modifications to $x$'s attributes.

When the client invoking \rev{d_1} provides no authorization to access to $d_1$'s diffs, then the
updates applied by $d_1$ to $x$ cannot be revealed.  However, even if the client 
grants access to $d_1$'s diffs, \sys will fail to reveal $d_1$'s updates to $x$ if $d_2$ and $d_1$
both modify $x$ in a conflicting manner: $\rev{d_1}$ must not restore data that was also disguised by
$d_2$, and until $d_2$ is revealed, invocations of \rev{d_1} will not reveal \xstart.

\subsection{Composition Implementation Techniques}
\sys's algorithm for disguising and disguise reversal in most scenarios is straightforward. 

To apply disguise $d_i$, \sys applies each \op{d_i} to all
data objects satisfying \op{d_i}'s predicate, while also taking into account the information in
accessible diffs(to \eg predicate against undisguised versions of objects). Each \op{d_i} produces
one or more diffs, which \sys stores either globally or encrypted by a data capability at
\lcapa{pi_i}.

To reveal disguise $d_i$, \sys reveals the undisguised version of data stored in
accessible diffs corresponding to $d_i$ by updating the relevant objects $x$ in the database.

However, \sys must be careful of two potential problems:
(1) \sys cannot accidentally reveal data that must be disguised; and (2) \sys cannot let global
diffs recording updates to data $x$ leak information if $x$ is subsequently (privately) disguised.

%%%%%%%%%%%%%%%%%%%%%%%%%%%%%%%%%%%%%%%%%%%%%%%%%%%%%%%%%%%%%%%%%%%%%%%%%%%%%%%%%%%%%%%%%%
\vspace{6pt}\noindent\textbf{\emph{Preventing Accidental Data Revelation.}}
\sys records the disguised state of data $x$ in each \tdiff{pd} recording a update performed by $d_i$
to $x$. If the state of $x$ does not match the disguised state in $d_i$'s diff for $x$, then \sys
knows the diff records a stale and overwritten update to $x$, and refuses to reveal the undisguised state of $x$
stored in the diff.

%%%%%%%%%%%%%%%%%%%%%%%%%%%%%%%%%%%%%%%%%%%%%%%%%%%%%%%%%%%%%%%%%%%%%%%%%%%%%%%%%%%%%%%%%%
\vspace{6pt}\noindent\textbf{\emph{Preventing Global Diff Information Leakage.}}
%The ONLY REASON we need to update diffs is if they're global!!! If they're private, we can use the
%retroactive application method... (optimization to update our "own" diffs though)
%
A disguise may optionally store diffs in global storage, accessible to \sys without any client
authorization.
Such a global disguise provides no privacy guarantees against an adversary who compromises 
the application server, but does transform the application database so external users see disguised
data.

If $d_1$ is a global disguise and $d_2$ a normal, privacy-preserving disguise,
\op{d_2} may update a data object $x$ previously updated by \op{d_1}. 

\op{d_1} produces a global diff \tdiff{pd_1}; \op{d_2} produces a normal,
encrypted and secured diff \tdiff{pd_2}.

If \sys simply performs \op{d_2}'s update to $x$, an adversary can still read undisguised data
from the \tdiff{pd_1} diff because the diff is global and records a now-stale state of $x$!
%
To prevent global diffs like \tdiff{pd_1} from leaking data that should be disguised, \sys
updates the data in \tdiff{pd_1} with \op{d_2}'s update. An update to a diff such as
\tdiff{pd_1} itself generates a (private) diff \tdiff{pd_2}, just as updating a data
object generates a diff.

Revealing $d_2$ uses \tdiff{pd_2} to reveal the modification to \tdiff{pd_1}. 
If $d_1$ has already been revealed, and the database state of $x$ updated to reflect the
contents of \tdiff{pd_1} (which were updated by the application of $d_2$), then \sys
simply uses \tdiff{pd_2} to reveal the current state of $x$ in the database. 

If a future $d_3$ has further updated \tdiff{pd_1} in a way that conflicts with $d_2$'s update, then
upon reversal of $d_2$, \sys will notice that \tdiff{pd_2} records an overwritten update to
\tdiff{pd_1}, and will not revert the state of \tdiff{pd_1} until the future disguise $d_3$ has been
revealed .

\lyt{Note that removal is just an interesting case of this, in which \tdiff{pd_1} is removed, by a
diff \tdiff{pd_2} stored in \lcapa{pd_2} that saves the removal of \tdiff{pd_1} for reversal.}

%\vspace{6pt}\noindent\textbf{\emph{diff Modification for Object Identification.}}
%diffs need to refer to the correct objects even if objects have been modified by future disguises.
%Prior diffs should be revised by future disguises so that they apply correctly to the disguised
%data.
%This allows an earlier disguise to be revealed correctly.
%An easy way to avoid needing to do this is to have all objects in the DB have unique, permanent
%identifiers.

\section{An Example Disguise}
\label{design:eg}
%
Consider disguising Bob when he deletes his HotCRP account.
%
Bob would prefer his papers and reviews to be unlinked from his identity.
%
HotCRP, on the other hand, would like to retain paper and review information that other users
find useful.
%
An application developer can easily achieve both with disguises.
%A careful selection of edge and object transformations achieves both.
%

The application developer writes a short specifcation of the disguise. Bob is unlinked from his
reviews via a predicated-transformation that decorrelates any reviews that reference Bob.
%
This transforms Bob into one unique user guise per review.
%
The disguise generates guise attribute values using suitable defaults;
%
in particular, HotCRP users' \texttt{disabled} attribute is set for the guises, ensuring that guises
have no permissions and never review papers.
%

%
Bob is linked to papers through conflicts, which can indicate coauthorship or a reviewer conflict.
%
These conflicts are not reassigned to the new guises, since preserved conflicts could reidentify Bob
as the likely author of a review. Thus, conflicts predicated on linking Bob to papers will be
removed.

%The disguise leaves all other edge types, ensuring that review and paper artifacts remain correctly
%linked: active reviewers still see the correct paper for their reviews, and active authors see the
%correct reviews for their papers, albeit potentially authored by anonymous, unlinkable guises of the
%original reviewer.
Unlike the current real-world HotCRP account deletion policy~\cite{hotcrp:privacy}, which deletes
all objects belonging to Bob, this disguise strikes a balance between decorrelating Bob's identity
from his reviews and papers, and maintaining useful information for other HotCRP users.
%
Furthermore, it is easy to imagine extending this disguise to automatically disguise Bob after some
time (\eg 2 years after the conference), protecting his future research career by hiding youthful
reviewing sins.
%

The tool performs the disguise when Bob invokes HotCRP account deletion. To do so correctly, Bob
provides his secret key to decrypt and read from his vault. The tool retrieves and temporarily reverses any conflicting
transformations that would make Bob's deletion fail: in particular, it recorrelates any of Bob's decorrelated
paper conflicts and then properly removes them. (Because the conflicts are removed, the
decorrelation is not reapplied). After performing the transformations specified in the disguise, the
tool records each transformation in the vault, and re-encrypts it, forgetting the key.

%%%%%%%%%%%%%%%%%%%%%%%%%%%%%%%%%%%%%%%%%%%%%%%%%%%%%%%%%%%%%%%%%%%%%%%%%%%%%%%%%%%%%%%%%%
\section{Implementing Disguising}
%%%%%%%%%%%%%%%%%%%%%%%%%%%%%%%%%%%%%%%%%%%%%%%%%%%%%%%%%%%%%%%%%%%%%%%%%%%%%%%%%%%%%%%%%%
\begin{table*}[t!]
\centering
\begin{tabular}{ c p{.7\linewidth} }
\textbf{Function} & \textbf{Description} \\
\hline
    \fn{ReadPrivateTokens(\symk{pd})} & Decrypts all of $p$'s private tokens produced by disguise
    $d$ using \symk{pd}. \\
    \fn{ReadGlobalTokens($d$)} & Retrieves all global tokens produced by disguise $d$. \\
    \fn{ApplyDisguise($d$,tokens)} & Applies disguise $d$, selectively composing $d$'s
    updates with prior disguises using the tokens's data. \\
    \fn{\op{d}.execute(tokens)} & Executes the disguise operation \op{d}, composing the operation
    with prior disguises using the tokens' data.\\
    \fn{ReverseDisguise($d$,tokens)} & Reverses disguise $d$ using the tokens' data.\\
    \fn{ReverseTokenOp(token)} & Reverses the data modification performed by the disguise operation
    that produced the token.\\
    \fn{StorePubKey($\pubk{p}$)} & Persistently saves the public key \pubk{p} indexed by $p$.\\
    \fn{LoadPubKey($p$)} & Retrieves public key \pubk{p} for $p$.\\
    \fn{LoadEncPrivKeyTokens($p$)} & Retrieves \tpriv{pdq'} ciphertexts for $p$.\\
    \fn{LoadEncSymKeys(caps)} & Retrieves \symk{pd} ciphertexts corresponding to the specified
    capabilities.\\
    \fn{StoreEncSymKey(\rptr{pd})} & Persistently saves the \symk{pd} ciphertext indexed by
    capability \rptr{pd}.\\
    \fn{LoadListTail}$(p,d)$ & Gets the first encrypted private token in \tokls{pd}, the list of
    tokens associated with $p$ produced by $d$.\\
    \fn{StoreListTail}$(p,d)$ & Persistently saves the first encrypted private token in \tokls{pd}
    indexed by $p$ and $d$.
\end{tabular}
    \vspace{12px}
\caption{Internal \sys functions}
\label{tab:funcs}
\end{table*}

Table~\ref{tab:funcs} describe \sys's internal functions run server-side to implement its API 
and apply or reverse disguises. 

Figures~\ref{fig:appdisg} and \ref{fig:revdisg} describe how \sys implements disguise application and
reversal respectively. Figure~\ref{fig:opexec} describes how disguise operations update application
state and produce and modify tokens, and 
Figure~\ref{fig:revtoken} describes how \sys uses a token's recorded
modification to reverse an applied disguise operation. 
Figure~\ref{fig:rpt} describes how \sys accesses the private tokens
corresponding to disguise $d$ and principal $p$ by recursively decrypting tokens and traversing
\tokls{pd}.

\sys's implementations of the remaining functions in Table~\ref{tab:funcs} simply read or write into
persistent maps indexed by principal and/or disguise.

\subsection{Composition Techniques}
\sys's algorithm for disguising and disguise reversal in most scenarios is straightforward. 

To apply disguise $d$, \sys applies each \op{d} to all
data objects satisfying \op{d}'s predicate, while also taking into account the information in
accessible tokens (to \eg predicate against undisguised versions of objects). Each \op{d} produces
one or more tokens, which \sys stores appropriately (globally, or encrypted in a \tokls{pd}).

For reversal of disguise $d$, \sys reveals the undisguised version of data stored in accessible tokens
corresponding to $d$ by updating the relevant objects $O$ in the database.

However, \sys must be careful of two potential problems:
(1) \sys cannot accidentally reveal data that must be disguised; and (2) \sys cannot let global
tokens recording updates to data $x$ leak information if $x$ is subsequently (privately) disguised.

%%%%%%%%%%%%%%%%%%%%%%%%%%%%%%%%%%%%%%%%%%%%%%%%%%%%%%%%%%%%%%%%%%%%%%%%%%%%%%%%%%%%%%%%%%
\vspace{6pt}\noindent\textbf{\emph{Preventing Accidental Data Revealing.}}
\sys records the disguised state of data $x$ in each \tdata{pd} recording a update performed by $d$
to $x$. If the state of $x$ does not match the disguised state in $d$'s token for $x$, then \sys
knows the token records an overwritten update to $x$, and refuses to reveal the undisguised state of $x$
stored in the token.

%%%%%%%%%%%%%%%%%%%%%%%%%%%%%%%%%%%%%%%%%%%%%%%%%%%%%%%%%%%%%%%%%%%%%%%%%%%%%%%%%%%%%%%%%%
\vspace{6pt}\noindent\textbf{\emph{Preventing Global Token Information Leakage.}}
%The ONLY REASON we need to update tokens is if they're global!!! If they're private, we can use the
%retroactive application method... (optimization to update our "own" tokens though)
%
Let $d_1$ be a global disguise and $d_2$ be a private disguise applied in sequence.  Consider the
scenario in which \op{d_1} updates a data object $O$, producing a global token \tdata{pd_1}. Later,
\sys determines that \op{d_2} also should update $O$.
%by checking if the corresponding \op{d_2}
%predicate matches recorded object data stored in \tdata{pd_1}.

If \sys simply performs \op{d_2}'s update to $O$ in the application database, an adversary can still
read undisguised data from \tdata{pd_1} since is it global and records a now-stale state of $O$!
%
%Furthermore, this means that \sys cannot restore the state of $O$ in the database to
%\xhist{[\app{d_2}]} when reversing $d_1$, because the revealing the data in \tdata{pd_1} would
%reveal data that $d_2$ should have disguised.
%
To prevent global tokens like \tdata{pd_1} from leaking information that should be disguised, \sys
updates the data in \tdata{pd_1} with \op{d_2}'s update. An update to a token such as \tdata{pd_1}
itself generates a (private) token \tdata{pd_2}, just as updating a data object generates a token.

Reversing $d_2$ uses this \tdata{pd_2} to reverse the modification to \tdata{pd_1}. 
If $d_1$ has already been reversed, and the database state of $O$ updated to reflect the contents
of \tdata{pd_1} (which were updated by the application of $d_2$), 
then \sys simply uses \tdata{pd_2} to reverse the current state of $O$ in the database. 

If a future $d_3$ has further updated \tdata{pd_1} in a way that conflicts with $d_2$'s update, then
upon reversal of $d_2$, \sys will notice that \tdata{pd_2} records an overwritten update to
\tdata{pd_1}, and will not revert the state of \tdata{pd_1} until the future disguise $d_3$ has been
reversed.

\lyt{Note that removal is just an interesting case of this, in which \tdata{pd_1} is removed, by a
token \tdata{pd_2} is stored in \tokls{pd_2} that saves the removal of \tdata{pd_1} for reversal.}

%\vspace{6pt}\noindent\textbf{\emph{Token Modification for Object Identification.}}
%Tokens need to refer to the correct objects even if objects have been modified by future disguises.
%Prior tokens should be revised by future disguises so that they apply correctly to the disguised
%data.
%This allows an earlier disguise to be reversed correctly.
%An easy way to avoid needing to do this is to have all objects in the DB have unique, permanent
%identifiers.



%-------------------------------------------------------------------------------
\section{Evaluation}
%-------------------------------------------------------------------------------

We evaluate \sys{} on two metrics: (1) can desired decorrelation policies be easily specified using the
provided decorrelation primitives for a range of practical applications?, and (2) can decorrelation
be supported with low performance overhead?

Our evaluation exemplifies how a suite of applications (Lobste.rs, \lyt{TODO}) can express a diverse
range of privacy policies using \sys{} with low developer effort.

\subsection{Lobste.rs}
The performance experiments reported are run on Intel Xeon E5-2660 v3 CPUs, with a
single-threaded client and \sys{}'s shim layer each pinned to a single core. \sys{} utilizes the
trawler workload~\cite{trawler} for Lobste.rs, which emulates production Lobste.rs traffic according
to a recorded production workload. 

\section{Discussion}

\lyt{Should we note somewhere that pseudoprincipals and recursive disguising are why we need asymmetric
crypto; otherwise, we could just send the symmetric key to the original user being disguised?}

%-------------------------------------------------------------------------------
\section{Background and Related Work}
%-------------------------------------------------------------------------------

Companies have developed frameworks to avoid deletion bugs (e.g., DELF~\cite{delf} at Facebook), but many
applications decorrelate only coarsely by associating remnants of data with a global placeholder for
all deleted users, or do not decorrelate at all (e.g., replacing usernames with a pseudonym).

\lyt{(cite Reddit, Lobsters, others?)}

The right to be forgotten has also been formally defined by Garg et
al.~\cite{garg}, where correct deletion corresponds to the notion of
leave-no-trace: the state of the data collection system after a user requests to be forgotten should
be left (nearly) indistinguishable from that where the user never existed to begin with. While
\sys{} uses a similar comparison, their formalization assumes that users operate
independently, and that the centralized data collector prevents one user's data from influencing
another's.

Other related works:
\begin{itemize}
    \item Deceptive Deletions for protecting withdrawn posts: https://arxiv.org/abs/2005.14113
    \item "My Friend Wanted to Talk About It and I Didn't": Understanding Perceptions of
        Deletion Privacy in Social Platforms, user survey https://arxiv.org/pdf/2008.11317.pdf;
        talk about decoy deletion, prescheduled deletion strategies~\cite{myfw}
    \item Contextual Integrity
    \item ML Unlearning
    \item k-anonymization, pseudonymization
\end{itemize}

%%-------------------------------------------------------------------------------
\section{Conclusion}
%-------------------------------------------------------------------------------
We propose data masking, an approach that enables developers to write high-level data mask
specifications for privacy transformations. Data masking tools take a data mask and automatically
apply the appropriate data transformations. We show how our prototype applies data masks for
existing and new privacy policies in practical applications.

\iffalse
new paradigm of data ownership on the web that provides users with control over
when and how applications access their data, without overhauling the current web architecture and
business model. To realize \name, we design \sys, which minimizes the burden upon users to store or
manage their data, and makes it easy for developers to systematically express and automate the
subtle and complex data transformations needed for unsubscription and resubscription. We demonstrate
how \sys can be used in practical web applications with low overhead.
\fi


%-------------------------------------------------------------------------------
\printbibliography

%%%%%%%%%%%%%%%%%%%%%%%%%%%%%%%%%%%%%%%%%%%%%%%%%%%%%%%%%%%%%%%%%%%%%%%%%%%%%%%%
\end{document}
%%%%%%%%%%%%%%%%%%%%%%%%%%%%%%%%%%%%%%%%%%%%%%%%%%%%%%%%%%%%%%%%%%%%%%%%%%%%%%%%

%%  LocalWords:  endnotes includegraphics fread ptr nobj noindent
%%  LocalWords:  pdflatex acks
