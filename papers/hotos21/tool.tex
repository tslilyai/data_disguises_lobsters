%-------------------------------------------------------------------------------
\subsection{Data Disguising Tools}
\label{sec:composition}
%-------------------------------------------------------------------------------

A data disguising tool handles the complexity behind disguise composition, applying disguises in
sequence and generating the necessary storage operations to achieve the correct end state.
When applying a disguise, a disguising tool must both correctly apply the specified transformations, and also
manage the complexity caused by conflicts and shared dependencies between the new disguise and any
prior disguises.  For example, \texttt{ConfAnon} destroys information linking data records back to a
user, thus making it impossible for \texttt{GDPR} to properly remove these records.

To handle inter-disguise dependencies, a disguising tool relies on (1) the structured nature of disguises to
statically determine whether disguises share dependencies, and (2) the key abstraction of \emph{user
vaults}, namely per-user logs of disguise updates to that user's data.  User vaults solve the issue
that disguises inherently destroy data necessary to correctly achieve the end-state of future
disguises by providing a secure way to store the data. A disguising tool queries the user vault to
temporarily restore destroyed data (\eg decorrelated foreign key relationships) in order to apply
the disguise correctly.

As shown in Figure~\ref{fig:tool}, a disguising tool sits next to the application, and queries the
user vaults and the application database. The application performs disguises by invoking a
disguising tool.

%-------------------------------------------------------------------------------
\paragraph{Deploying User Vaults.}
%-------------------------------------------------------------------------------
User vaults can be flexibly configured and deployed. It remains important, however, that any
configuration of user vaults should not violate the guarantee that disguises indeed destroy data,
from the viewpoint of the application and any users.
We imagine several exciting directions to explore for designing vaults that are both
performant and secure.
%

%
Some possible configurations include a disguising tool storing vaults encrypted with a per-user key; this key
may be secret-shared using a (2, 3) threshold scheme~\cite{secretsharing} between the user, the
tool, and a trusted third party (\eg Amazon S3), so that the user can authorize a disguising tool and the
third party to restore the key if the user forgets their share.
%
The vault entries could be configured to expire after a certain time (requiring that a disguising tool either
prevent future conflicting disguise application, or a-priori prevent disguises that conflict with
legally required disguises such as GDPR deletion).
%
Or perhaps the vault entries are stored entirely by some third party or locally by the user, whose
server runs a corresponding interface to allow disguise tools to read and write the vault.

A disguising tool should, with user permission, be able to read and write the vaults: a user invoking GDPR
\texttt{GDPR} could provide the key to their vault, and the application invoking the
\texttt{ConfAnon} might notify each user for approval (and temporary vault access) prior to disguise
application. An alternative might be a multi-tier security design, in which a disguising tool and application
are trusted with access to vaults in one tier, allowing them to perform application-instigated
disguises, and untrusted in a second tier, which stores updates from user-invoked disguises.

%\lyt{Might want to say more about how often user vaults are
%queried---what if one disguise needs access to many user vaults, but is being done on behalf of a
%single user? Perhaps there could be multiple ``levels'' of security in vaults, and any
%application-imposed disguise would be protected by application-known keys.}

%-------------------------------------------------------------------------------
\paragraph{Composing and Applying Disguises.}
%-------------------------------------------------------------------------------
\lyt{Moved a bunch of this to \S\ref{sec:composition}}
%Composed disguises should achieve an end-state that combines, in some way, the end-states achieved by each disguise when applied to the original application database in isolation.
%Correct composition of multiple disguises achieves an end-state equivalent to combining the
%end-states achieved by each disguise when applied to the original application database in isolation.
%
%If a prior disguise is reversible, then a disguising tool can use user vaults to ensure that this
%prior disguise does not affect \emph{which} objects are updated
%a future disguises. 
%%In this case, a disguising tool allows developers to reason about multiple conflicting updates to
%the same object: 
%regardless of when the disguises occurred, if one disguise removes an object that the other disguise
%modified, then the removal takes precedence.
%%
%
%However, if they both modify the same object attribute, a disguising tool establishes no precedence
%between the modifications and applies them in chronological order.  Alternatively, we can imagine
%that the developer could specify a partial ordering between modifications, or our framework could
%restrict the set of possible modifications and establish a precedence order within this set.
%
%If the prior disguise is not reversible, however, then the disguising tool could prevent future
%conflicting disguise application, or perhaps a-priori prevent the application of such non-reversible
%disguises that conflict with legally required disguises such as GDPR deletion \lyt{not sure what to
%put here? May also want to include something about developer assertions}
%%: a disguise will update all objects that it would have updated if performed on
%the original, undisguised state of application data.


A disguising tool applies disguises in a five-phase procedure:
\begin{enumerate}[nosep]
    \item \emph{Prepare}: reconcile any data dependencies between this disguise and prior disguises.
            A disguising tool detects read-after-write dependencies between the new disguise's predicates and prior disguises'
            updates, and, using entries in the vault, undoes any writes that may affect the new disguise's predicates. As an
            optimization, vault entries recording object removals need not be reversed.
        \item \emph{Read}: get all objects that satisfy (per-type) developer-specified predicates.
        \item \emph{Update}: modify, decorrelate, or remove objects read in step (2) according to the
        developer's specification.
    \item \emph{Record}: store records of all updates in the appropriate per-user vaults. A
        disguising tool
        must be able to determine which user vault should record each modification. This can be
        developer-specified, or rely on a set of heuristics (\eg assigning ownership by traversing,
        starting from each user, the application's object graph expressed in an object-relational
        model (ORM)~\cite{orm}, or implicitly via foreign keys).
        \item \emph{Finalize}: After applying the new disguise updates, the disguising tool reapplies the temporarily reversed modifications from earlier disguises.
\end{enumerate}

%-------------------------------------------------------------------------------
%\paragraph{Reversing Disguises.}
%-------------------------------------------------------------------------------
%To correctly reverse a disguise and ensure that reintroduced data is not mistakenly revealed because
%it missed a subsequent disguise, A disguising tool keeps a persistent log of all disguises performed by the
%application. Any other disguises performed between this disguise's application and its reversal are 
%%applied to any reintroduced data.
