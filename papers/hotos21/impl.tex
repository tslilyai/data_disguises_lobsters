%-------------------------------------------------------------------------------
\section{Prototype}
%-------------------------------------------------------------------------------
\label{sec:proto}
%
Our prototype disguising tool, \sys, is written in Rust and 
provides data disguising for applications that use relational databases.
%
We believe that data disguising can be extended to other types of stores.
%

Disguises in \sys associate each table in the application schema with a set of
predicate-transformation pairs: predicates are arbitrary SQL \texttt{WHERE} clauses, which \sys uses
to select table rows to transform; a transformation is either a removal, a decorrelation of a
particular foreign key, or a modification of a particular column. A modification takes a
closure over the original column value that returns the updated value.
%
For tables that may be decorrelated, developers describe how to create new guises---new table
rows---by providing per-column closures over the original column value
that returns the new guise column value.

%
\sys represents vaults as (currently unencrypted) per-user database tables; revealing functions
stored in vaults use the original and updated states of objects touched by a reversible disguise to
generate the necessary operations to restore the original state. \sys also keeps a disguise history
table that logs all disguises performed. 
%While \sys
%currently has no per-user encryption of vaults and can read and write vaults at any point, we plan
%to add per-user encryption keys that are not stored by \sys.
%Currently, the tool uses the heuristic that if a user invokes the disguise (\eg GDPR deletion),
%modifications to data records will be stored only to that user's vault; however, if the application
%invokes a general disguise, then any users connected to modified data records will store the
%modification in their vault.

\iffalse
\sys also provides a specification of the end-state after it applies a sequence of
disguises; this allows developers to double-check that their disguises achieve their privacy
transformation goals.
%
To do so, \sys automatically generates a sequence of materialized view filters that, when applied on top of each
other, correspond to the state achieved by composing the disguises. 
Developers specify assertions (\eg all objects belonging to departing user Bob have
been removed) that \sys ensures holds on the materialized view specification.

\sys's composition technique may reorder filters from one disguise with those from a subsequent disguise in order to achieve the
correct result.
%
\sys maps these filters directly to physical database operations (where reordering filters corresponds to
temporarily undoing vault entries). Thus, the developer can use the materialized view filters as a way to
test the correctness of \sys and their disguises prior to permanently altering the application database. 
\fi
