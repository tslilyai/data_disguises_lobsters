%-------------------------------------------------------------------------------
\section{Prototype}
%-------------------------------------------------------------------------------
\label{sec:proto}
%
Our prototype disguising tool, \sys, is a disguising library written in 5K LoC of Rust.
%
\sys provides data disguising for applications that use relational databases, where
objects are table rows and edges are foreign key relationships.
%
We believe that the data disguising approach can be extended to other types of stores.
%

%
\sys represents user vaults as additional database tables, storing blobs of the prior and new state
of the object being updated. \sys also keeps a disguise history table in order to correctly 
disguise reintroduced data from disguise reversals.
\lyt{The current implementation of \sys has no per-user encryption of vaults, allowing \sys to read and
write vaults at any point. We plan to add per-user encryption keys that are not stored by \sys.}
%

%
Disguises written for \sys associate each table in the application schema with a set of
transformation-predicate pairs. Predicates are arbitrary SQL \texttt{WHERE} clauses, and allow \sys
to select the relevant table rows to trasnform. The
transformation is either a removal, a decorrelation of a particular foreign key, or a
modification of a particular column. To modify a column value, developers provide arbitrary closures that take as input the original column value, and return the updated value.
%
\sys also requires developers to describe how to create new rows---\eg generate fake users---for all
tables that may be decorrelated.
\lyt{Would it help to have pseudo-code here?}

\sys also provides a specification of the end-state after it applies a sequence of
disguises; this allows developers to double-check that their disguises achieve their privacy
transformation goals.
%
To do so, \sys automatically generates a sequence of materialized view filters that, when applied on top of each
other, correspond to the state achieved by composing the disguises. 
Developers specify assertions (\eg all objects belonging to departing user Bob have
been removed) that \sys ensures holds on the materialized view specification.

\sys's composition technique may reorder filters from one disguise with those from a subsequent disguise in order to achieve the
correct result.
%
\sys maps these filters directly to physical database operations (where reordering filters corresponds to
temporarily undoing vault entries). Thus, the developer can use the materialized view filters as a way to
test the correctness of \sys and their disguises prior to permanently altering the application database. 
