\section{Privacy Transformations}
\label{sec:survey}

We investigate the potential of privacy transformations as a first-class citizen in application
design, and the challenges we face to support these transformations. These challenges argue for the
need for a systematic approach to privacy transformations.

\subsection{The Potential of Privacy Transformations}
Widely-used web applications today~\cite{spotify:privacy, amazon:privacy,
strava:privacy, hotcrp:privacy, wikipedia:privacy, facebook:privacy, twitter:privacy,
reddit:privacy, github:privacy, lobsters:privacy} mainly support variants of an account deletion privacy transformation,
one that \eg the GDPR~\cite[Art.\ 17]{eu:gdpr} mandates.
However, privacy transformations can achieve much more.
%
Users and application developers can both benefit from \textbf{more nuanced} privacy policies:
%
for example, a confidential paper review system like HotCRP~\cite{hotcrp} keeps users'
contributions (papers, reviews) after they delete their account to preserve utility for others, but
could additionally associate each review with a different placeholder to avoid revealing the deleted
reviewer's identity.
%
%Likewise, contributions with a shared property (\eg Reddit posts with a common tag)
%might be removed entirely to avoid inference attacks, or retained and decorrelated from the
%property (\ie keeping the user's Reddit posts, but removing their tags).
%\eddie{Would like more explanation of that}
In some cases, a policy could preserve utility while reducing the efficacy of later inference
attacks by \eg modifying sensitive metadata (\eg tags, creation times).
%
%Some transformations should be performed only on sensitive metadata:
%Similar policies could apply \emph{only if} the property was created by the user (\eg keeping
%the user's Reddit posts, but removing any user-created tags),
% tag like “\#cat” is likely insensitive and useful to preserve, whereas one naming a person is not.
%
%(\eg remove the user's posts on Reddit with tag $t$ if these posts comprise more than 10\%
%of all posts with tag $t$).
%
Individual users could even specify different preferences for their data.
%
%% A privacy transformation framework is necessary to turn these preferences into concrete
%% operations without undue developer burden.
%

%
Applications could go beyond simple account deletion and support \textbf{data
expiration} policies, which anonymize a user's contributions after the user has been inactive for a
period of time, possibly restoring the user's profile and contributions if the user ever logs back
in.
%
Or the application could support gradual \textbf{data decay} policies that ``decay'' sensitive data
by applying incremental privacy transformations over time.

Furthermore, many applications might wish to employ \textbf{reversible} transformations to, for
example, support account reactivation instead of permanent and irrevocable account deletion.
%
After all, if services must allow users to remove their data on request, it is in the operator's
interest as well as the user's to make it easy for a user to change their mind and return, bringing
their data along.

\subsection{Challenges}
%
Although privacy transformations offer many potential benefits, practically supporting them is hard.
First, modifying or deleting data must not compromise application correctness: for example, privacy
transformations must maintain referential integrity of an application's relational database.
Developers must be careful to properly transform user data without violating application invariants,
%
%Even describing the transformation may not be straightforward: 
%for example, on account deletion, surveyed applications only selectively delete user data (\eg
%passwords) from underlying databases, and retain the remainder for legal or necessary business purposes
%(\eg Spotify fraud detection~\cite{spotify:privacy}, Amazon
%orders~\cite{amazon:privacy}).
%%
%%Some services keep most deleted users' contributions indefinitely available (\eg
%%StackOverflow answers~\cite{stackoverflow:privacy}), sometimes associated with the original user
%%%name (\eg Wikipedia edit history~\cite{wikipedia:privacy}).
%%
%Platforms like Facebook and Twitter delete posts, but keep private messages unanonymized and visible to
%their recipients~\cite{facebook:privacy, twitter:privacy};
%others keep posts visible but anonymized, reattributing the contribution to a placeholder user
%(\eg GitHub's ``@ghost''~\cite{github:privacy}, Reddit/Lobsters'
%``[deleted]''~\cite{reddit:privacy, lobsters:privacy}).
%%In the open-source applications surveyed, developers implement privacy transformations
%%via ad-hoc database operations in their application code.
%%, and only trigger them on explicit, user-initiated account
%%deletion.
which they do today via ad-hoc database operations in their application code.

Furthermore, one application may apply several privacy transformations, intermixed with
transformation reversals.  When these transformations may share dependencies or contradict each
other, correctly composing transformations to achieve the desired privacy properties is a daunting
task.
%
Finally, practical privacy transformations should require minimal application changes.
%

%
To make these challenges concrete, suppose we were to implement the following two privacy
transformations in HotCRP. 
%
(1) \texttt{ConfAnon} implements universal conference anonymization, a form of data decay to
prevent reviewer identification in case of a data breach.
%
(2) \texttt{GDPR} implements GDPR's ``right-to-be-forgotten'' removal of user data.
%
\texttt{ConfAnon} decorrelates reviews and review metadata (\eg review preferences) from reviewers
by ensuring that no review references a real user's account (via a foreign key relationship).
To maintain referential integrity, decorrelation requires updating foreign key
attributes to reference fake user profiles.
\texttt{GDPR} finds and deletes the leaving user's data.
%

%
In isolation, each transformation may seem simple; however, composing them is tricky.
\texttt{ConfAnon} destroys information linking a user's data to their account by relinking this data
to anonymous, fake users (\eg rewriting the foreign key linking a review preference to a user to
point to a fake user profile).  If the user later invokes \texttt{GDPR}, the
disguise needs to know which data to remove, which is impossible without the original links.
%
Due to data dependencies between these transformations, \texttt{GDPR} fails to achieve its desired
privacy properties, namely the complete removal of the user's data.
%
