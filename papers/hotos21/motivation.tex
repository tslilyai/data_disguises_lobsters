\section{Challenges of Privacy Transformations}
\label{sec:motivation}
%
%We investigate the potential of privacy transformations as a first-class
%citizen in application design, and the challenges to support these
%transformations.
%%
%These challenges argue for a systematic approach to privacy transformations.
%%

%
Privacy transformations have two key challenges: the broad
range of possible policy choices for transformations, and the difficulty of
implementing those policies.
%
The first challenge can be appreciated by surveying current applications' policies
around account deletion.
%
Though many applications support deletion,
some retain
data for legal or necessary business purposes (\eg Spotify fraud detection~\cite{spotify:privacy},
Amazon orders~\cite{amazon:privacy}); some delete public contributions, but keep private
messages unanonymized and visible to their recipients, reflecting the
shared nature of such messages~\cite{facebook:privacy, twitter:privacy};
and yet others keep public contributions visible but anonymize them, reattributing the contribution
to a placeholder user (\eg GitHub's ``@ghost''~\cite{github:privacy}, Reddit/Lobsters'
``[deleted]''~\cite{reddit:privacy, lobsters:privacy}).
%
A system aiming to simplify privacy transformations must thus support a range
of operations, and implement application-defined policy.


The second challenge arises because privacy transformations are
inherently difficult (they require extensive tracing of user identities through
application data schemas), but are also secondary to normal application
functionality.
%
Modifying or deleting data must not compromise application functionality and preserve, for example,
referential integrity of an application's relational database and other application invariants.
%
Furthermore, one application may support several interacting privacy
transformations.


The complexity of implementing basic transformations may have discouraged
developers from supporting more nuanced transformations.
%
This is too bad, because where privacy is concerned, nuance often benefits
users.
%
For example, consider these useful policies that few applications
support:
%

\textbf{Reversible account deletion.}
%
GDPR and related laws focus on irrevocable account deletion, where
a user departing a platform causes permanent deletion of their information.
%
Users may be more likely to proactively protect their privacy if they
can return, \ie if they can reverse their account deletion.
%
Facilitating easy return is also in the web service's commercial interest.
%
A weak form of reversible transformation might preserve user data in the
application's database; though this hides the data from external view,
it leaves it at risk to breaches and does not satisfy most legal requirements
(\eg GDPR).
%
Stronger forms are possible, however: the records required to reverse account
deletion might be in offline storage, or even encrypted and passed to the
user themselves or to their authorized third-party cloud storage.
%

%Furthermore, many applications might wish to employ \textbf{reversible}
%transformations to, for example, support account reactivation instead of
%permanent and irrevocable account deletion.

\textbf{Expiration.}
%
Inactive users' accounts and data can make a data breach much worse.
%
Data expiration policies could proactively anonymize or sanitize user
contributions for long-inactive users.
%
Expiration policies should likely be reversible to support user return.
%

\textbf{Data decay.}
%
Gradual loss of fidelity in old data might strike a balance between the need
to preserve historic information and its inherent dangers.
%
Gradual data decay policies could apply increasingly strict privacy
transformations over time, aging out sensitive but outdated user data.
%

%
These policies, and especially reversible transformations, were central to our
thinking as we developed the data disguise paradigm.
%
But they come with serious technical challenges. Once applications support
multiple privacy trans\-form\-ations, some which users trigger explicitly (\eg
account deletion) and others that happen automatically and apply across users
(\eg data decay), as well as reversible transformations, applications must correctly
handle different interleaving of transformations that touch the same data.
%


\begin{comment}

\eddie{END SECTION HERE?}

%
%
%Finally, practical privacy transformations should require minimal application changes.
%

%
To make these challenges concrete, consider implementing two privacy
transformations in HotCRP~\cite{hotcrp}:
%
(1) \gdpr implements GDPR's ``right-to-be-forgotten'' removal of user
data; and
%
(2) \ca anonymizes reviewers in the database some time after the
reviewing concludes, preventing reviewer identification in case of a data breach.
%
\ca needs to decorrelate reviews and review metadata (\eg review preferences)
from reviewers by ensuring that nothing associated with a review references a real user's
account by foreign key.
%
To maintain referential integrity, the developer must take care to generate fake user profiles
and update foreign key attributes to reference them.
%
The \gdpr transformation, by contrast, must identify all data related to a user and
remove it.%, though taking care to leave behind shared data such as review text.
%

%
In isolation, each transformation may seem simple; however, composing them is tricky.
%
\ca destroys information that links a user's data to their account by changing
that data to reference anonymous, fake users (\eg rewriting the foreign key for a review's
author to point to a fake user profile).
%
If this reviewer user later invokes \gdpr, the transformation needs to know which
data to remove, which is impossible without the original foreign keys.
%
%Due to data dependencies between these transformations, \texttt{GDPR} fails to achieve its desired
%privacy properties, namely the complete removal of the user's data.
%
When applied the other way around (\gdpr, then \ca) the transformations
compose more easily, as there is no need to anonymize deleted data.
%
But it falls to the application developer to write privacy transformation code that handles
both sequences of disguises correctly.
%

%\subsection{Utility of Privacy Transformations}
%If we can address these challenges, privacy transformations have the potential to provide many
%privacy benefits to web applications and their users.
%Widely-used web applications today~\cite{spotify:privacy, amazon:privacy,
%strava:privacy, hotcrp:privacy, wikipedia:privacy, facebook:privacy, twitter:privacy,
%reddit:privacy, github:privacy, lobsters:privacy} mainly support variants of an account
%deletion privacy transformation, as requried by, \eg the GDPR~\cite[Art.\ 17]{eu:gdpr}.
%
These challenges only become more urgent if we consider other desirable privacy transformations.
%
%But many applications would benefit from support for other privacy transformations.
%
%Users and application developers can both benefit from \textbf{more nuanced} privacy policies:
%for example, a confidential paper review system like HotCRP~\cite{hotcrp} keeps users'
%contributions (papers, reviews) after they delete their account to preserve utility for others, but
%could additionally associate each review with a different placeholder to avoid revealing the deleted
%reviewer's identity.
%
%Likewise, contributions with a shared property (\eg Reddit posts with a common tag)
%might be removed entirely to avoid inference attacks, or retained and decorrelated from the
%property (\ie keeping the user's Reddit posts, but removing their tags).
%\eddie{Would like more explanation of that}
%In some cases, a policy could preserve utility while reducing the efficacy of later inference
%attacks by \eg modifying sensitive metadata (\eg tags, creation times).
%
%Some transformations should be performed only on sensitive metadata:
%Similar policies could apply \emph{only if} the property was created by the user (\eg keeping
%the user's Reddit posts, but removing any user-created tags),
% tag like “\#cat” is likely insensitive and useful to preserve, whereas one naming a person is not.
%
%(\eg remove the user's posts on Reddit with tag $t$ if these posts comprise more than 10\%
%of all posts with tag $t$).
%
%Individual users could even specify different preferences for their data.
%
%% A privacy transformation framework is necessary to turn these preferences into concrete
%% operations without undue developer burden.
%
%
Applications could go beyond simple account deletion and support \textbf{data
expiration} policies, which anonymize a user's contributions after the user has been inactive for a
period of time, possibly restoring the user's profile and contributions if the user ever logs back
in.
%
Or the application could support gradual \textbf{data decay} policies that ``decay'' sensitive data
by applying incremental privacy transformations over time.

Furthermore, many applications might wish to employ \textbf{reversible} transformations to, for
example, support account reactivation instead of permanent and irrevocable account deletion.
%
After all, if services must allow users to remove their data on request, it is in the operator's
interest as well as the user's to make it easy for a user to change their mind and return, bringing
their data along.

We envision that a systematic approach to privacy transformations can help address the challenges
identified; if we can overcome these challenges, privacy transformations have the potential to enhance application
privacy beyond what we have discussed here.

\end{comment}
