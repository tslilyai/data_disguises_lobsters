%-------------------------------------------------------------------------------
\section{The Need for a Systematic Approach to Privacy Transformations}
%-------------------------------------------------------------------------------
Properly supporting privacy transformations poses several challenges in practice. First, modifying or
deleting data must not compromise application correctness: for example, privacy transformations must
maintain referential integrity of an application's relational database.
%
Furthermore, one application may apply several privacy transformations. Composing
transformations---applying multiple transformations in sequence---should achieve a state satisfying
every transformations' privacy properties, even when these transformations may share dependencies or
contradict each other.
%
Finally, to make privacy transformations feasible in practice, application changes should be kept to
a minimum.
%

%
To make these challenges concrete, suppose we were to implement the following two privacy
transformations in HotCRP as SQL transactions. (1) \texttt{ConfAnon} implements universal conference
anonymization by reassociating reviews and other metadata (e.g., review preferences) with
placeholder users; this is a form of data decay, preventing leaking reviewer identities for historic data in
the case of a data breach. (2) \texttt{GDPR} implements GDPR's ``right-to-be-forgotten'' removal of user data.
%
\texttt{ConfAnon} decorrelates reviewer profiles from their contributions: no review should reference a
user's account (via a foreign key relationship).
Furthermore, decorrelation must maintain referential integrity, requiring changes to foreign key
attributes to reference some fake user profile.
\texttt{GDPR} must properly remove the leaving user's owned data.

While both of these transformations seem relatively straightforward to implement, composing even
these simple transformations is tricky. \texttt{ConfAnon} destroys links between a user's data and
their profile, making it impossible for a subsequent \texttt{GDPR} application to correctly remove
all of the user's data. 
%
For instance, \texttt{ConfAnon} removes the link between a user and their bookmarked papers by
relinking the bookmarks to anonymous, fake users; if the user later invokes \texttt{GDPR}, the
disguise needs to know which bookmarks to remove, which is impossible without the 
original links.
%
Data dependencies between the two transformations may thus cause the application of multiple
transformations to fail to achieve the desired privacy properties.
%

To solve these challenges, we propose \emph{data disguising}, a systematic approach to privacy
transformations. Data disguising reasons about \emph{disguises}, namely privacy transformations that
follow an organized structure and perform a restricted set of actions (\S\ref{sec:disguises}).
The structured nature of disguises allows a data disguising tool to automatically determine how to correctly
compose multiple disguises, and to check that the end-state achieves the desired privacy properties.
The tool uses the abstraction of \emph{per-user vaults}, which allow the tool to
reintroduce destroyed data from prior disguises as necessary, while still ensuring that user data
remains protected (\S\ref{sec:composition}).

A data disguising tool sits alongside the application database, and exposes an API to the
application allowing it to invoke disguises when necessary. Developers only need to determine when
to invoke the tool's API, and to provide the (necessarily application-specific) schema annotations
that describe each disguise; the tool then performs the necessary physical modifications to the application's
database to transform data as necessary.
