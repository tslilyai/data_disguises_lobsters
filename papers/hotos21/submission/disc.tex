%-------------------------------------------------------------------------------
\section{Discussion}
%-------------------------------------------------------------------------------
%
Data disguising makes it easier to explore and implement privacy transformations.
%
This may support new privacy paradigms like data decay and account reactivation after
deletion.
%
%Data disguising also goes beyond privacy, and can help developers moderate harmful
%content~\cite{contentmod, sasb}, for which they may soon be held legally
%liable~\cite{nytimes:230}.
%
%Instead of targeting users, disguises would target harmful entities (\eg misinformation posts).
%
Data disguising also has some clear limitations.
%
It assumes that transformations on the object graph capture all data that needs transforming;
application snapshots, backups, or external data copies are out of scope, and require
\eg taint tracking techniques~\cite{schengendb}.
%\name assumes that legal and societal incentives will sufficiently convince applications to adopt
%\name, and that
Data disguising also importantly does not provide privacy \emph{guarantees}.
%
Disguises are only as good as the specification a developer writes for them, and we assume
that developers capture application-specific needs to retain, remove, and decorrelate data in
their policies.
%Without removing all application data, even the best-intentioned developer may write policies that
%still retain sensitive information or correlations.
We imagine that data analysis tools and heuristics can help developers improve or catch
errors in disguise specifications, similar to techniques used by Facebook to detect incorrect
deletion~\cite{delf}.
%
Users of applications that use data disguises cannot blindly assume that the disguises provide
complete privacy, although they can perhaps vet application privacy properties via the disguise
specification.
%

\sys currently works only with relational databases, and does not support dynamic
updates to the application schema or disguise specification.
%
Database schema evolution research~\cite{schema:evo} may offer insights
into supporting disguises and disguise reversals after such updates.
%, since \sys cannot restore data removed using the old policy upon resubscription.



\iffalse
\subsection{Limitations}

Like all applications that provide some type of post-unsubscription privacy guarantees, \name assumes
that an adversary who aims to relink decorrelated entities to unsubscribed users, is limited in the
following ways:
\begin{itemize}
    \item An adversary can perform only those queries allowed by the application API,
i.e.,\ can access the application only via its public interface.
%\lyt{Alternatively, an adversary could perform arbitrary queries on some public subset of the
%application schema (e.g., all tables other than the mapping table, or all tables marked with some
%compliance policy); arbitrary queries over the
%entirety of the table are out of scope, unless ``private'' tables are removed and stored by
%unsubscribing users.}

    \item An adversary cannot perform application queries to the past or search web archives:
    information from prior application snapshots may reveal
    exactly how data records were decorrelated from unsubscribed users.

    \item An adversary cannot gain identifying information from re-shared or arbitrary
        user-generated content (for example, a reposted screenshot, or text in user stories or
        comments) that developers have left unghosted.
        \name gives developers the option to specify that unsubscription removes or replaces
        user-generated data and application metadata (e.g., date of postings, database ID columns),
        at the cost of losing this data.
\end{itemize}

User key / encrypted data storage scheme relies on application wanting to store user data (wanting
users to return). Also requires user to have some trusted third party.

Other limitations:
single-threaded;
recovery / eventual consistency during un/resubscribe;
subset of MySql;
no support for schema updates / updates to policy

\subsection{Future work.}
\paragraph{Maintaining Aggregate Accuracy.}
As future work, \sys can optionally allow for entities to be decorrelated without affecting queries which
specifically return aggregation results.

Queries that specifically perform aggregations and return statistical measures (e.g.,
the count of number of users in the system, or the number of stories per user), can return
significantly different results. This affects the utility of the data for the application: for
example, if the application relies on the number of stories per tag to determine hot topics, these
would be heavily changed if ghost tags were created.  In addition, the adversary may learn which
entities are ghosts: for example, an abnormally low count of stories per tag might indicate to an
adversary that these tags are ghost tags.  \lyt{But perhaps it's ok if an adversary can tell what's
a ghost, as long as it can't tell which user each ghost is correlated with.}

\sys stores and separately updates answers to aggregation queries;
these answers are updated when queries update the data tables, and these queries do not read from
the application tables (which may contain ghost records).

An alternate solution might analyze the aggregations performed by application queries, and then
introduce ghosts that lead to the same (or close-enough) aggregation result. For example, if a tag
is split into ghost tags, one per story associated with the tag, but the application still would
like the count of stories for this tag to be high, one of the ghost tags can be populated with many
ghost stories to retain the count of stories per tag.  \sys would remove any ghost stories that
were created upon recorrelation. Note that this solution 1) requires that generating ghosts is
admissible, and 2) may be impossible for certain combinations of aggregations (e.g., queries that
return both the average stories per tag and also the total number of stories).

\lyt{I don't *think* differential privacy really should be applied here, because we'd also face the
issue of running out of privacy budget. Adding noise might ensure that the impact of any one
(real/ghost) user is very little, but it has its own noise/utility tradeoff. Furthermore, the amount
of noise necessary if many ghost users are created might be too large.}

\lyt{Note that the application can also store this information if it wants...}
\fi
