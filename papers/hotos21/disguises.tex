%-------------------------------------------------------------------------------
\section{A Nuanced Policy}
%-------------------------------------------------------------------------------
\label{design:eg}


To further motivate our work, we describe a specific nuanced privacy
transformation that an application we're familiar with should support, namely
\emph{user scrubbing} from paper review software such as HotCRP.
%
In the current HotCRP system, when a user deletes their account, that
transitively deletes all of the user's data, including their reviews.
%
However, the scientific review process generally requires that the text of
reviews be retained for some time (for reference by authors and to justify
decisions).
%
Preserving reviews while deleting their owner requires a nuanced policy
that combines anonymization with deletion.


Specifically, user scrubbing for a user Bea should:
%
(1)~Delete Bea's user account.
%
(2)~Delete information that's only relevant to Bea, such as Bea's review
preferences.
%
(3)~Select or create a set of \emph{placeholder users}.
%
(4)~Modify Bea's other retained data, such as reviews, to refer to the
placeholder users instead of Bea.


After the scrubbing completes, Bea's review texts are still in the system, but
they are linked to different anonymous placeholders, making them difficult to
reassociate with one another or with Bea.
%
Placeholder users have suitable default values; for example, HotCRP users have
a \texttt{disabled} attribute, and the tool sets this for placeholder users,
ensuring they have no permissions and cannot log in.
%
The policy preserves Bea's reviews but does not preserve Bea's relationship to
her submissions. (No placeholder replaces Bea as submission “contact author”.)
%
This choice is the application's; a policy might go even further and
automatically delete a submission whose last author is scrubbed.
%

Already, this policy is better than policies implemented in HotCRP or other systems.
%
However, we hope to improve user experience further by making such policies
\emph{reversible}.
%
For example, a scrubbed user loses the ability to edit their reviews; a
scrubbed user might decide to temporarily reveal their identity to HotCRP in
order to fix a typo.
%
Furthermore, it is easy to imagine automatically applying such policies
after some time (\eg 2 years after the conference), protecting future
research careers by hiding their youthful reviewing sins.
%

%The tool disguises Bob when he invokes HotCRP account deletion.
%%
%Bob provides his secret key to decrypt and read from his vault, allowing the tool
%to retrieve and temporarily reverse any conflicting, reversible transformations
%that would make Bob's deletion fail: in particular, it recorrelates any
%of Bob's decorrelated paper conflicts and then properly removes them.
%%
%(Because the conflicts are removed, the decorrelation is not reapplied).
%%
%Bob wants his account deletion disguise to be reversible, so after performing the
%specified disguise updates, the tool records each transformation in the vault,
%re-encrypts the vault, and forgets the key.
