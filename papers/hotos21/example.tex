\section{An Example Disguise}
\label{design:eg}
%
Consider disguising Bob when he deletes his HotCRP account.
%
Bob would prefer his papers and reviews to be unlinked from his identity.
%
HotCRP, on the other hand, would like to retain paper and review information that other users
find useful.
%
A careful selection of edge and object transformations achieves both.
%

%
To decorrelate reviews from Bob, the disguise \texttt{Decorrelate}s user-to-review edges.
%
This requires transforming Bob into one unique user guise per review.
%
The disguise generates guise attribute values using suitable defaults;
%
in particular, HotCRP users' \texttt{disabled} attribute is set for the guises,
ensuring that guises have no permissions and never review papers.
%

%
Bob is further linked to papers through conflicts, which can indicate coauthorship or a
reviewer conflict.
%
These conflicts are not reassigned to the new guises, since preserved
conflicts could reidentify Bob as the likely author of a review. Thus, the
conflict edges that link a disguised user to papers need a \texttt{Delete} annotation.
%
%% Edge directionality matters here: paper-to-conflict edges should not be removed, as doing so
%% could incorrectly allow conflicted users to see the paper!

The disguise \texttt{Retain}s all other edge types, ensuring that review and paper
artifacts remain correctly linked. Active reviewers still see the correct paper for their reviews,
and active authors see the correct reviews for their papers, albeit potentially authored by
anonymous, unlinkable guises of the original reviewer.
%
%Review and paper guises copy the original object, retaining paper and review information.
%
%\ms{Does this mean duplicate papers/reviews can show up?}

%
Unlike the current real-world HotCRP account deletion policy~\cite{hotcrp:privacy}, which
deletes all objects belonging to Bob, this disguise strikes a balance between decorrelating
Bob's identity from his reviews and papers, and maintaining useful information for other
HotCRP users.
%
Furthermore, it is easy to imagine extending this disguise to automatically disguise Bob
after some time (\eg 2 years after the conference), protecting his future research career
by hiding youthful reviewing sins.
%


