\iffalse
\begin{figure}[t!]
    \centering
    \includegraphics[width=0.5\textwidth]{img/disguises}

    \caption{Disguises move the target object (in this example, a user Bob) from an identity-revealing
    guise to privacy-preserving guises.}
    \label{fig:example}
\end{figure}
\fi

\begin{figure*}[t!]
    \centering
    \footnotesize
\begin{tabular}{@{}c|c|c|c@{}}
\textbf{User Transformation Spec} & \textbf{User Object} & \textbf{Guise 1} &
    \textbf{Guise 2} \\
\begin{lstlisting}[language=Rust]
"id":       IDAttribute,
"name":     Gen(Random),
"active":   Gen(Default(false)),
"darkmode": CopyAll,
"notifs":   CopyOnce+Gen(Default(false)),
"tag_id":   GenForeignKey,
\end{lstlisting}
    &
\begin{lstlisting}[language=Rust]
"id":       19,
"name":     BobParr,
"active":   true,
"darkmode": false,
"notifs":   true,
"tag_id":   11
\end{lstlisting}
&
\begin{lstlisting}[language=Rust]
"id":       295,
"name":     MrIncredible,
"active":   false,
"darkmode": false,
"notifs":   true,
"tag_id":   81483
\end{lstlisting}
&
\begin{lstlisting}[language=Rust]
"id":       918,
"name":     SuperDad,
"active":   false,
"darkmode": false,
"notifs":   false,
"tag_id":   15592
\end{lstlisting}
\end{tabular}
    \caption{Creating two guises of an example user (of a synthetic application schema).}
    \label{fig:guises}
\end{figure*}

