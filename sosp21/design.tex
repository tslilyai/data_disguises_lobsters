%-------------------------------------------------------------------------------
\section{Data Disguising}
%-------------------------------------------------------------------------------
\begin{figure}[h!]
    \footnotesize
    \begin{tabular}{p{0.5\textwidth}}
Keep user contributions visible to the intended audience, but anonymized by reassociation with a global
placeholder user.
\begin{lstlisting}[language=Rust]
EdgeTransforms {
  "User-Contrib": Retain
}
EntityTransforms {
  "User":    Gen(Default("placeholder")) 
  "Contrib": Copy
}
\end{lstlisting}

    \\

Keep user's contributions, but associate each contribution with a different user account.
\begin{lstlisting}[language=Rust]
EdgeTransforms {
  "User-Contrib": Decorrelate
}
EntityTransforms {
  "User":    Gen(Random)
  "Contrib": Copy
}
\end{lstlisting}

\\
Retain users' contributions with a shared property, but decorrelate the contribution
from the property \emph{only if} the property was created by
the user.
\begin{lstlisting}[language=Rust]
EdgeTransforms {
  "User-Contrib":     Decorrelate
  "Contrib-Property": Decorrelate(Filter(fn(p,c) {
                        p.user == c.user
                      })
}
EntityTransforms {
  "User":     Gen(Random)
  "Contrib":  Copy
  "Property": CopyOnce+Gen(Default(None)) 
}
\end{lstlisting}

        \\

Remove users' contributions if they comprise more than $p$ percent of the contributions
with a shared property.
\begin{lstlisting}[language=Rust]
EdgeTransforms {
  "User-Contrib":     Decorrelate
  "Contrib-Property": Decorrelate(Sensitivity(p))
}
EntityTransforms {
  "User":     Gen(Random)
  "Contrib":  Copy
  "Property": CopyOnce+Gen(Default(None)) 
}
\end{lstlisting}
\end{tabular}
\caption{Example masks for a variety of privacy transformations (only relevant parts of mask
specification shown).}
\label{fig:masks}
\end{figure}


The key idea behind \emph{data disguising} is to associate multiple \emph{guises} with a target
data object. Guises vary in how they reveal identities or preserve privacy.
%
Objects move between different guises by means of privacy transformations.
%
%Figure~\ref{fig:example} illustrates this with the example of user account deletion.
%
When his account is active, user Bob's profile is associated with his true identity and all his
contributions to the site (an identity-revealing guise).
%
When Bob deletes his account, his profile and contributions move to different, privacy-preserving
guises: his name has been anonymized, his email address has been redacted, and his contributions
have been decorrelated and attributed to individual, unidentified user guises.
%

%
Data disguising builds on the existing structure of web applications.
%
Web applications are often structured as object graphs, either explicitly~\cite{tao, delf},
through an object-relational model (ORM)~\cite{orm}, or implicitly via foreign keys (edges)
between tables (vertices) in relational databases.
%
Data disguises transform this object graph.
%

%
The application developer writes a disguise specification for each privacy transformation needed
in the application.
%
This specification is a declarative statement similar to a relational schema, with entries for
graph vertices (objects) and directed edges (relationships between pairs of objects)
to be transformed (see \S\ref{sec:policies}).
%
We assume that:
\begin{enumerate}[nosep]
  \item developers use their domain knowledge to write correct and complete disguises;
  \item application code handles the different guises appropriately (\eg in
    displaying them); and
  \item different guises of the same object have the same structure (\eg they can be
    rows in the same table).
\end{enumerate}
%
A data disguising tool takes the disguise specification and turns it into storage operations that
apply the transformation (disguise) or its reverse (reveal).
%
At any given moment, an application's data object graph comprises a mix of
identity-revealing guises and privacy-preserving ones. Privacy transformations split
and combine individual guises when triggered.

%-------------------------------------------------------------------------------
\section{Specifying Data Disguises}
%-------------------------------------------------------------------------------
\label{sec:policies}

A data disguise is written once by the developer, and applied in the context of a specific target
object to disguise, such as a departing user, and a specific instance of the object graph.
%
The disguise consists of context-specific transformations that turn objects into one or
more guises (or remove the object completely). 
%Developers specify disguises in two parts. The
%first specifies how to create guises of a given object type (\S\ref{sec:guises}). The second specifies in which
%contexts objects should be transformed into guises or removed (\S\ref{sec:context}).
%
A disguising tool then applies the disguise by first determining the current context of objects in the object
graph, and then applying the appropriate transformation for that context.
%\ie from what type of edge, and sensitivity context.

\subsection{Context-Specific Transformations}
\label{sec:context}

Developers specify \textbf{contexts}, and transformations to the objects that match each context. 
%
% SOURCE = CHILD
% DEST = PARENT
%
%In the following, we refer to edge types in the object graph, which have a \emph{source} object type and a \emph{destination} object type,
%where the source references the destination (\eg via a foreign key column in a relational database).
%
A context consists of:
\begin{enumerate}[nosep]
    \item An object type, and
    \item A set of constraints on the objects' attributes 
\end{enumerate}
\noindent Given an instance of the object graph, a set of objects in the graph
satisfy the context if (1) the objects are of the specified object type and (2) the objects'
attributes satisfy the constraints.
%
We assume that objects have identifier attributes (\eg \texttt{id} in Figure~\ref{fig:guises}); and
some attributes are reference attributes that form edges in the graph to referred objects (\eg
\texttt{tag\_id} is a foreign key constraint to tag objects).

\vspace{\baselineskip}
\noindent\textbf{Object Attribute Constraints.} 
\begin{itemize}[nosep]
\item \textbf{Reference content} constraints on reference attributes are functions that
    take both the object and the referenced object as input, and returns true or false. 
    %
    %For example, perhaps we want to decorrelate paper-tag edge \emph{only if} the tags were created by the user who is leaving.
\item \textbf{Value content} constraints on non-reference (value) attributes are functions that take the object as
input, and returns true or false.  
\item \textbf{Sensitive reference} constraints on reference attributes match only objects that refer
    to sensitive objects (\ie connects back to the target object of the disguise).
\end{itemize}

\vspace{\baselineskip}
\noindent\textbf{Single-Guise Transformations.}
For every context, developers specify to either remove matching objects (which also removes all
descendents---objects that refer to these objects---as well), or to transform each object into
a single guise. For the latter, the developer specifies a single-guise transformation at
attribute-level granularity. 
%

Each non-reference (value) attribute has an associated value transformation:
\begin{itemize}[nosep]
    \item \textbf{Copy object content}: the guise copies the object's attribute value.
    \item \textbf{Generate new content}: the developer specifies a function that takes the object
        attribute value as input, and generates a new guise value (\eg hashing the value, generating
        random values, or setting the value to some default).
        %
        The guise attribute value depends at most on the value of the object's attribute.
\end{itemize}

Similarly, each reference attribute has an associated edge transformation:
\begin{itemize}[nosep]
    %
    \item \textbf{Retain} the reference (attribute is copied).
    %
    \item \textbf{Decorrelate} from the reference: create a reference guise for the referenced
        object (\S\ref{sec:reference_guises}); the guise's reference attribute refers to the
        created reference.
\end{itemize}

Identifier attributes are always unique and random values.

\subsection{Reference Guise Creation}
\label{sec:reference_guises} 
%
Decorrelating references requires creating guises for the referenced objects. One referenced object may turn into many
reference guises (one for each context-matching object that all should be decorrelated).  Because
multiple reference guises may be created to decorrelate objects from the reference object,
developers needs to specify how to transform the reference object into a guise \emph{for each guise
created}.

%
Developers specify \textbf{many-guise transformations} for each object type that may be a reference;
these transformations are \emph{independent} of any single context, because many-guise
transformations only occur for the purpose of decorrelation, and can occur in many contexts.
%
Figure~\ref{fig:guises} shows an example, producing guises for user objects.

A many-guise transformation for an object type consists of a single-guise transformation (as
described above, \ie copying or generating new object attributes); and additional per-attribute
annotations that specify whether the transformation should be applied to create all guises, or
applied to create all but $n$ guises. In the latter case, $n$ guises will simply copy the
object's attributes, and the other guises will transform the object's attributes according to the
single-guise transformation.

Note that creating reference object guises may also require recursively creating guises for references of
the reference object, using the (context-independent) many-guise transformation for those ancestral reference
object types.
%
%This enables the application to retain the original object semantics (\eg a count of how many
%users want notifications) without creating duplicates.
%

\subsection{Threshold Constraints}
\label{sec:threshold} 

Developers may want to transform attributes of only a subset of all context-matching objects when
changing them into guises.  
For each context, developers can associate a \emph{threshold} with each attribute of the context's
object type. Consider all objects of the context's object type that share the same attribute value
(\eg a referenced tag, or location value); this set contains both context-matching objects, and
objects ouside the context. The threshold determines the maximum proportion of these objects that
are both context-matching and should not transform the attribute value when becoming a guise.

For example, let the context match only papers written by the targeted user. These papers should be
decorrelated from a tag only enough to ensure that these sensitive papers comprise less than 10\% of
all papers with that tag: the developer thus specifies a threshold of 10\% for the tag reference
attribute.

All context-matching objects falling above the threshold transform the attribute according to the
associated attribute transformation, while all objects falling below the threshold leave the
attributes unchanged. 
%

Thresholds can also be associated with a set of
attributes, for example determining the maximum proportion of papers referencing a particular tag
\emph{and} a particular location attribute value that can match the context.

%same attribute(s), see k-anonymity?
%     \lyt{Threshold is sort of separate / above --- you first constrain the edges, and then you want
%     to determine how many of these edges fall above the threshold out of the unconstrained
%     edges}
%%
    %For example, the proportion of papers that are connected tags.
    %
    %Figure~\ref{fig:algo} illustrates a constraint that sets a threshold (0.5) for the
    %maximum proportion of papers connected to a tag that a targer-user's papers can make up.
    %Because the target user's papers make up more than 0.5 of all papers connected to the tag,
    %\sys decorrelates the sensitive paper sources from the tag until the user's papers fall
    %below or at the threshold.
    %%
    %If the threshold is negative, then \sys decorrelates or deletes \emph{all} sources of that
    %destination, including sources that were not traversed from the target object.

\lyt{TODO: What if context matches after you decorrelate? (infinitely matching contexts?)}

\subsection{An Example Disguise}
\label{design:eg}
%
Consider disguising Bob when he deletes his HotCRP account.
%
Bob would prefer his papers and reviews to be unlinked from his identity.
%
HotCRP, on the other hand, would like to retain paper and review information that other users
find useful.
%
A careful selection of edge and object transformations achieves both.
%

%
To decorrelate reviews from Bob, the disguise \texttt{Decorrelate}s user-to-review edges.
%
This requires transforming Bob into one unique user guise per review.
%
The disguise generates guise attribute values using suitable defaults;
%
in particular, HotCRP users' \texttt{disabled} attribute is set for the guises,
ensuring that guises have no permissions and never review papers.
%

%
Bob is further linked to papers through conflicts, which can indicate coauthorship or a
reviewer conflict.
%
These conflicts are not reassigned to the new guises, since preserved
conflicts could reidentify Bob as the likely author of a review. Thus, the
conflict edges that link a disguised user to papers need a \texttt{Delete} annotation.
%
%% Edge directionality matters here: paper-to-conflict edges should not be removed, as doing so
%% could incorrectly allow conflicted users to see the paper!

The disguise \texttt{Retain}s all other edge types, ensuring that review and paper
artifacts remain correctly linked. Active reviewers still see the correct paper for their reviews,
and active authors see the correct reviews for their papers, albeit potentially authored by
anonymous, unlinkable guises of the original reviewer.
%
%Review and paper guises copy the original object, retaining paper and review information.
%
%\ms{Does this mean duplicate papers/reviews can show up?}

%
Unlike the current real-world HotCRP account deletion policy~\cite{hotcrp:privacy}, which
deletes all objects belonging to Bob, this disguise strikes a balance between decorrelating
Bob's identity from his reviews and papers, and maintaining useful information for other
HotCRP users.
%
Furthermore, it is easy to imagine extending this disguise to automatically disguise Bob
after some time (\eg 2 years after the conference), protecting his future research career
by hiding youthful reviewing sins.
