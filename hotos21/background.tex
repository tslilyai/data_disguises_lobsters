\section{Subscription Support in Applications Today}
\begin{table*}[t!]
    \centering
    \footnotesize
\begin{tabular}{@{}rccccl@{}}
\multicolumn{1}{c}{}       &
\textbf{Automated} & \textbf{Retains Data} & \textbf{Removes User Identifiers} & \textbf{Resubscription} &  
  \\ \cmidrule(r){1-5}
\textbf{Lobsters}~\cite{lobsters:privacy},
    \textbf{Facebook/Instagram/Twitter Account Deletion}~\cite{facebook:privacy, twitter:privacy}
  & \CIRCLE  & \CIRCLE & \LEFTcircle &    &  
  \\ \cmidrule(r){1-5}
    \textbf{Facebook/Instagram/Twitter Account Deactivation}~\cite{facebook:privacy, twitter:privacy}
  & \CIRCLE & \CIRCLE & & \CIRCLE    &  
  \\ \cmidrule(r){1-5}
\textbf{Reddit}~\cite{reddit:privacy}, \textbf{GitHub}~\cite{github:privacy}        
  & \CIRCLE  & \CIRCLE  & \CIRCLE  &    
  &  \\ \cmidrule(r){1-5}
\textbf{PrestaShop}~\cite{prestashop:privacy}, \textbf{Spotify}~\cite{spotify:privacy}
  &   & \CIRCLE  & \CIRCLE  &    &  
  \\ \cmidrule(r){1-5}
 \textbf{Amazon}~\cite{amazon:privacy}, \textbf{HotCRP}~\cite{hotcrp:privacy},
    \textbf{Strava}~\cite{strava:privacy}, \textbf{Hacker News}~\cite{hackernews:privacy}
    &   & \CIRCLE & \LEFTcircle &    &  
  \\ \cmidrule(r){1-5}
\end{tabular}
 \caption{Unsubscription characteristics of a range of web applications.
    \LEFTcircle~indicates that the privacy policy permits some data to display the user 
    identifier post-unsubscription, but makes efforts to remove it from other data.}
    \label{tab:apps}
\end{table*}

%Unsubscription and resubscription support is often impossible
%or cumbersome (users must contact the developers or customer support directly in order to
%unsubscribe)~\cite{jdm}.  Developer's manual unsubscription implementions are often coarse-grained
%and fail to completely de-identify the user. Furthermore, no existing application (to our knowledge)
%supports privacy-preserving unsubscription with the possiblity of resubscription. Application
%unsubscription thus permanently deletes years or even decades of accumulated application data,
%costing both users and the web application, and creating little incentive for the application to
%support easy unsubscription. 
%including social media and news-feed applications, e-Commerce
%applications, and other specialized services.
%
Legal incentives (GDPR/CCPA) and user interests have pushed applications to support at least
unsubscription. To see how large a gap remains between current data ownership paradigms and \name, we
study support for unsubscription and resubscription in a range of widely-used enterprise and
open-source applications. 
%
Table~\ref{tab:apps} summarizes four characteristics of each application: 
\begin{enumerate}
    \item Is unsubscription automated, or does the user need to email or call the
application developers or customer service directly? 
    \item Does the application retain any of the unsubscribing user's data, or is all data removed?
    \item Does the application promise to remove all traces of user identifiers (\eg username, user
    IDs, emails) either by removing it or replacing it with an anonymized identifier?  
    \item Can users resubscribe, \ie regain control of their account and account data?
\end{enumerate}
Fewer than half of the services surveyed provide automated unsubscription; all others require the
user to email or call support, or actively make dissuade users from unsubscribing~\cite{nytimes:amazonsub}. For a more comprehensive study,
JustDeleteMe~\cite{jdm} provides a growing list of applications (currently 868) and rates ease
of unsubscription (account deletion) for each; 27\% of applications required users to manually
contact the company or developers, and 12\% did not support unsubscription. 

All applications retain some of the unsubscribed user' data: Facebook retains copies of private
messages for other users in the conversation; Reddit retains all posts and comments made by the
user; Lobsters retains the user's unique username and logs of the user's actions; e-commerce sites
retain a user's order history; and Strava retains geolocation data and any ``necessary''
information. All applications specify that any ``required'' information can be retained, for legal
or applications correctness purposes (e.g., to process payments, or to keep the coherence of private
messaging threads).  

Many applications state that they will remove any user identifiers from retained data.  For example,
Reddit, PrestaShop, and GitHub explicitly state that all remaining posts, orders, and comments will
be decorrelated and reassociated with a global placeholder identifier.  However, as indicated in
Table~\ref{tab:apps}, many applications either leave some identifiers unchanged (\eg private
messages in Facebook and Twitter), or fail to specify whether they rewrite identifiers at all for
some or all of the retained information.

Of the applications surveyed, only Facebook/Instagram and Twitter support some form of
resubscription via account reactivation. However, account deactivation only hides user-added content from
other users: all user identifiers remain visible in the application database. All other forms
of unsubscription across all applications permanently remove access to the user's account; even if
the application retains user data (such as orders on Amazon, or posts on Reddit), the user cannot
regain ownership of these records.

%Reddit~\cite{reddit:privacy} account deletion reassociates user data with a global ghost user, but
%all user posts, comments, and messages that have not been explicitly been removed by the user remain
%present but anonymized in the system. Users have no ability to resubscribe and regain
%ownership of these pieces of data that remain in the system. 
%
%Hacker News~\cite{hackernews:privacy} allows users to submit requests via email or phone to delete
%personal information, but reserves the right to refuse to delete content or remove associations with
%a user's identity.  Users cannot resubscribe and regain access to their prior profiles.
%
%Lobsters has no privacy policy~\cite{lobsters:privacy}. All user actions and moderations are
%%logged, enforcing Lobster's culture of transparent user accountability. A user can request to delete
%their account, which deletes only a users' private messages and negatively-scored comments; all
%stories and comments can optionally be reassociated with a global ghost user instead of the original
%user. In all cases, Lobsters privately retains the user's unique username, and leave log entries of users'
%actions unmodified.
%
%Amazon~\cite{amazon:privacy} allows users to request to delete their personal data by contacting
%Customer Service, with the warning that services become limited or unavailable when the user does
%so. There is no option to unsubscribe or delete an Amazon account from a user's Amazon account
%management page.  A user can also choose to permanently close her account~\cite{amazon:close}. If
%so, Amazon specifies that content is deleted from Photos and Drive, but potentially retains all other types
%of data, such as order history. Even though Amazon can retain profile data, a user can never
%again regain access to her account, losing access to purchased e-books or movies, gift card
%balances, Prime shipping, and more.
%
%%PrestaShop~\cite{prestashop:privacy} is an open-source e-commerce web service, and has explicit GDPR
%policies for its merchants when a customer requests to delete their account: their personal account
%details (age, email, address) are removed, but order invoices and abandoned carts are transferred to
%an anonymous account. Users cannot resubscribe.
%
%HotCRP's current privacy policy~\cite{hotcrp:privacy} allows site managers (e.g.,\ program chairs)
%to indefinitely store and distribute submissions and reviews. Each HotCRP.com user has an associated
%global profile and a profile for every HotCRP.com site, and must contact the HotCRP maintainers
%directly to remove these profiles.
%
%GitHub~\cite{github:privacy} reassociates some retained user content (contributions, comments on
%issues) with a global ghost user, and removes all other account information. However, user email
%addresses associated with commits remain unchanged. A user cannot resubscribe.
%
%Spotify~\cite{spotify:privacy} allows users to close their accounts, at the cost of permanently
%losing all account data (playlists, followers, etc.) Spotify promises to ``delete or anonymise...
%personal data so that it no longer identifies [the user], unless [they] are legally allowed or
%required to maintain certain personal data.'' Spotify does not specify exactly what data is
%retained.
%
%Strava~\cite{strava:privacy} users can request account deletion, which ``permanently and
%irreversibly'' deletes all personal data. Resubscription is not supported. Any content that is
%shared with others remains unanonymized after account deletion, as well as any ``necessary''
%information. Strava also retains unnecessary user data such as geolocation information, and promises
%to de-identify this data.

