\subsection{Background}

\begin{table*}[]
    \centering
    \footnotesize
\begin{tabular}{@{}rccccl@{}}
\multicolumn{1}{c}{}       &
\textbf{Automated} & \textbf{De-identifies} & \textbf{Retains Data} & \textbf{Resubscription} &  
  \\ \cmidrule(r){1-5}
\textbf{Facebook/Instagram Account Deletion}~\cite{facebook:privacy} 
  & \CIRCLE  & \LEFTcircle & \CIRCLE  &    &  
  \\ \cmidrule(r){1-5}
  \textbf{Facebook/Instagram Account Deactivation}~\cite{facebook:privacy} 
  & \CIRCLE & & \CIRCLE  & \CIRCLE    &  
  \\ \cmidrule(r){1-5}
\textbf{Twitter Account Deletion}~\cite{twitter:privacy}  
  & \CIRCLE &  \LEFTcircle & \CIRCLE  &    &  
  \\ \cmidrule(r){1-5}
\textbf{Twitter Account Deactivation}~\cite{twitter:privacy} 
  & \CIRCLE  &    & \CIRCLE  & \CIRCLE    &  
  \\ \cmidrule(r){1-5}
\textbf{Reddit}~\cite{reddit:privacy}        
  & \CIRCLE  & \CIRCLE  & \CIRCLE  &    
  &  \\ \cmidrule(r){1-5}
 \textbf{Hacker News}~\cite{hackernews:privacy}
  &   & \LEFTcircle & \CIRCLE  &    &  
  \\ \cmidrule(r){1-5}
 \textbf{Lobsters}~\cite{lobsters:privacy}
  & \CIRCLE  & \LEFTcircle & \CIRCLE  &    &  
  \\ \cmidrule(r){1-5}
 \textbf{Amazon}~\cite{amazon:privacy}        
  &   & \LEFTcircle    & \CIRCLE  &    &  
  \\ \cmidrule(r){1-5}
 \textbf{PrestaShop}~\cite{prestashop:privacy}       
  &   & \CIRCLE  & \CIRCLE  &    &  
  \\ \cmidrule(r){1-5}
 \textbf{HotCRP}~\cite{hotcrp:privacy}        
  &   & \LEFTcircle    & \CIRCLE  &    &  
  \\ \cmidrule(r){1-5}
 \textbf{GitHub}~\cite{github:privacy}        
  & \CIRCLE  & \CIRCLE  & \CIRCLE  &    &  
  \\ \cmidrule(r){1-5}
 \textbf{Spotify}~\cite{spotify:privacy}       
  &   & \CIRCLE  & \CIRCLE  &    &  
  \\ \cmidrule(r){1-5}
 \textbf{Strava}~\cite{strava:privacy}        
  &   & \LEFTcircle    & \CIRCLE  &    &  
  \\ \cmidrule(r){1-5}
\end{tabular}
 \caption{The characteristics of unsubscription in a range of web applications.
    \LEFTcircle indicates that the application permits some user data to remain identifiable
    post-subscription, but de-identifies other data in its privacy policy.}
    \label{tab:apps}
\end{table*}

We study support for unsubscription and resubscription in a range of web
services, including enterprise and open-source social media and news-feed applications such as
Facebook, Instagram, Twitter, Reddit, Hacker News, and Lobsters; e-Commerce applications such as
Amazon and PrestaShop; and other specialized services such as HotCRP, GitHub, Spotify, and Strava.
Table~\ref{tab:apps} summarizes the unsubscription characteristics of each application. We focus on
four characteristics: 
\begin{enumerate}
    \item Is unsubscription automated, or does the user need to email or call the
application developers or customer service directly? 
    \item Does the application's privacy policy (or
        code, when available) promise to de-identify all the unsubscribing user's data, 
        either via deletion or
        decorrelation of the data from the user's identity? 
    \item Does the application retain any of the unsubscribing user data, or is all data removed?
    \item Does unsubscription allow for resubscription, in which unsubscribed users can regain control of their
        account and account data?
\end{enumerate}

Fewer than half of the services surveyed provide automated unsubscription; all others require the
user to email or call the application maintainers or support. This indicates the difficulty of
automating unsubscription correctly, and consequently makes it more  
inconvenient for users to unsubscribe.

All applications retain some user information in some form: for example, Facebook retains copies of
private messages for other users in the conversation; Reddit retains all posts and comments made by
the user; Lobsters retains the user's unique username and logs of the user's actions; e-commerce
sites retain a user's order history; and Strava retains geolocation data and any
``necessary'' information. All applications specify that any
``required'' information can be retained, for legal or applications correctness purposes (e.g., to
process payments, or to keep the coherence of private messaging threads).

Many applications state that some portions of the retained data will be decorrelated from the
unsubscribed user's identity. For example, Reddit, PrestaShop, and GitHub explicitly state that all
remaining posts, orders, and comments will be reassociated with a global placeholder user.
However, as indicated in Table~\ref{tab:apps}, many applications either leave some data identifiable
(e.g., private messages in Facebook and Twitter), or fail to specify whether some or all of the ``required''
information will be de-identified.

Of the applications surveyed, only Facebook/Instagram and Twitter support some form of
resubscription. However, unsubscribing via account deactivation only hides user-added content from
other users: all user data remains in the system without being de-identified.  All other forms of
unsubscription across all applications permanently remove access to the user's account; even if the
application retains user data (such as orders on Amazon, or posts on Reddit), the user cannot regain
ownership of these records.

\subsection{Challenges of Unsubscription and Resubscription}
While many web applications make obvious efforts to comply with legal requirements (GDPR and CCPA),
unsubscription clearly requires more than merely deleting a user's data. Many of these services
require that some information remains, in transformed or anonymized form, after a user departs, for
necessary application or legal reasons. 
%
Developers must determine how to selectively retain, decorrelate, and remove an unsubscribing user's data,
without interfering with application correctness, and while preserving the user's privacy. They must 
implement the database transformations required to correctly unsubscribe a user while
preserving application semantics, and continously maintain this feature. 
%
From the lack of automatic unsubscription, the vague promises to decorrelate retained data, and the
failure to support privacy-compliant resubscription, it is clear this is no easy task.
%

We use HotCRP as an example to make these challenges concrete. HotCRP is open-source software for
managing conference reviews and paper submissions.  Reviewers and authors create user accounts to
submit or review papers, and HotCRP supports program committee conflict detection, paper rebuttals
and comments, review delegation, and other features necessary for the conference review process.

HotCRP's current privacy policy~\cite{hotcrp:privacy} allows site managers (e.g.,\ program chairs)
to indefinitely store and distribute submissions and reviews.  This is by request of its users:
users want to be able to reference prior conference reviews and paper metadata.  However, many
users---particularly reviewers---would prefer that their blinded reviews never become publically
associated with their identity in the case of a data compromise. Currently, the only way to ensure
this is to contact the HotCRP maintainers directly to manually remove their user profile: the HotCRP
developers considered automating support for unsubscription and found it too cumbersome to
implement. 
%
\lyt{(This isn't exactly what I want to say---sorry to put words in your mouth, Eddie!---but something like this is
probably useful.)} 
%
When user accounts are removed, every piece of data ever associated with that
user is deleted from the system: this model clearly would not work if all users freely
unsubscribed at will, as the system would be left with no useful data for subscribed users.

The desired HotCRP unsubscription behavior would both retain useful data for HotCRP and its users,
and satisfy the desire of HotCRP users to de-identify their data. In particular, the HotCRP
developers should be able to specify that unsubscription deletes only the user profile, and
\emph{decorrelates} the links between the user and their reviews (and other user-associated data)
while retaining the blinded reviews and data for other users to view. While correlations between
users and their reviews are obviously identifying, unsubscription must also handle more subtle
correlations: a user's reviewer conflicts can identify the user's affiliation and recent
collaborations, and in many cases, the user themselves. Implementing unsubscription that handles
these complex data relationships correctly would not be trivial.

%%%%%%%%%%%%%%%%%
To solve these challenges, we present \sys, a system that provides
abstractions and mechanisms to automate unsubscription and resubscription, and 
help developers of databased-backed web applications achieve correct, privacy-compliant user
unsubscription and resubscription without onerous labor.

\sys is based on the key observation that users' data can leak identifying information in two ways:
(1) directly via their content, and (2) indirectly via the structural \emph{correlations} they have
to other data in the system. 
Unsubscription must properly handle these information leaks, either through the removal of such data
or the \emph{decorrelation} of the data from the user's identity, and do so while preserving
application correctness.

%
Using \sys, developers pinpoint exactly which data and correlations may be identifying, and
specify the post-unsubscription state of this data. Developers provide a declarative
\emph{unsubscription policy} that indicates how the database contents need to change to meet
de-identification requirements on user unsubscription.
%
\sys then turns this policy into a set of concrete, executed database operations that remove,
anonymize, and structurally \emph{decorrelate} user data. \sys explicitly returns decorrelated or
removed user data to the unsubscribing user, and deletes it from the application.
%

%
When a user resubscribes, \sys re-imports missing data and \emph{recorrelates} the data with the
user account and other system data, restoring the user to her original subscribed state as much as
possible.

Furthermore, \sys transparently supports unsubscription and resubscription while still achieving performance
comparable to today’s widely-used databases and requiring no modification of application schemas.

%Reddit~\cite{reddit:privacy} account deletion reassociates user data with a global ghost user, but
%all user posts, comments, and messages that have not been explicitly been removed by the user remain
%present but anonymized in the system. Users have no ability to resubscribe and regain
%ownership of these pieces of data that remain in the system. 
%
%Hacker News~\cite{hackernews:privacy} allows users to submit requests via email or phone to delete
%personal information, but reserves the right to refuse to delete content or remove associations with
%a user's identity.  Users cannot resubscribe and regain access to their prior profiles.
%
%Lobsters has no privacy policy~\cite{lobsters:privacy}. All user actions and moderations are
%%logged, enforcing Lobster's culture of transparent user accountability. A user can request to delete
%their account, which deletes only a users' private messages and negatively-scored comments; all
%stories and comments can optionally be reassociated with a global ghost user instead of the original
%user. In all cases, Lobsters privately retains the user's unique username, and leave log entries of users'
%actions unmodified.
%
%Amazon~\cite{amazon:privacy} allows users to request to delete their personal data by contacting
%Customer Service, with the warning that services become limited or unavailable when the user does
%so. There is no option to unsubscribe or delete an Amazon account from a user's Amazon account
%management page.  A user can also choose to permanently close her account~\cite{amazon:close}. If
%so, Amazon specifies that content is deleted from Photos and Drive, but potentially retains all other types
%of data, such as order history. Even though Amazon can retain profile data, a user can never
%again regain access to her account, losing access to purchased e-books or movies, gift card
%balances, Prime shipping, and more.
%
%%PrestaShop~\cite{prestashop:privacy} is an open-source e-commerce web service, and has explicit GDPR
%policies for its merchants when a customer requests to delete their account: their personal account
%details (age, email, address) are removed, but order invoices and abandoned carts are transferred to
%an anonymous account. Users cannot resubscribe.
%
%HotCRP's current privacy policy~\cite{hotcrp:privacy} allows site managers (e.g.,\ program chairs)
%to indefinitely store and distribute submissions and reviews. Each HotCRP.com user has an associated
%global profile and a profile for every HotCRP.com site, and must contact the HotCRP maintainers
%directly to remove these profiles.
%
%GitHub~\cite{github:privacy} reassociates some retained user content (contributions, comments on
%issues) with a global ghost user, and removes all other account information. However, user email
%addresses associated with commits remain unchanged. A user cannot resubscribe.
%
%Spotify~\cite{spotify:privacy} allows users to close their accounts, at the cost of permanently
%losing all account data (playlists, followers, etc.) Spotify promises to ``delete or anonymise...
%personal data so that it no longer identifies [the user], unless [they] are legally allowed or
%required to maintain certain personal data.'' Spotify does not specify exactly what data is
%retained.
%
%Strava~\cite{strava:privacy} users can request account deletion, which ``permanently and
%irreversibly'' deletes all personal data. Resubscription is not supported. Any content that is
%shared with others remains unanonymized after account deletion, as well as any ``necessary''
%information. Strava also retains unnecessary user data such as geolocation information, and promises
%to de-identify this data.

