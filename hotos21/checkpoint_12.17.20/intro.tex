%-------------------------------------------------------------------------------
\section{Introduction}
%-------------------------------------------------------------------------------

%
Web application operators increasingly face pressure from legal requirements
that grant users new rights over their data.
%
For example, both the European Union's General Data Protection Regulation
(GDPR)~\cite{eu:gdpr} and California's Consumer Privacy Act
(CCPA)~\cite{ca:privacy-act} grant users the right to request erasure of
information related to them.
%
These laws, for the first time, require companies to support \emph{unsubscription}
of users from their services on request.
%

%
Unsubscription requires more than merely deleting a user's data: correct operation
of a service often requires that some information remains, in transformed or
anonymized form, after a user departs.
%
\ms{Give an example!}
%
It is difficult for developers to determine and implement the database
transformations required to correctly unsubscribe a user, and to continously
maintain this feature.
%
Consequently, the majority of web services today either lack unsubscription support
(requiring manual labor to handle \eg GDPR deletion requests), or leave substantial
information behind that potentially harms user privacy.
%

%
In this paper, we describe \sys, a system that provides abstractions and mechanisms
that help developers of databased-backed web applications achieve correct,
privacy-compliant user unsubscription without onerous labor.
%
Application developers using \sys specify a declarative \emph{unsubscription policy},
which indicates how the database contents need to change to meet de-identification
requirements on user unsubscription.
%
\sys then turns this policy into a set of concrete database operations that remove,
anonymize, and structurally decorrelate user data, and executes them.
%
But \sys goes beyond simply unsubscribing users: it makes it possible for users to
switch between a privacy-preserving unsubscribed mode and an
identity-revealing subscribed mode at any time without permanently losing their data.
%
This facilitates new web service designs, such as the user granting a time-limited
``lease'' of data to a web service instead of having a permanent service account.
%
When a user resubscribes, \sys both re-imports missing data and
\emph{recorrelates} the remnants of the user's prior data with the user account,
restoring the user to her original subscribed state as much as possible.
%

\sys makes the key observation that a user's data entities (e.g.,\ posts and upvotes) can leak identifying
information in two ways: (1) directly via its content, and (2) indirectly via the correlations it has to
other data entities in the system.
Correlations can range from obviously identifying (e.g.,\ posts correlated
with a user clearly belong to that user), to subtly perilous: posts correlated with a
particular user-generated tag most likely belong to the tag's author, and posts liked by the
same group of users likely belong to a friend of the group.

Many web applications (e.g.,\ Lobst.rs and Reddit) have come up with solutions to de-identify direct
content by anonymizing unique identifiers, leaving arbitrary user-generated content out of scope.
Other applications (e.g.,\ Facebook and Twitter) delete (most of the) user's data. The former
optimizes the amount of data the application keeps, but also retains all correlations between data
entities; the latter can lead to confusing semantics for the application (e.g.,\ nonsensical comment
threads) and the loss of useful application data. In both cases, accounts cannot be restored after
deletion.
\lyt{Applications such as Facebook or Twitter do support user account deactivation by hiding (most) of the
user's data, but retain all user data with their identifiers in their databases.}

Developers using \sys can specify policies that both anonymize unique identifiers, and remove
identifying information stemming from correlations between data entities. \sys's support for
fine-grained decorrelation policies gives developers options beyond either completely retaining
these links or deleting of all of the user's data and their dependees (while still supporting
these options). The complexity of implementating such policies is hidden from the developer:
\sys automatically performs decorrelation in a way that allows for recorrelation upon
resubscription, while still achieving performance comparable to today’s widely-used databases and
requiring no modification of application schemas.

\subsection{The goal of decorrelation}
Decorrelation ideally breaks connections between data entities that would otherwise allow
an adversary to determine that two entities belong to the same (unsubscribed) user:
post-decorrelation, an adversary should not be able to distinguish between a scenario in which two data entities
have been generated by two distinct users, and one in which they were both generated by the same
(unsubscribed) user.  If two of a user's entities cannot be correlated back to the user, then at
most one of these entities may leak identifying information about the user.\footnote{Assume for a
contradition that both entities leak identifying information: then both entities are more likely to
have been generated by a particular identity than any other, contradicting our assumption.}  Thus,
if the content of individual data entities is appropriately anonymized, then perfect decorrelation prevents
identifying information from being leaked via correlations.

\subsubsection{Threat Model}
An adversary aims to relink decorrelated entities to unsubscribed users after \sys
performs decorrelation. We make the following assumptions about such an adversary:
\begin{itemize}
    \item An adversary can perform only those queries allowed by the application API,
i.e.,\ can access the application only via its public interface.
%\lyt{Alternatively, an adversary could perform arbitrary queries on some public subset of the
%application schema (e.g., all tables other than the mapping table, or all tables marked with some
%compliance policy); arbitrary queries over the
%entirety of the table are out of scope, unless ``private'' tables are removed and stored by
%unsubscribing users.}

    \item An adversary cannot perform application queries to the past or search web archives:
    information from prior application snapshots may reveal
    exactly how data records were decorrelated from unsubscribed users.

    \item An adversary cannot gain identifying information from arbitrary user-generated content (for
        example, a reposted screenshot, or text in user stories or comments). Decorrelation seeks to
        remove identifying information from user-generated data that can be enumerated or follows a
        specific pattern (e.g., a birthday or email address), and application metadata (e.g., date
        of postings, database ID columns).
\end{itemize}

\subsubsection{Problems with decorrelation taken too far.}
Decorrelation taken to the extreme breaks all correlations between entities recursively related to
the user being anonymized, replacing these correlations with ghost correlations.  This removes as
much correlation-based identifying information as possible while keeping data entities present, but
in practice, completely decorrelated data entities may be useless to the application.  Applications
may lack a meaningful way to generate ghost entities: what does it mean for a story to become many
ghost stories?  And even if ghosts can be generated, the noise and data pollution from ghost
entities may instead affect the accuracy and semantics of the application: users may see meaningless
content, comment threads may be disjoint and scattered, and highly-ranked content may suddenly lose
votes.
%On the other extreme, however, performing no decorrelation at all fails to adequately de-identify

\sys offers developers a way to decorrelate as much as possible \emph{while still retaining
application semantics}. As the next section describes, \sys provides decorrelation specification
primitives with which developers can specify which (and how) entities can be decorrelated from other
entities, and which correlations must be kept for application correctness.

