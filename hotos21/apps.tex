\subsection{Unsubscription \& Resubscription in Today's Web Applications}
\lyt{
    Lily's plan for how this section should go:
}
\begin{itemize}
    \item Many applications now claim to be GDPR compliant and have privacy policies allowing for
    erasure of personal data 
    \item Many of these applications do so manually, and still retain user data
    (anonymized/unanonymized) for ``necessary'' application use.
    \item A table showing applications and whether they have the following features: 
    \begin{itemize}
        \item GDPR-compliant unsubscription: yes-automated/yes-manual/no
        \item Resubscription: yes (privacy preserving?) / no, 
        \item Retains user data for necessary application use: yes (anonymized/not anonymized?) / no
        \item Clear specification of what data remains and in what state (yes/no)
    \end{itemize}
    \item Argument that automated unsubscription is hard (only supported by large companies),
    coarse-grained and underspecified (users cannot determine exactly what information is left, and
    in what state), and too costly for users to freely unsubscribe. 
    \item What we need is a way to allow developers to specify, at a high level, the state of the
    application data post-unsubscription; and a system that automates the process of achieving this
    state (unsubscribing) and restoring the original state (resubscription).
\end{itemize}

--------------------------------------

Although more applications now claim to be GDPR-compliant, the state of proper support for
unsubscription and resubscription looks grim. For each of the following applications, we evaluate
whether the application supports unsubscription and, if so, how user data is removed or
de-identified; and whether the application supports account resubscription.

We look at a range of web services, including social media and news-feed applications such as
Facebook, Instagram, Twitter, Reddit, Hacker News, and Lobsters; e-Commerce applications such as
Amazon and PrestaShop; and other specialized services such as HotCRP, GitHub, Spotify, and Strava.

%- Resubscription support
%- Unsubscription: remove or retain user data (anonymized / unanonymized?)
    %- Requires identifying information for functionality

Facebook and Instagram~\cite{facebook:privacy} allow for account deletion and account deactivation:
the former permanently removes all user content and has no option for resubscription, and the latter
allows users to resubscribe at the cost of leaving all identifying information in the application.
More specifically, account deletion deletes all user-added content, but retains some (unanonymized)
data, including messages to friends. Any functionalities linked to having a user account are permanently lost,
including any logins to Oculus and other applications. Account deactivation, on the other hand,
allows users to resubscribe, but only hides user-added content from other users and all user data
remains in the system. 

Twitter~\cite{twitter:privacy} has similar account deactivation policies to Facebook, retaining all
user data but hiding the content from other users. After 30 days of deactivation, users can no
longer reactivate their account, and user data is deleted from their system.  However, like
Facebook, Twitter still retains unanonymized private messages.

Reddit~\cite{reddit:privacy} account deletion reassociates user data with a global ghost user, but
all user posts, comments, and messages that have not been explicitly been removed by the user remain
present but anonymized in the system. Users have no ability to resubscribe and regain
ownership of these pieces of data that remain in the system. 

Hacker News~\cite{hackernews:privacy} allows users to submit requests via email or phone to delete
personal information, but reserves the right to refuse to delete content or remove associations with
a user's identity.  Users cannot resubscribe and regain access to their prior profiles.

Lobsters has no privacy policy~\cite{lobsters:privacy}. All user actions and moderations are
logged, enforcing Lobster's culture of transparent user accountability. A user can request to delete
their account, which deletes only a users' private messages and negatively-scored comments; all
stories and comments can optionally be reassociated with a global ghost user instead of the original
user. In all cases, Lobsters privately retains the user's unique username, and leave log entries of users'
actions unmodified.

Amazon~\cite{amazon:privacy} allows users to request to delete their personal data by contacting
Customer Service, with the warning that services become limited or unavailable when the user does
so. There is no option to unsubscribe or delete an Amazon account from a user's Amazon account
management page.  A user can also choose to permanently close her account~\cite{amazon:close}. If
so, Amazon specifies that content is deleted from Photos and Drive, but potentially retains all other types
of data, such as order history. Even though Amazon can retain profile data, a user can never
again regain access to her account.

PrestaShop~\cite{prestashop:privacy} is an open-source e-commerce web service, and has explicit GDPR
policies for its merchants when a customer requests to delete their account: their personal account
details (age, email, address) are removed, but order invoices and abandoned carts are transferred to
an anonymous account. Users cannot resubscribe.

HotCRP's current privacy policy~\cite{hotcrp:privacy} allows site managers (e.g.,\ program chairs)
to indefinitely store and distribute submissions and reviews. Each HotCRP.com user has an associated
global profile and a profile for every HotCRP.com site, and must contact the HotCRP maintainers
directly to remove these profiles.

GitHub~\cite{github:privacy} reassociates some retained user content (contributions, comments on
issues) with a global ghost user, and removes all other account information. However, user email
addresses associated with commits remain unchanged. A user cannot resubscribe.

Spotify~\cite{spotify:privacy} allows users to close their accounts, at the cost of permanently
losing all account data (playlists, followers, etc.) Spotify promises to ``delete or anonymise...
personal data so that it no longer identifies [the user], unless [they] are legally allowed or
required to maintain certain personal data.'' Spotify does not specify exactly what data is
retained.

Strava~\cite{strava:privacy} users can request account deletion, which ``permanently and
irreversibly'' deletes all personal data. Resubscription is not supported. Any content that is
shared with others remains unanonymized after account deletion, as well as any ``necessary''
information. Strava also retains unnecessary user data such as geolocation information, and promises
to de-identify this data.

\subsection{Addressing the Challenges of Unsubscription and Resubscription}

While many web applications make obvious efforts to comply with legal requirements (GDPR and CCPA), 
unsubscription clearly requires more than merely deleting a user's data: many of these services
require that some information remains, in transformed or anonymized form, after a user departs.
%
Developers must determine and implement the database transformations required to
correctly unsubscribe a user, and continously maintain this feature.
%
Consequently, the majority of web services today either lack automated unsubscription support
(requiring manual labor to handle \eg GDPR deletion requests), or potentially leave substantial
information behind that harms user privacy.

Furthermore, all surveyed applications fail to support resubscription in a privacy-preserving way.
When unsubscribing, users today must choose to either permanently lose their account functionality,
or allow the application to retain all their personal data to be able to resubscribe.

To help developers of databased-backed web applications achieve correct,
privacy-compliant user unsubscription and resubscription without onerous labor, we present 
\sys, a system that provides abstractions and mechanisms to automate unsubscription and
resubscription.
%
Application developers using \sys specify a declarative \emph{unsubscription policy},
which indicates how the database contents need to change to meet de-identification
requirements on user unsubscription.
%
\sys then turns this policy into a set of concrete, executed database operations that remove,
anonymize, and structurally \emph{decorrelate} user data.
%
When a user resubscribes, \sys both re-imports missing data and
\emph{recorrelates} the remnants of the user's prior data with the user account,
restoring the user to her original subscribed state as much as possible.

Furthermore, \sys transparently supports unsubscription and resubscription while still achieving performance
comparable to today’s widely-used databases and requiring no modification of application schemas.

