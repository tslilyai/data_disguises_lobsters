\section{Challenges}
While many web applications make obvious efforts to comply with legal requirements (GDPR and CCPA),
unsubscription clearly requires more than merely deleting a user's data.  
Developers must implement the database transformations required to correctly unsubscribe a user while
preserving application semantics, and continously maintain this feature. 
%
They must ensure that any required information, for necessary application or legal reasons,
remains in transformed or anonymized form after a user departs.
Furthermore, developers must make these transformations reversible to support resubscription,
while still properly de-identifing users.
%
From the lack of automatic unsubscription, the vague promises to decorrelate retained data, and the
failure to support privacy-compliant resubscription, it is clear this is no easy task.
%

Developers face three main challenges when implementing proper support for unsubscription and
resubscription.
\begin{enumerate}
    \item \textbf{Selective Retention:}. Unsubscribed users' data needs to be selectively retained for necessary legal or application purposes.
    \item \textbf{De-Identification:} Unsubscribed users must be properly de-identified: any
        information that could possibly identify the user should be removed or, if retained, decorrelated from the user's identity.
\item \textbf{Resubscription:} Unsubscribed users should be given the option to resubscribe without permanently losing their account
data.
\end{enumerate}

\paragraph{Selective Retention.}
Retained data can be categorized into two functionality groups. First, retained data preserves legal or
necessary information about an individual user in isolation. The order history retained by
e-commerce applications such as Amazon and PrestaShop for legal purposes falls into this category.

Second, retained data also maintains application utility for other subscribed users of the
application. Social media and news feed applications fit into this category: Facebook or Twitter
preserve privates messages so that other subscribed users still see coherent conversations. Reddit
and Lobsters retain all previously posted user content, as the essence of the application's utility
is its site content.  Other non-feed applications also retain data for this reason. HotCRP,
open-source software for managing conference reviews and paper submissions, indefinitely stores and
distributes submissions and reviews from all conferences by request of its users: users want to be
able to reference their past prior conference reviews and paper metadata. Strava, a social network
for tracking workouts, retains users' geolocation data on public routes for other users to see which
routes are popular.

The application-specific nature of retained data means that any solution to unsubscription
should enable developers to easily specify exactly which data to retain, and prevent developers
from accidentally removing or retaining data unnecessarily.

\paragraph{De-Identification.}
While applications may want to retain data, this data should be properly de-identified. As shown in
Table~\ref{tab:apps}, applications often fail to do so, and retain data still associated
with users' identities. Other applications such as PrestaShop, Reddit, GitHub, and Lobsters make
attempts to decorrelate retained data, replacing user identifiers with a placeholder user identifier. 

However, it is insufficient to decorrelate only the direct correlations from users to their data:
correlations from retained data to \emph{other} application data or metadata can still identify
unsubscribed users. For example, PrestaShop orders with the same timestamp, or ordered from the same
location likely belong to the same user, and it is well-known that a user's order history can be
clearly identifying. Anonymized posts with the same flairs or tags on Reddit or Lobsters can
identify which posts probably belong to the same user. 

Due to these subtle information leaks, some applications, such as Facebook, Twitter, and HotCRP
simply delete the entire user profile and all associated data. When HotCRP users manually request to
be de-associated from their blinded reviews or submissions (preventing public disclosure in case of
a data compromise or hack), the HotCRP developers choose the nuclear option and remove the user's
data completely, at the cost of removing useful review and paper data for other HotCRP users.  While
the HotCRP developers would prefer to retain review and paper data, doing so while de-identifying
the user is not easy. HotCRP cannot only decorrelate direct user correlations: 
correlations between any retained data and conflicts with other reviewers can indirectly identify the user's
affiliation and recent collaborations, and in many cases, the user themselves.

\paragraph{Resubscription.}
Resubscription must recorrelate any retained data with the original owner and restore any removed
data. Today, applications either retain all data and forgo data de-identification entirely
(Facebook/Twitter Account Deactivation) to support resubscription, or remove and anonymize data
permanently without allowing users to resubscribe. Correctly supporting resubscription requires the
application or user to store additional information necessary for recorrelation, but must do so
without unduly burdening the application or user, and without adding any identifying information
about unsubscribing users to the system.

%HotCRP
%users---particularly reviewers---would prefer that their blinded reviews never become publically
%associated with their identity in the case of a data compromise. 
%

%Reviewers and authors create user accounts to
%ubmit or review papers, and HotCRP supports program committee conflict detection, paper rebuttals
%nd comments, review delegation, and other features necessary for the conference review process.
%HotCRP's current privacy policy~\cite{hotcrp:privacy} allows site managers (e.g.,\ program chairs)
%to indefinitely store and distribute submissions and reviews.  This is by request of its users:
%users want to be able to reference prior conference reviews and paper metadata.  

%Currently, the only way to ensure this is to contact the HotCRP maintainers directly to manually
%remove their user profile: the HotCRP developers considered automating support for unsubscription
%and found it too cumbersome to implement. 
%
%\lyt{(This isn't exactly what I want to say---sorry to put words in your mouth, Eddie!---but something like this is
%probably useful.)} 
%
%When user accounts are removed, every piece of data ever associated with that
%user is deleted from the system: this model clearly would not work if all users freely
%unsubscribed at will, as the system would be left with no useful data for subscribed users.
%
%The desired HotCRP unsubscription behavior would both retain useful data for HotCRP and its users,
%and satisfy the desire of HotCRP users to de-identify their data. In particular, the HotCRP
%developers should be able to specify that unsubscription deletes only the user profile, and
%\emph{decorrelates} the links between the user and their reviews (and other user-associated data)
%while retaining the blinded reviews and data for other users to view. While correlations between
%users and their reviews are obviously identifying, unsubscription must also handle more subtle
%correlations: a user's reviewer conflicts can identify the user's affiliation and recent
%collaborations, and in many cases, the user themselves. Implementing unsubscription that handles
%these complex data relationships correctly would not be trivial.
%

