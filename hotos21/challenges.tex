\section{\sys Requirements \& Challenges}

To achieve \sys, a system must (1) enable applications to \emph{selectively retain} unsubscribed
users' data for necessary legal or application purposes; (2) properly \emph{de-identify} the user
upon unsubscription by removing any possibly identifiable information, or decorrelating this
information from the user's identity; and (3) allow users to \emph{resubscribe} without losing their
account history or data, so unsubscription has no permanent consequences. We next describe each
requirement and the challenges to achieve it in greater detail.

\paragraph{Selective Retention.}
Retained data has two main purposes, to preserve legal or
necessary information about an individual user (\eg order history in e-commerce applications), and
to maintains application utility for other subscribed users of the application.  Social media
applications retain private messages for thread coherence; news feed applications like Reddit retain
previous posts, the main source of utility for other users of the sites.  HotCRP, conference review
and paper submission software, indefinitely stores and distributes submissions and reviews so users
can reference their past reviews and paper feedback. Strava, a workout tracking social network,
retains geolocation data for other users to see which routes are popular.

Under \sys, developers should be able to to specify exactly what data to retain 
post-unsubscription, and prevent accidental data deletion or retention.

\paragraph{De-Identification.}
If applications retain data, this data should be properly de-identified. 
Some applications make efforts to do so by replacing user identifiers with a placeholder user
identifier, or removing data with user identifiers. However, this often fails to sufficiently
de-identify retained data. Correlations from retained data to \emph{other} application data or
metadata can still identify unsubscribed users. For example, orders with the same
timestamp or same address likely belong to the same user, and a user's order
history can be clearly identifying. Anonymized posts with the same flairs or custom tags 
can identify which user made the posts.

Because of these subtle and complex indirect information leaks, many applications choose to forgo
data retention completely in order to ensure proper de-identification. For example, when HotCRP
users unsubscribe, 
%While HotCRP developers would prefer to retain review and paper data, doing so while de-identifying
%the user is not easy. 
HotCRP cannot just remove user identifiers: correlations between papers and conflicting  reviewers
can indirectly identify the user's affiliation and recent collaborations, and often the user
themselves. Instead, HotCRP chooses to reduce its utility by completely removing the unsubscribing
user's data, a solution that clearly would not work in a world where users frequently unsubscribe at
any time.
%HotCRP removes the user's data completely, at the cost of other users who reference those reviews
%and papers. 

Under \sys, developers should be able to easily detect, specify, and modify or remove problematic
content or correlations, even when these may be subtle; users should know exactly what identifying
information may be left behind. However, the \sys paradigm does not aim to completely solve the open
problem of guaranteeing complete de-identification.

\paragraph{Resubscription.}
Resubscription must recorrelate any retained data with the original owner, and restore any removed
data. Today, applications either retain all data and forgo de-identification entirely
(Facebook/Twitter Account Deactivation) in order to support resubscription, or permanently remove
and anonymize data without resubscription support. Supporting resubscription requires storing
information necessary for recorrelation and restoring users' accounts, but should not unduly burden
either the application or user, nor add any identifying information about unsubscribing users to the
system.

%HotCRP
%users---particularly reviewers---would prefer that their blinded reviews never become publically
%associated with their identity in the case of a data compromise. 
%

%Reviewers and authors create user accounts to
%ubmit or review papers, and HotCRP supports program committee conflict detection, paper rebuttals
%nd comments, review delegation, and other features necessary for the conference review process.
%HotCRP's current privacy policy~\cite{hotcrp:privacy} allows site managers (e.g.,\ program chairs)
%to indefinitely store and distribute submissions and reviews.  This is by request of its users:
%users want to be able to reference prior conference reviews and paper metadata.  

%Currently, the only way to ensure this is to contact the HotCRP maintainers directly to manually
%remove their user profile: the HotCRP developers considered automating support for unsubscription
%and found it too cumbersome to implement. 
%
%\lyt{(This isn't exactly what I want to say---sorry to put words in your mouth, Eddie!---but something like this is
%probably useful.)} 
%
%When user accounts are removed, every piece of data ever associated with that
%user is deleted from the system: this model clearly would not work if all users freely
%unsubscribed at will, as the system would be left with no useful data for subscribed users.
%
%The desired HotCRP unsubscription behavior would both retain useful data for HotCRP and its users,
%and satisfy the desire of HotCRP users to de-identify their data. In particular, the HotCRP
%developers should be able to specify that unsubscription deletes only the user profile, and
%\emph{decorrelates} the links between the user and their reviews (and other user-associated data)
%while retaining the blinded reviews and data for other users to view. While correlations between
%users and their reviews are obviously identifying, unsubscription must also handle more subtle
%correlations: a user's reviewer conflicts can identify the user's affiliation and recent
%collaborations, and in many cases, the user themselves. Implementing unsubscription that handles
%these complex data relationships correctly would not be trivial.
%

