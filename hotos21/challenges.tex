\section{Goals and Challenges}
A subscription paradigm should achieve the three following goals:
\begin{enumerate}
    \item \textbf{Selective Retention:}. Unsubscribed users' data needs to be selectively retained for necessary legal or application purposes.
    \item \textbf{De-Identification:} Unsubscribed users must be properly de-identified: any
        information that could possibly identify the user should be removed or, if retained, decorrelated from the user's identity.
\item \textbf{Resubscription:} Unsubscribed users should be given the option to resubscribe without permanently losing their account
data.
\end{enumerate}

We next describe each goal and the challenges of achieving it in greater detail.

\paragraph{Selective Retention.}
Retained data has two main purposes. First, retained data preserves legal or
necessary information about an individual user in isolation. For example, e-commerce applications
retain order history for legal purposes.
Retained data also maintains application utility for other subscribed users of the application.
Social media and news feed applications preserve private messages for thread coherence; Reddit and
Lobsters retain all previously posted user content, as the essence of the application's utility is
its site content.  HotCRP, open-source software for managing conference reviews and paper
submissions, indefinitely stores and distributes submissions and reviews from all conferences so
that users can reference their past prior conference reviews and paper metadata. Strava, a social
network for tracking workouts, retains users' geolocation data on public routes for other users to
see which routes are popular.

Developers must use their understanding of application semantics to pinpoint exactly what 
to retain post-unsubscription, and prevent accidental deletion or retention of application
data.

\paragraph{De-Identification.}
While applications may want to retain data, this data should be properly de-identified. As shown in
Table~\ref{tab:apps}, applications often fail to do so, and retain user identifiers along with their
data. 

Some applications do make efforts to de-identify retained data by replacing user identifiers with a
placeholder user identifier, or removing data associated with user identifiers.  However, simply
removing or anonymizing user identifiers often fails to sufficiently de-identify retained data.
Correlations from retained data to \emph{other} application data or metadata can still identify
unsubscribed users. For example, PrestaShop orders with the same timestamp, or ordered to the same
location likely belong to the same user, and a user's order history can be clearly identifying.
Anonymized posts with the same flairs or custom tags on Reddit or Lobsters can identify which user
made the posts.

Because of these subtle and complex indirect information leaks from data relationships, many
applications choose to forgo data retention completely in order to ensure proper de-identification.
For example, when HotCRP users unsubscribe, 
%While HotCRP developers would prefer to
%retain review and paper data, doing so while de-identifying the user is not easy. 
HotCRP cannot only remove user identifiers: correlations with data such as
conflicts with other reviewers can indirectly identify the user's affiliation and recent
collaborations, and the user themselves. Instead, HotCRP chooses to reduce its
utility for others users by completely removing the unsubscribing user's data, a solution that clearly would not work in a world where
users frequently unsubscribe at any time.
%HotCRP removes the user's data completely, at the
%cost of other users who reference those reviews and papers. 

\paragraph{Resubscription.}
Resubscription must recorrelate any retained data with the original owner, and restore any removed
data. Today, applications either retain all data and forgo de-identification entirely
(Facebook/Twitter Account Deactivation) to support resubscription, or remove and anonymize data
permanently without allowing users to resubscribe. Correctly supporting resubscription requires the
application or user to store additional information necessary for recorrelation, but must do so
without unduly burdening the application or user, and without adding any identifying information
about unsubscribing users to the system.

%HotCRP
%users---particularly reviewers---would prefer that their blinded reviews never become publically
%associated with their identity in the case of a data compromise. 
%

%Reviewers and authors create user accounts to
%ubmit or review papers, and HotCRP supports program committee conflict detection, paper rebuttals
%nd comments, review delegation, and other features necessary for the conference review process.
%HotCRP's current privacy policy~\cite{hotcrp:privacy} allows site managers (e.g.,\ program chairs)
%to indefinitely store and distribute submissions and reviews.  This is by request of its users:
%users want to be able to reference prior conference reviews and paper metadata.  

%Currently, the only way to ensure this is to contact the HotCRP maintainers directly to manually
%remove their user profile: the HotCRP developers considered automating support for unsubscription
%and found it too cumbersome to implement. 
%
%\lyt{(This isn't exactly what I want to say---sorry to put words in your mouth, Eddie!---but something like this is
%probably useful.)} 
%
%When user accounts are removed, every piece of data ever associated with that
%user is deleted from the system: this model clearly would not work if all users freely
%unsubscribed at will, as the system would be left with no useful data for subscribed users.
%
%The desired HotCRP unsubscription behavior would both retain useful data for HotCRP and its users,
%and satisfy the desire of HotCRP users to de-identify their data. In particular, the HotCRP
%developers should be able to specify that unsubscription deletes only the user profile, and
%\emph{decorrelates} the links between the user and their reviews (and other user-associated data)
%while retaining the blinded reviews and data for other users to view. While correlations between
%users and their reviews are obviously identifying, unsubscription must also handle more subtle
%correlations: a user's reviewer conflicts can identify the user's affiliation and recent
%collaborations, and in many cases, the user themselves. Implementing unsubscription that handles
%these complex data relationships correctly would not be trivial.
%

