\section{Challenges}
While many web applications make obvious efforts to comply with legal requirements (GDPR and CCPA),
unsubscription clearly requires more than merely deleting a user's data. Many of these services
require that some information remains, in transformed or anonymized form, after a user departs, for
necessary application or legal reasons. 
%
Developers must determine how to selectively retain, decorrelate, and remove an unsubscribing user's data,
without interfering with application correctness, and while preserving the user's privacy. They must 
implement the database transformations required to correctly unsubscribe a user while
preserving application semantics, and continously maintain this feature. 
%
From the lack of automatic unsubscription, the vague promises to decorrelate retained data, and the
failure to support privacy-compliant resubscription, it is clear this is no easy task.
%

We use HotCRP as an example to make these challenges concrete. HotCRP is open-source software for
managing conference reviews and paper submissions.  Reviewers and authors create user accounts to
submit or review papers, and HotCRP supports program committee conflict detection, paper rebuttals
and comments, review delegation, and other features necessary for the conference review process.

HotCRP's current privacy policy~\cite{hotcrp:privacy} allows site managers (e.g.,\ program chairs)
to indefinitely store and distribute submissions and reviews.  This is by request of its users:
users want to be able to reference prior conference reviews and paper metadata.  However, many
users---particularly reviewers---would prefer that their blinded reviews never become publically
associated with their identity in the case of a data compromise. Currently, the only way to ensure
this is to contact the HotCRP maintainers directly to manually remove their user profile: the HotCRP
developers considered automating support for unsubscription and found it too cumbersome to
implement. 
%
\lyt{(This isn't exactly what I want to say---sorry to put words in your mouth, Eddie!---but something like this is
probably useful.)} 
%
When user accounts are removed, every piece of data ever associated with that
user is deleted from the system: this model clearly would not work if all users freely
unsubscribed at will, as the system would be left with no useful data for subscribed users.

The desired HotCRP unsubscription behavior would both retain useful data for HotCRP and its users,
and satisfy the desire of HotCRP users to de-identify their data. In particular, the HotCRP
developers should be able to specify that unsubscription deletes only the user profile, and
\emph{decorrelates} the links between the user and their reviews (and other user-associated data)
while retaining the blinded reviews and data for other users to view. While correlations between
users and their reviews are obviously identifying, unsubscription must also handle more subtle
correlations: a user's reviewer conflicts can identify the user's affiliation and recent
collaborations, and in many cases, the user themselves. Implementing unsubscription that handles
these complex data relationships correctly would not be trivial.


