\subsection{Addressing the Challenges of Unsubscription and Resubscription}

While many web applications make obvious efforts to comply with legal requirements (GDPR and CCPA), 
unsubscription clearly requires more than merely deleting a user's data: many of these services
require that some information remains, in transformed or anonymized form, after a user departs.

TODO why

%
Developers must determine and implement the database transformations required to
correctly unsubscribe a user, and continously maintain this feature.
%
Consequently, the majority of web services today either lack automated unsubscription support
(requiring manual labor to handle \eg GDPR deletion requests), or potentially leave substantial
information behind that harms user privacy.

Furthermore, all surveyed applications fail to support resubscription in a privacy-preserving way.
When unsubscribing, users today must choose to either permanently lose their account functionality,
or allow the application to retain all their personal data to be able to resubscribe.

To help developers of databased-backed web applications achieve correct,
privacy-compliant user unsubscription and resubscription without onerous labor, we present 
\sys, a system that provides abstractions and mechanisms to automate unsubscription and
resubscription.
%
Application developers using \sys specify a declarative \emph{unsubscription policy},
which indicates how the database contents need to change to meet de-identification
requirements on user unsubscription.
%
\sys then turns this policy into a set of concrete, executed database operations that remove,
anonymize, and structurally \emph{decorrelate} user data.
%
When a user resubscribes, \sys both re-imports missing data and
\emph{recorrelates} the remnants of the user's prior data with the user account,
restoring the user to her original subscribed state as much as possible.

Furthermore, \sys transparently supports unsubscription and resubscription while still achieving performance
comparable to today’s widely-used databases and requiring no modification of application schemas.

