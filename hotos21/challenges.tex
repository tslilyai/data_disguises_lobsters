\section{\name Requirements \& Challenges}

To achieve \name, a system must (1) enable applications to \emph{selectively retain} unsubscribed
users' data for necessary legal or application purposes; (2) properly \emph{de-identify} the user
upon unsubscription by removing any possibly identifiable information, or decorrelating this
information from the user's identity; and (3) allow users to \emph{resubscribe} without losing their
account history or data, so unsubscription has no permanent consequences. We next describe each
requirement and the challenges to achieve it in greater detail.

\paragraph{Selective Retention.}
Retained data serves two purposes: to preserve legal or
necessary information about an individual user (\eg order history in e-commerce applications), and
to maintains application utility for subscribed users (\eg private messages
for thread coherence, or crowdsourced reviews and popular news posts).
Developers must ensure that post-unsubscription state avoids accidental data deletion or retention.
%Developers should easily determine what data to retain %post-unsubscription, and 

\paragraph{De-Identification.}
Any retained data must be properly de-identified. The surveyed applications often remove or replace
user identifiers with a placeholder user identifier. However, this can still fail to sufficiently
de-identify retained data: \emph{structural correlations} from or to retained data can still identify
unsubscribed users. For example, orders tagged with the same location likely belong to the same
user, and order history can identify the user. 
%Anonymized posts with the same flairs or custom tags can identify which user made the posts.

Detecting and correctly modifying even subtly problematic content and correlations pose technical
challenges that can prevent developers from retaining data in privacy-preserving ways, or force
developers to forgo data retention completely.  For example, HotCRP cannot just remove unsubscribing
users' identifiers: correlations between users' papers and conflicting reviewers indirectly identify
the user via the conflicts' affiliations and recent collaborations.  HotCRP chooses to reduce its
utility by completely removing the papers, rather than deal with these potentially identifying
correlations.
%, a solution that clearly would not
%work in a world where users frequently unsubscribe at any time.

These complex data transformations also make it difficult for developers to specify exactly 
what identifying information may be left behind for the user, leading to vague privacy policies.

\paragraph{Resubscription.}
Resubscription must recorrelate any retained data with the original owner, and restore any removed
data. Today, applications either retain all data and forgo de-identification entirely in order to
support resubscription, or permanently remove and anonymize data.
Supporting resubscription requires storing information necessary for recorrelation and restoring
users' accounts, but should not unduly burden either the application or user, nor add any
identifying information about unsubscribing users to the system.

%HotCRP
%users---particularly reviewers---would prefer that their blinded reviews never become publically
%associated with their identity in the case of a data compromise. 
%

%Reviewers and authors create user accounts to
%ubmit or review papers, and HotCRP supports program committee conflict detection, paper rebuttals
%nd comments, review delegation, and other features necessary for the conference review process.
%HotCRP's current privacy policy~\cite{hotcrp:privacy} allows site managers (e.g.,\ program chairs)
%to indefinitely store and distribute submissions and reviews.  This is by request of its users:
%users want to be able to reference prior conference reviews and paper metadata.  

%Currently, the only way to ensure this is to contact the HotCRP maintainers directly to manually
%remove their user profile: the HotCRP developers considered automating support for unsubscription
%and found it too cumbersome to implement. 
%
%\lyt{(This isn't exactly what I want to say---sorry to put words in your mouth, Eddie!---but something like this is
%probably useful.)} 
%
%When user accounts are removed, every piece of data ever associated with that
%user is deleted from the system: this model clearly would not work if all users freely
%unsubscribed at will, as the system would be left with no useful data for subscribed users.
%
%The desired HotCRP unsubscription behavior would both retain useful data for HotCRP and its users,
%and satisfy the desire of HotCRP users to de-identify their data. In particular, the HotCRP
%developers should be able to specify that unsubscription deletes only the user profile, and
%\emph{decorrelates} the links between the user and their reviews (and other user-associated data)
%while retaining the blinded reviews and data for other users to view. While correlations between
%users and their reviews are obviously identifying, unsubscription must also handle more subtle
%correlations: a user's reviewer conflicts can identify the user's affiliation and recent
%collaborations, and in many cases, the user themselves. Implementing unsubscription that handles
%these complex data relationships correctly would not be trivial.
%

