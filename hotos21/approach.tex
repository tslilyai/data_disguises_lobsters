\section{Approach}
As a first step in solving these challenges, we present \sys, a system that provides abstractions
and mechanisms to help developers of databased-backed web applications achieve correct,
privacy-compliant, and automatic unsubscription and resubscription without onerous labor.

With \sys, developers write an \emph{unsubscription policy} that specifies the high-level,
post-unsubscription state of application data.  The policy indicates how the database contents need
to change to meet de-identification requirements on user unsubscription, while still retaining data
necessary for the application to function. \sys then turns this policy into a set of concrete,
executed database operations that systematically remove, anonymize, and structurally
\emph{decorrelate} user data upon unsubscription to meet the resulting spec. \sys 
encrypts decorrelated or removed user data with a user-specific key, which can be either
password-encrypted by the application, or secret shared~\cite{secretsharing} with a trusted third
party. \sys provides a way for the application to store the encrypted blob and deletes the plaintext
data.

\sys's policy abstractions require knowledge of only the application schema and the high-level
application's semantics. Using these abstractions, developers pinpoint how identifying information
leaks either (1) directly via data content, or (2) indirectly via the structural \emph{correlations}
(foreign key relationships) between data records. Developers use their application expertise to
select from a set of policy choices that alter these contents and correlations, choosing the policy
that provides the greatest amount of de-identification while still preserving application
utility and correctness.

When a user resubscribes, \sys decrypts and inserts missing data, automatically
\emph{recorrelating} the data with the user account and other system data, and restoring the user to her
original subscribed state as much as possible.

Furthermore, \sys transparently supports unsubscription and resubscription while still achieving performance
comparable to today’s widely-used databases and requiring no modification to application schemas or
architecture.
