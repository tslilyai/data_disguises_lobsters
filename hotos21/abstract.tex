\begin{abstract}
Web applications face increasing legal and user demands to provide better privacy for their users,
whether through supporting user's right to be forgotten, or removing unnecessary or old user
data.
%
Meeting these demands requires applications to delete, anonymize, and decorrelate their data, 
while still retaining data required for legal purposes or to preserve application utility.
%
Meeting these requirements grows only more challenging as privacy policies become more complex;
    consequently, today's applications often support only simple, coarse-grained privacy policies
    implemented with ad-hoc methods.
%forgo these transformations, or implement ad-hoc transformations with unclear guarantees.
%\ms{sets us up to specify what guarantees we make}
%

To enable developers to better support existing and new, more expressive, privacy policies without
    onerous manual labor, we propose \emph{data masking}, a systematic approach that helps
    developers generate \emph{data masks}---concrete privacy transformations---for database-backed
    web applications from a high-level specification.
%
Data masks encompass the privacy transformations done by applications today, and go beyond,
supporting fine-grained and nuanced policies that would be cumbersome to implement manually, and 
enabling reversible transformations that permit returning to an identity-revealing state at any time.
%
Furthermore, data masking tools automate the application of data masks to reduce developer burden to
only the high-level mask specification. A prototype tool demonstrates that data masking can be practical.

%To help developers support data masks in new and existing applications, we design and
%implement \sys, a system that takes as input a high-level specification of a data mask, and
%automatically applies the mask to the application database.  \sys relies on the key abstractions of
%application \emph{entity graphs} and \emph{ghost entities}, which allow developers to flexibly express
%and automate practical data masks without onerous labor.
\end{abstract}

\iffalse
%
\name is a new data ownership paradigm for web services in which users subscribe to
services by granting a \emph{lease} of their data, instead of having a permanent account.
%
When the lease ends, the user switches from an identity-revealing
subscribed mode into a privacy-preserving unsubscribed mode, where the
service only retains anonymized data about them.
%
This paradigm strikes a balance between current practices that give services control over user
    data with low user privacy; and the opposite extreme that decouples user data from
    services for strong privacy, but breaks current business models and reduces service utility.
%achieves better user privacy and data protection than today's services,
%but without making impractical assumptions or breaking current business models.
%
Under \name, users withdraw from a service without permanently losing access,
as they can resubscribe at any time, and one user's withdrawal does not affect
the service's utility for others.

%
A prototype design and implementation for \name suggests it can be realized
efficiently and with low burden for application developers.
%
%\ms{We should give our paradigm a name!}

%The paper proposes a new web application \emph{subscription paradigm},  Users flexibly switch between a privacy-preserving unsubscribed mode and an
%identity-revealing subscribed mode at any time without permanently losing their data.
%This subscription paradigm strikes a balance between applications having complete ownership of user
%data, and completely decoupling user data from applications at the cost of application utility.
%To solve the complex data retention and de-identification challenges associated with subscription,
%we design \name{}, a practical system that provides abstractions and mechanisms to help
%developers of databased-backed web applications achieve correct, privacy-compliant user
%unsubscription and resubscription without onerous labor.
\fi
