\begin{abstract}
%
We propose \name, a new data ownership paradigm for web services that achieves
better user privacy and data protection than today's services, but without
making impractical assumptions or breaking the business model.
%

%
Users subscribe to a service by granting a \emph{lease} to their data,
instead of having a permanent account.
%
When the lease ends, the service switches the user from the identity-revealing
subscribed mode into a privacy-preserving unsubscribed mode, where the
service only retains anonymized data about them.
%
Users can thus withdraw from a service without permanently losing access,
as they can resubscribe at any time, and a user's withdrawal does not affect
the service's utility for other users.
%
%This subscription paradigm strikes a balance between applications having
%complete ownership of user data, and users owning and storing their data at
%the cost of application utility.
%

%
A prototype design and implementation for \name suggests it can be realized
efficiently and with low burden for application developers.
%
\ms{We should give our paradigm a name!}

%The paper proposes a new web application \emph{subscription paradigm}, in which users subscribe to
%applications by granting a time-limited ``lease'' of their data, instead of having a permanent
%account. Users flexibly switch between a privacy-preserving unsubscribed mode and an
%identity-revealing subscribed mode at any time without permanently losing their data. 
%This subscription paradigm strikes a balance between applications having complete ownership of user
%data, and completely decoupling user data from applications at the cost of application utility. 
%To solve the complex data retention and de-identification challenges associated with subscription,
%we design \name{}, a practical system that provides abstractions and mechanisms to help
%developers of databased-backed web applications achieve correct, privacy-compliant user
%unsubscription and resubscription without onerous labor.
\end{abstract}
