%\begin{abstract}
\paragraph{Abstract.}

Conscientious application developers want to provide better user privacy;
legal requirements and user demand increasingly require it.
%
Unfortunately, privacy in complex, data-rich applications is fundamentally
hard. Consider a user who wants to remove their account from a service. Even
finding the corresponding data is nontrivial, but once it is found, only some
of it should be removed; other data should be anonymized or decorrelated (for
legal reasons, or to maintain application utility for other users), and some
of these transformations should be reversible (in case a user wants to
return).
%
% Conscientious application developers face increasing legal and user demands to provide better privacy for their users.
% %    \lyt{Need some defn of privacy-enhancing data transformations?}
% % Malte: defined by example (account del)
% %
% But a conscientious developer has a tough job: privacy-enhancing transformations
% like account deletion require deleting, anonymizing, and decorrelating user data, while
% maintaining application utility for other users and any data retained for legal reasons.
%
% Consequently, today's applications support only simple, coarse-grained, and ad-hoc privacy
% transformations.
%

We propose \emph{data disguising}, a systematic approach that helps developers generate
privacy transformations for database-backed web applications from a high-level specification
and preexisting data relationships.
%
Data disguising simplifies privacy transformations that applications use today, support
fine-grained and nuanced policies that would be cumbersome to implement manually, and
enables reversible transformations for users who change their mind.
%
%Furthermore, data reduce developer burden, as developers must no longer manually implement
%privacy transformations, and enable key new use cases, such as user-specific privacy modes.
%
A prototype tool for data disguising demonstrates that our approach is practical.
%\end{abstract}

\iffalse
%
\name is a new data ownership paradigm for web services in which users subscribe to
services by granting a \emph{lease} of their data, instead of having a permanent account.
%
When the lease ends, the user switches from an identity-revealing
subscribed mode into a privacy-preserving unsubscribed mode, where the
service only retains anonymized data about them.
%
This paradigm strikes a balance between current practices that give services control over user
    data with low user privacy; and the opposite extreme that decouples user data from
    services for strong privacy, but breaks current business models and reduces service utility.
%achieves better user privacy and data protection than today's services,
%but without making impractical assumptions or breaking current business models.
%
Under \name, users withdraw from a service without permanently losing access,
as they can resubscribe at any time, and one user's withdrawal does not affect
the service's utility for others.

%
A prototype design and implementation for \name suggests it can be realized
efficiently and with low burden for application developers.
%
%\ms{We should give our paradigm a name!}

%The paper proposes a new web application \emph{subscription paradigm},  Users flexibly switch between a privacy-preserving unsubscribed mode and an
%identity-revealing subscribed mode at any time without permanently losing their data.
%This subscription paradigm strikes a balance between applications having complete ownership of user
%data, and completely decoupling user data from applications at the cost of application utility.
%To solve the complex data retention and de-identification challenges associated with subscription,
%we design \name{}, a practical system that provides abstractions and mechanisms to help
%developers of databased-backed web applications achieve correct, privacy-compliant user
%unsubscription and resubscription without onerous labor.
\fi
