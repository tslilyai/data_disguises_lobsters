\begin{abstract}
    Web applications face increasing legal and user demands to properly protect and manage user
    data, whether through supporting user's right to be forgotten, removing unnecessary or old user
    data, or moderating user content.
    %
    Meeting these demands requires that applications perform the appropriate data transformations to
    sufficiently de-identify or modify content, while retaining data required for application utility or legal
    purposes (\eg purchase order history).  Because such transformations are necessarily
    application-specific and require handling complex data correlations, today's applications
    often forgo support for these demands, or use ad-hoc methods with unclear guarantees to achieve
    them.
    %
    To help developers flexibly support data protection transformations in new and existing
    applications, we design and implement \sys, a system that takes as input a high-level
    specification of a transformation, and automatically applies it to the application database.
    \sys relies on the key abstractions of application \emph{entity graphs} and \emph{ghost
    entities}, which allow developers to express and automate practical data protection transformations without
    onerous labor.
\end{abstract}

\iffalse
%
\name is a new data ownership paradigm for web services in which users subscribe to
services by granting a \emph{lease} of their data, instead of having a permanent account.
%
When the lease ends, the user switches from an identity-revealing
subscribed mode into a privacy-preserving unsubscribed mode, where the
service only retains anonymized data about them.
%
This paradigm strikes a balance between current practices that give services control over user
    data with low user privacy; and the opposite extreme that decouples user data from
    services for strong privacy, but breaks current business models and reduces service utility.
%achieves better user privacy and data protection than today's services, 
%but without making impractical assumptions or breaking current business models.
%
Under \name, users withdraw from a service without permanently losing access,
as they can resubscribe at any time, and one user's withdrawal does not affect
the service's utility for others.

%
A prototype design and implementation for \name suggests it can be realized
efficiently and with low burden for application developers.
%
%\ms{We should give our paradigm a name!}

%The paper proposes a new web application \emph{subscription paradigm},  Users flexibly switch between a privacy-preserving unsubscribed mode and an
%identity-revealing subscribed mode at any time without permanently losing their data. 
%This subscription paradigm strikes a balance between applications having complete ownership of user
%data, and completely decoupling user data from applications at the cost of application utility. 
%To solve the complex data retention and de-identification challenges associated with subscription,
%we design \name{}, a practical system that provides abstractions and mechanisms to help
%developers of databased-backed web applications achieve correct, privacy-compliant user
%unsubscription and resubscription without onerous labor.
\fi
