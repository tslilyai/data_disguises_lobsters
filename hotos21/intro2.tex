%-------------------------------------------------------------------------------
\section{Introduction}
%-------------------------------------------------------------------------------

Web application users are more aware of their data privacy rights than ever before. 
%
This increased awareness has driven demand for user-based ownership, as opposed to application-based
ownership, of private data, and has led to the creation of laws such as the European Union's General
Data Protection Regulation (GDPR)~\cite{eu:gdpr} and California's Consumer Privacy Act
(CCPA)~\cite{ca:privacy-act}, which grant users the right to request erasure of information related
to them.
%
Users also increasingly perceive the dangers of leaving private data on the
web: others can dig up embarrassing or compromising details from the past, web applications may
suffer data leaks or hacks, and private data may be shared by web applications without the user's
knowledge or explicit consent\cite{nytimes_fb, npr_data}.
%

%
For the first time, companies face both legal and societal demand to support \emph{unsubscription}
of users from their services. Users can request to unsubscribe at any time, preventing 
idle or unused accounts from retaining personal data indefinitely. 
%

%
Unsubscription, however, is only half the story. Users should not pay to regain ownership of their
data, which is rightly theirs: the cost of unsubscription cannot unduly restrict users' freedom to exercise their data
rights. To meet the flexible privacy demands of users, web applications must allow users to
unsubscribe \emph{and resubscribe} as they wish, switching between a privacy-preserving unsubscribed
mode and an identity-revealing subscribed mode at any time without permanently losing their data, or
needing to maintain their account data themselves. 
%
With flexible privacy, we can imagine a world where web services acquire time-limited ``leases'' of user
data, instead of permanent user accounts. 
\lyt{Users could do this without application support by themselves by programmatically setting a
personal reminder to ``unsubscribe after x days'' every time they create an account.}
%

%%%%%%%%%%%%%%%%%%%%%%%%%%%%%%%%%%%%%%%%%%
The current failure of web applications to support privacy-preserving unsubscription and
resubscription  highlights the many challenges of doing so correctly.
%
Unsubscription requires more than merely deleting a user's data: correct operation of a service
often requires that some information remains, in transformed or anonymized form, after a user
departs.
%
\ms{Give an example!}
%
It is difficult for developers to determine and implement the database transformations required to
correctly unsubscribe a user, and to continously maintain this feature.
%
Consequently, the majority of web services today either lack unsubscription support (requiring
manual labor to handle \eg GDPR deletion requests), or leave substantial information behind that
potentially harms user privacy.
%
Resubscription either is impossible after a user leaves (\eg Reddit), or requires for all user data
to be kept by the application (e.g.,\ account reactivation in Facebook and Twitter).

%%%%%%%%%%%%%%%%%%%%%%%%%%%%%%%%%%%%%%%%%%
In this paper, we describe \sys, a system that provides abstractions and mechanisms
that help developers of databased-backed web applications achieve correct,
privacy-compliant user unsubscription and resubscription without onerous labor.
%
Application developers using \sys specify a declarative \emph{unsubscription policy},
which indicates how the database contents need to change to meet de-identification
requirements on user unsubscription.
%
\sys then turns this policy into a set of concrete, executed database operations that remove,
anonymize, and structurally \emph{decorrelate} user data.

\sys's abstractions stem from the key observation that users' data can leak identifying information
in two ways: (1) directly via their content, and (2) indirectly via the structural
\emph{correlations} they have to other data in the system.  Correlations can range from obviously
identifying, such as posts written by a user, to subtly perilous: posts tagged with user-generated
tags most likely belong to the tag's author, and posts liked by the same group of users likely
belong to a friend of the group.

%
\sys provides a menu of unsubscription policy choices that allow developers to choose how to
\emph{ghost} individual data records (at column-level granularity), and how to \emph{decorrelate}
sensitive correlations. Specifying the policy requires nothing more than the application schema:
ghosting policies act on application datatables and on foreign key relationships between tables.
Table column values can be ``ghosted''---removed, anonymized, or modified---in application-specific
ways; and correlations can be broken, removed, or desensitized by adding noise. This gives
developers flexibility to choose policies other than deleting of all of the user's data, or
retaining all the user's pseudonymized data, without needing to deal with the complexity of
implementating such policies.

%
\sys goes beyond simply supporting unsubscription by automating resubscription as well.
When a user resubscribes, \sys both re-imports missing data and
\emph{recorrelates} the remnants of the user's prior data with the user account,
restoring the user to her original subscribed state as much as possible.

Furthermore, \sys transparently supports unsubscription and resubscription while still achieving performance
comparable to today’s widely-used databases and requiring no modification of application schemas.

%%%%%%%%%%%%%%%%%%%%%%%%%%%%%%%%%%%%%%%%%%

\subsection{Threat Model}
\lyt{This seems important to put *somewhere*, but I'm not sure where it goes...}
An adversary aims to relink decorrelated entities to unsubscribed users after \sys
performs decorrelation. We make the following assumptions about such an adversary:
\begin{itemize}
    \item An adversary can perform only those queries allowed by the application API,
i.e.,\ can access the application only via its public interface.
%\lyt{Alternatively, an adversary could perform arbitrary queries on some public subset of the
%application schema (e.g., all tables other than the mapping table, or all tables marked with some
%compliance policy); arbitrary queries over the
%entirety of the table are out of scope, unless ``private'' tables are removed and stored by
%unsubscribing users.}

    \item An adversary cannot perform application queries to the past or search web archives:
    information from prior application snapshots may reveal
    exactly how data records were decorrelated from unsubscribed users.

    \item An adversary cannot gain identifying information from arbitrary user-generated content (for
        example, a reposted screenshot, or text in user stories or comments). Decorrelation seeks to
        remove identifying information from user-generated data that can be enumerated or follows a
        specific pattern (e.g., a birthday or email address), and application metadata (e.g., date
        of postings, database ID columns).
\end{itemize}

%\subsubsection{Problems with decorrelation taken too far.}
%Decorrelation ideally breaks connections between data entities that would otherwise allow
%an adversary to determine that two entities belong to the same (unsubscribed) user:
%post-decorrelation, an adversary should not be able to distinguish between a scenario in which two data entities
%have been generated by two distinct users, and one in which they were both generated by the same
%(unsubscribed) user.  If two of a user's entities cannot be correlated back to the user, then at
%most one of these entities may leak identifying information about the user.\footnote{Assume for a
%contradition that both entities leak identifying information: then both entities are more likely to
%have been generated by a particular identity than any other, contradicting our assumption.}  Thus,
%if the content of individual data entities is appropriately anonymized, then perfect decorrelation prevents
%identifying information from being leaked via correlations.
%
%Decorrelation taken to the extreme breaks all correlations between entities recursively related to
%the user being anonymized, replacing these correlations with ghost correlations.  This removes as
%much correlation-based identifying information as possible while keeping data entities present, but
%in practice, completely decorrelated data entities may be useless to the application.  Applications
%may lack a meaningful way to generate ghost entities: what does it mean for a story to become many
%ghost stories?  And even if ghosts can be generated, the noise and data pollution from ghost
%entities may instead affect the accuracy and semantics of the application: users may see meaningless
%content, comment threads may be disjoint and scattered, and highly-ranked content may suddenly lose
%votes.
%On the other extreme, however, performing no decorrelation at all fails to adequately de-identify

%\sys offers developers a way to decorrelate as much as possible \emph{while still retaining
%application semantics}. As the next section describes, \sys provides decorrelation specification
%primitives with which developers can specify which (and how) entities can be decorrelated from other
%entities, and which correlations must be kept for application correctness.

