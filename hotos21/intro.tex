%-------------------------------------------------------------------------------
\section{Introduction}
%-------------------------------------------------------------------------------

Web application users are more aware of their data privacy rights than ever before. 
Laws such as the European Union's General Data Protection Regulation (GDPR)~\cite{eu:gdpr} and
California's Consumer Privacy Act (CCPA)~\cite{ca:privacy-act} codify users' rights to 
data ownership, granting users the right to request erasure of information related to them.
%
Public news around web services increasingly emphasize the dangers of leaving private data on the
web: embarrassing or compromising details from the past has reemerged, web applications have
suffered data leaks or hacks~\cite{breach:twitter, breach:fb, breach:marriott, breach:quora}, and
private data has be shared by web applications without the user's knowledge or explicit
consent~\cite{nytimes:fb, npr:data}.

%
These legal and societal pressures increasingly demand that companies support \emph{unsubscription}
of users from their services.  Users should have the right to unsubscribe at any time, preventing
idle or unused accounts from retaining personal data indefinitely, and regaining control over who
can access their data. 
%

%
Unsubscription, however, is only half the story. Users cannot truly regain their data rights if they
are impeded from exercising them. In order to claim that users have regained ownership of their
data, users must be able to \emph{freely} exercise their right to be forgotten. In particular, they
should not have to pay a high price in order in order to unsubscribe whenever they wish: they should
not encounter barriers to unsubscription that prevent them from choosing to do so, when they would have
otherwise. 

Unfortunately, today's web services face many challenges that make this ideal of free and easy
unsubscription hard.  Many do not support automated unsubscription, requiring users to go through a
manual process of contacting the developers or customer support directly in order to unsubscribe.
Services that do support de-identifying unsubscription permanently delete of years or decades of
accumulated application data, a cost many users are not willing to pay.  Although penalties for
compromising or mishandling user data create greater incentives to remove unnecessary data than ever
before, web services do not want to permanently lose their users.

To empower users to freely exercise their data rights, web applications must solve these challenges
and allow users to easily unsubscribe \emph{and resubscribe} as they wish, switching between a
privacy-preserving unsubscribed mode and an identity-revealing subscribed mode at any time without
permanently losing their data, or needing to maintain their account data themselves. 

%
With this model of flexible privacy, the balance of data ownership flips. Applications no longer
control users' data choices by demanding that users choose between indefinitely giving up their data
to use the application, or permanently losing their data and application utility.
Instead, users control when applications can obtain their data in exchange for the right to use the application, and do not permanently lose application utility when they remove access to their data.

In the rest of this paper, we describe the current support for unsubscription and resubscription in
today's web applications, and the challenges of implementing it correctly.  We then describe \sys, a
system that provides abstractions and mechanisms that help developers of databased-backed web
applications achieve correct, privacy-compliant user unsubscription and resubscription without
onerous labor, and without adding undue overheads.
