%-------------------------------------------------------------------------------
\section{Introduction}
%-------------------------------------------------------------------------------
Web application developers today have more incentives than ever to provide better privacy for their
users.
%
Laws like the EU's General Data Protection Regulation (GDPR)~\cite{eu:gdpr} and California's
Consumer Privacy Act (CCPA)~\cite{ca:privacy-act} codify users' right to be forgotten, and restrict
any data retention to to anonymized information.
%
Legal consequences and the reputational damage associated with data breaches~\cite{breach:amazon,
breach:twitter, breach:fb, breach:marriott, breach:quora} make it good practice to minimize the user
data retained at any point.
%

%
Although many developers are well-intentioned, they must today implement \emph{privacy
transformations} (such as user unsubscription) manually, which results in simplistic and ad-hoc
solutions (\S\ref{sec:survey}).
%
To get privacy transformations right, developers must carefully map the high-level privacy
policy to operations that delete or rewrite data objects, ensuring that the application preserves
utility for other users, retains legally-mandated anonymized data, and avoids violating
application invariants.
%
For example, a user's unsubscription should not allow another user to view content they could
not originally access, nor should it make non-sensitive shared content disappear.
%

%
Privacy transformations must correctly handle identifying data as well as subtly identifying
correlations between data objects.
%
For example, anonymized public running routes correlated with the same location can identify the
user's hometown and be reassociated with the user~\ms{cite Strava privacy issues?};
%anonymized posts on Reddit correlated with a subreddit with very few subscribers can be
%associated back to a single user;
papers' affiliation and reviewer conflicts in HotCRP can reidentify the author; and an anonymized
order history of an e-commerce site can reidentify the buyer.
%

%
The burden of implementing privacy transformations grows with the underlying privacy policy's
complexity.
%
But more nuanced privacy policies are important and useful!
%
Such policies can help protect users against data correlation attacks; they could give more
control to individuals by allowing them to choose their own, fine-grained privacy semantics; and
they might enable new privacy modes such as data that gradually becomes less identifyable over
time.
%
Likewise, a ``reversible'' privacy notion strikes a sweet spot: it might allow users to remove
identifiable information temporarily without deleting their account, helping service
providers make it easy for those users to return.
%
But the developer burden imposed by implementing the transformations required to enact these
policies is prohibitive today.
%

%We next describe a wide range of privacy transformations, some from existing applications' privacy policies,
%and others that demonstrate the potential for better, more nuanced privacy policies. Implementing
%these transformations using ad-hoc methods places undue labor on the developer and, as policies grow
%more complex, becomes more error-prone.

%
To systematically address these challenges, we propose \emph{data disguising}, a new framework
for specifying and implementing privacy transformations.
%
With data disguising, developers specify transformations required in privacy policies as
high-level \emph{data disguises} over existing application data types and associations.
%
Applying a disguise transforms the state of application data to, \eg delete or hide a users'
identifiers or decorrelate identifying object relationships, while preserving application
invariants and utility.
%
%Disguises consist of transformations performed on the high-level object graph embedded in
%database-based applications (encoded by \eg foreign key relationships in relational
%databases)~\cite{orms}.
%
A data disguising tool takes a disguise and its target, and automatically generates the
appropriate database transformations to achieve the disguised state, alleviating the developer
burden.
%


\section{The Need for Privacy Transformations}
\label{sec:survey}

%
We surveyed several widely-used web applications to understand what privacy-increasing operations
they apply on user unsubscription.
%
A set of common themes emerged.
%~\cite{facebook:privacy, twitter:privacy, hotcrp:privacy, reddit:privacy,
%github:privacy, hackernews:privacy, strava:privacy, linkedin:privacy, stackoverflow:privacy,
%wikipedia:privacy, amazon:privacy, prestashop:privacy, spotify:privacy, lobsters:privacy}:
%
Some services that publicly display user contributions (\eg Wikipedia edit
history~\cite{wikipedia:privacy}, StackOverflow answers~\cite{stackoverflow:privacy},
Strava routes~\cite{strava:privacy}) keep them publicly and indefinitely available even if a user deletes
their account.
%
Social networking platforms, which fundamentally thrive off users' \emph{shared} data, keep
contributions directly shared with another user unanonymized and visible to the recipient
(\eg Facebook/Twitter private messages~\cite{facebook:privacy, twitter:privacy},
LinkedIn updates~\cite{linkedin:privacy}).
%
Other platforms with mostly public content keep user contributions visible to the intended audience,
but anonymize them by reattributing the contribution to a placeholder user (\eg GitHub's
@ghost~\cite{github:privacy}, Reddit and Lobsters'
[deleted]~\cite{reddit:privacy, lobsters:privacy}).
%
%    \item Keep certain user contributions unanonymized and visible to its intended audience (\eg
%        HotCRP, Lobsters, Wikipedia, HackerNews~\cite{hotcrp:privacy, lobsters:privacy,
%        hackernews:privacy, wikipedia:privacy}).
%    \item Delete user contributions on user profile or feed (\eg Facebook,
%        Twitter~\cite{facebook:privacy, twitter:privacy}).
%\end{itemize}
%
All applications surveyed retain some information for legal or necessary business purposes (\eg
Spotify fraud detection~\cite{spotify:privacy}, PrestaShop/Amazon orders~\cite{amazon:privacy,
prestashop:privacy}).
%
For open-source applications, inspection shows that developers implement these transformations
via ad-hoc database operations, and only support explicit, user-user initiated account deletion
(a rare event).
%
Generally, developers appear to pay little attention to identifying correlations
within the remaining data.
%

%
We argue for a more \emph{systematic} treatment of privacy transformations, making them a
first-class citizen in application design.
%
In particular, we imagine developers declaratively specify the above and other transformation
policies for their application's data, similar to a relational schema.
%
This also allows for new policies and use cases that benefit both end-users and service operators.
%
Our \emph{data disguising} approach supports the following new policies and concepts,
which are missing from today's applications but easily described via disguises.
%

\paragraph{Nuanced policies.}
%
Users---and application developers---can benefit from more nuanced privacy policies.
%
For example, a confidential paper review system like HotCRP must keep a user's contributions
(papers, reviews) to preserve utility for others, but may associate each review with a different
placeholder to avoid accidentally revealing the unsubscribed reviewer's identity.
%
Likewise, contributions with a shared property (\eg posts on Reddit that share a common tag)
might be removed entirely to avoid inference attacks, or retained and decorrelated from the
property (\eg keeping the user's Reddit posts, but removing their tags).
%
Similar policies could apply \emph{only} if the property was created by the user (\eg keeping
the user's Reddit posts, but removing any user-created tags), or if the user's contributions
comprise more than a threshold percentage of the contributions with a shared property.
%(\eg remove the user's posts on Reddit with tag $t$ if these posts comprise more than 10\%
%of all posts with tag $t$).
%
Individual users may even specify different preferences for their data.
%
A privacy transformation framework is necessary to turn these preferences into concrete
operations without undue developer burden.
%

\paragraph{Data decomposition.}
%
Applications could go beyond simple account deletion and support a data expiration policy that
anonymizes a user's contributions after the user has been inactive for a period of time, and
restores the user's profile and contributions if the user ever logs back in.
%
Or the application could gradually ``decompose'' sensitive data by applying a series of
privacy transformations that incrementally remove more identifiable information from it as it
ages.
%

\paragraph{Reversibility.}
%
Many applications might wish to employ \emph{reversible} transformations, going beyond permanent
and irrevokable unsubscription.
%
After all, if services must allow users to remove their data on request, it's in the operator's
interest to make it easy for users to change their mind and return.
%
An advanced reversible transformation might, for example, record all actions performed in an
encrypted log, and offer that log for download or push it to third-party cloud storage.
%
If the user wishes to return, they simply supply the log and the transformation reverses.
%
To ensure access to the log even if the user loses their key, the transformation might
secret-share the encryption key~\cite{secretsharing} among the user, the service, and a trusted
third party (\eg the ACLU or EFF).
%
