%-------------------------------------------------------------------------------
\section{Introduction}
%-------------------------------------------------------------------------------
%\begin{figure*}[ht!]
%    \centering
%    \includegraphics[width=\textwidth]{img/worlds}
%
%    \caption{\textbf{(a)} The current web application paradigm, in which applications maintain and
%    own user data; \textbf{(b)} A paradigm that decouples user data from web applications, giving users ownership of their data;
%    \textbf{(c)} \name, which allows users to switch between privacy-preserving unsubscribed mode (right) and identity-revealing subscribed mode (left).}
%    \label{fig:world}
%\end{figure*}
%
Web applications today own, store, and sell user data, often without the user's knowledge or
explicit consent~\cite{nytimes:fb, npr:data}. This both violates users' right for privacy, and has
dangerous consequences for both users and application developers as data leaks lead to loss of
livelihoods and lawsuits~\cite{breach:amazon,breach:twitter, breach:fb, breach:marriott,
breach:quora}. 
%Granting web applications complete ownership of personal data clearly fails to
%protect users' privacy. 

Although users want stronger privacy, completely decoupling user data from applications results in a
potentially even less desirable world. While possible~\cite{solid, amber, w5, blockstack, bstore}, such a
model hinders service-side computation and application performance, and requires users to manage
long-time security and storage of their data, leading to a lack of adoption in practice (Section~\ref{sec:related}).  

This paper proposes \name, a new paradigm that grants users flexible privacy when using web
applications, balancing users' desire for privacy with their desire for application utility. In
\name, users subscribe to applications by granting a time-limited lease to their data, with the
provision that the application may retain only de-identified information once the user unsubscribes.
Users flexibly switch between a privacy-preserving unsubscribed mode and an identity-revealing
subscribed mode at any time without permanently losing their data. \name contrasts
with the current web application paradigm for data ownership, in which applications have complete
ownership, and the other extreme in which the user has complete data ownership.% (see Figure~\ref{fig:world}). 

\name benefits users: they can choose privacy at any time, without
permanently losing their accounts or affecting the utility of the applications for others.  Just as
importantly, \name also benefits application developers. Recent laws such as the
European Union's General Data Protection Regulation (GDPR)~\cite{eu:gdpr} and California's Consumer
Privacy Act (CCPA)~\cite{ca:privacy-act} codify users' rights to data ownership, granting users the
right to request erasure of information related to them. Supporting \name enables
applications to comply with these legal mandates, while still allowing its departing users to easily
come back: if applications must let users leave, it is in their best interest to make it easy for
them to return.  

Furthermore, applications can continue to operate using their current revenue model, maintaining
performance, reliability, and utility for their users.  Because applications retain use of
subscribed users' data, and de-identified data of unsubscribed users, applications optimize the
amount of data available to generate profit and provide utility for subscribed users. The
application holds only identifying data for subscribed users, reducing the amount of
compromising data in the system to only those users who have actively agreed to temporarily give up
their privacy.

Realizing \name poses a number of technical challenges.
Unsubscription and resubscription requires complex and fragile data transformations: developers must
selectively retain unsubscribed users' data for legal or application purposes while properly
de-identifying this data, a non-trivial task in the face of subtle inference attacks (\eg tags on a
user's post can identify the user). De-identification and data removal needs to be reversible,
allowing the user to resubscribe at any time to their last-known subscribed state.

To make \name a reality, we need new tools that allow developers to easily and systematically
automate the process for new and existing applications. As a first step, we design \sys, a practical
system that introduces \emph{ghost entity} and \emph{ghost policy} abstractions to help developers
of databased-backed web applications automatically achieve correct, privacy-compliant user unsubscription and
resubscription without onerous labor.
