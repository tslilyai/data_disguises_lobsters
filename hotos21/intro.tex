%-------------------------------------------------------------------------------
\section{Introduction}
%-------------------------------------------------------------------------------

Web application users are more aware of their data privacy rights than ever before. 
Laws such as the European Union's General Data Protection Regulation (GDPR)~\cite{eu:gdpr} and
California's Consumer Privacy Act (CCPA)~\cite{ca:privacy-act} codify users' rights to 
data ownership, granting users the right to request erasure of information related to them.
%
Public news around web services increasingly emphasize the dangers of leaving private data on the
web: embarrassing or compromising details from the past has reemerged, web applications have 
suffered data leaks or hacks, and private data has be shared by web applications without the user's
knowledge or explicit consent~\cite{nytimes:fb, npr:data}.

%
For the first time, companies face both legal and societal demand to support \emph{unsubscription}
of users from their services. Users have the right to unsubscribe at any time, preventing 
idle or unused accounts from retaining personal data indefinitely, and regaining control over who
can access their data. 
%

%
Unsubscription, however, is only half the story. Users cannot truly regain their data rights if they
are impeded from exercising them.  In order to claim that users have regained ownership of their
data, users must be able to \emph{freely} exercise their right to be forgotten. In particular, they should
not have to pay a high price in order in order to unsubscribe whenever they wish: web services
should not raise barriers to unsubscription that prevent users from choosing to do so, when they
would have otherwise. These barriers can take several forms, from UI choices that make
unsubscription cumbersome, to the threat of permanent deletion of years or decades of
accumulated application data.

To empower users to freely exercise their data rights, web applications must allow users to easily
unsubscribe \emph{and resubscribe} as they wish, switching between a privacy-preserving unsubscribed
mode and an identity-revealing subscribed mode at any time without permanently losing their data, or
needing to maintain their account data themselves. 
%
With this model of flexible privacy, the balance of data ownership flips. Applications no longer
control users' data choices by demanding that users choose between indefinitely giving up their data
to use the application, or permanently losing their data and application utility.
Instead, users lease out their data for limited periods of time in exchange for the ability to use the application for 
that period, and lose no application utility between leases.
%\lyt{Users could do this without
%application support by themselves by programmatically setting a personal reminder to ``unsubscribe
%after x days'' every time they create an account.}
%

In the rest of this paper, we describe the current unsubscription and resubscription support (and
lack of support) in today's web applications, and the challenges of doing so correctly.
We then describe \sys, a system that provides abstractions and mechanisms
that help developers of databased-backed web applications achieve correct,
privacy-compliant user unsubscription and resubscription without onerous labor, and without adding
undue overheads.
