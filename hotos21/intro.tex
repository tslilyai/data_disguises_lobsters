%-------------------------------------------------------------------------------
\section{Introduction}
%-------------------------------------------------------------------------------

Web application users are more aware of their data privacy rights than ever before. 
Laws such as the European Union's General Data Protection Regulation (GDPR)~\cite{eu:gdpr} and
California's Consumer Privacy Act (CCPA)~\cite{ca:privacy-act} codify users' rights to 
data ownership, granting users the right to request erasure of information related to them.
%
Public news around web services increasingly emphasize the dangers of leaving private data on the
web: embarrassing or compromising details from the past has reemerged, web applications have
suffered data leaks or hacks~\cite{breach:twitter, breach:fb, breach:marriott, breach:quora}, and
private data has be shared by web applications without the user's knowledge or explicit
consent~\cite{nytimes:fb, npr:data}.

%
These legal and societal pressures increasingly demand that companies support \emph{unsubscription}
of users from their services.  Users should have the right to unsubscribe at any time, preventing
idle or unused accounts from retaining personal data indefinitely, and regaining control over who
can access their data. 
%

%
Unsubscription, however, is only half the story. To empower users to freely exercise their data
rights, web applications must allow users to easily unsubscribe \emph{and resubscribe} as they wish,
switching between a privacy-preserving unsubscribed mode and an identity-revealing subscribed mode
at any time without permanently losing their data, or needing to maintain their account data
themselves.  Unsubscription without resubscription unduly impedes users from exercising their right
to be forgotten: in order to claim that users have regained ownership of their data, users should
not have to pay a high price in order in order to unsubscribe whenever they wish. They should not
encounter barriers, such as permanently losing their data, that prevent them from choosing to
unsubscribe when they would have otherwise. 

With this model of flexible privacy, the balance of data ownership flips. Applications no longer
control users' data choices by demanding that users choose between indefinitely giving up their data
to use the application, or permanently losing their data and application utility.
Instead, users control when applications can obtain their data in exchange for the right to use the application, and do not permanently lose application utility when they remove access to their data.

Unfortunately, today's web services face many challenges that make free and easy unsubscription and
resubscription hard. Many do not support automated unsubscription, requiring users to 
contact the developers or customer support directly in order to unsubscribe.
Developers must manually implement unsubscription, leading to coarse-grained, error-prone data
handling that can fail to completely de-identify the user. Many applications require that some
information remains post-unsubscription, for legal or necessary application use, but retained data
can identify a user in complex ways: both user identifiers (\eg usernames) and structural
\emph{correlations} between application data records (\eg between users and posts, or posts and
tags) may reveal identifying information.  Without a systematic way to anonymize user identifiers
and \emph{decorrelate} these potentially identifying structural correlations, many developers have
chosen to either entirely delete all sensitive data at the expense of other subscribed users, or
leave identifying information unchanged at the expense of unsubscribed users.

Even if de-identification were easy, services that do provide some amount of de-identification upon
unsubscription fail to support resubscription, thus permanently deleting years or even decades of
accumulated application data.  This permanent loss of user data costs both users and the web
service, and creates little incentive to lower barriers to unsubscription.  

Legal penalties today create greater incentives to remove unnecessary data than ever before.
However, these incentives alone are not enough. The challenges and complexity of
correct unsubscription highlight the need for new tools that allow developers to easily and
systematically express how application data and structural correlations need to change when a user
unsubscribes, and support automatic resubscription and recorrelation of users with their account
data at any time.
%

In the rest of this paper, we describe the current support for unsubscription and resubscription in
today's web applications, and the challenges of implementing it correctly.  We then describe \sys, a
system that provides abstractions and mechanisms to help developers of databased-backed web
applications achieve correct, privacy-compliant user unsubscription and resubscription without
onerous labor, and without adding undue overheads.
